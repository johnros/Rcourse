\documentclass[]{book}
\usepackage{lmodern}
\usepackage{amssymb,amsmath}
\usepackage{ifxetex,ifluatex}
\usepackage{fixltx2e} % provides \textsubscript
\ifnum 0\ifxetex 1\fi\ifluatex 1\fi=0 % if pdftex
  \usepackage[T1]{fontenc}
  \usepackage[utf8]{inputenc}
\else % if luatex or xelatex
  \ifxetex
    \usepackage{mathspec}
  \else
    \usepackage{fontspec}
  \fi
  \defaultfontfeatures{Ligatures=TeX,Scale=MatchLowercase}
\fi
% use upquote if available, for straight quotes in verbatim environments
\IfFileExists{upquote.sty}{\usepackage{upquote}}{}
% use microtype if available
\IfFileExists{microtype.sty}{%
\usepackage{microtype}
\UseMicrotypeSet[protrusion]{basicmath} % disable protrusion for tt fonts
}{}
\usepackage[unicode=true]{hyperref}
\hypersetup{
            pdftitle={R (BGU course)},
            pdfauthor={Jonathan D. Rosenblatt},
            pdfborder={0 0 0},
            breaklinks=true}
\urlstyle{same}  % don't use monospace font for urls
\usepackage{natbib}
\bibliographystyle{apalike}
\usepackage{color}
\usepackage{fancyvrb}
\newcommand{\VerbBar}{|}
\newcommand{\VERB}{\Verb[commandchars=\\\{\}]}
\DefineVerbatimEnvironment{Highlighting}{Verbatim}{commandchars=\\\{\}}
% Add ',fontsize=\small' for more characters per line
\usepackage{framed}
\definecolor{shadecolor}{RGB}{248,248,248}
\newenvironment{Shaded}{\begin{snugshade}}{\end{snugshade}}
\newcommand{\KeywordTok}[1]{\textcolor[rgb]{0.13,0.29,0.53}{\textbf{{#1}}}}
\newcommand{\DataTypeTok}[1]{\textcolor[rgb]{0.13,0.29,0.53}{{#1}}}
\newcommand{\DecValTok}[1]{\textcolor[rgb]{0.00,0.00,0.81}{{#1}}}
\newcommand{\BaseNTok}[1]{\textcolor[rgb]{0.00,0.00,0.81}{{#1}}}
\newcommand{\FloatTok}[1]{\textcolor[rgb]{0.00,0.00,0.81}{{#1}}}
\newcommand{\ConstantTok}[1]{\textcolor[rgb]{0.00,0.00,0.00}{{#1}}}
\newcommand{\CharTok}[1]{\textcolor[rgb]{0.31,0.60,0.02}{{#1}}}
\newcommand{\SpecialCharTok}[1]{\textcolor[rgb]{0.00,0.00,0.00}{{#1}}}
\newcommand{\StringTok}[1]{\textcolor[rgb]{0.31,0.60,0.02}{{#1}}}
\newcommand{\VerbatimStringTok}[1]{\textcolor[rgb]{0.31,0.60,0.02}{{#1}}}
\newcommand{\SpecialStringTok}[1]{\textcolor[rgb]{0.31,0.60,0.02}{{#1}}}
\newcommand{\ImportTok}[1]{{#1}}
\newcommand{\CommentTok}[1]{\textcolor[rgb]{0.56,0.35,0.01}{\textit{{#1}}}}
\newcommand{\DocumentationTok}[1]{\textcolor[rgb]{0.56,0.35,0.01}{\textbf{\textit{{#1}}}}}
\newcommand{\AnnotationTok}[1]{\textcolor[rgb]{0.56,0.35,0.01}{\textbf{\textit{{#1}}}}}
\newcommand{\CommentVarTok}[1]{\textcolor[rgb]{0.56,0.35,0.01}{\textbf{\textit{{#1}}}}}
\newcommand{\OtherTok}[1]{\textcolor[rgb]{0.56,0.35,0.01}{{#1}}}
\newcommand{\FunctionTok}[1]{\textcolor[rgb]{0.00,0.00,0.00}{{#1}}}
\newcommand{\VariableTok}[1]{\textcolor[rgb]{0.00,0.00,0.00}{{#1}}}
\newcommand{\ControlFlowTok}[1]{\textcolor[rgb]{0.13,0.29,0.53}{\textbf{{#1}}}}
\newcommand{\OperatorTok}[1]{\textcolor[rgb]{0.81,0.36,0.00}{\textbf{{#1}}}}
\newcommand{\BuiltInTok}[1]{{#1}}
\newcommand{\ExtensionTok}[1]{{#1}}
\newcommand{\PreprocessorTok}[1]{\textcolor[rgb]{0.56,0.35,0.01}{\textit{{#1}}}}
\newcommand{\AttributeTok}[1]{\textcolor[rgb]{0.77,0.63,0.00}{{#1}}}
\newcommand{\RegionMarkerTok}[1]{{#1}}
\newcommand{\InformationTok}[1]{\textcolor[rgb]{0.56,0.35,0.01}{\textbf{\textit{{#1}}}}}
\newcommand{\WarningTok}[1]{\textcolor[rgb]{0.56,0.35,0.01}{\textbf{\textit{{#1}}}}}
\newcommand{\AlertTok}[1]{\textcolor[rgb]{0.94,0.16,0.16}{{#1}}}
\newcommand{\ErrorTok}[1]{\textcolor[rgb]{0.64,0.00,0.00}{\textbf{{#1}}}}
\newcommand{\NormalTok}[1]{{#1}}
\usepackage{longtable,booktabs}
\usepackage{graphicx,grffile}
\makeatletter
\def\maxwidth{\ifdim\Gin@nat@width>\linewidth\linewidth\else\Gin@nat@width\fi}
\def\maxheight{\ifdim\Gin@nat@height>\textheight\textheight\else\Gin@nat@height\fi}
\makeatother
% Scale images if necessary, so that they will not overflow the page
% margins by default, and it is still possible to overwrite the defaults
% using explicit options in \includegraphics[width, height, ...]{}
\setkeys{Gin}{width=\maxwidth,height=\maxheight,keepaspectratio}
\IfFileExists{parskip.sty}{%
\usepackage{parskip}
}{% else
\setlength{\parindent}{0pt}
\setlength{\parskip}{6pt plus 2pt minus 1pt}
}
\setlength{\emergencystretch}{3em}  % prevent overfull lines
\providecommand{\tightlist}{%
  \setlength{\itemsep}{0pt}\setlength{\parskip}{0pt}}
\setcounter{secnumdepth}{5}
% Redefines (sub)paragraphs to behave more like sections
\ifx\paragraph\undefined\else
\let\oldparagraph\paragraph
\renewcommand{\paragraph}[1]{\oldparagraph{#1}\mbox{}}
\fi
\ifx\subparagraph\undefined\else
\let\oldsubparagraph\subparagraph
\renewcommand{\subparagraph}[1]{\oldsubparagraph{#1}\mbox{}}
\fi
\usepackage{booktabs}
\usepackage{amsthm}

\usepackage[margin=0.7in]{geometry}
\usepackage{fancyhdr}
\pagestyle{fancy}


\makeatletter
\def\thm@space@setup{%
  \thm@preskip=8pt plus 2pt minus 4pt
  \thm@postskip=\thm@preskip
}
\makeatother

\title{R (BGU course)}
\author{Jonathan D. Rosenblatt}
\date{2017-04-23}

\usepackage{amsthm}
\newtheorem{theorem}{Theorem}[chapter]
\newtheorem{lemma}{Lemma}[chapter]
\theoremstyle{definition}
\newtheorem{definition}{Definition}[chapter]
\newtheorem{corollary}{Corollary}[chapter]
\newtheorem{proposition}{Proposition}[chapter]
\theoremstyle{definition}
\newtheorem{example}{Example}[chapter]
\theoremstyle{remark}
\newtheorem*{remark}{Remark}
\let\BeginKnitrBlock\begin \let\EndKnitrBlock\end
\begin{document}
\maketitle

{
\setcounter{tocdepth}{1}
\tableofcontents
}
\chapter{Preface}\label{preface}

This book accompanies BGU's ``R'' course, at the department of
Industrial Engineering and Management. It has several purposes:

\begin{itemize}
\tightlist
\item
  Help me organize and document the course material.
\item
  Help students during class so that they may focus on listening and not
  writing.
\item
  Help students after class, so that they may self-study.
\end{itemize}

At its current state it is experimental. It can thus be expected to
change from time to time, and include mistakes. I will be enormously
grateful to whoever decides to share with me any mistakes found.

I am enormously grateful to Yihui Xie, who's \emph{bookdown} R package
made it possible to easily write a book which has many mathematical
formulae, and R output.

I hope the reader will find this text interesting and useful.

For reproducing my results you will want to run \texttt{set.seed(1)}.

\section{Notation Conventions}\label{notation-conventions}

In this text we use the following conventions: Lower case \(x\) may be a
vector or a scalar, random of fixed, as implied by the context. Upper
case \(A\) will stand for matrices. Equality \(=\) is an equality, and
\(:=\) is a definition. Norm functions are denoted with
\(\Vert x \Vert\) for vector norms, and \(\Vert A \Vert\) for matrix
norms. The type of norm is indicated in the subscript; e.g.
\(\Vert x \Vert_2\) for the Euclidean (\(l_2\)) norm. Tag, \(x'\) is a
transpose. The distribution of a random vector is \(\sim\).

\section{Acknowledgements}\label{acknowledgements}

I have consulted many people during the writing of this text. I would
like to thank \href{https://kesslerlab.wordpress.com/}{Yoav Kessler},
\href{http://fohs.bgu.ac.il/research/profileBrief.aspx?id=VeeMVried}{Lena
Novack}, Efrat Vilenski, Ron Sarfian, and Liad Shekel in particular, for
their valuable inputs.

\chapter{Introduction}\label{intro}

\section{What is R?}\label{what-r}

R was not designed to be a bona-fide programming language. It is an
evolution of the S language, developed at Bell labs (later Lucent) as a
wrapper for the endless collection of statistical libraries they wrote
in Fortran.

As of 2011, half of R's libraries are
\href{https://wrathematics.github.io/2011/08/27/how-much-of-r-is-written-in-r/}{actually
written in C}.

For more on the history of R see
\href{http://www.research.att.com/articles/featured_stories/2013_09/201309_SandR.html?fbid=Yxy4qyQzmMa}{AT\&T's
site}, John Chamber's talk at
\href{https://www.youtube.com/watch?v=_hcpuRB5nGs}{UserR! 2014} or the
Introduction to the excellent \citet{venables2013modern}.

\hypertarget{ecosystem}{\section{The R Ecosystem}\label{ecosystem}}

A large part of R's success is due to the ease in which a user, or a
firm, can augment it. This led to a large community of users,
developers, and protagonists. Some of the most important parts of R's
ecosystem include:

\begin{itemize}
\item
  \href{https://cran.r-project.org/}{CRAN}: a repository for R packages,
  mirrored worldwide.
\item
  \href{https://www.r-project.org/mail.html}{R-help}: an immensely
  active mailing list. Noways being replaced by StackExchange meta-site.
  Look for the R tags in the
  \href{http://stackoverflow.com/}{StackOverflow} and
  \href{http://stats.stackexchange.com/}{CrossValidated} sites.
\item
  \href{https://cran.r-project.org/web/views/}{TakViews}: part of CRAN
  that collects packages per topic.
\item
  \href{https://www.bioconductor.org/}{Bioconductor}: A CRAN-like
  repository dedicated to the life sciences.
\item
  \href{https://www.neuroconductor.org/}{Neuroconductor}: A CRAN-like
  repository dedicated to neuroscience, and neuroimaging.
\item
  \href{https://www.r-project.org/doc/bib/R-books.html}{Books}: An
  insane amount of books written on the language. Some are free, some
  are not.
\item
  \href{https://groups.google.com/forum/\#!forum/israel-r-user-group}{The
  Israeli-R-user-group}: just like the name suggests.
\item
  Commercial R: being open source and lacking support may seem like a
  problem that would prohibit R from being adopted for commercial
  applications. This void is filled by several very successful
  commercial versions such as
  \href{https://mran.microsoft.com/open/}{Microsoft R}, with its
  accompanying CRAN equivalent called
  \href{https://mran.microsoft.com/}{MRAN},
  \href{http://spotfire.tibco.com/discover-spotfire/what-does-spotfire-do/predictive-analytics/tibco-enterprise-runtime-for-r-terr}{Tibco's
  Spotfire}, and
  \href{https://en.wikipedia.org/wiki/R_(programming_language)\#Commercial_support_for_R}{others}.
\item
  \href{https://www.rstudio.com/products/rstudio/download-server/}{RStudio}:
  since its earliest days R came equipped with a minimal text editor. It
  later received plugins for major integrated development environments
  (IDEs) such as Eclipse, WinEdit and even
  \href{https://www.visualstudio.com/vs/rtvs/}{VisualStudio}. None of
  these, however, had the impact of the RStudio IDE. Written completely
  in JavaScript, the RStudio IDE allows the seamless integration of
  cutting edge web-design technologies, remote access, and other killer
  features, making it today's most popular IDE for R.
\end{itemize}

\section{Bibliographic Notes}\label{bibliographic-notes}

\section{Practice Yourself}\label{practice-yourself}

\chapter{R Basics}\label{basics}

We now start with the basics of R. If you have any experience at all
with R, you can probably skip this section.

First, make sure you work with the RStudio IDE. Some useful pointers for
this IDE include:

\begin{itemize}
\tightlist
\item
  Ctrl+Return to run lines from editor.
\item
  Alt+Shift+k for RStudio keyboard shortcuts.
\item
  Alt+Shift+j to navigate between sections
\item
  tab for auto-completion
\item
  Ctrl+1 to skip to editor.
\item
  Ctrl+2 to skip to console.
\item
  Ctrl+8 to skip to the environment list.
\item
  Code Folding:

  \begin{itemize}
  \tightlist
  \item
    Alt+l collapse chunk.
  \item
    Alt+Shift+l unfold chunk.
  \item
    Alt+o collapse all.
  \item
    Alt+Shift+o unfold all.
  \end{itemize}
\end{itemize}

\section{Simple calculator}\label{simple-calculator}

R can be used as a simple calculator.

\begin{Shaded}
\begin{Highlighting}[]
\DecValTok{10+5}
\end{Highlighting}
\end{Shaded}

\begin{verbatim}
## [1] 15
\end{verbatim}

\begin{Shaded}
\begin{Highlighting}[]
\DecValTok{70}\NormalTok{*}\DecValTok{81}
\end{Highlighting}
\end{Shaded}

\begin{verbatim}
## [1] 5670
\end{verbatim}

\begin{Shaded}
\begin{Highlighting}[]
\DecValTok{2}\NormalTok{**}\DecValTok{4}
\end{Highlighting}
\end{Shaded}

\begin{verbatim}
## [1] 16
\end{verbatim}

\begin{Shaded}
\begin{Highlighting}[]
\DecValTok{2}\NormalTok{^}\DecValTok{4}
\end{Highlighting}
\end{Shaded}

\begin{verbatim}
## [1] 16
\end{verbatim}

\begin{Shaded}
\begin{Highlighting}[]
\KeywordTok{log}\NormalTok{(}\DecValTok{10}\NormalTok{)                         }
\end{Highlighting}
\end{Shaded}

\begin{verbatim}
## [1] 2.302585
\end{verbatim}

\begin{Shaded}
\begin{Highlighting}[]
\KeywordTok{log}\NormalTok{(}\DecValTok{16}\NormalTok{, }\DecValTok{2}\NormalTok{)                      }
\end{Highlighting}
\end{Shaded}

\begin{verbatim}
## [1] 4
\end{verbatim}

\begin{Shaded}
\begin{Highlighting}[]
\KeywordTok{log}\NormalTok{(}\DecValTok{1000}\NormalTok{, }\DecValTok{10}\NormalTok{)                   }
\end{Highlighting}
\end{Shaded}

\begin{verbatim}
## [1] 3
\end{verbatim}

\section{Probability calculator}\label{probability-calculator}

R can be used as a probability calculator. You probably wish you knew
this when you did your Intro To Probability.

The binomial distribution function:

\begin{Shaded}
\begin{Highlighting}[]
\KeywordTok{dbinom}\NormalTok{(}\DataTypeTok{x=}\DecValTok{3}\NormalTok{, }\DataTypeTok{size=}\DecValTok{10}\NormalTok{, }\DataTypeTok{prob=}\FloatTok{0.5}\NormalTok{)  }\CommentTok{# Compute P(X=3) for X~B(n=10, p=0.5) }
\end{Highlighting}
\end{Shaded}

\begin{verbatim}
## [1] 0.1171875
\end{verbatim}

Notice that arguments do not need to be named explicitly

\begin{Shaded}
\begin{Highlighting}[]
\KeywordTok{dbinom}\NormalTok{(}\DecValTok{3}\NormalTok{, }\DecValTok{10}\NormalTok{, }\FloatTok{0.5}\NormalTok{)}
\end{Highlighting}
\end{Shaded}

\begin{verbatim}
## [1] 0.1171875
\end{verbatim}

The binomial cumulative distribution function (CDF):

\begin{Shaded}
\begin{Highlighting}[]
\KeywordTok{pbinom}\NormalTok{(}\DataTypeTok{q=}\DecValTok{3}\NormalTok{, }\DataTypeTok{size=}\DecValTok{10}\NormalTok{, }\DataTypeTok{prob=}\FloatTok{0.5}\NormalTok{) }\CommentTok{# Compute P(X<=3) for X~B(n=10, p=0.5)   }
\end{Highlighting}
\end{Shaded}

\begin{verbatim}
## [1] 0.171875
\end{verbatim}

The binomial quantile function:

\begin{Shaded}
\begin{Highlighting}[]
\KeywordTok{qbinom}\NormalTok{(}\DataTypeTok{p=}\FloatTok{0.1718}\NormalTok{, }\DataTypeTok{size=}\DecValTok{10}\NormalTok{, }\DataTypeTok{prob=}\FloatTok{0.5}\NormalTok{) }\CommentTok{# For X~B(n=10, p=0.5) returns k such that P(X<=k)=0.1718}
\end{Highlighting}
\end{Shaded}

\begin{verbatim}
## [1] 3
\end{verbatim}

Generate random variables:

\begin{Shaded}
\begin{Highlighting}[]
\KeywordTok{rbinom}\NormalTok{(}\DataTypeTok{n=}\DecValTok{10}\NormalTok{, }\DataTypeTok{size=}\DecValTok{10}\NormalTok{, }\DataTypeTok{prob=}\FloatTok{0.5}\NormalTok{)}
\end{Highlighting}
\end{Shaded}

\begin{verbatim}
##  [1] 4 4 5 7 4 7 7 6 6 3
\end{verbatim}

R has many built-in distributions. Their names may change, but the
prefixes do not:

\begin{itemize}
\tightlist
\item
  \textbf{d} prefix for the \emph{distribution} function.
\item
  \textbf{p} prefix for the \emph{cummulative distribution} function
  (CDF).
\item
  \textbf{q} prefix for the \emph{quantile} function (i.e., the inverse
  CDF).
\item
  \textbf{r} prefix to generate random samples.
\end{itemize}

Demonstrating this idea, using the CDF of several popular distributions:

\begin{itemize}
\tightlist
\item
  \texttt{pbinom()} for the binomial CDF.
\item
  \texttt{ppois()} for the Poisson CDF.
\item
  \texttt{pnorm()} for the Gaussian CDF.
\item
  \texttt{pexp()} for the exponential CDF.
\end{itemize}

For more information see \texttt{?distributions}.

\section{Getting Help}\label{getting-help}

One of the most important parts of working with a language, is to know
where to find help. R has several in-line facilities, besides the
various help resources in the R
\protect\hyperlink{ecosystem}{ecosystem}.

Get help for a particular function.

\begin{Shaded}
\begin{Highlighting}[]
\NormalTok{?dbinom }
\KeywordTok{help}\NormalTok{(dbinom)}
\end{Highlighting}
\end{Shaded}

If you don't know the name of the function you are looking for, search
local help files for a particular string:

\begin{Shaded}
\begin{Highlighting}[]
\NormalTok{??binomial}
\KeywordTok{help.search}\NormalTok{(}\StringTok{'dbinom'}\NormalTok{) }
\end{Highlighting}
\end{Shaded}

Or load a menu where you can navigate local help in a web-based fashion:

\begin{Shaded}
\begin{Highlighting}[]
\KeywordTok{help.start}\NormalTok{() }
\end{Highlighting}
\end{Shaded}

\section{Variable Asignment}\label{variable-asignment}

Assignment of some output into an object named ``x'':

\begin{Shaded}
\begin{Highlighting}[]
\NormalTok{x =}\StringTok{ }\KeywordTok{rbinom}\NormalTok{(}\DataTypeTok{n=}\DecValTok{10}\NormalTok{, }\DataTypeTok{size=}\DecValTok{10}\NormalTok{, }\DataTypeTok{prob=}\FloatTok{0.5}\NormalTok{) }\CommentTok{# Works. Bad style.}
\NormalTok{x <-}\StringTok{ }\KeywordTok{rbinom}\NormalTok{(}\DataTypeTok{n=}\DecValTok{10}\NormalTok{, }\DataTypeTok{size=}\DecValTok{10}\NormalTok{, }\DataTypeTok{prob=}\FloatTok{0.5}\NormalTok{) }
\end{Highlighting}
\end{Shaded}

If you are familiar with other programming languages you may prefer the
\texttt{=} assignment rather than the \texttt{\textless{}-} assignment.
We recommend you make the effort to change your preferences. This is
because thinking with \texttt{\textless{}-} helps to read your code,
distinguishes between assignments and function arguments: think of
\texttt{function(argument=value)} versus
\texttt{function(argument\textless{}-value)}. It also helps understand
special assignment operators such as \texttt{\textless{}\textless{}-}
and \texttt{-\textgreater{}}.

\BeginKnitrBlock{remark}
\iffalse {Remark. } \fi \textbf{Style}: We do not discuss style
guidelines in this text, but merely remind the reader that good style is
extremely important. When you write code, think of other readers, but
also think of future self. See
\href{http://adv-r.had.co.nz/Style.html}{Hadley's style guide} for more.
\EndKnitrBlock{remark}

To print the contents of an object just type its name

\begin{Shaded}
\begin{Highlighting}[]
\NormalTok{x}
\end{Highlighting}
\end{Shaded}

\begin{verbatim}
##  [1] 7 4 6 3 4 5 2 5 7 4
\end{verbatim}

which is an implicit call to

\begin{Shaded}
\begin{Highlighting}[]
\KeywordTok{print}\NormalTok{(x)  }
\end{Highlighting}
\end{Shaded}

\begin{verbatim}
##  [1] 7 4 6 3 4 5 2 5 7 4
\end{verbatim}

Alternatively, you can assign and print simultaneously using
parenthesis.

\begin{Shaded}
\begin{Highlighting}[]
\NormalTok{(x <-}\StringTok{ }\KeywordTok{rbinom}\NormalTok{(}\DataTypeTok{n=}\DecValTok{10}\NormalTok{, }\DataTypeTok{size=}\DecValTok{10}\NormalTok{, }\DataTypeTok{prob=}\FloatTok{0.5}\NormalTok{))  }\CommentTok{# Assign and print.}
\end{Highlighting}
\end{Shaded}

\begin{verbatim}
##  [1] 5 5 5 4 6 6 6 3 6 5
\end{verbatim}

Operate on the object

\begin{Shaded}
\begin{Highlighting}[]
\KeywordTok{mean}\NormalTok{(x)  }\CommentTok{# compute mean}
\end{Highlighting}
\end{Shaded}

\begin{verbatim}
## [1] 5.1
\end{verbatim}

\begin{Shaded}
\begin{Highlighting}[]
\KeywordTok{var}\NormalTok{(x)  }\CommentTok{# compute variance}
\end{Highlighting}
\end{Shaded}

\begin{verbatim}
## [1] 0.9888889
\end{verbatim}

\begin{Shaded}
\begin{Highlighting}[]
\KeywordTok{hist}\NormalTok{(x) }\CommentTok{# plot histogram}
\end{Highlighting}
\end{Shaded}

\includegraphics[width=0.5\linewidth]{Rcourse_files/figure-latex/unnamed-chunk-15-1}

R saves every object you create in RAM\footnote{S and S-Plus used to
  save objects on disk. Working from RAM has advantages and
  disadvantages. More on this in Chapter \ref{memory}.}. The collection
of all such objects is the \textbf{workspace} which you can inspect with

\begin{Shaded}
\begin{Highlighting}[]
\KeywordTok{ls}\NormalTok{()}
\end{Highlighting}
\end{Shaded}

\begin{verbatim}
## [1] "x"
\end{verbatim}

or with Ctrl+8 in RStudio.

If you lost your object, you can use \texttt{ls} with a text pattern to
search for it

\begin{Shaded}
\begin{Highlighting}[]
\KeywordTok{ls}\NormalTok{(}\DataTypeTok{pattern=}\StringTok{'x'}\NormalTok{)}
\end{Highlighting}
\end{Shaded}

\begin{verbatim}
## [1] "x"
\end{verbatim}

To remove objects from the workspace:

\begin{Shaded}
\begin{Highlighting}[]
\KeywordTok{rm}\NormalTok{(x) }\CommentTok{# remove variable}
\KeywordTok{ls}\NormalTok{() }\CommentTok{# verify}
\end{Highlighting}
\end{Shaded}

\begin{verbatim}
## character(0)
\end{verbatim}

You may think that if an object is removed then its memory is freed.
This is almost true, and depends on a negotiation mechanism between R
and the operating system. R's memory management is discussed in Chapter
\ref{memory}.

\section{Piping}\label{piping}

Because R originates in Unix and Linux environments, it inherits much of
its flavor.
\href{http://ryanstutorials.net/linuxtutorial/piping.php}{Piping} is an
idea taken from the Linux shell which allows to use the output of one
expression as the input to another. Piping thus makes code easier to
read and write.

\BeginKnitrBlock{remark}
\iffalse {Remark. } \fi Volleyball fans may be confused with the idea of
spiking a ball from the 3-meter line, also called
\href{https://www.youtube.com/watch?v=GWW15Nr1lQM}{piping}. So: (a)
These are very different things. (b) If you can pipe,
\href{http://in.bgu.ac.il/sport/Pages/asa.aspx}{ASA-BGU} is looking for
you!
\EndKnitrBlock{remark}

Prerequisites:

\begin{Shaded}
\begin{Highlighting}[]
\KeywordTok{library}\NormalTok{(magrittr) }\CommentTok{# load the piping functions}
\NormalTok{x <-}\StringTok{ }\KeywordTok{rbinom}\NormalTok{(}\DataTypeTok{n=}\DecValTok{1000}\NormalTok{, }\DataTypeTok{size=}\DecValTok{10}\NormalTok{, }\DataTypeTok{prob=}\FloatTok{0.5}\NormalTok{) }\CommentTok{# generate some toy data}
\end{Highlighting}
\end{Shaded}

Examples

\begin{Shaded}
\begin{Highlighting}[]
\NormalTok{x %>%}\StringTok{ }\KeywordTok{var}\NormalTok{() }\CommentTok{# Instead of var(x)}
\NormalTok{x %>%}\StringTok{ }\KeywordTok{hist}\NormalTok{()  }\CommentTok{# Instead of hist(x)}
\NormalTok{x %>%}\StringTok{ }\KeywordTok{mean}\NormalTok{() %>%}\StringTok{ }\KeywordTok{round}\NormalTok{(}\DecValTok{2}\NormalTok{) %>%}\StringTok{ }\KeywordTok{add}\NormalTok{(}\DecValTok{10}\NormalTok{) }
\end{Highlighting}
\end{Shaded}

The next example\footnote{Taken from
  \url{http://cran.r-project.org/web/packages/magrittr/vignettes/magrittr.html}}
demonstrates the benefits of piping. The next two chunks of code do the
same thing. Try parsing them in your mind:

\begin{Shaded}
\begin{Highlighting}[]
\CommentTok{# Functional (onion) style}
\NormalTok{car_data <-}\StringTok{ }
\StringTok{  }\KeywordTok{transform}\NormalTok{(}\KeywordTok{aggregate}\NormalTok{(. ~}\StringTok{ }\NormalTok{cyl, }
                      \DataTypeTok{data =} \KeywordTok{subset}\NormalTok{(mtcars, hp >}\StringTok{ }\DecValTok{100}\NormalTok{), }
                      \DataTypeTok{FUN =} \NormalTok{function(x) }\KeywordTok{round}\NormalTok{(}\KeywordTok{mean}\NormalTok{(x, }\DecValTok{2}\NormalTok{))), }
            \DataTypeTok{kpl =} \NormalTok{mpg*}\FloatTok{0.4251}\NormalTok{)}
\end{Highlighting}
\end{Shaded}

\begin{Shaded}
\begin{Highlighting}[]
\CommentTok{# Piping (magrittr) style}
\NormalTok{car_data <-}\StringTok{ }
\StringTok{  }\NormalTok{mtcars %>%}
\StringTok{  }\KeywordTok{subset}\NormalTok{(hp >}\StringTok{ }\DecValTok{100}\NormalTok{) %>%}
\StringTok{  }\KeywordTok{aggregate}\NormalTok{(. ~}\StringTok{ }\NormalTok{cyl, }\DataTypeTok{data =} \NormalTok{., }\DataTypeTok{FUN =} \NormalTok{. %>%}\StringTok{ }\NormalTok{mean %>%}\StringTok{ }\KeywordTok{round}\NormalTok{(}\DecValTok{2}\NormalTok{)) %>%}
\StringTok{  }\KeywordTok{transform}\NormalTok{(}\DataTypeTok{kpl =} \NormalTok{mpg %>%}\StringTok{ }\KeywordTok{multiply_by}\NormalTok{(}\FloatTok{0.4251}\NormalTok{)) %>%}
\StringTok{  }\NormalTok{print}
\end{Highlighting}
\end{Shaded}

Tip: RStudio has a keyboard shortcut for the \texttt{\%\textgreater{}\%}
operator. Try Ctrl+Shift+m.

\section{Vector Creation and
Manipulation}\label{vector-creation-and-manipulation}

The most basic building block in R is the \textbf{vector}. We will now
see how to create them, and access their elements (i.e.~subsetting).
Here are three ways to create the same arbitrary vector:

\begin{Shaded}
\begin{Highlighting}[]
\KeywordTok{c}\NormalTok{(}\DecValTok{10}\NormalTok{, }\DecValTok{11}\NormalTok{, }\DecValTok{12}\NormalTok{, }\DecValTok{13}\NormalTok{, }\DecValTok{14}\NormalTok{, }\DecValTok{15}\NormalTok{, }\DecValTok{16}\NormalTok{, }\DecValTok{17}\NormalTok{, }\DecValTok{18}\NormalTok{, }\DecValTok{19}\NormalTok{, }\DecValTok{20}\NormalTok{, }\DecValTok{21}\NormalTok{) }\CommentTok{# manually}
\DecValTok{10}\NormalTok{:}\DecValTok{21} \CommentTok{# the `:` operator                            }
\KeywordTok{seq}\NormalTok{(}\DataTypeTok{from=}\DecValTok{10}\NormalTok{, }\DataTypeTok{to=}\DecValTok{21}\NormalTok{, }\DataTypeTok{by=}\DecValTok{1}\NormalTok{) }\CommentTok{# the seq() function}
\end{Highlighting}
\end{Shaded}

Let's assign it to the object named ``x'':

\begin{Shaded}
\begin{Highlighting}[]
\NormalTok{x <-}\StringTok{ }\KeywordTok{c}\NormalTok{(}\DecValTok{10}\NormalTok{, }\DecValTok{11}\NormalTok{, }\DecValTok{12}\NormalTok{, }\DecValTok{13}\NormalTok{, }\DecValTok{14}\NormalTok{, }\DecValTok{15}\NormalTok{, }\DecValTok{16}\NormalTok{, }\DecValTok{17}\NormalTok{, }\DecValTok{18}\NormalTok{, }\DecValTok{19}\NormalTok{, }\DecValTok{20}\NormalTok{, }\DecValTok{21}\NormalTok{)  }
\end{Highlighting}
\end{Shaded}

\BeginKnitrBlock{remark}
\iffalse {Remark. } \fi In line with the linux look and feel, variables
starting with a dot (.) are saved but are hidden. To show them see
\texttt{?ls}.
\EndKnitrBlock{remark}

Operations usually work element-wise:

\begin{Shaded}
\begin{Highlighting}[]
\NormalTok{x}\DecValTok{+2}
\end{Highlighting}
\end{Shaded}

\begin{verbatim}
##  [1] 12 13 14 15 16 17 18 19 20 21 22 23
\end{verbatim}

\begin{Shaded}
\begin{Highlighting}[]
\NormalTok{x*}\DecValTok{2}    
\end{Highlighting}
\end{Shaded}

\begin{verbatim}
##  [1] 20 22 24 26 28 30 32 34 36 38 40 42
\end{verbatim}

\begin{Shaded}
\begin{Highlighting}[]
\NormalTok{x^}\DecValTok{2}    
\end{Highlighting}
\end{Shaded}

\begin{verbatim}
##  [1] 100 121 144 169 196 225 256 289 324 361 400 441
\end{verbatim}

\begin{Shaded}
\begin{Highlighting}[]
\KeywordTok{sqrt}\NormalTok{(x)  }
\end{Highlighting}
\end{Shaded}

\begin{verbatim}
##  [1] 3.162278 3.316625 3.464102 3.605551 3.741657 3.872983 4.000000
##  [8] 4.123106 4.242641 4.358899 4.472136 4.582576
\end{verbatim}

\begin{Shaded}
\begin{Highlighting}[]
\KeywordTok{log}\NormalTok{(x)   }
\end{Highlighting}
\end{Shaded}

\begin{verbatim}
##  [1] 2.302585 2.397895 2.484907 2.564949 2.639057 2.708050 2.772589
##  [8] 2.833213 2.890372 2.944439 2.995732 3.044522
\end{verbatim}

\section{Search Paths and Packages}\label{search-paths-and-packages}

R can be easily extended with packages, which are merely a set of
documented functions, which can be loaded or unloaded conveniently.
Let's look at the function \texttt{read.csv}. We can see its contents by
calling it without arguments:

\begin{Shaded}
\begin{Highlighting}[]
\NormalTok{read.csv}
\end{Highlighting}
\end{Shaded}

\begin{verbatim}
## function (file, header = TRUE, sep = ",", quote = "\"", dec = ".", 
##     fill = TRUE, comment.char = "", ...) 
## read.table(file = file, header = header, sep = sep, quote = quote, 
##     dec = dec, fill = fill, comment.char = comment.char, ...)
## <bytecode: 0x47f3150>
## <environment: namespace:utils>
\end{verbatim}

Never mind what the function does. Note the
\texttt{environment:\ namespace:utils} line at the end. It tells us that
this function is part of the \textbf{utils} package. We did not need to
know this because it is loaded by default. Here are the packages that
are currently loaded:

\begin{Shaded}
\begin{Highlighting}[]
\KeywordTok{head}\NormalTok{(}\KeywordTok{search}\NormalTok{())}
\end{Highlighting}
\end{Shaded}

\begin{verbatim}
## [1] ".GlobalEnv"         "package:doSNOW"     "package:snow"      
## [4] "package:doParallel" "package:parallel"   "package:iterators"
\end{verbatim}

Other packages can be loaded via the \texttt{library} function, or
downloaded from the internet using the \texttt{install.packages}
function before loading with \texttt{library}. R's package import
mechanism is quite powerful, and is one of the reasons for R's success.

\section{Simple Plotting}\label{simple-plotting}

R has many plotting facilities as we will further detail in the Plotting
Chapter \ref{plotting}. We start with the simplest facilities, namely,
the \texttt{plot} function from the \textbf{graphics} package, which is
loaded by default.

\begin{Shaded}
\begin{Highlighting}[]
\NormalTok{x<-}\StringTok{ }\DecValTok{1}\NormalTok{:}\DecValTok{100}
\NormalTok{y<-}\StringTok{ }\DecValTok{3}\NormalTok{+}\KeywordTok{sin}\NormalTok{(x) }
\KeywordTok{plot}\NormalTok{(}\DataTypeTok{x =} \NormalTok{x, }\DataTypeTok{y =} \NormalTok{y) }\CommentTok{# x,y syntax                         }
\end{Highlighting}
\end{Shaded}

\includegraphics[width=0.5\linewidth]{Rcourse_files/figure-latex/unnamed-chunk-30-1}

Given an \texttt{x} argument and a \texttt{y} argument, \texttt{plot}
tries to present a scatter plot. We call this the \texttt{x,y} syntax. R
has another unique syntax to state functional relations. We call
\texttt{y\textasciitilde{}x} the ``tilde'' syntax, which originates in
works of \citet{wilkinson1973symbolic} and was adopted in the early days
of S.

\begin{Shaded}
\begin{Highlighting}[]
\KeywordTok{plot}\NormalTok{(y ~}\StringTok{ }\NormalTok{x) }\CommentTok{# y~x syntax }
\end{Highlighting}
\end{Shaded}

\includegraphics[width=0.5\linewidth]{Rcourse_files/figure-latex/unnamed-chunk-31-1}

The syntax \texttt{y\textasciitilde{}x} is read as ``y is a function of
x''. We will prefer the \texttt{y\textasciitilde{}x} syntax over the
\texttt{x,y} syntax since it is easier to read, and will be very useful
when we discuss more complicated models.

Here are some arguments that control the plot's appearance. We use
\texttt{type} to control the plot type, \texttt{main} to control the
main title.

\begin{Shaded}
\begin{Highlighting}[]
\KeywordTok{plot}\NormalTok{(y~x, }\DataTypeTok{type=}\StringTok{'l'}\NormalTok{, }\DataTypeTok{main=}\StringTok{'Plotting a connected line'}\NormalTok{) }
\end{Highlighting}
\end{Shaded}

\includegraphics[width=0.5\linewidth]{Rcourse_files/figure-latex/unnamed-chunk-32-1}

We use \texttt{xlab} for the x-axis label, \texttt{ylab} for the y-axis.

\begin{Shaded}
\begin{Highlighting}[]
\KeywordTok{plot}\NormalTok{(y~x, }\DataTypeTok{type=}\StringTok{'h'}\NormalTok{, }\DataTypeTok{main=}\StringTok{'Sticks plot'}\NormalTok{, }\DataTypeTok{xlab=}\StringTok{'Insert x axis label'}\NormalTok{, }\DataTypeTok{ylab=}\StringTok{'Insert y axis label'}\NormalTok{) }
\end{Highlighting}
\end{Shaded}

\includegraphics[width=0.5\linewidth]{Rcourse_files/figure-latex/unnamed-chunk-33-1}

We use \texttt{pch} to control the point type.

\begin{Shaded}
\begin{Highlighting}[]
\KeywordTok{plot}\NormalTok{(y~x, }\DataTypeTok{pch=}\DecValTok{5}\NormalTok{) }\CommentTok{# Point type with pcf}
\end{Highlighting}
\end{Shaded}

\includegraphics[width=0.5\linewidth]{Rcourse_files/figure-latex/unnamed-chunk-34-1}

We use \texttt{col} to control the color, \texttt{cex} for the point
size, and \texttt{abline} to add a straight line.

\begin{Shaded}
\begin{Highlighting}[]
\KeywordTok{plot}\NormalTok{(y~x, }\DataTypeTok{pch=}\DecValTok{10}\NormalTok{, }\DataTypeTok{type=}\StringTok{'p'}\NormalTok{, }\DataTypeTok{col=}\StringTok{'blue'}\NormalTok{, }\DataTypeTok{cex=}\DecValTok{4}\NormalTok{) }
\KeywordTok{abline}\NormalTok{(}\DecValTok{3}\NormalTok{, }\FloatTok{0.002}\NormalTok{) }
\end{Highlighting}
\end{Shaded}

\includegraphics[width=0.5\linewidth]{Rcourse_files/figure-latex/unnamed-chunk-35-1}

For more plotting options run these

\begin{Shaded}
\begin{Highlighting}[]
\KeywordTok{example}\NormalTok{(plot)}
\KeywordTok{example}\NormalTok{(points)}
\NormalTok{?plot}
\KeywordTok{help}\NormalTok{(}\DataTypeTok{package=}\StringTok{'graphics'}\NormalTok{)}
\end{Highlighting}
\end{Shaded}

When your plotting gets serious, go to Chapter \ref{plotting}.

\section{Object Types}\label{object-types}

We already saw that the basic building block of R objects is the vector.
Vectors can be of the following types:

\begin{itemize}
\tightlist
\item
  \textbf{character} Where each element is a string, i.e., a sequence of
  alphanumeric symbols.
\item
  \textbf{numeric} Where each element is a
  \href{https://en.wikipedia.org/wiki/Real_number}{real number} in
  \href{https://en.wikipedia.org/wiki/Double-precision_floating-point_format}{double
  precision} floating point format.
\item
  \textbf{integer} Where each element is an
  \href{https://en.wikipedia.org/wiki/Integer}{integer}.
\item
  \textbf{logical} Where each element is either TRUE, FALSE, or
  NA\footnote{R uses a
    \href{https://en.wikipedia.org/wiki/Three-valued_logic}{\textbf{three}
    valued logic} where a missing value (NA) is neither TRUE, nor FALSE.}
\item
  \textbf{complex} Where each element is a complex number.
\item
  \textbf{list} Where each element is an arbitrary R object.
\item
  \textbf{factor} Factors are not actually vector objects, but they feel
  like such. They are used to encode any finite set of values. This will
  be very useful when fitting linear model, but may be confusing if you
  think you are dealing with a character vector when in fact you are
  dealing with a factor. Be alert!
\end{itemize}

Vectors can be combined into larger objects. A \texttt{matrix} can be
thought of as the binding of several vectors of the same type. In
reality, a matrix is merely a vector with a dimension attribute, that
tells R to read it as a matrix and not a vector.

If vectors of different types (but same length) are binded, we get a
\texttt{data.frame} which is the most fundamental object in R for data
analysis.

\section{Data Frames}\label{data-frames}

Creating a simple data frame:

\begin{Shaded}
\begin{Highlighting}[]
\NormalTok{x<-}\StringTok{ }\DecValTok{1}\NormalTok{:}\DecValTok{10}
\NormalTok{y<-}\StringTok{ }\DecValTok{3} \NormalTok{+}\StringTok{ }\KeywordTok{sin}\NormalTok{(x) }
\NormalTok{frame1 <-}\StringTok{ }\KeywordTok{data.frame}\NormalTok{(}\DataTypeTok{x=}\NormalTok{x, }\DataTypeTok{sin=}\NormalTok{y)    }
\end{Highlighting}
\end{Shaded}

Let's inspect our data frame:

\begin{Shaded}
\begin{Highlighting}[]
\KeywordTok{head}\NormalTok{(frame1)}
\end{Highlighting}
\end{Shaded}

\begin{verbatim}
##   x      sin
## 1 1 3.841471
## 2 2 3.909297
## 3 3 3.141120
## 4 4 2.243198
## 5 5 2.041076
## 6 6 2.720585
\end{verbatim}

Now using the RStudio Excel-like viewer:

\begin{Shaded}
\begin{Highlighting}[]
\NormalTok{frame1 %>%}\StringTok{ }\KeywordTok{View}\NormalTok{() }
\end{Highlighting}
\end{Shaded}

We highly advise against editing the data this way since there will be
no documentation of the changes you made.

Verifying this is a data frame:

\begin{Shaded}
\begin{Highlighting}[]
\KeywordTok{class}\NormalTok{(frame1) }\CommentTok{# the object is of type data.frame}
\end{Highlighting}
\end{Shaded}

\begin{verbatim}
## [1] "data.frame"
\end{verbatim}

Check the dimension of the data

\begin{Shaded}
\begin{Highlighting}[]
\KeywordTok{dim}\NormalTok{(frame1)                             }
\end{Highlighting}
\end{Shaded}

\begin{verbatim}
## [1] 10  2
\end{verbatim}

Note that checking the dimension of a vector is different than checking
the dimension of a data frame.

\begin{Shaded}
\begin{Highlighting}[]
\KeywordTok{length}\NormalTok{(x)}
\end{Highlighting}
\end{Shaded}

\begin{verbatim}
## [1] 10
\end{verbatim}

The length of a \texttt{data.frame} is merely the number of columns.

\begin{Shaded}
\begin{Highlighting}[]
\KeywordTok{length}\NormalTok{(frame1) }
\end{Highlighting}
\end{Shaded}

\begin{verbatim}
## [1] 2
\end{verbatim}

\section{Exctraction}\label{exctraction}

R provides many ways to subset and extract elements from vectors and
other objects. The basics are fairly simple, but not paying attention to
the ``personality'' of each extraction mechanism may cause you a lot of
headache.

For starters, extraction is done with the \texttt{{[}} operator. The
operator can take vectors of many types.

Extracting element with by integer index:

\begin{Shaded}
\begin{Highlighting}[]
\NormalTok{frame1[}\DecValTok{1}\NormalTok{, }\DecValTok{2}\NormalTok{]  }\CommentTok{# exctract the element in the 1st row and 2nd column.}
\end{Highlighting}
\end{Shaded}

\begin{verbatim}
## [1] 3.841471
\end{verbatim}

Extract \textbf{column} by index:

\begin{Shaded}
\begin{Highlighting}[]
\NormalTok{frame1[,}\DecValTok{1}\NormalTok{]                              }
\end{Highlighting}
\end{Shaded}

\begin{verbatim}
##  [1]  1  2  3  4  5  6  7  8  9 10
\end{verbatim}

Extract column by name:

\begin{Shaded}
\begin{Highlighting}[]
\NormalTok{frame1[, }\StringTok{'sin'}\NormalTok{]}
\end{Highlighting}
\end{Shaded}

\begin{verbatim}
##  [1] 3.841471 3.909297 3.141120 2.243198 2.041076 2.720585 3.656987
##  [8] 3.989358 3.412118 2.455979
\end{verbatim}

As a general rule, extraction with \texttt{{[}} will conserve the class
of the parent object. There are, however, exceptions. Notice the
extraction mechanism and the class of the output in the following
examples.

\begin{Shaded}
\begin{Highlighting}[]
\KeywordTok{class}\NormalTok{(frame1[, }\StringTok{'sin'}\NormalTok{])  }\CommentTok{# extracts a column vector}
\end{Highlighting}
\end{Shaded}

\begin{verbatim}
## [1] "numeric"
\end{verbatim}

\begin{Shaded}
\begin{Highlighting}[]
\KeywordTok{class}\NormalTok{(frame1[}\StringTok{'sin'}\NormalTok{])  }\CommentTok{# extracts a data frame}
\end{Highlighting}
\end{Shaded}

\begin{verbatim}
## [1] "data.frame"
\end{verbatim}

\begin{Shaded}
\begin{Highlighting}[]
\KeywordTok{class}\NormalTok{(frame1[,}\DecValTok{1}\NormalTok{:}\DecValTok{2}\NormalTok{])  }\CommentTok{# extracts a data frame}
\end{Highlighting}
\end{Shaded}

\begin{verbatim}
## [1] "data.frame"
\end{verbatim}

\begin{Shaded}
\begin{Highlighting}[]
\KeywordTok{class}\NormalTok{(frame1[}\DecValTok{2}\NormalTok{])  }\CommentTok{# extracts a data frame}
\end{Highlighting}
\end{Shaded}

\begin{verbatim}
## [1] "data.frame"
\end{verbatim}

\begin{Shaded}
\begin{Highlighting}[]
\KeywordTok{class}\NormalTok{(frame1[}\DecValTok{2}\NormalTok{, ])  }\CommentTok{# extract a data frame}
\end{Highlighting}
\end{Shaded}

\begin{verbatim}
## [1] "data.frame"
\end{verbatim}

\begin{Shaded}
\begin{Highlighting}[]
\KeywordTok{class}\NormalTok{(frame1$sin)  }\CommentTok{# extracts a column vector}
\end{Highlighting}
\end{Shaded}

\begin{verbatim}
## [1] "numeric"
\end{verbatim}

The \texttt{subset()} function does the same

\begin{Shaded}
\begin{Highlighting}[]
\KeywordTok{subset}\NormalTok{(frame1, }\DataTypeTok{select=}\NormalTok{sin) }
\KeywordTok{subset}\NormalTok{(frame1, }\DataTypeTok{select=}\DecValTok{2}\NormalTok{)}
\KeywordTok{subset}\NormalTok{(frame1, }\DataTypeTok{select=} \KeywordTok{c}\NormalTok{(}\DecValTok{2}\NormalTok{,}\DecValTok{0}\NormalTok{))}
\end{Highlighting}
\end{Shaded}

If you want to force the stripping of the class attribute when
extracting, try the \texttt{{[}{[}} mechanism instead of \texttt{{[}}.

\begin{Shaded}
\begin{Highlighting}[]
\NormalTok{a <-}\StringTok{ }\NormalTok{frame1[}\DecValTok{1}\NormalTok{] }\CommentTok{# [ extraction}
\NormalTok{b <-}\StringTok{ }\NormalTok{frame1[[}\DecValTok{1}\NormalTok{]] }\CommentTok{# [[ extraction}
\NormalTok{a==b }\CommentTok{# objects are element-wise identical }
\end{Highlighting}
\end{Shaded}

\begin{verbatim}
##          x
##  [1,] TRUE
##  [2,] TRUE
##  [3,] TRUE
##  [4,] TRUE
##  [5,] TRUE
##  [6,] TRUE
##  [7,] TRUE
##  [8,] TRUE
##  [9,] TRUE
## [10,] TRUE
\end{verbatim}

\begin{Shaded}
\begin{Highlighting}[]
\KeywordTok{class}\NormalTok{(a)==}\KeywordTok{class}\NormalTok{(b)}
\end{Highlighting}
\end{Shaded}

\begin{verbatim}
## [1] FALSE
\end{verbatim}

The different types of output classes cause different behaviors. Compare
the behavior of \texttt{{[}} on seemingly identical objects.

\begin{Shaded}
\begin{Highlighting}[]
\NormalTok{frame1[}\DecValTok{1}\NormalTok{][}\DecValTok{1}\NormalTok{]}
\end{Highlighting}
\end{Shaded}

\begin{verbatim}
##     x
## 1   1
## 2   2
## 3   3
## 4   4
## 5   5
## 6   6
## 7   7
## 8   8
## 9   9
## 10 10
\end{verbatim}

\begin{Shaded}
\begin{Highlighting}[]
\NormalTok{frame1[[}\DecValTok{1}\NormalTok{]][}\DecValTok{1}\NormalTok{]}
\end{Highlighting}
\end{Shaded}

\begin{verbatim}
## [1] 1
\end{verbatim}

If you want to learn more about subsetting see
\href{http://adv-r.had.co.nz/Subsetting.html}{Hadley's guide}.

\section{Data Import and Export}\label{data-import-and-export}

For any practical purpose, you will not be generating your data
manually. R comes with many importing and exporting mechanisms which we
now present. If, however, you do a lot of data ``munging'', make sure to
see Hadley-verse Chapter \ref{hadley}. If you work with MASSIVE data
sets, read the Memory Efficiency Chapter \ref{memory}.

\subsection{Import from WEB}\label{import-from-web}

The \texttt{read.table} function is the main importing workhorse. It can
import directly from the web.

\begin{Shaded}
\begin{Highlighting}[]
\NormalTok{URL <-}\StringTok{ 'http://statweb.stanford.edu/~tibs/ElemStatLearn/datasets/bone.data'}
\NormalTok{tirgul1 <-}\StringTok{ }\KeywordTok{read.table}\NormalTok{(URL)}
\end{Highlighting}
\end{Shaded}

Always look at the imported result!

\begin{Shaded}
\begin{Highlighting}[]
\KeywordTok{head}\NormalTok{(tirgul1)}
\end{Highlighting}
\end{Shaded}

\begin{verbatim}
##      V1    V2     V3          V4
## 1 idnum   age gender      spnbmd
## 2     1  11.7   male  0.01808067
## 3     1  12.7   male  0.06010929
## 4     1 13.75   male 0.005857545
## 5     2 13.25   male  0.01026393
## 6     2  14.3   male   0.2105263
\end{verbatim}

Ohh dear. The header row was not recognized. Fix with
\texttt{header=TRUE}:

\begin{Shaded}
\begin{Highlighting}[]
\NormalTok{tirgul1 <-}\StringTok{ }\KeywordTok{read.table}\NormalTok{(URL, }\DataTypeTok{header =} \OtherTok{TRUE}\NormalTok{) }
\KeywordTok{head}\NormalTok{(tirgul1)}
\end{Highlighting}
\end{Shaded}

\subsection{Export as CSV}\label{export-as-csv}

Let's write a simple file so that we have something to import

\begin{Shaded}
\begin{Highlighting}[]
\KeywordTok{head}\NormalTok{(airquality) }\CommentTok{#  examine the data to export}
\end{Highlighting}
\end{Shaded}

\begin{verbatim}
##   Ozone Solar.R Wind Temp Month Day
## 1    41     190  7.4   67     5   1
## 2    36     118  8.0   72     5   2
## 3    12     149 12.6   74     5   3
## 4    18     313 11.5   62     5   4
## 5    NA      NA 14.3   56     5   5
## 6    28      NA 14.9   66     5   6
\end{verbatim}

\begin{Shaded}
\begin{Highlighting}[]
\NormalTok{temp.file.name <-}\StringTok{ }\KeywordTok{tempfile}\NormalTok{() }\CommentTok{# get some arbitrary file name}
\KeywordTok{write.csv}\NormalTok{(}\DataTypeTok{x =} \NormalTok{airquality, }\DataTypeTok{file =} \NormalTok{temp.file.name) }\CommentTok{# export}
\end{Highlighting}
\end{Shaded}

Now let's import the exported file. Being a .csv file, I can use
\texttt{read.csv} instead of \texttt{read.table}.

\begin{Shaded}
\begin{Highlighting}[]
\NormalTok{my.data<-}\StringTok{ }\KeywordTok{read.csv}\NormalTok{(}\DataTypeTok{file=}\NormalTok{temp.file.name) }\CommentTok{# import}
\KeywordTok{head}\NormalTok{(my.data) }\CommentTok{# verify import}
\end{Highlighting}
\end{Shaded}

\begin{verbatim}
##   X Ozone Solar.R Wind Temp Month Day
## 1 1    41     190  7.4   67     5   1
## 2 2    36     118  8.0   72     5   2
## 3 3    12     149 12.6   74     5   3
## 4 4    18     313 11.5   62     5   4
## 5 5    NA      NA 14.3   56     5   5
## 6 6    28      NA 14.9   66     5   6
\end{verbatim}

\BeginKnitrBlock{remark}
\iffalse {Remark. } \fi Windows users may need to use
``\textbackslash{}'' instead of ``/''.
\EndKnitrBlock{remark}

\subsection{Reading From Text Files}\label{reading-from-text-files}

Some general notes on importing text files via the \texttt{read.table}
function. But first, we need to know what is the active directory. Here
is how to get and set R's active directory:

\begin{Shaded}
\begin{Highlighting}[]
\KeywordTok{getwd}\NormalTok{() }\CommentTok{#What is the working directory?}
\KeywordTok{setwd}\NormalTok{() }\CommentTok{#Setting the working directory in Linux}
\end{Highlighting}
\end{Shaded}

We can now call the \texttt{read.table} function to import text files.
If you care about your sanity, see \texttt{?read.table} before starting
imports. Some notable properties of the function:

\begin{itemize}
\tightlist
\item
  \texttt{read.table} will try to guess column separators (tab, comma,
  etc.)
\item
  \texttt{read.table} will try to guess if a header row is present.
\item
  \texttt{read.table} will convert character vectors to factors unless
  told not to.
\item
  The output of \texttt{read.table} needs to be explicitly assigned to
  an object for it to be saved.
\end{itemize}

\subsection{Writing Data to Text
Files}\label{writing-data-to-text-files}

The function \texttt{write.table} is the exporting counterpart of
\texttt{read.table}.

\subsection{.XLS(X) files}\label{xlsx-files}

Strongly recommended to convert to .csv in Excel, and then import as
csv. If you still insist see the \textbf{xlsx} package.

\subsection{Massive files}\label{massive-files}

The above importing and exporting mechanisms were not designed for
massive files. See the section on Sparse Representation (\ref{sparse})
and Out-of-Ram Algorithms (\ref{memory}) for more on working with
massive data files.

\subsection{Databases}\label{databases}

R does not need to read from text files; it can read directly from a
database. This is very useful since it allows the filtering, selecting
and joining operations to rely on the database's optimized algorithms.
See
\url{https://rforanalytics.wordpress.com/useful-links-for-r/odbc-databases-for-r/}
for more information.

\section{Functions}\label{functions}

One of the most basic building blocks of programming is the ability of
writing your own functions. A function in R, like everything else, is an
object accessible using its name. We first define a simple function that
sums its two arguments

\begin{Shaded}
\begin{Highlighting}[]
\NormalTok{my.sum <-}\StringTok{ }\NormalTok{function(x,y) \{}
  \KeywordTok{return}\NormalTok{(x+y)}
\NormalTok{\}}
\KeywordTok{my.sum}\NormalTok{(}\DecValTok{10}\NormalTok{,}\DecValTok{2}\NormalTok{)}
\end{Highlighting}
\end{Shaded}

\begin{verbatim}
## [1] 12
\end{verbatim}

From this example you may notice that:

\begin{itemize}
\item
  The function \texttt{function} tells R to construct a function object.
\item
  The arguments of the \texttt{function}, i.e. \texttt{(x,y)}, need to
  be named but we are not required to specify their type. This makes
  writing functions very easy, but it is also the source of many bugs,
  and slowness of R compared to type declaring languages (C,
  Fortran,Java,\ldots{}).
\item
  A typical R function does not change objects\footnote{This is a
    classical \emph{functional programming} paradigm. If you are used to
    \emph{object oriented} programming, you may want to read about
    \href{http://adv-r.had.co.nz/R5.html}{references classes} which may
    be required if you are planning to compute with very complicated
    objects.} but rather creates new ones. To save the output of
  \texttt{my.sum} we will need to assign it using the
  \texttt{\textless{}-} operator.
\end{itemize}

Here is a (slightly) more advanced example.

\begin{Shaded}
\begin{Highlighting}[]
\NormalTok{my.sum}\FloatTok{.2} \NormalTok{<-}\StringTok{ }\NormalTok{function(x, y , }\DataTypeTok{absolute=}\OtherTok{FALSE}\NormalTok{) \{}
  \NormalTok{if(absolute==}\OtherTok{TRUE}\NormalTok{) \{}
    \NormalTok{result <-}\StringTok{ }\KeywordTok{abs}\NormalTok{(x+y)}
  \NormalTok{\}}
  \NormalTok{else\{}
    \NormalTok{result <-}\StringTok{ }\NormalTok{x+y}
  \NormalTok{\} }
  \NormalTok{result}
\NormalTok{\}}
\KeywordTok{my.sum.2}\NormalTok{(-}\DecValTok{10}\NormalTok{,}\DecValTok{2}\NormalTok{, }\OtherTok{TRUE}\NormalTok{)}
\end{Highlighting}
\end{Shaded}

\begin{verbatim}
## [1] 8
\end{verbatim}

Things to note:

\begin{itemize}
\tightlist
\item
  The function will output its last evaluated expression. You don't need
  to use the \texttt{return} function explicitly.
\item
  Using \texttt{absolute=FALSE} sets the default value of
  \texttt{absolute} to \texttt{FALSE}. This is overriden if
  \texttt{absolute} is stated explicitly in the function call.
\end{itemize}

An important behavior of R is the \emph{scoping rules}. This refers to
the way R seeks for variables used in functions. As a rule of thumb, R
will first look for variables inside the function and if not found, will
search for the variable values in outer environments\footnote{More
  formally, this is called
  \href{https://darrenjw.wordpress.com/2011/11/23/lexical-scope-and-function-closures-in-r/}{Lexical
  Scoping}.}. Think of the next example.

\begin{Shaded}
\begin{Highlighting}[]
\NormalTok{a <-}\StringTok{ }\DecValTok{1}
\NormalTok{b <-}\StringTok{ }\DecValTok{2}
\NormalTok{x <-}\StringTok{ }\DecValTok{3}
\NormalTok{scoping <-}\StringTok{ }\NormalTok{function(a,b)\{}
  \NormalTok{a+b+x}
\NormalTok{\}}
\KeywordTok{scoping}\NormalTok{(}\DecValTok{10}\NormalTok{,}\DecValTok{11}\NormalTok{)}
\end{Highlighting}
\end{Shaded}

\begin{verbatim}
## [1] 24
\end{verbatim}

\section{Looping}\label{looping}

The real power of scripting is when repeated operations are done by
iteration. R supports the usual \texttt{for}, \texttt{while}, and
\texttt{repated} loops. Here is an embarrassingly simple example

\begin{Shaded}
\begin{Highlighting}[]
\NormalTok{for (i in }\DecValTok{1}\NormalTok{:}\DecValTok{5}\NormalTok{)\{}
    \KeywordTok{print}\NormalTok{(i)}
    \NormalTok{\}}
\end{Highlighting}
\end{Shaded}

\begin{verbatim}
## [1] 1
## [1] 2
## [1] 3
## [1] 4
## [1] 5
\end{verbatim}

\section{Apply}\label{apply}

For applying the same function to a set of elements, there is no need to
write an explicit loop. This is such en elementary operation that every
programming language will provide some facility to \textbf{apply}, or
\textbf{map} the function to all elements of a set. R provides several
facilities to perform this. The most basic of which is \texttt{lapply}
which applies a function over all elements of a list, and return a list
of outputs:

\begin{Shaded}
\begin{Highlighting}[]
\NormalTok{the.list <-}\StringTok{ }\KeywordTok{list}\NormalTok{(}\DecValTok{1}\NormalTok{,}\StringTok{'a'}\NormalTok{,mean)}
\KeywordTok{lapply}\NormalTok{(}\DataTypeTok{X =} \NormalTok{the.list, }\DataTypeTok{FUN =} \NormalTok{class)}
\end{Highlighting}
\end{Shaded}

\begin{verbatim}
## [[1]]
## [1] "numeric"
## 
## [[2]]
## [1] "character"
## 
## [[3]]
## [1] "standardGeneric"
## attr(,"package")
## [1] "methods"
\end{verbatim}

R provides many variations on \texttt{lapply} to facilitate programing.
Here is a partial list:

\begin{itemize}
\tightlist
\item
  \texttt{sapply}: The same as \texttt{lapply} but tries to arrange
  output in a vector or matrix, and not an unstructured list.
\item
  \texttt{vapply}: A safer version of \texttt{sapply}, where the output
  class is pre-specified.
\item
  \texttt{apply}: For applying over the rows or columns of matrices.
\item
  \texttt{mapply}: For applying functions with various inputs.
\item
  \texttt{tapply}: For splitting vectors and applying functions on
  subsets.
\item
  \texttt{rapply}: A recursive version of \texttt{lapply}.
\item
  \texttt{eapply}: Like \texttt{lapply}, only operates on
  \texttt{environments} instead of lists.
\item
  \texttt{Map}+\texttt{Reduce}: For a
  \href{https://en.wikipedia.org/wiki/Common_Lisp}{Common Lisp} look and
  feel of \texttt{lapply}.
\item
  \texttt{parallel::parLapply}: A parallel version of \texttt{lapply}.
\item
  \texttt{parallel::parLBapply}: A parallel version of \texttt{lapply},
  with load balancing.
\end{itemize}

\section{Recursion}\label{recursion}

The R compiler is really not designed for recursion, and you will rarely
need to do so.\\
See the RCpp Chapter \ref{rcpp} for linking C code, which is better
suited for recursion. If you really insist to write recursions in R,
make sure to use the \texttt{Recall} function, which, as the name
suggests, recalls the function in which it is place. Here is a
demonstration with the Fibonacci series.

\begin{Shaded}
\begin{Highlighting}[]
\NormalTok{fib<-function(n) \{}
    \NormalTok{if (n <}\StringTok{ }\DecValTok{2}\NormalTok{) fn<-}\DecValTok{1} 
    \NormalTok{else fn<-}\KeywordTok{Recall}\NormalTok{(n -}\StringTok{ }\DecValTok{1}\NormalTok{) +}\StringTok{ }\KeywordTok{Recall}\NormalTok{(n -}\StringTok{ }\DecValTok{2}\NormalTok{) }
    \KeywordTok{return}\NormalTok{(fn)}
\NormalTok{\} }
\KeywordTok{fib}\NormalTok{(}\DecValTok{5}\NormalTok{)}
\end{Highlighting}
\end{Shaded}

\begin{verbatim}
## [1] 8
\end{verbatim}

\section{Dates and Times}\label{dates-and-times}

{[}TODO{]}

\section{Bibliographic Notes}\label{bibliographic-notes-1}

There are endlessly many introductory texts on R. For a list of free
resources see
\href{http://stats.stackexchange.com/questions/138/free-resources-for-learning-r}{CrossValidated}.
I personally recommend the official introduction
\citet{venables2004introduction},
\href{https://cran.r-project.org/doc/manuals/r-release/R-intro.pdf}{available
online}, or anything else Bill Venables writes. For advanced R
programming see \citet{wickham2014advanced},
\href{http://adv-r.had.co.nz/}{available online}, or anything else
Hadley Wickham writes.

\section{Practice Yourself}\label{practice-yourself-1}

\chapter{Exploratory Data Analysis}\label{eda}

Exploratory Data Analysis (EDA) is a term cast by
\href{https://en.wikipedia.org/wiki/John_Tukey}{John W. Tukey} in his
seminal book \citep{tukey1977exploratory}. It is the practice of
inspecting, and exploring your data, before stating hypotheses, fitting
predictors, and other more ambitious inferential goals. It typically
includes the computation of simple \emph{summary statistics} which
capture some property of interest in the data, and \emph{visualization}.
EDA can be thought of as an assumption free, purely algorithmic
practice.

In this text we present EDA techniques along the following lines:

\begin{itemize}
\tightlist
\item
  How we explore: with summary statistics, or visually?
\item
  How many variables analyzed simultaneously: univariate, bivariate, or
  multivariate?
\item
  What type of variable: categorical or continuous?
\end{itemize}

\section{Summary Statistics}\label{summary-statistics}

\subsection{Categorical Data}\label{categorical-data}

Categorical variables do not admit any mathematical operations on them.
We cannot sum them, or even sort them. We can only \textbf{count} them.
As such, summaries of categorical variables will always start with the
counting of the frequency of each category.

\subsubsection{Summary of Univariate Categorical
Data}\label{summary-of-univariate-categorical-data}

\begin{Shaded}
\begin{Highlighting}[]
\NormalTok{gender <-}\StringTok{ }\KeywordTok{c}\NormalTok{(}\KeywordTok{rep}\NormalTok{(}\StringTok{'Boy'}\NormalTok{, }\DecValTok{10}\NormalTok{), }\KeywordTok{rep}\NormalTok{(}\StringTok{'Girl'}\NormalTok{, }\DecValTok{12}\NormalTok{))}
\NormalTok{drink <-}\StringTok{ }\KeywordTok{c}\NormalTok{(}\KeywordTok{rep}\NormalTok{(}\StringTok{'Coke'}\NormalTok{, }\DecValTok{5}\NormalTok{), }\KeywordTok{rep}\NormalTok{(}\StringTok{'Sprite'}\NormalTok{, }\DecValTok{3}\NormalTok{), }\KeywordTok{rep}\NormalTok{(}\StringTok{'Coffee'}\NormalTok{, }\DecValTok{6}\NormalTok{), }\KeywordTok{rep}\NormalTok{(}\StringTok{'Tea'}\NormalTok{, }\DecValTok{7}\NormalTok{), }\KeywordTok{rep}\NormalTok{(}\StringTok{'Water'}\NormalTok{, }\DecValTok{1}\NormalTok{))  }
\NormalTok{age <-}\StringTok{  }\KeywordTok{sample}\NormalTok{(}\KeywordTok{c}\NormalTok{(}\StringTok{'Young'}\NormalTok{, }\StringTok{'Old'}\NormalTok{), }\DataTypeTok{size =} \KeywordTok{length}\NormalTok{(gender), }\DataTypeTok{replace =} \OtherTok{TRUE}\NormalTok{)}

\KeywordTok{table}\NormalTok{(gender)}
\end{Highlighting}
\end{Shaded}

\begin{verbatim}
## gender
##  Boy Girl 
##   10   12
\end{verbatim}

\begin{Shaded}
\begin{Highlighting}[]
\KeywordTok{table}\NormalTok{(drink)}
\end{Highlighting}
\end{Shaded}

\begin{verbatim}
## drink
## Coffee   Coke Sprite    Tea  Water 
##      6      5      3      7      1
\end{verbatim}

\begin{Shaded}
\begin{Highlighting}[]
\KeywordTok{table}\NormalTok{(age)}
\end{Highlighting}
\end{Shaded}

\begin{verbatim}
## age
##   Old Young 
##    10    12
\end{verbatim}

If instead of the level counts you want the proportions, you can use
\texttt{prop.table}

\begin{Shaded}
\begin{Highlighting}[]
\KeywordTok{prop.table}\NormalTok{(}\KeywordTok{table}\NormalTok{(gender))}
\end{Highlighting}
\end{Shaded}

\begin{verbatim}
## gender
##       Boy      Girl 
## 0.4545455 0.5454545
\end{verbatim}

\subsubsection{Summary of Bivariate Categorical
Data}\label{summary-of-bivariate-categorical-data}

\begin{Shaded}
\begin{Highlighting}[]
\KeywordTok{library}\NormalTok{(magrittr)}
\KeywordTok{cbind}\NormalTok{(gender, drink) %>%}\StringTok{ }\NormalTok{head }\CommentTok{# inspect the raw data}
\end{Highlighting}
\end{Shaded}

\begin{verbatim}
##      gender drink   
## [1,] "Boy"  "Coke"  
## [2,] "Boy"  "Coke"  
## [3,] "Boy"  "Coke"  
## [4,] "Boy"  "Coke"  
## [5,] "Boy"  "Coke"  
## [6,] "Boy"  "Sprite"
\end{verbatim}

\begin{Shaded}
\begin{Highlighting}[]
\NormalTok{table1 <-}\StringTok{ }\KeywordTok{table}\NormalTok{(gender, drink) }
\NormalTok{table1                                      }
\end{Highlighting}
\end{Shaded}

\begin{verbatim}
##       drink
## gender Coffee Coke Sprite Tea Water
##   Boy       2    5      3   0     0
##   Girl      4    0      0   7     1
\end{verbatim}

\subsubsection{Summary of Multivariate Categorical
Data}\label{summary-of-multivariate-categorical-data}

You may be wondering how does R handle tables with more than two
dimensions. It is indeed not trivial, and R offers several solutions:
\texttt{table} is easier to compute with, and \texttt{ftable} is human
readable.

\begin{Shaded}
\begin{Highlighting}[]
\NormalTok{table2}\FloatTok{.1} \NormalTok{<-}\StringTok{ }\KeywordTok{table}\NormalTok{(gender, drink, age) }\CommentTok{# A multilevel table. }
\NormalTok{table2}\FloatTok{.1}
\end{Highlighting}
\end{Shaded}

\begin{verbatim}
## , , age = Old
## 
##       drink
## gender Coffee Coke Sprite Tea Water
##   Boy       0    5      1   0     0
##   Girl      1    0      0   3     0
## 
## , , age = Young
## 
##       drink
## gender Coffee Coke Sprite Tea Water
##   Boy       2    0      2   0     0
##   Girl      3    0      0   4     1
\end{verbatim}

\begin{Shaded}
\begin{Highlighting}[]
\NormalTok{table}\FloatTok{.2.2} \NormalTok{<-}\StringTok{ }\KeywordTok{ftable}\NormalTok{(gender, drink, age) }\CommentTok{# A human readable table.}
\NormalTok{table}\FloatTok{.2.2}
\end{Highlighting}
\end{Shaded}

\begin{verbatim}
##               age Old Young
## gender drink               
## Boy    Coffee       0     2
##        Coke         5     0
##        Sprite       1     2
##        Tea          0     0
##        Water        0     0
## Girl   Coffee       1     3
##        Coke         0     0
##        Sprite       0     0
##        Tea          3     4
##        Water        0     1
\end{verbatim}

If you want proportions instead of counts, you need to specify the
denominator, i.e., the margins.

\begin{Shaded}
\begin{Highlighting}[]
\KeywordTok{prop.table}\NormalTok{(table1, }\DataTypeTok{margin =} \DecValTok{1}\NormalTok{)}
\end{Highlighting}
\end{Shaded}

\begin{verbatim}
##       drink
## gender     Coffee       Coke     Sprite        Tea      Water
##   Boy  0.20000000 0.50000000 0.30000000 0.00000000 0.00000000
##   Girl 0.33333333 0.00000000 0.00000000 0.58333333 0.08333333
\end{verbatim}

\begin{Shaded}
\begin{Highlighting}[]
\KeywordTok{prop.table}\NormalTok{(table1, }\DataTypeTok{margin =} \DecValTok{2}\NormalTok{)}
\end{Highlighting}
\end{Shaded}

\begin{verbatim}
##       drink
## gender    Coffee      Coke    Sprite       Tea     Water
##   Boy  0.3333333 1.0000000 1.0000000 0.0000000 0.0000000
##   Girl 0.6666667 0.0000000 0.0000000 1.0000000 1.0000000
\end{verbatim}

\subsection{Continous Data}\label{continous-data}

Continuous variables admit many more operations than categorical. We can
thus compute sums, means, quantiles, and more.

\subsubsection{Summary of Univariate Continuous
Data}\label{summary-of-univariate-continuous-data}

We distinguish between several types of summaries, each capturing a
different property of the data.

\subsubsection{Summary of Location}\label{summary-of-location}

Capture the ``location'' of the data. These include:

\BeginKnitrBlock{definition}[Average]
\protect\hypertarget{def:unnamed-chunk-68}{}{\label{def:unnamed-chunk-68}
\iffalse (Average) \fi }The mean, or average, of a sample \(x\) of
length \(n\), denoted \(\bar x\) is defined as
\[ \bar x := n^{-1} \sum x_i. \]
\EndKnitrBlock{definition}

The sample mean is \textbf{non robust}. A single large observation may
inflate the mean indefinitely. For this reason, we define several other
summaries of location, which are more robust, i.e., less affected by
``contaminations'' of the data.

We start by defining the sample quantiles, themselves \textbf{not} a
summary of location.

\BeginKnitrBlock{definition}[Quantiles]
\protect\hypertarget{def:unnamed-chunk-69}{}{\label{def:unnamed-chunk-69}
\iffalse (Quantiles) \fi }The \(\alpha\) quantile of a sample \(x\),
denoted \(x_\alpha\), is (non uniquely) defined as a value above
\(100 \alpha \%\) of the sample, and below \(100 (1-\alpha) \%\).
\EndKnitrBlock{definition}

We emphasize that sample quantiles are non-uniquely defined. See
\texttt{?quantile} for the 9(!) different definitions that R provides.

We can now define another summary of location, the median.

\BeginKnitrBlock{definition}[Median]
\protect\hypertarget{def:unnamed-chunk-70}{}{\label{def:unnamed-chunk-70}
\iffalse (Median) \fi }The median of a sample \(x\), denoted \(x_{0.5}\)
is the \(\alpha=0.5\) quantile of the sample.
\EndKnitrBlock{definition}

A whole family of summaries of locations is the \textbf{alpha trimmed
mean}.

\BeginKnitrBlock{definition}[Alpha Trimmed Mean]
\protect\hypertarget{def:unnamed-chunk-71}{}{\label{def:unnamed-chunk-71}
\iffalse (Alpha Trimmed Mean) \fi }The \(\alpha\) trimmed mean of a
sample \(x\), denoted \(\bar x_\alpha\) is the average of the sample
after removing the \(\alpha\) proportion of largest and \(\alpha\)
proportion of smallest observations.
\EndKnitrBlock{definition}

The simple mean and median are instances of the alpha trimmed mean:
\(\bar x_0\) and \(\bar x_{0.5}\) respectively.

Here are the R implementations:

\begin{Shaded}
\begin{Highlighting}[]
\NormalTok{x <-}\StringTok{ }\KeywordTok{rexp}\NormalTok{(}\DecValTok{100}\NormalTok{)}
\KeywordTok{mean}\NormalTok{(x) }\CommentTok{# simple mean}
\end{Highlighting}
\end{Shaded}

\begin{verbatim}
## [1] 0.9291637
\end{verbatim}

\begin{Shaded}
\begin{Highlighting}[]
\KeywordTok{median}\NormalTok{(x) }\CommentTok{# median}
\end{Highlighting}
\end{Shaded}

\begin{verbatim}
## [1] 0.7343279
\end{verbatim}

\begin{Shaded}
\begin{Highlighting}[]
\KeywordTok{mean}\NormalTok{(x, }\DataTypeTok{trim =} \FloatTok{0.2}\NormalTok{) }\CommentTok{# alpha trimmed mean with alpha=0.2}
\end{Highlighting}
\end{Shaded}

\begin{verbatim}
## [1] 0.7600934
\end{verbatim}

\subsubsection{Summary of Scale}\label{summary-of-scale}

The \emph{scale} of the data, sometimes known as \emph{spread}, can be
thought of its variability.

\BeginKnitrBlock{definition}[Standard Deviation]
\protect\hypertarget{def:unnamed-chunk-73}{}{\label{def:unnamed-chunk-73}
\iffalse (Standard Deviation) \fi }The standard deviation of a sample
\(x\), denoted \(S(x)\), is defined as
\[ S(x):=\sqrt{(n-1)^{-1} \sum (x_i-\bar x)^2} . \]
\EndKnitrBlock{definition}

For reasons of robustness, we define other, more robust, measures of
scale.

\BeginKnitrBlock{definition}[MAD]
\protect\hypertarget{def:unnamed-chunk-74}{}{\label{def:unnamed-chunk-74}
\iffalse (MAD) \fi }The Median Absolute Deviation from the median,
denoted as \(MAD(x)\), is defined as
\[MAD(x):= c \: |x-x_{0.5}|_{0.5} . \]
\EndKnitrBlock{definition}

where \(c\) is some constant, typically set to \(c=1.4826\) so that the
MAD is a robust estimate of \(S(x)\).

\BeginKnitrBlock{definition}[IQR]
\protect\hypertarget{def:unnamed-chunk-75}{}{\label{def:unnamed-chunk-75}
\iffalse (IQR) \fi }The Inter Quantile Range of a sample \(x\), denoted
as \(IQR(x)\), is defined as \[ IQR(x):= x_{0.75}-x_{0.25} .\]
\EndKnitrBlock{definition}

Here are the R implementations

\begin{Shaded}
\begin{Highlighting}[]
\KeywordTok{sd}\NormalTok{(x) }\CommentTok{# standard deviation}
\end{Highlighting}
\end{Shaded}

\begin{verbatim}
## [1] 0.7855576
\end{verbatim}

\begin{Shaded}
\begin{Highlighting}[]
\KeywordTok{mad}\NormalTok{(x) }\CommentTok{# MAD}
\end{Highlighting}
\end{Shaded}

\begin{verbatim}
## [1] 0.71428
\end{verbatim}

\begin{Shaded}
\begin{Highlighting}[]
\KeywordTok{IQR}\NormalTok{(x) }\CommentTok{# IQR}
\end{Highlighting}
\end{Shaded}

\begin{verbatim}
## [1] 1.00403
\end{verbatim}

\subsubsection{Summary of Asymmetry}\label{summary-of-asymmetry}

The symmetry of a univariate sample is easily understood. Summaries of
asymmetry, also known as \emph{skewness}, quantify the departure of the
\(x\) from a symmetric sample.

\BeginKnitrBlock{definition}[Yule]
\protect\hypertarget{def:unnamed-chunk-77}{}{\label{def:unnamed-chunk-77}
\iffalse (Yule) \fi }The Yule measure of assymetry, denoted \(Yule(x)\)
is defined as
\[Yule(x) := \frac{1/2 \: (x_{0.75}+x_{0.25}) - x_{0.5} }{1/2 \: IQR(x)} .\]
\EndKnitrBlock{definition}

Here is an R implementation

\begin{Shaded}
\begin{Highlighting}[]
\NormalTok{yule <-}\StringTok{ }\NormalTok{function(x)\{}
  \NormalTok{numerator <-}\StringTok{ }\FloatTok{0.5} \NormalTok{*}\StringTok{ }\NormalTok{(}\KeywordTok{quantile}\NormalTok{(x,}\FloatTok{0.75}\NormalTok{) +}\StringTok{ }\KeywordTok{quantile}\NormalTok{(x,}\FloatTok{0.25}\NormalTok{))-}\KeywordTok{median}\NormalTok{(x) }
  \NormalTok{denominator <-}\StringTok{ }\FloatTok{0.5}\NormalTok{*}\StringTok{ }\KeywordTok{IQR}\NormalTok{(x)}
  \KeywordTok{c}\NormalTok{(numerator/denominator, }\DataTypeTok{use.names=}\OtherTok{FALSE}\NormalTok{)}
\NormalTok{\}}
\KeywordTok{yule}\NormalTok{(x)}
\end{Highlighting}
\end{Shaded}

\begin{verbatim}
## [1] 0.1776012
\end{verbatim}

\subsubsection{Summary of Bivariate Continuous
Data}\label{summary-of-bivariate-continuous-data}

When dealing with bivariate, or multivariate data, we can obviously
compute univariate summaries for each variable separately. This is not
the topic of this section, in which we want to summarize the association
\textbf{between} the variables, and not within them.

\BeginKnitrBlock{definition}[Covariance]
\protect\hypertarget{def:unnamed-chunk-78}{}{\label{def:unnamed-chunk-78}
\iffalse (Covariance) \fi }The covariance between two samples, \(x\) and
\(y\), of same length \(n\), is defined as
\[Cov(x,y):= (n-1)^{-1} \sum (x_i-\bar x)(y_i-\bar y)  \]
\EndKnitrBlock{definition}

We emphasize this is not the covariance you learned about in probability
classes, since it is not the covariance between two \emph{random
variables} but rather, between two \emph{samples}. For this reasons,
some authors call it the \emph{empirical} covariance.

\BeginKnitrBlock{definition}[Pearson's Correlation Coefficient]
\protect\hypertarget{def:unnamed-chunk-79}{}{\label{def:unnamed-chunk-79}
\iffalse (Pearson's Correlation Coefficient) \fi }Peasrson's correlation
coefficient, a.k.a. Pearson's moment product correlation, or simply, the
correlation, denoted \texttt{r(x,y)}, is defined as
\[r(x,y):=\frac{Cov(x,y)}{S(x)S(y)}. \]
\EndKnitrBlock{definition}

If you find this definition enigmatic, just think of the correlation as
the covariance between \(x\) and \(y\) after transforming each to the
unitless scale of z-scores.

\BeginKnitrBlock{definition}[Z-Score]
\protect\hypertarget{def:unnamed-chunk-80}{}{\label{def:unnamed-chunk-80}
\iffalse (Z-Score) \fi }The z-scores of a sample \(x\) are defined as
the mean-centered, scale normalized observations:
\[z_i(x):= \frac{x_i-\bar x}{S(x)}.\]
\EndKnitrBlock{definition}

We thus have that \(r(x,y)=Cov(z(x),z(y))\).

\subsubsection{Summary of Multivariate Continuous
Data}\label{summary-of-multivariate-continuous-data}

The covariance is a simple summary of association between two variables,
but it certainly may not capture the whole ``story''. Things get more
complicated when summarizing the relation between multiple variables.
The most common summary of relation, is the \textbf{covariance matrix},
but we warn that only the simplest multivariate relations are fully
summarized by this matrix.

\BeginKnitrBlock{definition}[Sample Covariance Matrix]
\protect\hypertarget{def:unnamed-chunk-81}{}{\label{def:unnamed-chunk-81}
\iffalse (Sample Covariance Matrix) \fi }Given \(n\) observations on
\(p\) variables, denote \(x_{i,j}\) the \(i\)'th observation of the
\(j\)'th variable. The \emph{sample covariance matrix}, denoted
\(\hat \Sigma\) is defined as
\[\hat \Sigma_{k,l}=(n-1)^{-1} \sum_i [(x_{i,k}-\bar x_k)(x_{i,l}-\bar x_l)].\]
Put differently, the \(k,l\)'th entry in \(\hat \Sigma\) is the sample
covariance between variables \(k\) and \(l\).
\EndKnitrBlock{definition}

\BeginKnitrBlock{remark}
\iffalse {Remark. } \fi \(\hat \Sigma\) is clearly non robust. How would
you define a robust covariance matrix?
\EndKnitrBlock{remark}

\section{Visualization}\label{visualization}

Summarizing the information in a variable to a single number clearly
conceals much of the story in the sample. This is akin to inspecting a
person using a caricature, instead of a picture. Visualizing the data,
when possible, is more informative.

\subsection{Categorical Data}\label{categorical-data-1}

Recalling that with categorical variables we can only count the
frequency of each level, the plotting of such variables are typically
variations on the \emph{bar plot}.

\subsubsection{Visualizing Univariate Categorical
Data}\label{visualizing-univariate-categorical-data}

\begin{Shaded}
\begin{Highlighting}[]
\KeywordTok{barplot}\NormalTok{(}\KeywordTok{table}\NormalTok{(age))}
\end{Highlighting}
\end{Shaded}

\includegraphics[width=0.5\linewidth]{Rcourse_files/figure-latex/barplot-1}

\subsubsection{Visualizing Bivariate Categorical
Data}\label{visualizing-bivariate-categorical-data}

There are several generalizations of the barplot, aimed to deal with the
visualization of bivariate categorical data. They are sometimes known as
the \emph{clustered bar plot} and the \emph{stacked bar plot}. In this
text, we advocate the use of the \emph{mosaic plot} which is also the
default in R.

\begin{Shaded}
\begin{Highlighting}[]
\KeywordTok{plot}\NormalTok{(table1, }\DataTypeTok{main=}\StringTok{'Bivariate mosaic plot'}\NormalTok{)}
\end{Highlighting}
\end{Shaded}

\includegraphics[width=0.5\linewidth]{Rcourse_files/figure-latex/unnamed-chunk-83-1}

\subsubsection{Visualizing Multivariate Categorical
Data}\label{visualizing-multivariate-categorical-data}

The \emph{mosaic plot} is not easy to generalize to more than two
variables, but it is still possible (at the cost of interpretability).

\begin{Shaded}
\begin{Highlighting}[]
\KeywordTok{plot}\NormalTok{(table2}\FloatTok{.1}\NormalTok{, }\DataTypeTok{main=}\StringTok{'Trivaraite mosaic plot'}\NormalTok{)}
\end{Highlighting}
\end{Shaded}

\includegraphics[width=0.5\linewidth]{Rcourse_files/figure-latex/unnamed-chunk-84-1}

\subsection{Continuous Data}\label{continuous-data}

\subsubsection{Visualizing Univariate Continuous
Data}\label{visualizing-univariate-continuous-data}

Unlike categirical variables, there are endlessly many way to visualize
continuous variables. The simplest way is to look at the raw data via
the \texttt{stripchart}.

\begin{Shaded}
\begin{Highlighting}[]
\NormalTok{sample1 <-}\StringTok{ }\KeywordTok{rexp}\NormalTok{(}\DecValTok{10}\NormalTok{)                             }
\KeywordTok{stripchart}\NormalTok{(sample1)}
\end{Highlighting}
\end{Shaded}

\includegraphics[width=0.5\linewidth]{Rcourse_files/figure-latex/unnamed-chunk-85-1}

Clearly, if there are many observations, the \texttt{stripchart} will be
a useless line of black dots. We thus bin them together, and look at the
frequency of each bin; this is the \emph{histogram}. R's
\texttt{histogram} function has very good defaults to choose the number
of bins. Here is a histogram showing the counts of each bin.

\begin{Shaded}
\begin{Highlighting}[]
\NormalTok{sample1 <-}\StringTok{ }\KeywordTok{rexp}\NormalTok{(}\DecValTok{100}\NormalTok{)                            }
\KeywordTok{hist}\NormalTok{(sample1, }\DataTypeTok{freq=}\NormalTok{T, }\DataTypeTok{main=}\StringTok{'Counts'}\NormalTok{)        }
\end{Highlighting}
\end{Shaded}

\includegraphics[width=0.5\linewidth]{Rcourse_files/figure-latex/unnamed-chunk-86-1}

The bin counts can be replaced with the proportion of each bin using the
\texttt{freq} argument.

\begin{Shaded}
\begin{Highlighting}[]
\KeywordTok{hist}\NormalTok{(sample1, }\DataTypeTok{freq=}\NormalTok{F, }\DataTypeTok{main=}\StringTok{'Proportion'}\NormalTok{)    }
\end{Highlighting}
\end{Shaded}

\includegraphics[width=0.5\linewidth]{Rcourse_files/figure-latex/unnamed-chunk-87-1}

The bins of a histogram are non overlapping. We can adopt a sliding
window approach, instead of binning. This is the \emph{density plot}
which is produced with the \texttt{density} function, and added to an
existing plot with the \texttt{lines} function. The \texttt{rug}
function adds the original data points as ticks on the axes, and is
strongly recommended to detect artifacts introduced by the binning of
the histogram, or the smoothing of the density plot.

\begin{Shaded}
\begin{Highlighting}[]
\KeywordTok{hist}\NormalTok{(sample1, }\DataTypeTok{freq=}\NormalTok{F, }\DataTypeTok{main=}\StringTok{'Frequencies'}\NormalTok{)   }
\KeywordTok{lines}\NormalTok{(}\KeywordTok{density}\NormalTok{(sample1))                     }
\KeywordTok{rug}\NormalTok{(sample1)}
\end{Highlighting}
\end{Shaded}

\includegraphics[width=0.5\linewidth]{Rcourse_files/figure-latex/unnamed-chunk-88-1}

\BeginKnitrBlock{remark}
\iffalse {Remark. } \fi Why would it make no sense to make a table, or a
barplot, of continuous data?
\EndKnitrBlock{remark}

One particularly useful visualization, due to John W. Tukey, is the
\emph{boxplot}. The boxplot is designed to capture the main phenomena in
the data, and simultaneously point to outliers.

\begin{Shaded}
\begin{Highlighting}[]
\KeywordTok{boxplot}\NormalTok{(sample1)    }
\end{Highlighting}
\end{Shaded}

\includegraphics[width=0.5\linewidth]{Rcourse_files/figure-latex/unnamed-chunk-90-1}

\subsubsection{Visualizing Bivariate Continuous
Data}\label{visualizing-bivariate-continuous-data}

The bivariate counterpart of the \texttt{stipchart} is the celebrated
scatter plot.

\begin{Shaded}
\begin{Highlighting}[]
\NormalTok{n <-}\StringTok{ }\DecValTok{20}
\NormalTok{x1 <-}\StringTok{ }\KeywordTok{rexp}\NormalTok{(n)}
\NormalTok{x2 <-}\StringTok{ }\DecValTok{2}\NormalTok{*}\StringTok{ }\NormalTok{x1 +}\StringTok{ }\DecValTok{4} \NormalTok{+}\StringTok{ }\KeywordTok{rexp}\NormalTok{(n)}
\KeywordTok{plot}\NormalTok{(x2~x1)}
\end{Highlighting}
\end{Shaded}

\includegraphics[width=0.5\linewidth]{Rcourse_files/figure-latex/unnamed-chunk-91-1}

Like the univariate \texttt{stripchart}, the scatter plot will be an
uninformative mess in the presence of a lot of data. A nice bivariate
counterpart of the univariate histogram is the \emph{hexbin plot}, which
tessellates the plane with hexagons, and reports their frequencies.

\begin{Shaded}
\begin{Highlighting}[]
\KeywordTok{library}\NormalTok{(hexbin)}
\NormalTok{n <-}\StringTok{ }\FloatTok{2e5}
\NormalTok{x1 <-}\StringTok{ }\KeywordTok{rexp}\NormalTok{(n)}
\NormalTok{x2 <-}\StringTok{ }\DecValTok{2}\NormalTok{*}\StringTok{ }\NormalTok{x1 +}\StringTok{ }\DecValTok{4} \NormalTok{+}\StringTok{ }\KeywordTok{rnorm}\NormalTok{(n)}
\KeywordTok{plot}\NormalTok{(}\KeywordTok{hexbin}\NormalTok{(}\DataTypeTok{x =} \NormalTok{x1, }\DataTypeTok{y =} \NormalTok{x2))}
\end{Highlighting}
\end{Shaded}

\includegraphics[width=0.5\linewidth]{Rcourse_files/figure-latex/unnamed-chunk-92-1}

\subsubsection{Visualizing Multivariate Continuous
Data}\label{visualizing-multivariate-continuous-data}

Visualizing multivariate data is a tremendous challenge given that we
cannot grasp \(4\) dimensional spaces, nor can the computer screen
present more than \(2\) dimensional spaces. We thus have several
options: (i) To project the data to 2D. This is discussed in the
Dimensionality Reduction Section \ref{dim-reduce}. (ii) To visualize not
the data, but the summaries, like the covariance matrix.

Since the covariance matrix, \(\hat \Sigma\) is a matrix, it can be
visualized as an image. Note the use of the \texttt{::} operator, which
is used to call a function from some package, without loading the whole
package. We will use the \texttt{::} operator when we want to emphasize
the package of origin of a function.

\begin{Shaded}
\begin{Highlighting}[]
\NormalTok{covariance <-}\StringTok{ }\KeywordTok{cov}\NormalTok{(longley) }\CommentTok{# The covariance of the longley dataset}
\NormalTok{lattice::}\KeywordTok{levelplot}\NormalTok{(covariance)}
\end{Highlighting}
\end{Shaded}

\includegraphics[width=0.5\linewidth]{Rcourse_files/figure-latex/unnamed-chunk-93-1}

\section{Bibliographic Notes}\label{bibliographic-notes-2}

Like any other topic in this book, you can consult
\citet{venables2013modern}. The seminal book on EDA, written long before
R was around, is \citet{tukey1977exploratory}. For an excellent text on
robust statistics see \citet{wilcox2011introduction}.

\section{Practice Yourself}\label{practice-yourself-2}

\chapter{Linear Models}\label{lm}

\section{Problem Setup}\label{problem-setup}

\BeginKnitrBlock{example}[Bottle Cap Production]
\protect\hypertarget{ex:cap-experiment}{}{\label{ex:cap-experiment}
\iffalse (Bottle Cap Production) \fi }Consider a randomized experiment
designed to study the effects of temperature and pressure on the
diameter of manufactured a bottle cap.
\EndKnitrBlock{example}

\BeginKnitrBlock{example}[Rental Prices]
\protect\hypertarget{ex:rental}{}{\label{ex:rental} \iffalse (Rental Prices)
\fi }Consider the prediction of rental prices given an appartment's
attributes.
\EndKnitrBlock{example}

Both examples require some statistical model, but they are very
different. The first is a \emph{causal inference} problem: we want to
design an intervention so that we need to recover the causal effect of
temperature and pressure. The second is a \emph{prediction} problem,
a.k.a. a \emph{forecasting} problem, in which we don't care about the
causal effects, we just want good predictions.

In this chapter we discuss the causal problem in Example
\ref{ex:cap-experiment}. This means that when we assume a model, we
assume it is the actual \emph{data generating process}, i.e., we assume
the \emph{sampling distribution} is well specified. The second type of
problems is discussed in the Supervised Learning Chapter
\ref{supervised}.

Here are some more examples of the types of problems we are discussing.

\BeginKnitrBlock{example}[Plant Growth]
\protect\hypertarget{ex:unnamed-chunk-94}{}{\label{ex:unnamed-chunk-94}
\iffalse (Plant Growth) \fi }Consider the treatment of various plants
with various fertilizers to study the fertilizer's effect on growth.
\EndKnitrBlock{example}

\BeginKnitrBlock{example}[Return to Education]
\protect\hypertarget{ex:unnamed-chunk-95}{}{\label{ex:unnamed-chunk-95}
\iffalse (Return to Education) \fi }Consider the study of return to
education by analyzing the incomes of individuals with different
education years.
\EndKnitrBlock{example}

\BeginKnitrBlock{example}[Drug Effect]
\protect\hypertarget{ex:unnamed-chunk-96}{}{\label{ex:unnamed-chunk-96}
\iffalse (Drug Effect) \fi }Consider the study of the effect of a new
drug, by analyzing the level of blood coagulation after the
administration of various amounts of the new drug.
\EndKnitrBlock{example}

Let's present the linear model. We assume that a response\footnote{The
  ``response'' is also know as the ``dependent'' variable in the
  statistical literature, or the ``labels'' in the machine learning
  literature.} variable is the sum of effects of some factors\footnote{The
  ``factors'' are also known as the ``independent variable'', or ``the
  design'', in the statistical literature, and the ``features'', or
  ``attributes'' in the machine learning literature.}. Denoting the
dependent by \(y\), the factors by \(x\), and the effects by \(\beta\)
the linear model assumption implies that

\begin{align}
  E[y]=\sum_j x_j \beta_j=x'\beta .
  \label{eq:linear-mean}
\end{align}

Clearly, there may be other factors that affect the the caps' diameters.
We thus introduce an error term\footnote{The ``error term'' is also
  known as the ``noise'', or the ``common causes of variability''.},
denoted by \(\varepsilon\), to capture the effects of all unmodeled
factors and measurement error\footnote{You may philosophize if the
  measurement error is a mere instance of unmodeled factors or not, but
  this has no real implication for our purposes.}. The implied
generative process of a sample of \(i=1,\dots,n\) observations it thus

\begin{align}
  y_i = \sum_j x_{i,j} \beta_j + \varepsilon_i , i=1,\dots,n .
  \label{eq:linear-observed}
\end{align}

or in matrix notation

\begin{align}
  y = X \beta + \varepsilon .
  \label{eq:linear-matrix}
\end{align}

Let's demonstrate Eq.\eqref{eq:linear-observed}. In our cap example
(\ref{ex:cap-experiment}), assuming that pressure and temperature have
two levels each (say, high and low), we would write \(x_{i,1}=1\) if the
pressure of the \(i\)'th measurement was set to high, and \(x_{i,1}=-1\)
if the \textbf{pressure} was set to low. Similarly, we would write
\(x_{i,2}=1\), and \(x_{i,2}=-1\), if the \textbf{temperature} was set
to high, or low, respectively. The coding with \(\{-1,1\}\) is known as
\emph{effect coding}. If you prefer coding with \(\{0,1\}\), this is
known as \emph{dummy coding}.

In
\href{https://en.wikipedia.org/wiki/Regression_toward_the_mean}{Galton's}
classical regression problem, where we try to seek the relation between
the heights of sons and fathers then \(p=1\), \(y_i\) is the height of
the \(i\)'th father, and \(x_i\) the height of the \(i\)'th son.

There are many reasons linear models are very popular:

\begin{enumerate}
\def\labelenumi{\arabic{enumi}.}
\item
  Before the computer age, these were pretty much the only models that
  could actually be computed\footnote{By ``computed'' we mean what
    statisticians call ``fitted'', or ``estimated'', and computer
    scientists call ``learned''.}. The whole Analysis of Variance
  (ANOVA) literature is an instance of linear models, that relies on
  sums of squares, which do not require a computer to work with.
\item
  For purposes of prediction, where the actual data generating process
  is not of primary importance, they are popular because they simply
  work. Why is that? They are simple so that they do not require a lot
  of data to be computed. Put differently, they may be biased, but their
  variance is small enough to make them more accurate than other models.
\item
  For categorical or factorial predictors, \textbf{any} functional
  relation can be cast as a linear model.
\item
  For the purpose of \emph{screening}, where we only want to show the
  existence of an effect, and are less interested in the magnitude of
  that effect, a linear model is enough.
\item
  If the true generative relation is not linear, but smooth enough, then
  the linear function is a good approximation via Taylor's theorem.
\end{enumerate}

There are still two matters we have to attend: (i) How the estimate
\(\beta\)? (ii) How to perform inference?

In the simplest linear models the estimation of \(\beta\) is done using
the method of least squares. A linear model with least squares
estimation is known as Ordinary Least Squares (OLS). The OLS problem:

\begin{align}
  \hat \beta:= argmin_\beta \{ \sum_i (y_i-x_i'\beta)^2 \},
  \label{eq:ols}
\end{align}

and in matrix notation

\begin{align}
  \hat \beta:= argmin_\beta \{ \Vert y-X\beta \Vert^2_2 \}.
  \label{eq:ols-matrix}
\end{align}

\BeginKnitrBlock{remark}
\iffalse {Remark. } \fi Personally, I prefer the matrix notation because
it is suggestive of the geometry of the problem. The reader is referred
to \citet{friedman2001elements}, Section 3.2, for more on the geometry
of OLS.
\EndKnitrBlock{remark}

Different software suits, and even different R packages, solve
Eq.\eqref{eq:ols} in different ways so that we skip the details of how
exactly it is solved. These are discussed in Chapters \ref{algebra} and
\ref{convex}.

The last matter we need to attend is how to do inference on
\(\hat \beta\). For that, we will need some assumptions on
\(\varepsilon\). A typical set of assumptions is the following:

\begin{enumerate}
\def\labelenumi{\arabic{enumi}.}
\tightlist
\item
  \textbf{Independence}: we assume \(\varepsilon_i\) are independent of
  everything else. Think of them as the measurement error of an
  instrument: it is independent of the measured value and of previous
  measurements.
\item
  \textbf{Centered}: we assume that \(E[\varepsilon]=0\), meaning there
  is no systematic error.
\item
  \textbf{Normality}: we will typically assume that
  \(\varepsilon \sim \mathcal{N}(0,\sigma^2)\), but we will later see
  that this is not really required.
\end{enumerate}

We emphasize that these assumptions are only needed for inference on
\(\hat \beta\) and not for the estimation itself, which is done by the
purely algorithmic framework of OLS.

Given the above assumptions, we can apply some probability theory and
linear algebra to get the distribution of the estimation error:

\begin{align}
  \hat \beta - \beta \sim \mathcal{N}(0, (X'X)^{-1} \sigma^2).
  \label{eq:ols-distribution}
\end{align}

The reason I am not too strict about the normality assumption above, is
that Eq.\eqref{eq:ols-distribution} is approximately correct even if
\(\varepsilon\) is not normal, provided that there are many more
observations than factors (\(n \gg p\)).

\section{OLS Estimation in R}\label{ols-estimation-in-r}

We are now ready to estimate some linear models with R. We will use the
\texttt{whiteside} data from the \textbf{MASS} package, recording the
outside temperature and gas consumption, before and after an
appartment's insulation.

\begin{Shaded}
\begin{Highlighting}[]
\KeywordTok{library}\NormalTok{(MASS)}
\KeywordTok{data}\NormalTok{(MASS::whiteside)}
\end{Highlighting}
\end{Shaded}

\begin{verbatim}
## Warning in data(MASS::whiteside): data set 'MASS::whiteside' not found
\end{verbatim}

\begin{Shaded}
\begin{Highlighting}[]
\KeywordTok{head}\NormalTok{(whiteside) }\CommentTok{# inspect the data}
\end{Highlighting}
\end{Shaded}

\begin{verbatim}
##    Insul Temp Gas
## 1 Before -0.8 7.2
## 2 Before -0.7 6.9
## 3 Before  0.4 6.4
## 4 Before  2.5 6.0
## 5 Before  2.9 5.8
## 6 Before  3.2 5.8
\end{verbatim}

We do the OLS estimation on the pre-insulation data with \texttt{lm}
function, possibly the most important function in R.

\begin{Shaded}
\begin{Highlighting}[]
\NormalTok{lm}\FloatTok{.1} \NormalTok{<-}\StringTok{ }\KeywordTok{lm}\NormalTok{(Gas~Temp, }\DataTypeTok{data=}\NormalTok{whiteside[whiteside$Insul==}\StringTok{'Before'}\NormalTok{,]) }\CommentTok{# OLS estimation }
\end{Highlighting}
\end{Shaded}

Things to note:

\begin{itemize}
\tightlist
\item
  We used the tilde syntax \texttt{Gas\textasciitilde{}Temp}, reading
  ``gas as linear function of temperature''.
\item
  The \texttt{data} argument tells R where to look for the variables Gas
  and Temp. We used only observations before the insulation.
\item
  The result is assigned to the object \texttt{lm.1}.
\end{itemize}

Alternative formulations with the same results would be

\begin{Shaded}
\begin{Highlighting}[]
\NormalTok{lm}\FloatTok{.1} \NormalTok{<-}\StringTok{ }\KeywordTok{lm}\NormalTok{(}\DataTypeTok{y=}\NormalTok{Gas, }\DataTypeTok{x=}\NormalTok{Temp, }\DataTypeTok{data=}\NormalTok{whiteside[whiteside$Insul==}\StringTok{'Before'}\NormalTok{,]) }
\NormalTok{lm}\FloatTok{.1} \NormalTok{<-}\StringTok{ }\KeywordTok{lm}\NormalTok{(}\DataTypeTok{y=}\NormalTok{whiteside[whiteside$Insul==}\StringTok{'Before'}\NormalTok{,]$Gas, }\DataTypeTok{x=}\NormalTok{whiteside[whiteside$Insul==}\StringTok{'Before'}\NormalTok{,]$Temp)  }
\end{Highlighting}
\end{Shaded}

The output is an object of class \texttt{lm}.

\begin{Shaded}
\begin{Highlighting}[]
\KeywordTok{class}\NormalTok{(lm}\FloatTok{.1}\NormalTok{)}
\end{Highlighting}
\end{Shaded}

\begin{verbatim}
## [1] "lm"
\end{verbatim}

Objects of class \texttt{lm} are very complicated. They store a lot of
information which may be used for inference, plotting, etc. The
\texttt{str} function, short for ``structure'', shows us the various
elements of the object.

\begin{Shaded}
\begin{Highlighting}[]
\KeywordTok{str}\NormalTok{(lm}\FloatTok{.1}\NormalTok{)}
\end{Highlighting}
\end{Shaded}

\begin{verbatim}
## List of 12
##  $ coefficients : Named num [1:2] 6.854 -0.393
##   ..- attr(*, "names")= chr [1:2] "(Intercept)" "Temp"
##  $ residuals    : Named num [1:26] 0.0316 -0.2291 -0.2965 0.1293 0.0866 ...
##   ..- attr(*, "names")= chr [1:26] "1" "2" "3" "4" ...
##  $ effects      : Named num [1:26] -24.2203 -5.6485 -0.2541 0.1463 0.0988 ...
##   ..- attr(*, "names")= chr [1:26] "(Intercept)" "Temp" "" "" ...
##  $ rank         : int 2
##  $ fitted.values: Named num [1:26] 7.17 7.13 6.7 5.87 5.71 ...
##   ..- attr(*, "names")= chr [1:26] "1" "2" "3" "4" ...
##  $ assign       : int [1:2] 0 1
##  $ qr           :List of 5
##   ..$ qr   : num [1:26, 1:2] -5.099 0.196 0.196 0.196 0.196 ...
##   .. ..- attr(*, "dimnames")=List of 2
##   .. .. ..$ : chr [1:26] "1" "2" "3" "4" ...
##   .. .. ..$ : chr [1:2] "(Intercept)" "Temp"
##   .. ..- attr(*, "assign")= int [1:2] 0 1
##   ..$ qraux: num [1:2] 1.2 1.35
##   ..$ pivot: int [1:2] 1 2
##   ..$ tol  : num 1e-07
##   ..$ rank : int 2
##   ..- attr(*, "class")= chr "qr"
##  $ df.residual  : int 24
##  $ xlevels      : Named list()
##  $ call         : language lm(formula = Gas ~ Temp, data = whiteside[whiteside$Insul == "Before",      ])
##  $ terms        :Classes 'terms', 'formula'  language Gas ~ Temp
##   .. ..- attr(*, "variables")= language list(Gas, Temp)
##   .. ..- attr(*, "factors")= int [1:2, 1] 0 1
##   .. .. ..- attr(*, "dimnames")=List of 2
##   .. .. .. ..$ : chr [1:2] "Gas" "Temp"
##   .. .. .. ..$ : chr "Temp"
##   .. ..- attr(*, "term.labels")= chr "Temp"
##   .. ..- attr(*, "order")= int 1
##   .. ..- attr(*, "intercept")= int 1
##   .. ..- attr(*, "response")= int 1
##   .. ..- attr(*, ".Environment")=<environment: R_GlobalEnv> 
##   .. ..- attr(*, "predvars")= language list(Gas, Temp)
##   .. ..- attr(*, "dataClasses")= Named chr [1:2] "numeric" "numeric"
##   .. .. ..- attr(*, "names")= chr [1:2] "Gas" "Temp"
##  $ model        :'data.frame':   26 obs. of  2 variables:
##   ..$ Gas : num [1:26] 7.2 6.9 6.4 6 5.8 5.8 5.6 4.7 5.8 5.2 ...
##   ..$ Temp: num [1:26] -0.8 -0.7 0.4 2.5 2.9 3.2 3.6 3.9 4.2 4.3 ...
##   ..- attr(*, "terms")=Classes 'terms', 'formula'  language Gas ~ Temp
##   .. .. ..- attr(*, "variables")= language list(Gas, Temp)
##   .. .. ..- attr(*, "factors")= int [1:2, 1] 0 1
##   .. .. .. ..- attr(*, "dimnames")=List of 2
##   .. .. .. .. ..$ : chr [1:2] "Gas" "Temp"
##   .. .. .. .. ..$ : chr "Temp"
##   .. .. ..- attr(*, "term.labels")= chr "Temp"
##   .. .. ..- attr(*, "order")= int 1
##   .. .. ..- attr(*, "intercept")= int 1
##   .. .. ..- attr(*, "response")= int 1
##   .. .. ..- attr(*, ".Environment")=<environment: R_GlobalEnv> 
##   .. .. ..- attr(*, "predvars")= language list(Gas, Temp)
##   .. .. ..- attr(*, "dataClasses")= Named chr [1:2] "numeric" "numeric"
##   .. .. .. ..- attr(*, "names")= chr [1:2] "Gas" "Temp"
##  - attr(*, "class")= chr "lm"
\end{verbatim}

At this point, we only want \(\hat \beta\) which can be extracted with
the \texttt{coef} function.

\begin{Shaded}
\begin{Highlighting}[]
\KeywordTok{coef}\NormalTok{(lm}\FloatTok{.1}\NormalTok{)}
\end{Highlighting}
\end{Shaded}

\begin{verbatim}
## (Intercept)        Temp 
##   6.8538277  -0.3932388
\end{verbatim}

Things to note:

\begin{itemize}
\item
  R automatically adds an \texttt{(Intercept)} term. This means we
  estimate \(y=\beta_0 + \beta_1 Gas + \varepsilon\) and not
  \(y=\beta_1 Gas + \varepsilon\). This makes sense because we are
  interested in the contribution of the temperature to the variability
  of the gas consumption about its \textbf{mean}, and not about zero.
\item
  The effect of temperature, i.e., \(\hat \beta_1\), is -0.39. The
  negative sign means that the higher the temperature, the less gas is
  consumed. The magnitude of the coefficient means that for a unit
  increase in the outside temperature, the gas consumption decreases by
  0.39 units.
\end{itemize}

We can use the \texttt{predict} function to make predictions, but we
emphasize that if the purpose of the model is to make predictions, and
not interpret coefficients, better skip to the Supervised Learning
Chapter \ref{supervised}.

\begin{Shaded}
\begin{Highlighting}[]
\KeywordTok{plot}\NormalTok{(}\KeywordTok{predict}\NormalTok{(lm}\FloatTok{.1}\NormalTok{)~whiteside[whiteside$Insul==}\StringTok{'Before'}\NormalTok{,]$Gas)}
\KeywordTok{abline}\NormalTok{(}\DecValTok{0}\NormalTok{,}\DecValTok{1}\NormalTok{, }\DataTypeTok{lty=}\DecValTok{2}\NormalTok{)}
\end{Highlighting}
\end{Shaded}

\includegraphics[width=0.5\linewidth]{Rcourse_files/figure-latex/unnamed-chunk-104-1}

The model seems to fit the data nicely. A common measure of the goodness
of fit is the \emph{coefficient of determination}, more commonly known
as the \(R^2\).



\BeginKnitrBlock{definition}[(ref:R2)]
\protect\hypertarget{def:unnamed-chunk-105}{}{\label{def:unnamed-chunk-105}
\iffalse (\(R^2\).) \fi }The coefficient of determination, denoted
\(R^2\), is defined as

\begin{align}
  R^2:= 1-\frac{\sum_i (y_i - \hat y_i)^2}{\sum_i (y_i - \bar y)^2},
\end{align}

where \(\hat y_i\) is the model's prediction,
\(\hat y_i = x_i \hat \beta\).
\EndKnitrBlock{definition}

It can be easily computed

\begin{Shaded}
\begin{Highlighting}[]
\NormalTok{R2 <-}\StringTok{ }\NormalTok{function(y, y.hat)\{}
  \NormalTok{numerator <-}\StringTok{ }\NormalTok{(y-y.hat)^}\DecValTok{2} \NormalTok\StringTok{ }\NormalTok{sum}
  \NormalTok{denominator <-}\StringTok{ }\NormalTok{(y-}\KeywordTok{mean}\NormalTok{(y))^}\DecValTok{2} \NormalTok\StringTok{ }\NormalTok{sum}
  \DecValTok{1}\NormalTok{-numerator/denominator}
\NormalTok{\}}
\KeywordTok{R2}\NormalTok{(}\DataTypeTok{y=}\NormalTok{whiteside[whiteside$Insul==}\StringTok{'Before'}\NormalTok{,]$Gas, }\DataTypeTok{y.hat=}\KeywordTok{predict}\NormalTok{(lm}\FloatTok{.1}\NormalTok{))}
\end{Highlighting}
\end{Shaded}

\begin{verbatim}
## [1] 0.9438081
\end{verbatim}

This is a nice result implying that about \(94\%\) of the variability in
gas consumption can be attributed to changes in the outside temperature.

Obviously, R does provide the means to compute something as basic as
\(R^2\), but I will let you find it for yourselves.

\section{Inference}\label{inference}

To perform inference on \(\hat \beta\), in order to test hypotheses and
construct confidence intervals, we need to quantify the uncertainly in
the reported \(\hat \beta\). This is exactly what
Eq.\eqref{eq:ols-distribution} gives us.

Luckily, we don't need to manipulate multivariate distributions
manually, and everything we need is already implemented. The most
important function is \texttt{summary} which gives us an overview of the
model's fit. We emphasize that that fitting a model with \texttt{lm} is
an assumption free algorithmic step. Inference using \texttt{summary} is
\textbf{not} assumption free, and requires the set of assumptions
leading to Eq.\eqref{eq:ols-distribution}.

\begin{Shaded}
\begin{Highlighting}[]
\KeywordTok{summary}\NormalTok{(lm}\FloatTok{.1}\NormalTok{)}
\end{Highlighting}
\end{Shaded}

\begin{verbatim}
## 
## Call:
## lm(formula = Gas ~ Temp, data = whiteside[whiteside$Insul == 
##     "Before", ])
## 
## Residuals:
##      Min       1Q   Median       3Q      Max 
## -0.62020 -0.19947  0.06068  0.16770  0.59778 
## 
## Coefficients:
##             Estimate Std. Error t value Pr(>|t|)    
## (Intercept)  6.85383    0.11842   57.88   <2e-16 ***
## Temp        -0.39324    0.01959  -20.08   <2e-16 ***
## ---
## Signif. codes:  0 '***' 0.001 '**' 0.01 '*' 0.05 '.' 0.1 ' ' 1
## 
## Residual standard error: 0.2813 on 24 degrees of freedom
## Multiple R-squared:  0.9438, Adjusted R-squared:  0.9415 
## F-statistic: 403.1 on 1 and 24 DF,  p-value: < 2.2e-16
\end{verbatim}

Things to note:

\begin{itemize}
\tightlist
\item
  The estimated \(\hat \beta\) is reported in the `Coefficients' table,
  which has point estimates, standard errors, t-statistics, and the
  p-values of a two-sided hypothesis test for each coefficient
  \(H_{0,j}:\beta_j=0, j=1,\dots,p\).
\item
  The \(R^2\) is reported at the bottom. The ``Adjusted R-squared'' is a
  variation that compensates for the model's complexity.
\item
  The original call to \texttt{lm} is saved in the \texttt{Call}
  section.
\item
  Some summary statistics of the residuals (\(y_i-\hat y_i\)) in the
  \texttt{Residuals} section.
\item
  The ``residuals standard error''\footnote{Sometimes known as the Root
    Mean Squared Error (RMSE).} is
  \(\sqrt{(n-p)^{-1} \sum_i (y_i-\hat y_i)^2}\). The denominator of this
  expression is the \emph{degrees of freedom}, \(n-p\), which can be
  thought of as the hardness of the problem.
\end{itemize}

As the name suggests, \texttt{summary} is merely a summary. The full
\texttt{summary(lm.1)} object is a monstrous object. Its various
elements can be queried using \texttt{str(sumary(lm.1))}.

Can we check the assumptions required for inference? Some. Let's start
with the linearity assumption. If we were wrong, and the data is not
arranged about a linear line, the residuals will have some shape. We
thus plot the residuals as a function of the predictor to diagnose
shape.

\begin{Shaded}
\begin{Highlighting}[]
\KeywordTok{plot}\NormalTok{(}\KeywordTok{residuals}\NormalTok{(lm}\FloatTok{.1}\NormalTok{)~whiteside[whiteside$Insul==}\StringTok{'Before'}\NormalTok{,]$Temp)}
\KeywordTok{abline}\NormalTok{(}\DecValTok{0}\NormalTok{,}\DecValTok{0}\NormalTok{, }\DataTypeTok{lty=}\DecValTok{2}\NormalTok{)}
\end{Highlighting}
\end{Shaded}

\includegraphics[width=0.5\linewidth]{Rcourse_files/figure-latex/unnamed-chunk-108-1}

I can't say I see any shape. Let's fit a \textbf{wrong} model, just to
see what ``shape'' means.

\begin{Shaded}
\begin{Highlighting}[]
\NormalTok{lm}\FloatTok{.1.1} \NormalTok{<-}\StringTok{ }\KeywordTok{lm}\NormalTok{(Gas~}\KeywordTok{I}\NormalTok{(Temp^}\DecValTok{2}\NormalTok{), }\DataTypeTok{data=}\NormalTok{whiteside[whiteside$Insul==}\StringTok{'Before'}\NormalTok{,])}
\KeywordTok{plot}\NormalTok{(}\KeywordTok{residuals}\NormalTok{(lm}\FloatTok{.1.1}\NormalTok{)~whiteside[whiteside$Insul==}\StringTok{'Before'}\NormalTok{,]$Temp); }\KeywordTok{abline}\NormalTok{(}\DecValTok{0}\NormalTok{,}\DecValTok{0}\NormalTok{, }\DataTypeTok{lty=}\DecValTok{2}\NormalTok{)}
\end{Highlighting}
\end{Shaded}

\includegraphics[width=0.5\linewidth]{Rcourse_files/figure-latex/unnamed-chunk-109-1}

Things to note:

\begin{itemize}
\tightlist
\item
  We used \texttt{I(Temp)\^{}2} to specify the model
  \(Gas=\beta_0 + \beta_1 Temp^2+ \varepsilon\).
\item
  The residuals have a ``belly''. Because they are not a cloud around
  the linear trend, and we have the wrong model.
\end{itemize}

To the next assumption. We assumed \(\varepsilon_i\) are independent of
everything else. The residuals, \(y_i-\hat y_i\) can be thought of a
sample of \(\varepsilon_i\). When diagnosing the linearity assumption,
we already saw their distribution does not vary with the \(x\)'s,
\texttt{Temp} in our case. They may be correlated with themselves; a
positive departure from the model, may be followed by a series of
positive departures etc. Diagnosing these \emph{auto-correlations} is a
real art, which is not part of our course.

The last assumption we required is normality. As previously stated, if
\(n \gg p\), this assumption can be relaxed. If \(n \sim p\), i.e.,
\(n\) is in the order of \(p\), we need to verify this assumption. My
favorite tool for this task is the \emph{qqplot}. A qqplot compares the
quantiles of the sample with the respective quantiles of the assumed
distribution. If quantiles align along a line, the assumed distribution
if OK. If quantiles depart from a line, then the assumed distribution
does not fit the sample.

\begin{Shaded}
\begin{Highlighting}[]
\KeywordTok{qqnorm}\NormalTok{(}\KeywordTok{resid}\NormalTok{((lm}\FloatTok{.1}\NormalTok{)))}
\end{Highlighting}
\end{Shaded}

\includegraphics[width=0.5\linewidth]{Rcourse_files/figure-latex/unnamed-chunk-110-1}

The \texttt{qqnorm} function plots a qqplot against a normal
distribution. Judging from the figure, the normality assumption is quite
plausible. Let's try the same on a non-normal sample, namely a uniformly
distributed sample, to see how that would look.

\begin{Shaded}
\begin{Highlighting}[]
\KeywordTok{qqnorm}\NormalTok{(}\KeywordTok{runif}\NormalTok{(}\DecValTok{100}\NormalTok{))}
\end{Highlighting}
\end{Shaded}

\includegraphics[width=0.5\linewidth]{Rcourse_files/figure-latex/unnamed-chunk-111-1}

\subsection{Testing a Hypothesis on a Single
Coefficient}\label{testing-a-hypothesis-on-a-single-coefficient}

The first inferential test we consider is a hypothesis test on a single
coefficient. In our gas example, we may want to test that the
temperature has no effect on the gas consumption. The answer for that is
given immediately by \texttt{summary(lm.1)}

\begin{Shaded}
\begin{Highlighting}[]
\NormalTok{summary.lm1 <-}\StringTok{ }\KeywordTok{summary}\NormalTok{(lm}\FloatTok{.1}\NormalTok{)}
\NormalTok{coefs.lm1 <-}\StringTok{ }\NormalTok{summary.lm1$coefficients}
\NormalTok{coefs.lm1}
\end{Highlighting}
\end{Shaded}

\begin{verbatim}
##               Estimate Std. Error   t value     Pr(>|t|)
## (Intercept)  6.8538277 0.11842341  57.87561 2.717533e-27
## Temp        -0.3932388 0.01958601 -20.07754 1.640469e-16
\end{verbatim}

We see that the p-value for \(H_{0,1}:\hat \beta_1=0\) against a two
sided alternative is effectively 0, so that \(\beta_1\) is unlikely to
be \(0\).

\subsection{Constructing a Confidence Interval on a Single
Coefficient}\label{constructing-a-confidence-interval-on-a-single-coefficient}

Since the \texttt{summary} function gives us the standard errors of
\(\hat \beta\), we can immediately compute
\(\hat \beta_j \pm 2 \sqrt{Var[\hat \beta_j]}\) to get ourselves a
(roughly) \(95\%\) confidence interval. In our example the interval is

\begin{Shaded}
\begin{Highlighting}[]
\NormalTok{coefs.lm1[}\DecValTok{2}\NormalTok{,}\DecValTok{1}\NormalTok{] +}\StringTok{ }\KeywordTok{c}\NormalTok{(-}\DecValTok{2}\NormalTok{,}\DecValTok{2}\NormalTok{) *}\StringTok{ }\NormalTok{coefs.lm1[}\DecValTok{2}\NormalTok{,}\DecValTok{2}\NormalTok{]}
\end{Highlighting}
\end{Shaded}

\begin{verbatim}
## [1] -0.4324108 -0.3540668
\end{verbatim}

\subsection{Multiple Regression}\label{multiple-regression}

\BeginKnitrBlock{remark}
\iffalse {Remark. } \fi \emph{Multiple regression} is not to be confused
with \emph{multivariate regression} discussed in Chapter
\ref{multivariate}.
\EndKnitrBlock{remark}

Our next example\footnote{The example is taken from
  \url{http://rtutorialseries.blogspot.co.il/2011/02/r-tutorial-series-two-way-anova-with.html}}
contains a hypothetical sample of \(60\) participants who are divided
into three stress reduction treatment groups (mental, physical, and
medical) and two gender groups (male and female). The stress reduction
values are represented on a scale that ranges from 1 to 5. This dataset
can be conceptualized as a comparison between three stress treatment
programs, one using mental methods, one using physical training, and one
using medication across genders. The values represent how effective the
treatment programs were at reducing participant's stress levels, with
larger effects indicating higher effectiveness.

\begin{Shaded}
\begin{Highlighting}[]
\NormalTok{data <-}\StringTok{ }\KeywordTok{read.csv}\NormalTok{(}\StringTok{'dataset_anova_twoWay_comparisons.csv'}\NormalTok{)}
\KeywordTok{head}\NormalTok{(data)}
\end{Highlighting}
\end{Shaded}

\begin{verbatim}
##   Treatment   Age StressReduction
## 1    mental young              10
## 2    mental young               9
## 3    mental young               8
## 4    mental   mid               7
## 5    mental   mid               6
## 6    mental   mid               5
\end{verbatim}

How many observations per group?

\begin{Shaded}
\begin{Highlighting}[]
\KeywordTok{table}\NormalTok{(data$Treatment, data$Age)}
\end{Highlighting}
\end{Shaded}

\begin{verbatim}
##           
##            mid old young
##   medical    3   3     3
##   mental     3   3     3
##   physical   3   3     3
\end{verbatim}

Since we have two factorial predictors, this multiple regression is
nothing but a \emph{two way ANOVA}. Let's fit the model and inspect it.

\begin{Shaded}
\begin{Highlighting}[]
\NormalTok{lm}\FloatTok{.2} \NormalTok{<-}\StringTok{ }\KeywordTok{lm}\NormalTok{(StressReduction~.-}\DecValTok{1}\NormalTok{,}\DataTypeTok{data=}\NormalTok{data)}
\KeywordTok{summary}\NormalTok{(lm}\FloatTok{.2}\NormalTok{)}
\end{Highlighting}
\end{Shaded}

\begin{verbatim}
## 
## Call:
## lm(formula = StressReduction ~ . - 1, data = data)
## 
## Residuals:
##    Min     1Q Median     3Q    Max 
##     -1     -1      0      1      1 
## 
## Coefficients:
##                   Estimate Std. Error t value Pr(>|t|)    
## Treatmentmedical    4.0000     0.3892  10.276 7.34e-10 ***
## Treatmentmental     6.0000     0.3892  15.414 2.84e-13 ***
## Treatmentphysical   5.0000     0.3892  12.845 1.06e-11 ***
## Ageold             -3.0000     0.4264  -7.036 4.65e-07 ***
## Ageyoung            3.0000     0.4264   7.036 4.65e-07 ***
## ---
## Signif. codes:  0 '***' 0.001 '**' 0.01 '*' 0.05 '.' 0.1 ' ' 1
## 
## Residual standard error: 0.9045 on 22 degrees of freedom
## Multiple R-squared:  0.9794, Adjusted R-squared:  0.9747 
## F-statistic:   209 on 5 and 22 DF,  p-value: < 2.2e-16
\end{verbatim}

Things to note:

\begin{itemize}
\item
  The \texttt{StressReduction\textasciitilde{}.} syntax is read as
  ``Stress reduction as a function of everything else''.
\item
  The \texttt{StressReduction\textasciitilde{}.-1} means that I do not
  want an intercept in the model, so that the baseline response is 0.
\item
  All the (main) effects seem to be significant.
\item
  The data has 2 factors, but the coefficients table has 4 predictors.
  This is because \texttt{lm} noticed that \texttt{Treatment} and
  \texttt{Age} are factors. The numerical values of the factors are
  meaningless. Instead, R has constructed a dummy variable for each
  level of each factor. The names of the effect are a concatenation of
  the factor's name, and its level. You can inspect these dummy
  variables with the \texttt{model.matrix} command.
\end{itemize}

\begin{Shaded}
\begin{Highlighting}[]
\KeywordTok{head}\NormalTok{(}\KeywordTok{model.matrix}\NormalTok{(lm}\FloatTok{.2}\NormalTok{))}
\end{Highlighting}
\end{Shaded}

\begin{verbatim}
##   Treatmentmedical Treatmentmental Treatmentphysical Ageold Ageyoung
## 1                0               1                 0      0        1
## 2                0               1                 0      0        1
## 3                0               1                 0      0        1
## 4                0               1                 0      0        0
## 5                0               1                 0      0        0
## 6                0               1                 0      0        0
\end{verbatim}

If you don't want the default dummy coding, look at \texttt{?contrasts}.

If you are more familiar with the ANOVA literature, or that you don't
want the effects of each level separately, but rather, the effect of
\textbf{all} the levels of each factor, use the \texttt{anova} command.

\begin{Shaded}
\begin{Highlighting}[]
\KeywordTok{anova}\NormalTok{(lm}\FloatTok{.2}\NormalTok{)}
\end{Highlighting}
\end{Shaded}

\begin{verbatim}
## Analysis of Variance Table
## 
## Response: StressReduction
##           Df Sum Sq Mean Sq F value Pr(>F)    
## Treatment  3    693 231.000  282.33 <2e-16 ***
## Age        2    162  81.000   99.00  1e-11 ***
## Residuals 22     18   0.818                   
## ---
## Signif. codes:  0 '***' 0.001 '**' 0.01 '*' 0.05 '.' 0.1 ' ' 1
\end{verbatim}

Things to note:

\begin{itemize}
\tightlist
\item
  The ANOVA table, unlike the \texttt{summary} function, tests if
  \textbf{any} of the levels of a factor has an effect, and not one
  level at a time.
\item
  The significance of each factor is computed using an F-test.
\item
  The degrees of freedom, encoding the number of levels of a factor, is
  given in the \texttt{Df} column.
\item
  The StressReduction seems to vary for different ages and treatments,
  since both factors are significant.
\end{itemize}

As in any two-way ANOVA, we may want to ask if different age groups
respond differently to different treatments. In the statistical
parlance, this is called an \emph{interaction}, or more precisely, an
\emph{interaction of order 2}.

\begin{Shaded}
\begin{Highlighting}[]
\NormalTok{lm}\FloatTok{.3} \NormalTok{<-}\StringTok{ }\KeywordTok{lm}\NormalTok{(StressReduction~Treatment+Age+Treatment:Age}\DecValTok{-1}\NormalTok{,}\DataTypeTok{data=}\NormalTok{data)}
\end{Highlighting}
\end{Shaded}

The syntax
\texttt{StressReduction\textasciitilde{}Treatment+Age+Treatment:Age-1}
tells R to include main effects of Treatment, Age, and their
interactions. Here are other ways to specify the same model.

\begin{Shaded}
\begin{Highlighting}[]
\NormalTok{lm}\FloatTok{.3} \NormalTok{<-}\StringTok{ }\KeywordTok{lm}\NormalTok{(StressReduction ~}\StringTok{ }\NormalTok{Treatment *}\StringTok{ }\NormalTok{Age -}\StringTok{ }\DecValTok{1}\NormalTok{,}\DataTypeTok{data=}\NormalTok{data)}
\NormalTok{lm}\FloatTok{.3} \NormalTok{<-}\StringTok{ }\KeywordTok{lm}\NormalTok{(StressReduction~(.)^}\DecValTok{2} \NormalTok{-}\StringTok{ }\DecValTok{1}\NormalTok{,}\DataTypeTok{data=}\NormalTok{data)}
\end{Highlighting}
\end{Shaded}

The syntax \texttt{Treatment\ *\ Age} means ``mains effects with second
order interactions''. The syntax \texttt{(.)\^{}2} means ``everything
with second order interactions''

Let's inspect the model

\begin{Shaded}
\begin{Highlighting}[]
\KeywordTok{summary}\NormalTok{(lm}\FloatTok{.3}\NormalTok{)}
\end{Highlighting}
\end{Shaded}

\begin{verbatim}
## 
## Call:
## lm(formula = StressReduction ~ Treatment + Age + Treatment:Age - 
##     1, data = data)
## 
## Residuals:
##    Min     1Q Median     3Q    Max 
##     -1     -1      0      1      1 
## 
## Coefficients:
##                              Estimate Std. Error t value Pr(>|t|)    
## Treatmentmedical            4.000e+00  5.774e-01   6.928 1.78e-06 ***
## Treatmentmental             6.000e+00  5.774e-01  10.392 4.92e-09 ***
## Treatmentphysical           5.000e+00  5.774e-01   8.660 7.78e-08 ***
## Ageold                     -3.000e+00  8.165e-01  -3.674  0.00174 ** 
## Ageyoung                    3.000e+00  8.165e-01   3.674  0.00174 ** 
## Treatmentmental:Ageold      4.246e-16  1.155e+00   0.000  1.00000    
## Treatmentphysical:Ageold    1.034e-15  1.155e+00   0.000  1.00000    
## Treatmentmental:Ageyoung   -3.126e-16  1.155e+00   0.000  1.00000    
## Treatmentphysical:Ageyoung  5.128e-16  1.155e+00   0.000  1.00000    
## ---
## Signif. codes:  0 '***' 0.001 '**' 0.01 '*' 0.05 '.' 0.1 ' ' 1
## 
## Residual standard error: 1 on 18 degrees of freedom
## Multiple R-squared:  0.9794, Adjusted R-squared:  0.9691 
## F-statistic:    95 on 9 and 18 DF,  p-value: 2.556e-13
\end{verbatim}

Things to note:

\begin{itemize}
\tightlist
\item
  There are still \(5\) main effects, but also \(4\) interactions. This
  is because when allowing a different average response for every
  \(Treatment*Age\) combination, we are effectively estimating \(3*3=9\)
  cell means, even if they are not parametrized as cell means, but
  rather as main effect and interactions.
\item
  The interactions do not seem to be significant.
\item
  The assumptions required for inference are clearly not met in this
  example, which is there just to demonstrate R's capabilities.
\end{itemize}

Asking if all the interactions are significant, is asking if the
different age groups have the same response to different treatments. Can
we answer that based on the various interactions? We might, but it is
possible that no single interaction is significant, while the
combination is. To test for all the interactions together, we can simply
check if the model without interactions is (significantly) better than a
model with interactions. I.e., compare \texttt{lm.2} to \texttt{lm.3}.
This is done with the \texttt{anova} command.

\begin{Shaded}
\begin{Highlighting}[]
\KeywordTok{anova}\NormalTok{(lm}\FloatTok{.2}\NormalTok{,lm}\FloatTok{.3}\NormalTok{, }\DataTypeTok{test=}\StringTok{'F'}\NormalTok{)}
\end{Highlighting}
\end{Shaded}

\begin{verbatim}
## Analysis of Variance Table
## 
## Model 1: StressReduction ~ (Treatment + Age) - 1
## Model 2: StressReduction ~ Treatment + Age + Treatment:Age - 1
##   Res.Df RSS Df Sum of Sq  F Pr(>F)
## 1     22  18                       
## 2     18  18  4         0  0      1
\end{verbatim}

We see that \texttt{lm.3} is \textbf{not} better than \texttt{lm.2}, so
that we can conclude that there are no interactions: different ages have
the same response to different treatments.

\subsection{Testing a Hypothesis on a Single
Contrast}\label{testing-a-hypothesis-on-a-single-contrast}

Returning to the model without interactions, \texttt{lm.2}.

\begin{Shaded}
\begin{Highlighting}[]
\KeywordTok{coef}\NormalTok{(}\KeywordTok{summary}\NormalTok{(lm}\FloatTok{.2}\NormalTok{))}
\end{Highlighting}
\end{Shaded}

\begin{verbatim}
##                   Estimate Std. Error   t value     Pr(>|t|)
## Treatmentmedical         4  0.3892495 10.276186 7.336391e-10
## Treatmentmental          6  0.3892495 15.414279 2.835706e-13
## Treatmentphysical        5  0.3892495 12.845233 1.064101e-11
## Ageold                  -3  0.4264014 -7.035624 4.647299e-07
## Ageyoung                 3  0.4264014  7.035624 4.647299e-07
\end{verbatim}

We see that the effect of the various treatments is rather similar. It
is possible that all treatments actually have the same effect. Comparing
the levels of a factor is called a \emph{contrast}. Let's test if the
medical treatment, has in fact, the same effect as the physical
treatment.

\begin{Shaded}
\begin{Highlighting}[]
\KeywordTok{library}\NormalTok{(multcomp)}
\NormalTok{my.contrast <-}\StringTok{ }\KeywordTok{matrix}\NormalTok{(}\KeywordTok{c}\NormalTok{(-}\DecValTok{1}\NormalTok{,}\DecValTok{0}\NormalTok{,}\DecValTok{1}\NormalTok{,}\DecValTok{0}\NormalTok{,}\DecValTok{0}\NormalTok{), }\DataTypeTok{nrow =}  \DecValTok{1}\NormalTok{)}
\NormalTok{lm}\FloatTok{.4} \NormalTok{<-}\StringTok{ }\KeywordTok{glht}\NormalTok{(lm}\FloatTok{.2}\NormalTok{, }\DataTypeTok{linfct=}\NormalTok{my.contrast)}
\KeywordTok{summary}\NormalTok{(lm}\FloatTok{.4}\NormalTok{)}
\end{Highlighting}
\end{Shaded}

\begin{verbatim}
## 
##   Simultaneous Tests for General Linear Hypotheses
## 
## Fit: lm(formula = StressReduction ~ . - 1, data = data)
## 
## Linear Hypotheses:
##        Estimate Std. Error t value Pr(>|t|)  
## 1 == 0   1.0000     0.4264   2.345   0.0284 *
## ---
## Signif. codes:  0 '***' 0.001 '**' 0.01 '*' 0.05 '.' 0.1 ' ' 1
## (Adjusted p values reported -- single-step method)
\end{verbatim}

Things to note:

\begin{itemize}
\tightlist
\item
  A contrast is a linear function of the coefficients. In our example
  \(H_0:\beta_1-\beta_3=0\), which justifies the construction of
  \texttt{my.contrast}.
\item
  We used the \texttt{glht} function (generalized linear hypothesis
  test) from the package \textbf{multcompt}.
\item
  The contrast is significant, i.e., the effect of a medical treatment,
  is different than that of a physical treatment.
\end{itemize}

\section{Bibliographic Notes}\label{bibliographic-notes-3}

Like any other topic in this book, you can consult
\citet{venables2013modern} for more on linear models. For the theory of
linear models, I like \citet{greene2003econometric}.

\section{Practice Yourself}\label{practice-yourself-3}

\chapter{Generalized Linear Models}\label{glm}

\BeginKnitrBlock{example}
\protect\hypertarget{ex:cigarettes}{}{\label{ex:cigarettes}}Consider the
relation between cigarettes smoked, and the occurance of lung cancer. Do
we expect the probability of cancer to be linear in the number of
cigarettes? Probably not. Do we expect the variability of events to be
constant about the trend? Probably not.
\EndKnitrBlock{example}

\section{Problem Setup}\label{problem-setup-1}

In the Linear Models Chapter \ref{lm}, we assumed the generative process
to be

\begin{align}
  y|x=x'\beta+\varepsilon
  \label{eq:linear-mean-again}
\end{align}

This does not allow for (assumingly) non-linear relations, nor does it
allow for the variability of \(\varepsilon\) to change with \(x\).
\emph{Generalize linear models} (GLM), as the name suggests, are a
generalization that allow that\footnote{Do not confuse \emph{generalized
  linear models} with
  \href{https://en.wikipedia.org/wiki/Nonlinear_regression}{\emph{non-linear
  regression}}, or
  \href{https://en.wikipedia.org/wiki/Generalized_least_squares}{\emph{generalized
  least squares}}. These are different things, that we do not discuss.}.

To understand GLM, we recall that with the normality of \(\varepsilon\),
Eq.\eqref{eq:linear-mean-again} implies that \[
 y|x \sim \mathcal{N}(x'\beta, \sigma^2)
\] For Example \ref{ex:cigarettes}, we would like something in the lines
of \[
 y|x \sim Binom(1,p(x))
\] More generally, for some distribution \(F(\theta)\), with a parameter
\(\theta\), we would like

\begin{align}
  y|x \sim F(\theta(x))
\end{align}

Possible examples include

\begin{align}
 y|x &\sim Poisson(\lambda(x)) \\
 y|x &\sim Exp(\lambda(x)) \\
 y|x &\sim \mathcal{N}(\mu(x),\sigma^2(x)) 
\end{align}

GLMs constrain \(\theta\) to be some known function, \(g\), of a linear
combination of the \(x\)'s. Formally, \[\theta(x)=g(x'\beta),\] where
\[x'\beta=\beta_0 + \sum_j x_j \beta_j.\] The function \(g\) is called
the \emph{link} function.

\section{Logistic Regression}\label{logistic-regression}

The best known of the GLM class of models is the \emph{logistic
regression} that deals with Binomial, or more precisely, Bernoulli
distributed data. The link function in the logistic regression is the
\emph{logistic function}

\begin{align}
  g(t)=\frac{e^t}{(1+e^t)}
  \label{eq:logistic-link}  
\end{align}

implying that

\begin{align}
  y|x \sim Binom \left( 1, p=\frac{e^{x'\beta}}{1+e^{x'\beta}} \right)
  \label{eq:logistic}
\end{align}

Before we fit such a model, we try to justify this construction, in
particular, the enigmatic link function in Eq.\eqref{eq:logistic-link}.
Let's look at the simplest possible case: the comparison of two groups
indexed by \(x\): \(x=0\) for the first, and \(x=1\) for the second. We
start with some definitions.

\BeginKnitrBlock{definition}[Odds]
\protect\hypertarget{def:unnamed-chunk-126}{}{\label{def:unnamed-chunk-126}
\iffalse (Odds) \fi }The \emph{odds}, of a binary random variable,
\(y\), is defined as \[\frac{P(y=1)}{P(y=0)}.\]
\EndKnitrBlock{definition}

Odds are the same as probabilities, but instead of of telling me there
is a \(66\%\) of success, they tell me the odds of success are ``2 to
1''. If you ever placed a bet, the language of ``odds'' should not be
unfamiliar to you.

\BeginKnitrBlock{definition}[Odds Ratio]
\protect\hypertarget{def:unnamed-chunk-127}{}{\label{def:unnamed-chunk-127}
\iffalse (Odds Ratio) \fi }The \emph{odds ratio} between two binary
random variables, \(y_1\) and \(y_2\), is defined as the ratio between
their odds. Formally:
\[OR(y_1,y_2):=\frac{P(y_1=1)/P(y_1=0)}{P(y_2=1)/P(y_2=0)}.\]
\EndKnitrBlock{definition}

Odds ratios (OR) compares between the probabilities of two groups, only
that it does not compare them in probability scale, but rather in odds
scale.

Under the logistic link assumption, the OR between two conditions
indexed by \(y|x=1\) and \(y|x=0\), returns:

\begin{align}
   OR(y|x=1,y|x=0) 
   = \frac{P(y=1|x=1)/P(y=0|x=1)}{P(y=1|x=0)/P(y=0|x=0)} 
   = e^{\beta_1}.  
\end{align}

The last equality demystifies the choice of the link function in the
logistic regression: \textbf{it allows us to interpret \(\beta\) of the
logistic regression as a measure of change of binary random variables,
namely, as the (log) odds-ratios due to a unit increase in \(x\)}.

\BeginKnitrBlock{remark}
\iffalse {Remark. } \fi Another popular link function is the normal
quantile function, a.k.a., the Gaussian inverse CDF, leading to
\emph{probit regression} instead of logistic regression.
\EndKnitrBlock{remark}

\subsection{Logistic Regression with
R}\label{logistic-regression-with-r}

Let's get us some data. The \texttt{PlantGrowth} data records the weight
of plants under three conditions: control, treatment1, and treatment2.

\begin{Shaded}
\begin{Highlighting}[]
\KeywordTok{head}\NormalTok{(PlantGrowth)}
\end{Highlighting}
\end{Shaded}

\begin{verbatim}
##   weight group
## 1   4.17  ctrl
## 2   5.58  ctrl
## 3   5.18  ctrl
## 4   6.11  ctrl
## 5   4.50  ctrl
## 6   4.61  ctrl
\end{verbatim}

We will now \texttt{attach} the data so that its contents is available
in the workspace (don't forget to \texttt{detach} afterwards, or you can
expect some conflicting object names). We will also use the \texttt{cut}
function to create a binary response variable for Light, and Heavy
plants (we are doing logistic regression, so we need a two-class
response). As a general rule of thumb, when we discretize continuous
variables, we lose information. For pedagogical reasons, however, we
will proceed with this bad practice.

\begin{Shaded}
\begin{Highlighting}[]
\KeywordTok{attach}\NormalTok{(PlantGrowth)}
\NormalTok{weight.factor<-}\StringTok{ }\KeywordTok{cut}\NormalTok{(weight, }\DecValTok{2}\NormalTok{, }\DataTypeTok{labels=}\KeywordTok{c}\NormalTok{(}\StringTok{'Light'}\NormalTok{, }\StringTok{'Heavy'}\NormalTok{))}
\KeywordTok{plot}\NormalTok{(}\KeywordTok{table}\NormalTok{(group, weight.factor))}
\end{Highlighting}
\end{Shaded}

\includegraphics[width=0.5\linewidth]{Rcourse_files/figure-latex/unnamed-chunk-130-1}

Let's fit a logistic regression, and inspect the output.

\begin{Shaded}
\begin{Highlighting}[]
\NormalTok{glm}\FloatTok{.1}\NormalTok{<-}\StringTok{ }\KeywordTok{glm}\NormalTok{(weight.factor~group, }\DataTypeTok{family=}\NormalTok{binomial)}
\KeywordTok{summary}\NormalTok{(glm}\FloatTok{.1}\NormalTok{)}
\end{Highlighting}
\end{Shaded}

\begin{verbatim}
## 
## Call:
## glm(formula = weight.factor ~ group, family = binomial)
## 
## Deviance Residuals: 
##     Min       1Q   Median       3Q      Max  
## -2.1460  -0.6681   0.4590   0.8728   1.7941  
## 
## Coefficients:
##             Estimate Std. Error z value Pr(>|z|)  
## (Intercept)   0.4055     0.6455   0.628   0.5299  
## grouptrt1    -1.7918     1.0206  -1.756   0.0792 .
## grouptrt2     1.7918     1.2360   1.450   0.1471  
## ---
## Signif. codes:  0 '***' 0.001 '**' 0.01 '*' 0.05 '.' 0.1 ' ' 1
## 
## (Dispersion parameter for binomial family taken to be 1)
## 
##     Null deviance: 41.054  on 29  degrees of freedom
## Residual deviance: 29.970  on 27  degrees of freedom
## AIC: 35.97
## 
## Number of Fisher Scoring iterations: 4
\end{verbatim}

Things to note:

\begin{itemize}
\tightlist
\item
  The \texttt{glm} function is our workhorse for all GLM models.
\item
  The \texttt{family} argument of \texttt{glm} tells R the output is
  binomial, thus, performing a logistic regression.
\item
  The \texttt{summary} function is content aware. It gives a different
  output for \texttt{glm} class objects than for other objects, such as
  the \texttt{lm} we saw in Chapter \ref{lm}. In fact, what
  \texttt{summary} does is merely call \texttt{summary.glm}.
\item
  As usual, we get the coefficients table, but recall that they are to
  be interpreted as (log) odd-ratios.
\item
  As usual, we get the significance for the test of no-effect, versus a
  two-sided alternative.
\item
  The residuals of \texttt{glm} are slightly different than the
  \texttt{lm} residuals, and called \emph{Deviance Residuals}.
\item
  For help see \texttt{?glm}, \texttt{?family}, and
  \texttt{?summary.glm}.
\end{itemize}

Like in the linear models, we can use an ANOVA table to check if
treatments have any effect, and not one treatment at a time. In the case
of GLMs, this is called an \emph{analysis of deviance} table.

\begin{Shaded}
\begin{Highlighting}[]
\KeywordTok{anova}\NormalTok{(glm}\FloatTok{.1}\NormalTok{, }\DataTypeTok{test=}\StringTok{'LRT'}\NormalTok{)}
\end{Highlighting}
\end{Shaded}

\begin{verbatim}
## Analysis of Deviance Table
## 
## Model: binomial, link: logit
## 
## Response: weight.factor
## 
## Terms added sequentially (first to last)
## 
## 
##       Df Deviance Resid. Df Resid. Dev Pr(>Chi)   
## NULL                     29     41.054            
## group  2   11.084        27     29.970 0.003919 **
## ---
## Signif. codes:  0 '***' 0.001 '**' 0.01 '*' 0.05 '.' 0.1 ' ' 1
\end{verbatim}

Things to note:

\begin{itemize}
\tightlist
\item
  The \texttt{anova} function, like the \texttt{summary} function, are
  content-aware and produce a different output for the \texttt{glm}
  class than for the \texttt{lm} class. All that \texttt{anova} does is
  call \texttt{anova.glm}.
\item
  In GLMs there is no canonical test (like the F test for \texttt{lm}).
  We thus specify the type of test desired with the \texttt{test}
  argument.
\item
  The distribution of the weights of the plants does vary with the
  treatment given, as we may see from the significance of the
  \texttt{group} factor.
\item
  Readers familiar with ANOVA tables, should know that we computed the
  GLM equivalent of a type I sum- of-squares. Run
  \texttt{drop1(glm.1,\ test=\textquotesingle{}Chisq\textquotesingle{})}
  for a GLM equivalent of a type III sum-of-squares.
\item
  For help see \texttt{?anova.glm}.
\end{itemize}

Let's predict the probability of a heavy plant for each treatment.

\begin{Shaded}
\begin{Highlighting}[]
\KeywordTok{predict}\NormalTok{(glm}\FloatTok{.1}\NormalTok{, }\DataTypeTok{type=}\StringTok{'response'}\NormalTok{)}
\end{Highlighting}
\end{Shaded}

\begin{verbatim}
##   1   2   3   4   5   6   7   8   9  10  11  12  13  14  15  16  17  18 
## 0.6 0.6 0.6 0.6 0.6 0.6 0.6 0.6 0.6 0.6 0.2 0.2 0.2 0.2 0.2 0.2 0.2 0.2 
##  19  20  21  22  23  24  25  26  27  28  29  30 
## 0.2 0.2 0.9 0.9 0.9 0.9 0.9 0.9 0.9 0.9 0.9 0.9
\end{verbatim}

Things to note:

\begin{itemize}
\tightlist
\item
  Like the \texttt{summary} and \texttt{anova} functions, the
  \texttt{predict} function is aware that its input is of \texttt{glm}
  class. All that \texttt{predict} does is call \texttt{predict.glm}.
\item
  In GLMs there are many types of predictions. The \texttt{type}
  argument controls which type is returned.
\item
  How do I know we are predicting the probability of a heavy plant, and
  not a light plant? Just run \texttt{contrasts(weight.factor)} to see
  which of the categories of the factor \texttt{weight.factor} is
  encoded as 1, and which as 0.
\item
  For help see \texttt{?predict.glm}.
\end{itemize}

Let's detach the data so it is no longer in our workspace, and object
names do not collide.

\begin{Shaded}
\begin{Highlighting}[]
\KeywordTok{detach}\NormalTok{(PlantGrowth)}
\end{Highlighting}
\end{Shaded}

We gave an example with a factorial (i.e.~discrete) predictor. We can do
the same with multiple continuous predictors.

\begin{Shaded}
\begin{Highlighting}[]
\KeywordTok{data}\NormalTok{(}\StringTok{'Pima.te'}\NormalTok{, }\DataTypeTok{package=}\StringTok{'MASS'}\NormalTok{) }\CommentTok{# Loads data}
\KeywordTok{head}\NormalTok{(Pima.te)}
\end{Highlighting}
\end{Shaded}

\begin{verbatim}
##   npreg glu bp skin  bmi   ped age type
## 1     6 148 72   35 33.6 0.627  50  Yes
## 2     1  85 66   29 26.6 0.351  31   No
## 3     1  89 66   23 28.1 0.167  21   No
## 4     3  78 50   32 31.0 0.248  26  Yes
## 5     2 197 70   45 30.5 0.158  53  Yes
## 6     5 166 72   19 25.8 0.587  51  Yes
\end{verbatim}

\begin{Shaded}
\begin{Highlighting}[]
\NormalTok{glm}\FloatTok{.2}\NormalTok{<-}\StringTok{ }\KeywordTok{step}\NormalTok{(}\KeywordTok{glm}\NormalTok{(type~., }\DataTypeTok{data=}\NormalTok{Pima.te, }\DataTypeTok{family=}\NormalTok{binomial))}
\end{Highlighting}
\end{Shaded}

\begin{verbatim}
## Start:  AIC=301.79
## type ~ npreg + glu + bp + skin + bmi + ped + age
## 
##         Df Deviance    AIC
## - skin   1   286.22 300.22
## - bp     1   286.26 300.26
## - age    1   286.76 300.76
## <none>       285.79 301.79
## - npreg  1   291.60 305.60
## - ped    1   292.15 306.15
## - bmi    1   293.83 307.83
## - glu    1   343.68 357.68
## 
## Step:  AIC=300.22
## type ~ npreg + glu + bp + bmi + ped + age
## 
##         Df Deviance    AIC
## - bp     1   286.73 298.73
## - age    1   287.23 299.23
## <none>       286.22 300.22
## - npreg  1   292.35 304.35
## - ped    1   292.70 304.70
## - bmi    1   302.55 314.55
## - glu    1   344.60 356.60
## 
## Step:  AIC=298.73
## type ~ npreg + glu + bmi + ped + age
## 
##         Df Deviance    AIC
## - age    1   287.44 297.44
## <none>       286.73 298.73
## - npreg  1   293.00 303.00
## - ped    1   293.35 303.35
## - bmi    1   303.27 313.27
## - glu    1   344.67 354.67
## 
## Step:  AIC=297.44
## type ~ npreg + glu + bmi + ped
## 
##         Df Deviance    AIC
## <none>       287.44 297.44
## - ped    1   294.54 302.54
## - bmi    1   303.72 311.72
## - npreg  1   304.01 312.01
## - glu    1   349.80 357.80
\end{verbatim}

\begin{Shaded}
\begin{Highlighting}[]
\KeywordTok{summary}\NormalTok{(glm}\FloatTok{.2}\NormalTok{)}
\end{Highlighting}
\end{Shaded}

\begin{verbatim}
## 
## Call:
## glm(formula = type ~ npreg + glu + bmi + ped, family = binomial, 
##     data = Pima.te)
## 
## Deviance Residuals: 
##     Min       1Q   Median       3Q      Max  
## -2.9845  -0.6462  -0.3661   0.5977   2.5304  
## 
## Coefficients:
##              Estimate Std. Error z value Pr(>|z|)    
## (Intercept) -9.552177   1.096207  -8.714  < 2e-16 ***
## npreg        0.178066   0.045343   3.927  8.6e-05 ***
## glu          0.037971   0.005442   6.978  3.0e-12 ***
## bmi          0.084107   0.021950   3.832 0.000127 ***
## ped          1.165658   0.444054   2.625 0.008664 ** 
## ---
## Signif. codes:  0 '***' 0.001 '**' 0.01 '*' 0.05 '.' 0.1 ' ' 1
## 
## (Dispersion parameter for binomial family taken to be 1)
## 
##     Null deviance: 420.30  on 331  degrees of freedom
## Residual deviance: 287.44  on 327  degrees of freedom
## AIC: 297.44
## 
## Number of Fisher Scoring iterations: 5
\end{verbatim}

Things to note:

\begin{itemize}
\tightlist
\item
  We used the \texttt{\textasciitilde{}.} syntax to tell R to fit a
  model with all the available predictors.
\item
  Since we want to focus on significant predictors, we used the
  \texttt{step} function to perform a \emph{step-wise} regression,
  i.e.~sequentially remove non-significant predictors. The function
  reports each model it has checked, and the variable it has decided to
  remove at each step.
\item
  The output of \texttt{step} is a single model, with the subset of
  selected predictors.
\end{itemize}

\section{Poisson Regression}\label{poisson-regression}

Poisson regression means we fit a model assuming
\(y|x \sim Poisson(\lambda(x))\). Put differently, we assume that for
each treatment, encoded as a combinations of predictors \(x\), the
response is Poisson distributed with a rate that depends on the
predictors.

The typical link function for Poisson regression is \(g(t)=e^t\). This
means that we assume \(y|x \sim Poisson(\lambda(x) = e^{x'\beta})\). Why
is this a good choice? We again resort to the two-group case, encoded by
\(x=1\) and \(x=0\), to understand this model:
\(\lambda(x=1)=e^{\beta_0+\beta_1}=e^{\beta_0} \; e^{\beta_1}= \lambda(x=0) \; e^{\beta_1}\).
We thus see that this link function implies that a change in \(x\)
\textbf{multiples} the rate of events. For our example\footnote{Taken
  from
  \url{http://www.theanalysisfactor.com/generalized-linear-models-in-r-part-6-poisson-regression-count-variables/}}
we inspect the number of infected high-school kids, as a function of the
days since an outbreak.

\begin{Shaded}
\begin{Highlighting}[]
\NormalTok{cases <-}\StringTok{  }
\KeywordTok{structure}\NormalTok{(}\KeywordTok{list}\NormalTok{(}\DataTypeTok{Days =} \KeywordTok{c}\NormalTok{(1L, 2L, 3L, 3L, 4L, 4L, 4L, 6L, 7L, 8L, }
\NormalTok{8L, 8L, 8L, 12L, 14L, 15L, 17L, 17L, 17L, 18L, 19L, 19L, 20L, }
\NormalTok{23L, 23L, 23L, 24L, 24L, 25L, 26L, 27L, 28L, 29L, 34L, 36L, 36L, }
\NormalTok{42L, 42L, 43L, 43L, 44L, 44L, 44L, 44L, 45L, 46L, 48L, 48L, 49L, }
\NormalTok{49L, 53L, 53L, 53L, 54L, 55L, 56L, 56L, 58L, 60L, 63L, 65L, 67L, }
\NormalTok{67L, 68L, 71L, 71L, 72L, 72L, 72L, 73L, 74L, 74L, 74L, 75L, 75L, }
\NormalTok{80L, 81L, 81L, 81L, 81L, 88L, 88L, 90L, 93L, 93L, 94L, 95L, 95L, }
\NormalTok{95L, 96L, 96L, 97L, 98L, 100L, 101L, 102L, 103L, 104L, 105L, }
\NormalTok{106L, 107L, 108L, 109L, 110L, 111L, 112L, 113L, 114L, 115L), }
    \DataTypeTok{Students =} \KeywordTok{c}\NormalTok{(6L, 8L, 12L, 9L, 3L, 3L, 11L, 5L, 7L, 3L, 8L, }
    \NormalTok{4L, 6L, 8L, 3L, 6L, 3L, 2L, 2L, 6L, 3L, 7L, 7L, 2L, 2L, 8L, }
    \NormalTok{3L, 6L, 5L, 7L, 6L, 4L, 4L, 3L, 3L, 5L, 3L, 3L, 3L, 5L, 3L, }
    \NormalTok{5L, 6L, 3L, 3L, 3L, 3L, 2L, 3L, 1L, 3L, 3L, 5L, 4L, 4L, 3L, }
    \NormalTok{5L, 4L, 3L, 5L, 3L, 4L, 2L, 3L, 3L, 1L, 3L, 2L, 5L, 4L, 3L, }
    \NormalTok{0L, 3L, 3L, 4L, 0L, 3L, 3L, 4L, 0L, 2L, 2L, 1L, 1L, 2L, 0L, }
    \NormalTok{2L, 1L, 1L, 0L, 0L, 1L, 1L, 2L, 2L, 1L, 1L, 1L, 1L, 0L, 0L, }
    \NormalTok{0L, 1L, 1L, 0L, 0L, 0L, 0L, 0L)), }\DataTypeTok{.Names =} \KeywordTok{c}\NormalTok{(}\StringTok{"Days"}\NormalTok{, }\StringTok{"Students"}
\NormalTok{), }\DataTypeTok{class =} \StringTok{"data.frame"}\NormalTok{, }\DataTypeTok{row.names =} \KeywordTok{c}\NormalTok{(}\OtherTok{NA}\NormalTok{, -109L))}
\KeywordTok{attach}\NormalTok{(cases)}
\KeywordTok{head}\NormalTok{(cases) }
\end{Highlighting}
\end{Shaded}

\begin{verbatim}
##   Days Students
## 1    1        6
## 2    2        8
## 3    3       12
## 4    3        9
## 5    4        3
## 6    4        3
\end{verbatim}

And visually:

\begin{Shaded}
\begin{Highlighting}[]
\KeywordTok{plot}\NormalTok{(Days, Students, }\DataTypeTok{xlab =} \StringTok{"DAYS"}\NormalTok{, }\DataTypeTok{ylab =} \StringTok{"STUDENTS"}\NormalTok{, }\DataTypeTok{pch =} \DecValTok{16}\NormalTok{)}
\end{Highlighting}
\end{Shaded}

\includegraphics[width=0.5\linewidth]{Rcourse_files/figure-latex/unnamed-chunk-137-1}

We now fit a model to check for the change in the rate of events as a
function of the days since the outbreak.

\begin{Shaded}
\begin{Highlighting}[]
\NormalTok{glm}\FloatTok{.3} \NormalTok{<-}\StringTok{ }\KeywordTok{glm}\NormalTok{(Students ~}\StringTok{ }\NormalTok{Days, }\DataTypeTok{family =} \NormalTok{poisson)}
\KeywordTok{summary}\NormalTok{(glm}\FloatTok{.3}\NormalTok{)}
\end{Highlighting}
\end{Shaded}

\begin{verbatim}
## 
## Call:
## glm(formula = Students ~ Days, family = poisson)
## 
## Deviance Residuals: 
##      Min        1Q    Median        3Q       Max  
## -2.00482  -0.85719  -0.09331   0.63969   1.73696  
## 
## Coefficients:
##              Estimate Std. Error z value Pr(>|z|)    
## (Intercept)  1.990235   0.083935   23.71   <2e-16 ***
## Days        -0.017463   0.001727  -10.11   <2e-16 ***
## ---
## Signif. codes:  0 '***' 0.001 '**' 0.01 '*' 0.05 '.' 0.1 ' ' 1
## 
## (Dispersion parameter for poisson family taken to be 1)
## 
##     Null deviance: 215.36  on 108  degrees of freedom
## Residual deviance: 101.17  on 107  degrees of freedom
## AIC: 393.11
## 
## Number of Fisher Scoring iterations: 5
\end{verbatim}

Things to note:

\begin{itemize}
\tightlist
\item
  We used \texttt{family=poisson} in the \texttt{glm} function to tell R
  that we assume a Poisson distribution.
\item
  The coefficients table is there as usual. When interpreting the table,
  we need to recall that the effect, i.e.~the \(\hat \beta\), are
  \textbf{multiplicative} due to the assumed link function.
\item
  Each day \textbf{decreases} the rate of events by a factor of about
  0.02.
\item
  For more information see \texttt{?glm} and \texttt{?family}.
\end{itemize}

\section{Extensions}\label{extensions}

As we already implied, GLMs are a very wide class of models. We do not
need to use the default link function,but more importantly, we are not
constrained to Binomial, or Poisson distributed response. For
exponential, gamma, and other response distributions, see \texttt{?glm}
or the references in the Bibliographic Notes section.

\section{Bibliographic Notes}\label{bibliographic-notes-4}

The ultimate reference on GLMs is \citet{mccullagh1984generalized}. For
a less technical exposition, we refer to the usual
\citet{venables2013modern}.

\section{Practice Yourself}\label{practice-yourself-4}

\begin{enumerate}
\def\labelenumi{\arabic{enumi}.}
\tightlist
\item
  Try using \texttt{lm} for analyzing the plant growth data in
  \texttt{weight.factor} as a function of \texttt{group} in the
  \texttt{PlantGrowth} data.
\end{enumerate}

\chapter{Linear Mixed Models}\label{lme}

\BeginKnitrBlock{example}[Fixed and Random Machine Effect]
\protect\hypertarget{ex:fixed-effects}{}{\label{ex:fixed-effects}
\iffalse (Fixed and Random Machine Effect) \fi }Consider the problem of
testing for a change in the distribution of manufactured bottle caps.
Bottle caps are produced by several machines. We could standardize over
machines by removing each machine's average. This implies the
within-machine variability is the only source of variability we care
about. Alternatively, we could ignore the machine of origin. This second
practice implies there are two sources of variability we care about: the
within-machine variability, and the between-machine variability. The
former practice is known as a \emph{fixed effects} model. The latter as
a \emph{random effects} model.
\EndKnitrBlock{example}

\BeginKnitrBlock{example}[Fixed and Random Subject Effect]
\protect\hypertarget{ex:random-effects}{}{\label{ex:random-effects}
\iffalse (Fixed and Random Subject Effect) \fi }Consider a
crossover\footnote{If you are unfamiliar with design of experiments,
  have a look at Chapter 6 of my Quality Engineering
  \href{https://github.com/johnros/qualityEngineering/blob/master/Class_notes/notes.pdf}{class
  notes}.} experimenal design where each subject is given 2 types of
diets, and his health condition is recorded. We could standardize over
subjects by removing the subject-wise average, before comparing diets.
This is what a paired t-test does, and implies the within-subject
variability is the only source of variability we care about.
Alternatively, we could ignore the subject of origin. This second
practice implies there are two sources of variability we care about: the
within-subject variability and the betwee-subject variability.
\EndKnitrBlock{example}

The unifying theme of the above two examples, is that the variability we
want to infer against has several sources. Which are the sources of
variability that need to concern us? It depends on your purpose\ldots{}

Mixed models are so fundamental, that they have earned many names:

\begin{itemize}
\item
  \textbf{Mixed Effects}: Because we may have both \emph{fixed effects}
  we want to estimate and remove, and \emph{random effects} which
  contribute to the variability to infer against.
\item
  \textbf{Variance Components}: Because as the examples show, variance
  has more than a single source (like in the Linear Models of Chapter
  \ref{lm}).
\item
  \textbf{Hirarchial Models}: Because as Example \ref{ex:random-effects}
  demonstrates, we can think of the sampling as hierarchical-- first
  sample a subject, and then sample its response.
\item
  \textbf{Repeated Measures}: Because we many have several measurements
  from each unit, like in \ref{ex:random-effects}.
\item
  \textbf{Longitudinal Data}: Because we follow units over time, like in
  Example \ref{ex:random-effects}.
\item
  \textbf{Panel Data}: Is the term typically used in econometric for
  such longitudinal data.
\end{itemize}

We now emphasize:

\begin{enumerate}
\def\labelenumi{\arabic{enumi}.}
\tightlist
\item
  Mixed effect models are a way to infer against the right level of
  variability. Using a naive linear model (which assumes a single source
  of variability) instead of a mixed effects model, probably means your
  inference is overly anti-conservative, i.e., the estimation error is
  higher than you think.
\item
  A mixed effect models, as we will later see, is typically specified
  via its fixed and random effects. It is possible, however, to specify
  a mixed effects model by putting all the fixed effects into a linear
  model, and putting all the random effects into the covariance between
  \(\varepsilon\). This is known as \emph{multivariate regression}, or
  \emph{multivariate analysis of variance} (MANOVA). For more on this
  view, see Chapter 8 in (the excellent) \citet{weiss2005modeling}.
\item
  Like in previous chapters, by ``model'' we refer to the assumed
  generative distribution, i.e., the sampling distribution.
\item
  If you are using the model merely for predictions, and not for
  inference on the fixed effects or variance components, then stating
  the generative distribution may be be useful, but not necessarily. See
  the Supervised Learning Chapter \ref{supervised} for more on
  prediction problems.
\end{enumerate}

\section{Problem Setup}\label{problem-setup-2}

\begin{align}
  y|x,u = x'\beta + z'u + \varepsilon
  \label{eq:mixed-model}  
\end{align}

where \(x\) are the factors with fixed effects, \(\beta\), which we may
want to study. The factors \(z\), with effects \(u\), are the random
effects which contribute to variability. Put differently, we state
\(y|x,u\) merely as a convenient way to do inference on \(y|x\), instead
of directly specifying \(Var[y|x]\).

Given a sample of \(n\) observations \((y_i,x_i,z_i)\) from model
\eqref{eq:mixed-model}, we will want to estimate \((\beta,u)\). Under some
assumption on the distribution of \(\varepsilon\) and \(z\), we can use
\emph{maximum likelihood} (ML). In the context of mixed-models, however,
ML is typically replaced with \emph{restricted maximum likelihood}
(ReML), because it returns unbiased estimates of \(Var[y|x]\) and ML
does not.

\section{Mixed Models with R}\label{mixed-models-with-r}

We will fit mixed models with the \texttt{lmer} function from the
\textbf{lme4} package, written by the mixed-models Guru
\href{http://www.stat.wisc.edu/~bates/}{Douglas Bates}. We start with a
small simulation demonstrating the importance of acknowledging your
sources of variability, by fitting a linear model when a mixed model is
appropriate. We start by creating some synthetic data.

\begin{Shaded}
\begin{Highlighting}[]
\NormalTok{n.groups <-}\StringTok{ }\DecValTok{10}
\NormalTok{n.repeats <-}\StringTok{ }\DecValTok{2}
\NormalTok{groups <-}\StringTok{ }\KeywordTok{gl}\NormalTok{(}\DataTypeTok{n =} \NormalTok{n.groups, }\DataTypeTok{k =} \NormalTok{n.repeats)}
\NormalTok{n <-}\StringTok{ }\KeywordTok{length}\NormalTok{(groups)}
\NormalTok{z0 <-}\StringTok{ }\KeywordTok{rnorm}\NormalTok{(}\DecValTok{10}\NormalTok{,}\DecValTok{0}\NormalTok{,}\DecValTok{10}\NormalTok{)}
\NormalTok{z <-}\StringTok{ }\NormalTok{z0[}\KeywordTok{as.numeric}\NormalTok{(groups)] }\CommentTok{# create the random effect vector.}
\NormalTok{epsilon <-}\StringTok{ }\KeywordTok{rnorm}\NormalTok{(n,}\DecValTok{0}\NormalTok{,}\DecValTok{1}\NormalTok{) }\CommentTok{# create the measurement error vector.}
\NormalTok{beta0 <-}\StringTok{ }\DecValTok{2} \CommentTok{# create the global mean}
\NormalTok{y <-}\StringTok{ }\NormalTok{beta0 +}\StringTok{ }\NormalTok{z  +}\StringTok{ }\NormalTok{epsilon }\CommentTok{# generate synthetic sample}
\end{Highlighting}
\end{Shaded}

We can now fit the linear and mixed models.

\begin{Shaded}
\begin{Highlighting}[]
\NormalTok{lm}\FloatTok{.5} \NormalTok{<-}\StringTok{ }\KeywordTok{lm}\NormalTok{(y~z)  }\CommentTok{# fit a linear model}
\KeywordTok{library}\NormalTok{(lme4)}
\NormalTok{lme}\FloatTok{.5} \NormalTok{<-}\StringTok{ }\KeywordTok{lmer}\NormalTok{(y~}\DecValTok{1}\NormalTok{|z) }\CommentTok{# fit a mixed-model}
\end{Highlighting}
\end{Shaded}

The summary of the linear model

\begin{Shaded}
\begin{Highlighting}[]
\NormalTok{summary.lm}\FloatTok{.5} \NormalTok{<-}\StringTok{ }\KeywordTok{summary}\NormalTok{(lm}\FloatTok{.5}\NormalTok{)}
\NormalTok{summary.lm}\FloatTok{.5}
\end{Highlighting}
\end{Shaded}

\begin{verbatim}
## 
## Call:
## lm(formula = y ~ z)
## 
## Residuals:
##     Min      1Q  Median      3Q     Max 
## -2.3494 -0.4272  0.0312  0.6227  1.3482 
## 
## Coefficients:
##             Estimate Std. Error t value Pr(>|t|)    
## (Intercept)   2.0417     0.2029   10.06 8.12e-09 ***
## z             1.0309     0.0221   46.64  < 2e-16 ***
## ---
## Signif. codes:  0 '***' 0.001 '**' 0.01 '*' 0.05 '.' 0.1 ' ' 1
## 
## Residual standard error: 0.8902 on 18 degrees of freedom
## Multiple R-squared:  0.9918, Adjusted R-squared:  0.9913 
## F-statistic:  2175 on 1 and 18 DF,  p-value: < 2.2e-16
\end{verbatim}

The summary of the mixed-model

\begin{Shaded}
\begin{Highlighting}[]
\NormalTok{summary.lme}\FloatTok{.5} \NormalTok{<-}\StringTok{ }\KeywordTok{summary}\NormalTok{(lme}\FloatTok{.5}\NormalTok{)}
\NormalTok{summary.lme}\FloatTok{.5}
\end{Highlighting}
\end{Shaded}

\begin{verbatim}
## Linear mixed model fit by REML ['lmerMod']
## Formula: y ~ 1 | z
## 
## REML criterion at convergence: 104.1
## 
## Scaled residuals: 
##      Min       1Q   Median       3Q      Max 
## -1.50174 -0.52792 -0.03625  0.41093  1.48802 
## 
## Random effects:
##  Groups   Name        Variance Std.Dev.
##  z        (Intercept) 95.5094  9.7729  
##  Residual              0.9869  0.9934  
## Number of obs: 20, groups:  z, 10
## 
## Fixed effects:
##             Estimate Std. Error t value
## (Intercept)   0.2008     3.0984   0.065
\end{verbatim}

Look at the standard error of the global mean, i.e., the intercept: for
\texttt{lm} it is 0.202921, and for \texttt{lme} it is 3.0984323. Why
this difference? Because \texttt{lm} discounts the group
effect\footnote{A.k.a. the \emph{cluster effect} in the epidemiological
  literature.}, while it should treat it as another source of
variability. Clearly, inference using \texttt{lm} is overly optimistic.

\subsection{A Single Random Effect}\label{a-single-random-effect}

We will use the \texttt{Dyestuff} data from the \textbf{lme4} package,
which encodes the yield, in grams, of a coloring solution (dyestuff),
produced in 6 batches using 5 different preparations.

\begin{Shaded}
\begin{Highlighting}[]
\KeywordTok{data}\NormalTok{(Dyestuff, }\DataTypeTok{package=}\StringTok{'lme4'}\NormalTok{)}
\KeywordTok{attach}\NormalTok{(Dyestuff)}
\KeywordTok{head}\NormalTok{(Dyestuff)}
\end{Highlighting}
\end{Shaded}

\begin{verbatim}
##   Batch Yield
## 1     A  1545
## 2     A  1440
## 3     A  1440
## 4     A  1520
## 5     A  1580
## 6     B  1540
\end{verbatim}

And visually

\begin{Shaded}
\begin{Highlighting}[]
\KeywordTok{plot}\NormalTok{(Yield~Batch)}
\end{Highlighting}
\end{Shaded}

\includegraphics[width=0.5\linewidth]{Rcourse_files/figure-latex/unnamed-chunk-142-1}

If we want to do inference on the mean yield, we need to account for the
two sources of variability: the batch effect, and the measurement error.
We thus fit a mixed model, with an intercept and random batch effect,
which means this is it not a bona-fide mixed-model, but rather, a simple
random-effect model.

\begin{Shaded}
\begin{Highlighting}[]
\NormalTok{lme}\FloatTok{.1}\NormalTok{<-}\StringTok{ }\KeywordTok{lmer}\NormalTok{( Yield ~}\StringTok{ }\DecValTok{1}  \NormalTok{|}\StringTok{ }\NormalTok{Batch  , Dyestuff )}
\KeywordTok{summary}\NormalTok{(lme}\FloatTok{.1}\NormalTok{)}
\end{Highlighting}
\end{Shaded}

\begin{verbatim}
## Linear mixed model fit by REML ['lmerMod']
## Formula: Yield ~ 1 | Batch
##    Data: Dyestuff
## 
## REML criterion at convergence: 319.7
## 
## Scaled residuals: 
##     Min      1Q  Median      3Q     Max 
## -1.4117 -0.7634  0.1418  0.7792  1.8296 
## 
## Random effects:
##  Groups   Name        Variance Std.Dev.
##  Batch    (Intercept) 1764     42.00   
##  Residual             2451     49.51   
## Number of obs: 30, groups:  Batch, 6
## 
## Fixed effects:
##             Estimate Std. Error t value
## (Intercept)  1527.50      19.38    78.8
\end{verbatim}

Things to note:

\begin{itemize}
\tightlist
\item
  As usual, \texttt{summary} is content aware and has a different
  behavior for \texttt{lme} class objects.
\item
  The syntax \texttt{Yield\ \textasciitilde{}\ 1\ \ \textbar{}\ Batch}
  tells R to fit a model with a global intercept (\texttt{1}) and a
  random Batch effect (\texttt{\textbar{}Batch}). More on that later.
\item
  The output distinguishes between random effects, a source of
  variability, and fixed effect, which's coefficients we want to study.
\item
  Were we not interested in the variance components, and only in the
  coefficients or predictions, an (almost) equivalent \texttt{lm}
  formulation is \texttt{lm(Yield\ \textasciitilde{}\ Batch)}.
\end{itemize}

Some utility functions let us query the \texttt{lme} object. The
function \texttt{coef} will work, but will return a cumbersome output.
Better use \texttt{fixef} to extract the fixed effects, and
\texttt{ranef} to extract the random effects. The model matrix (of the
fixed effects alone), can be extracted with \texttt{model.matrix}, and
predictions made with \texttt{predict}. Note, however, that predictions
with mixed-effect models are (i) a delicate matter, and (ii) better
treated as prediction problems as in the Supervised Learning Chapter
\ref{supervised}.

\subsection{Multiple Random Effects}\label{multiple-random-effects}

Let's make things more interesting by allowing more than one random
effect. One-way ANOVA can be thought of as the fixed-effects counterpart
of the single random effect. Our next example, involving two random
effects, can be though of as the Two-way ANOVA counterpart.

In the \texttt{Penicillin} data, we measured the diameter of spread of
an organism, along the plate used (a to x), and penicillin type (A to
F).

\begin{Shaded}
\begin{Highlighting}[]
\KeywordTok{detach}\NormalTok{(Dyestuff)}
\KeywordTok{head}\NormalTok{(Penicillin)}
\end{Highlighting}
\end{Shaded}

\begin{verbatim}
##   diameter plate sample
## 1       27     a      A
## 2       23     a      B
## 3       26     a      C
## 4       23     a      D
## 5       23     a      E
## 6       21     a      F
\end{verbatim}

One sample per combination:

\begin{Shaded}
\begin{Highlighting}[]
\KeywordTok{attach}\NormalTok{(Penicillin)}
\KeywordTok{table}\NormalTok{(sample, plate)}
\end{Highlighting}
\end{Shaded}

\begin{verbatim}
##       plate
## sample a b c d e f g h i j k l m n o p q r s t u v w x
##      A 1 1 1 1 1 1 1 1 1 1 1 1 1 1 1 1 1 1 1 1 1 1 1 1
##      B 1 1 1 1 1 1 1 1 1 1 1 1 1 1 1 1 1 1 1 1 1 1 1 1
##      C 1 1 1 1 1 1 1 1 1 1 1 1 1 1 1 1 1 1 1 1 1 1 1 1
##      D 1 1 1 1 1 1 1 1 1 1 1 1 1 1 1 1 1 1 1 1 1 1 1 1
##      E 1 1 1 1 1 1 1 1 1 1 1 1 1 1 1 1 1 1 1 1 1 1 1 1
##      F 1 1 1 1 1 1 1 1 1 1 1 1 1 1 1 1 1 1 1 1 1 1 1 1
\end{verbatim}

And visually:

\begin{Shaded}
\begin{Highlighting}[]
\NormalTok{lattice::}\KeywordTok{dotplot}\NormalTok{(}\KeywordTok{reorder}\NormalTok{(plate, diameter) ~}\StringTok{ }\NormalTok{diameter,}\DataTypeTok{data=}\NormalTok{Penicillin,}
              \DataTypeTok{groups =} \NormalTok{sample,}
              \DataTypeTok{ylab =} \StringTok{"Plate"}\NormalTok{, }\DataTypeTok{xlab =} \StringTok{"Diameter of growth inhibition zone (mm)"}\NormalTok{,}
              \DataTypeTok{type =} \KeywordTok{c}\NormalTok{(}\StringTok{"p"}\NormalTok{, }\StringTok{"a"}\NormalTok{), }\DataTypeTok{auto.key =} \KeywordTok{list}\NormalTok{(}\DataTypeTok{columns =} \DecValTok{6}\NormalTok{, }\DataTypeTok{lines =} \OtherTok{TRUE}\NormalTok{))}
\end{Highlighting}
\end{Shaded}

\includegraphics[width=0.5\linewidth]{Rcourse_files/figure-latex/unnamed-chunk-145-1}

Let's fit a mixed-effects model with a random plate effect, and a random
sample effect:

\begin{Shaded}
\begin{Highlighting}[]
\NormalTok{lme}\FloatTok{.2} \NormalTok{<-}\StringTok{ }\KeywordTok{lmer} \NormalTok{( diameter ~}\StringTok{  }\DecValTok{1}\NormalTok{+}\StringTok{ }\NormalTok{(}\DecValTok{1}\NormalTok{|}\StringTok{ }\NormalTok{plate ) +}\StringTok{ }\NormalTok{(}\DecValTok{1}\NormalTok{|}\StringTok{ }\NormalTok{sample ) , Penicillin )}
\KeywordTok{fixef}\NormalTok{(lme}\FloatTok{.2}\NormalTok{) }\CommentTok{# Fixed effects}
\end{Highlighting}
\end{Shaded}

\begin{verbatim}
## (Intercept) 
##    22.97222
\end{verbatim}

\begin{Shaded}
\begin{Highlighting}[]
\KeywordTok{ranef}\NormalTok{(lme}\FloatTok{.2}\NormalTok{) }\CommentTok{# Random effects}
\end{Highlighting}
\end{Shaded}

\begin{verbatim}
## $plate
##   (Intercept)
## a  0.80454704
## b  0.80454704
## c  0.18167191
## d  0.33739069
## e  0.02595313
## f -0.44120322
## g -1.37551591
## h  0.80454704
## i -0.75264078
## j -0.75264078
## k  0.96026582
## l  0.49310948
## m  1.42742217
## n  0.49310948
## o  0.96026582
## p  0.02595313
## q -0.28548443
## r -0.28548443
## s -1.37551591
## t  0.96026582
## u -0.90835956
## v -0.28548443
## w -0.59692200
## x -1.21979713
## 
## $sample
##   (Intercept)
## A  2.18705797
## B -1.01047615
## C  1.93789946
## D -0.09689497
## E -0.01384214
## F -3.00374417
\end{verbatim}

Things to note:

\begin{itemize}
\tightlist
\item
  The syntax
  \texttt{1+\ (1\textbar{}\ plate\ )\ +\ (1\textbar{}\ sample\ )} fits a
  global intercept (mean), a random plate effect, and a random sample
  effect.
\item
  Were we not interested in the variance components, an (almost)
  equivalent \texttt{lm} formulation is
  \texttt{lm(diameter\ \textasciitilde{}\ plate\ +\ sample)}.
\end{itemize}

Since we have two random effects, we may compute the variability of the
global mean (the only fixed effect) as we did before. Perhaps more
interestingly, we can compute the variability in the response, for a
particular plate or sample type.

\begin{Shaded}
\begin{Highlighting}[]
\NormalTok{random.effect.lme2 <-}\StringTok{ }\KeywordTok{ranef}\NormalTok{(lme}\FloatTok{.2}\NormalTok{, }\DataTypeTok{condVar =} \OtherTok{TRUE}\NormalTok{) }
\NormalTok{qrr2 <-}\StringTok{ }\NormalTok{lattice::}\KeywordTok{dotplot}\NormalTok{(random.effect.lme2, }\DataTypeTok{strip =} \OtherTok{FALSE}\NormalTok{)}
\end{Highlighting}
\end{Shaded}

Things to note:

\begin{itemize}
\tightlist
\item
  The \texttt{condVar} argument of the \texttt{ranef} function tells R
  to compute the variability in response conditional on each random
  effect at a time.
\item
  The \texttt{dotplot} function, from the \textbf{lattice} package, is
  only there for the fancy plotting.
\end{itemize}

Variability in response for each plate, over various sample types:

\begin{Shaded}
\begin{Highlighting}[]
\KeywordTok{print}\NormalTok{(qrr2[[}\DecValTok{1}\NormalTok{]]) }
\end{Highlighting}
\end{Shaded}

\includegraphics[width=0.5\linewidth]{Rcourse_files/figure-latex/unnamed-chunk-148-1}

Variability in response for each sample type, over the various plates:

\begin{Shaded}
\begin{Highlighting}[]
\KeywordTok{print}\NormalTok{(qrr2[[}\DecValTok{2}\NormalTok{]])  }
\end{Highlighting}
\end{Shaded}

\includegraphics[width=0.5\linewidth]{Rcourse_files/figure-latex/unnamed-chunk-149-1}

\subsection{A Full Mixed-Model}\label{a-full-mixed-model}

In the \texttt{sleepstudy} data, we recorded the reaction times to a
series of tests (\texttt{Reaction}), after various subject
(\texttt{Subject}) underwent various amounts of sleep deprivation
(\texttt{Day}).

\begin{Shaded}
\begin{Highlighting}[]
\KeywordTok{data}\NormalTok{(sleepstudy)}
\NormalTok{lattice::}\KeywordTok{xyplot}\NormalTok{(Reaction ~}\StringTok{ }\NormalTok{Days |}\StringTok{ }\NormalTok{Subject, sleepstudy, }\DataTypeTok{aspect =} \StringTok{"xy"}\NormalTok{,}
             \DataTypeTok{layout =} \KeywordTok{c}\NormalTok{(}\DecValTok{9}\NormalTok{,}\DecValTok{2}\NormalTok{), }\DataTypeTok{type =} \KeywordTok{c}\NormalTok{(}\StringTok{"g"}\NormalTok{, }\StringTok{"p"}\NormalTok{, }\StringTok{"r"}\NormalTok{),}
             \DataTypeTok{index.cond =} \NormalTok{function(x,y) }\KeywordTok{coef}\NormalTok{(}\KeywordTok{lm}\NormalTok{(y ~}\StringTok{ }\NormalTok{x))[}\DecValTok{1}\NormalTok{],}
             \DataTypeTok{xlab =} \StringTok{"Days of sleep deprivation"}\NormalTok{,}
             \DataTypeTok{ylab =} \StringTok{"Average reaction time (ms)"}\NormalTok{)}
\end{Highlighting}
\end{Shaded}

\includegraphics[width=0.5\linewidth]{Rcourse_files/figure-latex/unnamed-chunk-150-1}

We now want to estimate the (fixed) effect of the days of deprivation,
while allowing each subject to have his/hers own effect. Put
differently, we want to estimate a \emph{random slope} for variable
\texttt{day} for each subject. The fixed \texttt{Days} effect can be
thought of as the average slope over subjects.

\begin{Shaded}
\begin{Highlighting}[]
\NormalTok{lme}\FloatTok{.3} \NormalTok{<-}\StringTok{ }\KeywordTok{lmer} \NormalTok{( Reaction ~}\StringTok{ }\NormalTok{Days +}\StringTok{ }\NormalTok{( Days |}\StringTok{ }\NormalTok{Subject ) , }\DataTypeTok{data=} \NormalTok{sleepstudy )}
\end{Highlighting}
\end{Shaded}

Things to note:

\begin{itemize}
\tightlist
\item
  We used the \texttt{Days\textbar{}Subect} syntax to tell R we want to
  fit the model \texttt{\textasciitilde{}Days} within each subject.
\item
  Were we fitting the model for purposes of prediction only, an (almost)
  equivalent \texttt{lm} formulation is
  \texttt{lm(Reaction\textasciitilde{}Days*Subject)}.
\end{itemize}

The fixed (i.e.~average) day effect is:

\begin{Shaded}
\begin{Highlighting}[]
\KeywordTok{fixef}\NormalTok{(lme}\FloatTok{.3}\NormalTok{)}
\end{Highlighting}
\end{Shaded}

\begin{verbatim}
## (Intercept)        Days 
##   251.40510    10.46729
\end{verbatim}

The variability in the average response (intercept) and day effect is

\begin{Shaded}
\begin{Highlighting}[]
\KeywordTok{ranef}\NormalTok{(lme}\FloatTok{.3}\NormalTok{)}
\end{Highlighting}
\end{Shaded}

\begin{verbatim}
## $Subject
##     (Intercept)        Days
## 308   2.2585654   9.1989719
## 309 -40.3985770  -8.6197032
## 310 -38.9602459  -5.4488799
## 330  23.6904985  -4.8143313
## 331  22.2602027  -3.0698946
## 332   9.0395259  -0.2721707
## 333  16.8404312  -0.2236244
## 334  -7.2325792   1.0745761
## 335  -0.3336959 -10.7521591
## 337  34.8903509   8.6282839
## 349 -25.2101104   1.1734143
## 350 -13.0699567   6.6142050
## 351   4.5778352  -3.0152572
## 352  20.8635925   3.5360133
## 369   3.2754530   0.8722166
## 370 -25.6128694   4.8224646
## 371   0.8070397  -0.9881551
## 372  12.3145394   1.2840297
\end{verbatim}

Did we really need the whole \texttt{lme} machinery to fit a
within-subject linear regression and then average over subjects? The
answer is yes. The assumptions on the distribution of random effect,
namely, that they are normally distributed, allows us to pool
information from one subject to another. In the words of John Tukey:
``we borrow strength over subjects''. Is this a good thing? If the
normality assumption is true, it certainly is. If, on the other hand,
you have a lot of samples per subject, and you don't need to ``borrow
strength'' from one subject to another, you can simply fit
within-subject linear models without the mixed-models machinery.

To demonstrate the ``strength borrowing'', here is a comparison of the
subject-wise intercepts of the mixed-model, versus a subject-wise linear
model. They are not the same.

\includegraphics[width=0.5\linewidth]{Rcourse_files/figure-latex/unnamed-chunk-153-1}

Here is a comparison of the random-day effect from \texttt{lme} versus a
subject-wise linear model. They are not the same.

\includegraphics[width=0.5\linewidth]{Rcourse_files/figure-latex/unnamed-chunk-154-1}

\begin{Shaded}
\begin{Highlighting}[]
\KeywordTok{detach}\NormalTok{(Penicillin)}
\end{Highlighting}
\end{Shaded}

\section{The Variance-Components
View}\label{the-variance-components-view}

\section{Bibliographic Notes}\label{bibliographic-notes-5}

Most of the examples in this chapter are from the documentation of the
\textbf{lme4} package \citep{lme4}. For a more theoretical view see
\citet{weiss2005modeling} or \citet{searle2009variance}. As usual, a
hands on view can be found in \citet{venables2013modern}.

\section{Practice Yourself}\label{practice-yourself-5}

\chapter{Multivariate Data Analysis}\label{multivariate}

The term ``multivariate data analysis'' is so broad and so overloaded,
that we start by clarifying what is discussed and what is not discussed
in this chapter. Broadly speaking, we will discuss statistical
\emph{inference}, and leave more ``exploratory flavored'' matters like
clustering, and visualization, to the Unsupervised Learning Chapter
\ref{unsupervised}.

More formally, let \(y\) be a \(p\) variate random vector, with
\(E[y]=\mu\). Here is the set of problems we will discuss, in order of
their statistical difficulty.

\begin{itemize}
\tightlist
\item
  \textbf{Signal detection}: a.k.a. \emph{multivariate hypothesis
  testing}, i.e., testing if \(\mu\) equals \(\mu_0\) and for
  \(\mu_0=0\) in particular.
\item
  \textbf{Signal counting}: Counting the number of elements in \(\mu\)
  that differ from \(\mu_0\), and for \(\mu_0=0\) in particular.
\item
  \textbf{Signal identification}: a.k.a. \emph{multiple testing}, i.e.,
  testing which of the elements in \(\mu\) differ from \(\mu_0\) and for
  \(\mu_0=0\) in particular.
\item
  \textbf{Signal estimation}: a.k.a. \emph{selective inference}, i.e.,
  estimating the magnitudes of the departure of \(\mu\) from \(\mu_0\),
  and for \(\mu_0=0\) in particular.
\item
  \textbf{Multivariate Regression}: a.k.a. \emph{MANOVA} in statistical
  literature, and \emph{structured learning} in the machine learning
  literature.
\item
  \textbf{Graphical Models}: Learning \emph{graphical models} deals with
  the fitting/learning the multivariate distribution of \(y\). In
  particular, it deals with the identification of independencies between
  elements of \(y\).
\end{itemize}

\BeginKnitrBlock{example}
\protect\hypertarget{ex:icu}{}{\label{ex:icu}}Consider the problem of a
patient monitored in the intensive care unit. At every minute the
monitor takes \(p\) physiological measurements: blood pressure, body
temperature, etc. The total number of minutes in our data is \(n\), so
that in total, we have \(n \times p\) measurements, arranged in a
matrix. We also know the typical measurements for this patient when
healthy: \(\mu_0\).

Signal detection means testing if the patient's measurement depart in
any way from his healthy state, \(\mu_0\). Signal counting means
measuring \emph{how many} measurement depart from the healthy state.
Signal identification means pin-pointing which of the physiological
measurements depart from his healthy state. Signal estimation means
estimating the magnitude of the departure from the healthy state.
Multivaraite regression means finding the factors which many explain the
departure from the healthy state. Fitting a distribution means finding
the joint distribution of the physiological measurements, and in
particular, their dependencies and independenceis.
\EndKnitrBlock{example}

\BeginKnitrBlock{remark}
\iffalse {Remark. } \fi In the above, ``signal'' is defined in terms of
\(\mu\). It is possible that the signal is not in the location, \(\mu\),
but rather in the covariance, \(\Sigma\). We do not discuss these
problems here, and refer the reader to \citet{nadler2008finite}.
\EndKnitrBlock{remark}

\section{Signal Detection}\label{signal-detection}

Signal detection deals with the detection of the departure of \(\mu\)
from some \(\mu_0\), and especially, \(\mu_0=0\). This problem can be
thought of as the multivariate counterpart of the univariate hypothesis
test. Indeed, the most fundamental approach is a mere generalization of
the t-test, known as \emph{Hotelling's \(T^2\) test}.

Recall the univariate t-statistic of a data vector \(x\) of length
\(n\):

\begin{align}
  t^2(x):= \frac{(\bar{x}-\mu_0)^2}{Var[\bar{x}]}= (\bar{x}-\mu_0)Var[\bar{x}]^{-1}(\bar{x}-\mu_0),
  \label{eq:t-test}
\end{align}

where \(Var[\bar{x}]=S^2(x)/n\), and \(S^2(x)\) is the unbiased variance
estimator \(S^2(x):=(n-1)^{-1}\sum (x_i-\bar x)^2\).

Generalizing Eq\eqref{eq:t-test} to the multivariate case: \(\mu_0\) is a
\(p\)-vector, \(\bar x\) is a \(p\)-vector, and \(Var[\bar x]\) is a
\(p \times p\) matrix of the covariance between the \(p\) coordinated of
\(\bar x\). When operating with vectors, the squaring becomes a
quadratic form, and the division becomes a matrix inverse. We thus have

\begin{align}
  T^2(x):= (\bar{x}-\mu_0)' Var[\bar{x}]^{-1} (\bar{x}-\mu_0),
  \label{eq:hotelling-test}
\end{align}

which is the definition of Hotelling's \(T^2\) test statistic. We
typically denote the covariance between coordinates in \(x\) with
\(\hat \Sigma(x)\), so that
\(\widehat \Sigma_{k,l}:=\widehat {Cov}[x_k,x_l]=(n-1)^{-1} \sum (x_{k,i}-\bar x_k)(x_{l,i}-\bar x_l)\).
Using the \(\Sigma\) notation, Eq.\eqref{eq:hotelling-test} becomes

\begin{align}
  T^2(x):= n (\bar{x}-\mu_0)' \hat \Sigma(x)^{-1} (\bar{x}-\mu_0),
\end{align}

which is the standard notation of Hotelling's test statistic.

To discuss the distribution of Hotelling's test statistic we need to
introduce some vocabulary\footnote{This vocabulary is not standard in
  the literature, so when you read a text, you need to verify yourself
  what the author means.}:

\begin{enumerate}
\def\labelenumi{\arabic{enumi}.}
\tightlist
\item
  \textbf{Low Dimension}: We call a problem \emph{low dimensional} if
  \(n \gg p\), i.e. \(p/n \approx 0\). This means there are many
  observations per estimated parameter.
\item
  \textbf{High Dimension}: We call a problem \emph{high dimensional} if
  \(p/n \to c\), where \(c\in (0,1)\). This means there are more
  observations than parameters, but not many.
\item
  \textbf{Very High Dimension}: We call a problem \emph{very high
  dimensional} if \(p/n \to c\), where \(1<c<\infty\). This means there
  are less observations than parameter.
\item
  \textbf{Extremely high dimensional}: We call a problem
  \textbf{extremely high dimensional} if \(p/n \to \infty\). This means
  there are many more parameters than observations.
\end{enumerate}

Hotelling's \(T^2\) test can only be used in the low dimensional regime.
For some intuition on this statement, think of taking \(n=20\)
measurements of \(p=100\) physiological variables. We seemingly have
\(20\) observations, but there are \(100\) unknown quantities in
\(\mu\). Would you trust your conclusion that \(\bar x\) is different
than \(\mu_0\) based on merely \(20\) observations.

Put formally: We cannot compute Hotelling's test when \(n<p\) because
\(\hat \Sigma\) is simply not invertible-- this is an algebraic problem.
We cannot compute Hotelling's test when \(p/n \to c > 0\) because the
signal-to-noise is very low-- this is a statistical problem.

Only in the low dimensional case can we compute and trust Hotelling's
test. When \(n \gg p\) then \(T^2(x)\) is roughly \(\chi^2\) distributed
with \(p\) degrees of freedom. The F distribution may also be found in
the literature in this context, and will appear if assuming the \(n\)
\(p\)-vectors are independent, and \(p\)-variate Gaussian. This F
distribution is non-robust the underlying assumptions, so from a
practical point of view, I would not trust the Hotelling test unless
\(n \gg p\).

\subsection{Signal Detection with R}\label{signal-detection-with-r}

Let's generate some data with no signal.

\begin{Shaded}
\begin{Highlighting}[]
\KeywordTok{library}\NormalTok{(mvtnorm)}
\NormalTok{n <-}\StringTok{ }\FloatTok{1e2}
\NormalTok{p <-}\StringTok{ }\FloatTok{1e1}
\NormalTok{mu <-}\StringTok{ }\KeywordTok{rep}\NormalTok{(}\DecValTok{0}\NormalTok{,p) }\CommentTok{# no signal}
\NormalTok{x <-}\StringTok{ }\KeywordTok{rmvnorm}\NormalTok{(}\DataTypeTok{n =} \NormalTok{n, }\DataTypeTok{mean =} \NormalTok{mu)}
\KeywordTok{dim}\NormalTok{(x)}
\end{Highlighting}
\end{Shaded}

\begin{verbatim}
## [1] 100  10
\end{verbatim}

\begin{Shaded}
\begin{Highlighting}[]
\KeywordTok{image}\NormalTok{(x)}
\end{Highlighting}
\end{Shaded}

\includegraphics[width=0.5\linewidth]{Rcourse_files/figure-latex/unnamed-chunk-157-1}

Now make our own Hotelling function.

\begin{Shaded}
\begin{Highlighting}[]
\NormalTok{hotellingOneSample <-}\StringTok{ }\NormalTok{function(x, }\DataTypeTok{mu0=}\KeywordTok{rep}\NormalTok{(}\DecValTok{0}\NormalTok{,}\KeywordTok{ncol}\NormalTok{(x)))\{}
  \NormalTok{n <-}\StringTok{ }\KeywordTok{nrow}\NormalTok{(x)}
  \NormalTok{p <-}\StringTok{ }\KeywordTok{ncol}\NormalTok{(x)}
  \KeywordTok{stopifnot}\NormalTok{(n >}\StringTok{ }\DecValTok{5}\NormalTok{*}\StringTok{ }\NormalTok{p)}
  \NormalTok{bar.x <-}\StringTok{ }\KeywordTok{colMeans}\NormalTok{(x)}
  \NormalTok{Sigma <-}\StringTok{ }\KeywordTok{var}\NormalTok{(x)}
  \NormalTok{Sigma.inv <-}\StringTok{ }\KeywordTok{solve}\NormalTok{(Sigma)}
  \NormalTok{T2 <-}\StringTok{ }\NormalTok{n *}\StringTok{ }\NormalTok{(bar.x-mu0) %*%}\StringTok{ }\NormalTok{Sigma.inv %*%}\StringTok{ }\NormalTok{(bar.x-mu0)}
  \NormalTok{p.value <-}\StringTok{ }\KeywordTok{pchisq}\NormalTok{(}\DataTypeTok{q =} \NormalTok{T2, }\DataTypeTok{df =} \NormalTok{p, }\DataTypeTok{lower.tail =} \OtherTok{FALSE}\NormalTok{)}
  \KeywordTok{return}\NormalTok{(}\KeywordTok{list}\NormalTok{(}\DataTypeTok{statistic=}\NormalTok{T2, }\DataTypeTok{pvalue=}\NormalTok{p.value))}
\NormalTok{\}}

\KeywordTok{hotellingOneSample}\NormalTok{(x)}
\end{Highlighting}
\end{Shaded}

\begin{verbatim}
## $statistic
##          [,1]
## [1,] 17.75369
## 
## $pvalue
##            [,1]
## [1,] 0.05926339
\end{verbatim}

Things to note:

\begin{itemize}
\tightlist
\item
  \texttt{stopifnot(n\ \textgreater{}\ 5\ *\ p)} is a little
  verification to check that the problem is indeed low dimensional.
  Otherwise, the \(\chi^2\) approximation cannot be trusted.\\
\item
  \texttt{solve} returns a matrix inverse.
\item
  \texttt{\%*\%} is the matrix product operator (see also
  \texttt{crossprod()}).
\item
  A function may return only a single object, so we wrap the statistic
  and its p-value in a \texttt{list} object.
\end{itemize}

Just for verification, we compare our home made Hotelling's test, to the
implementation in the \textbf{rrcov} package. The results look good!

\begin{Shaded}
\begin{Highlighting}[]
\NormalTok{rrcov::}\KeywordTok{T2.test}\NormalTok{(x)}
\end{Highlighting}
\end{Shaded}

\begin{verbatim}
## 
##  One-sample Hotelling test
## 
## data:  x
## T2 = 17.754, F = 1.614, df1 = 10, df2 = 90, p-value = 0.1152
## alternative hypothesis: true mean vector is not equal to (0, 0, 0, 0, 0, 0, 0, 0, 0, 0)' 
## 
## sample estimates:
##                     [,1]       [,2]      [,3]        [,4]        [,5]
## mean x-vector -0.0948489 0.07538331 0.1969828 -0.04134792 -0.03286212
##                     [,6]       [,7]      [,8]      [,9]     [,10]
## mean x-vector 0.02524591 0.07800582 -0.238338 0.2412012 0.1198553
\end{verbatim}

\section{Signal Counting}\label{signal-counting}

There are many ways to approach the \emph{signal counting} problem. For
the purposes of this book, however, we will not discuss them directly,
and solve the signal counting problem as a signal identification
problem: if we know \textbf{where} \(\mu\) departs from \(\mu_0\), we
only need to count coordinates to solve the signal counting problem.

\section{Signal Identification}\label{identification}

The problem of \emph{signal identification} is also known as
\emph{selective testing}, or more commonly as \emph{multiple testing}.

In the ANOVA literature, an identification stage will typically follow a
detection stage. These are known as the \emph{omnibus F test}, and
\emph{post-hoc} tests, respectively. In the multiple testing literature
there will typically be no preliminary detection stage. It is typically
assumed that signal is present, and the only question is ``where?''

The first question when approaching a multiple testing problem is ``what
is an error''? Is an error declaring a coordinate in \(\mu\) to be
different than \(\mu_0\) when it is actually not? Is an error the
proportion of such false declarations. The former is known as the
\emph{family wise error rate} (FWER), and the latter as the \emph{false
discovery rate} (FDR).

\subsection{Signal Identification in
R}\label{signal-identification-in-r}

One (of many) ways to do signal identification involves the
\texttt{stats::p.adjust} function. {[}TODO: clarify why use
\texttt{p.adjust}?{]} The function takes as inputs a \(p\)-vector of
\textbf{p-values}. This implies that: (i) you are assumed to be able to
compute the p-value of each the \(p\) hypothesis tested; one hypothesis
for every coordinate in \(\mu\). (ii) unlike the Hotelling test, we do
not try to estimate the covariance between coordinates. Not because it
is not important, but rather, because the methods we will use apply to a
wide variety of covariances, so the covariance does not need to be
estimated.

We start be generating some multivariate data and computing the
coordinate-wise (i.e.~hypothesis-wise) p-value.

\begin{Shaded}
\begin{Highlighting}[]
\KeywordTok{library}\NormalTok{(mvtnorm)}
\NormalTok{n <-}\StringTok{ }\FloatTok{1e1}
\NormalTok{p <-}\StringTok{ }\FloatTok{1e2}
\NormalTok{mu <-}\StringTok{ }\KeywordTok{rep}\NormalTok{(}\DecValTok{0}\NormalTok{,p)}
\NormalTok{x <-}\StringTok{ }\KeywordTok{rmvnorm}\NormalTok{(}\DataTypeTok{n =} \NormalTok{n, }\DataTypeTok{mean =} \NormalTok{mu)}
\KeywordTok{dim}\NormalTok{(x)}
\end{Highlighting}
\end{Shaded}

\begin{verbatim}
## [1]  10 100
\end{verbatim}

\begin{Shaded}
\begin{Highlighting}[]
\KeywordTok{image}\NormalTok{(x)}
\end{Highlighting}
\end{Shaded}

\includegraphics[width=0.5\linewidth]{Rcourse_files/figure-latex/unnamed-chunk-160-1}

We now compute the pvalues of each coordinate. We use a coordinate-wise
t-test. Why a t-test? Because for the purpose of demonstration we want a
simple test. In reality, you may use any test that returns valid
p-values.

\begin{Shaded}
\begin{Highlighting}[]
\NormalTok{p.values <-}\StringTok{ }\KeywordTok{apply}\NormalTok{(}\DataTypeTok{X =} \NormalTok{x, }\DataTypeTok{MARGIN =} \DecValTok{2}\NormalTok{, }\DataTypeTok{FUN =} \NormalTok{function(y) }\KeywordTok{t.test}\NormalTok{(y)$p.value) }
\KeywordTok{plot}\NormalTok{(p.values, }\DataTypeTok{type=}\StringTok{'h'}\NormalTok{)}
\end{Highlighting}
\end{Shaded}

\includegraphics[width=0.5\linewidth]{Rcourse_files/figure-latex/unnamed-chunk-161-1}

Things to note:

\begin{itemize}
\tightlist
\item
  We used the \texttt{apply} function to apply the same function to each
  column of \texttt{x}.
\item
  The output, \texttt{p.values}, is a vector of 100 p-values.
\end{itemize}

We are now ready to do the identification, i.e., find which coordinate
of \(\mu\) is different than \(\mu_0=0\). The workflow is: (i) Compute
an \texttt{adjusted\ p-value}. (ii) Compare the adjusted p-value to the
desired error level.

If we want \(FWER \leq 0.05\), meaning that we allow a \(5\%\)
probability of making any mistake, we will use the
\texttt{method="holm"} argument of \texttt{p.adjust}.

\begin{Shaded}
\begin{Highlighting}[]
\NormalTok{alpha <-}\StringTok{ }\FloatTok{0.05}
\NormalTok{p.values.holm <-}\StringTok{ }\KeywordTok{p.adjust}\NormalTok{(p.values, }\DataTypeTok{method =} \StringTok{'holm'} \NormalTok{)}
\KeywordTok{table}\NormalTok{(p.values.holm <}\StringTok{ }\NormalTok{alpha)}
\end{Highlighting}
\end{Shaded}

\begin{verbatim}
## 
## FALSE  TRUE 
##    99     1
\end{verbatim}

If we want \(FDR \leq 0.05\), meaning that we allow the proportion of
false discoveries to be no larger than \(5\%\), we use the
\texttt{method="BH"} argument of \texttt{p.adjust}.

\begin{Shaded}
\begin{Highlighting}[]
\NormalTok{alpha <-}\StringTok{ }\FloatTok{0.05}
\NormalTok{p.values.BH <-}\StringTok{ }\KeywordTok{p.adjust}\NormalTok{(p.values, }\DataTypeTok{method =} \StringTok{'BH'} \NormalTok{)}
\KeywordTok{table}\NormalTok{(p.values.BH <}\StringTok{ }\NormalTok{alpha)}
\end{Highlighting}
\end{Shaded}

\begin{verbatim}
## 
## FALSE  TRUE 
##    99     1
\end{verbatim}

We now inject some signal in \(\mu\) just to see that the process works.
We will artificially inject signal in the first 10 coordinates.

\begin{Shaded}
\begin{Highlighting}[]
\NormalTok{mu[}\DecValTok{1}\NormalTok{:}\DecValTok{10}\NormalTok{] <-}\StringTok{ }\DecValTok{2} \CommentTok{# inject signal}
\NormalTok{x <-}\StringTok{ }\KeywordTok{rmvnorm}\NormalTok{(}\DataTypeTok{n =} \NormalTok{n, }\DataTypeTok{mean =} \NormalTok{mu) }\CommentTok{# generate data}
\NormalTok{p.values <-}\StringTok{ }\KeywordTok{apply}\NormalTok{(}\DataTypeTok{X =} \NormalTok{x, }\DataTypeTok{MARGIN =} \DecValTok{2}\NormalTok{, }\DataTypeTok{FUN =} \NormalTok{function(y) }\KeywordTok{t.test}\NormalTok{(y)$p.value) }
\NormalTok{p.values.BH <-}\StringTok{ }\KeywordTok{p.adjust}\NormalTok{(p.values, }\DataTypeTok{method =} \StringTok{'BH'} \NormalTok{)}
\KeywordTok{which}\NormalTok{(p.values.BH <}\StringTok{ }\NormalTok{alpha)}
\end{Highlighting}
\end{Shaded}

\begin{verbatim}
##  [1]  1  2  3  4  5  6  7  8  9 10
\end{verbatim}

Indeed- we are now able to detect that the first coordinates carry
signal, because their respective coordinate-wise null hypotheses have
been rejected.

\section{Signal Estimation}\label{signal-estimation}

The estimation of the elements of \(\mu\) is a seemingly straightforward
task. This is not the case, however, if we estimate only the elements
that were selected because they were significant (or any other
data-dependent criterion). Clearly, estimating only significant entries
will introduce a bias in the estimation. In the statistical literature,
this is known as \emph{selection bias}. Selection bias also occurs when
you perform inference on regression coefficients after some model
selection, say, with a lasso, or a forward search\footnote{You might
  find this shocking, but it does mean that you cannot trust the
  \texttt{summary} table of a model that was selected from a multitude
  of models.}.

Selective inference is a complicated and active research topic so we
will not offer any off-the-shelf solution to the matter. The curious
reader is invited to read \citet{rosenblatt2014selective},
\citet{javanmard2014confidence}, or
\href{http://www.stat.berkeley.edu/~wfithian/}{Will Fithian's} PhD
thesis \citep{fithian2015topics} for more on the topic.

\section{Multivariate Regression}\label{multivariate-regression}

\emph{Multivaraite regression}, a.k.a. \emph{MANOVA}, similar to
\href{https://en.wikipedia.org/wiki/Structured_prediction}{\emph{structured
learning}} in machine learning, is simply a regression problem where the
outcome, \(y\), is not scalar values but vector valued. It is not to be
confused with \emph{multiple regression} where the predictor, \(x\), is
vector valued, but the outcome is scalar.

If the linear models generalize the two-sample t-test from two, to
multiple populations, then multivariate regression generalizes
Hotelling's test in the same way.

\subsection{Multivariate Regression with
R}\label{multivariate-regression-with-r}

TODO

\section{Graphical Models}\label{graphical-models}

Fitting a multivariate distribution, i.e.~learning a \emph{graphical
model}, is a very hard task. To see why, consider the problem of \(p\)
continuous variables. In the simplest case, where we can assume
normality, fitting a distributions means estimating the \(p\) parameters
in the expectation, \(\mu\), and \(p(p+1)/2\) parameters in the
covariance, \(\Sigma\). The number of observations required for this
task, \(n\), may be formidable.

A more humble task, is to identify \textbf{independencies}, known as
\emph{structure learning} in the machine learning literature. Under the
multivariate normality assumption, this means identifying zero entries
in \(\Sigma\), or more precisely, zero entries in \(\Sigma^{-1}\). This
task can be approached as a \textbf{signal identification} problem
(\ref{identification}). The same solutions may be applied even if
dealing with \(\Sigma\) instead of \(\mu\).

If multivariate normality cannot be assumed, then identifying
independencies cannot be done via the covariance matrix \(\Sigma\) and
more elaborate algorithms are required.

\subsection{Graphical Models in R}\label{graphical-models-in-r}

TODO

\section{Biblipgraphic Notes}\label{biblipgraphic-notes}

For a general introduction to multivariate data analysis see
\citet{anderson2004introduction}. For an R oriented introduction, see
\citet{everitt2011introduction}. For more on the difficulties with high
dimensional problems, see \citet{bai1996effect}. For more on multiple
testing, and signal identification, see \citet{efron2012large}. For more
on the choice of your error rate see \citet{rosenblatt2013practitioner}.
For an excellent reivew on graphical models see
\citet{kalisch2014causal}. Everything you need on graphical models,
Bayesian belief networks, and structure learning in R, is collected in
the \href{https://cran.r-project.org/web/views/gR.html}{Task View}.

\section{Practice Yourself}\label{practice-yourself-6}

\chapter{Supervised Learning}\label{supervised}

Machine learning is very similar to statistics, but it is certainly not
the same. As the name suggests, in machine learning we want machines to
learn. This means that we want to replace hard-coded expert algorithm,
with data-driven self-learned algorithm.

There are many learning setups, that depend on what information is
available to the machine. The most common setup, discussed in this
chapter, is \emph{supervised learning}. The name takes from the fact
that by giving the machine data samples with known inputs (a.k.a.
features) and desired outputs (a.k.a. labels), the human is effectively
supervising the learning. If we think of the inputs as predictors, and
outcomes as predicted, it is no wonder that supervised learning is very
similar to statistical prediction. When asked ``are these the same?'' I
like to give the example of internet fraud. If you take a sample of
fraud ``attacks'', a statistical formulation of the problem is highly
unlikely. This is because fraud events are not randomly drawn from some
distribution, but rather, arrive from an adversary learning the defenses
and adapting to it. This instance of supervised learning is more similar
to game theory than statistics.

Other types of machine learning problems include
\citep{sammut2011encyclopedia}:

\begin{itemize}
\item
  \textbf{Unsupervised learning}: See Chapter \ref{unsupervised}.
\item
  \textbf{Semi supervised learning}: Where only part of the samples are
  labeled. A.k.a. \emph{co-training}, \emph{learning from labeled and
  unlabeled data}, \emph{transductive learning}.
\item
  \textbf{Active learning}: Where the machine is allowed to query the
  user for labels. Very similar to \emph{adaptive design of
  experiments}.
\item
  \textbf{Learning on a budget}: A version of active learning where
  querying for labels induces variable costs.
\item
  \textbf{Reinforcement learning}:\\
  Similar to active learning, in that the machine may query for labels.
  Different from active learning, in that the machine does not receive
  labels, but \emph{rewards}.
\item
  \textbf{Structure learning}: When predicting objects with structure
  such as dependent vectors, graphs, images, tensors, etc.
\item
  \textbf{Manifold learning}: An instance of unsupervised learning,
  where the goal is to reduce the dimension of the data by embedding it
  into a lower dimensional manifold. A.k.a. \emph{support estimation}.
\item
  \textbf{Learning to learn}: Deals with the carriage of ``experience''
  from one learning problem to another. A.k.a. \emph{cummulative
  learning}, \emph{knowledge transfer}, and \emph{meta learning}.
\end{itemize}

\section{Problem Setup}\label{problem-setup-3}

We now present the \emph{empirical risk minimization} (ERM) approach to
supervised learning, a.k.a. \emph{M-estimation} in the statistical
literature.

\BeginKnitrBlock{remark}
\iffalse {Remark. } \fi We do not discuss purely algorithmic approaches
such as K-nearest neighbour and \emph{kernel smoothing} due to space
constraints. For a broader review of supervised learning, see the
Bibliographic Notes.
\EndKnitrBlock{remark}

Given \(n\) samples with inputs \(x\) from some space \(\mathcal{X}\)
and desired outcome, \(y\), from some space \(\mathcal{Y}\). Samples,
\((x,y)\) have some distribution we denote \(P\). We want to learn a
function that maps inputs to outputs. This function is called a
\emph{hypothesis}, or \emph{predictor}, or \emph{classifier}, denoted
\(f\), that belongs to a hypothesis class \(\mathcal{F}\) such that
\(f:\mathcal{X} \to \mathcal{Y}\). We also choose some other function
that fines us for erroneous prediction. This function is called the
\emph{loss}, and we denote it by
\(l:\mathcal{Y}\times \mathcal{Y} \to \mathbb{R}^+\).

\BeginKnitrBlock{remark}
\iffalse {Remark. } \fi The \emph{hypothesis} in machine learning is
only vaguely related the \emph{hypothesis} in statistical testing, which
is quite confusing.
\EndKnitrBlock{remark}

\BeginKnitrBlock{remark}
\iffalse {Remark. } \fi The \emph{hypothesis} in machine learning is not
a bona-fide \emph{statistical model} since we don't assume it is the
data generating process, but rather some function which we choose for
its good predictive performance.
\EndKnitrBlock{remark}

The fundamental task in supervised (statistical) learning is to recover
a hypothesis that minimizes the average loss in the sample, and not in
the population. This is know as the \emph{risk minimization problem}.

\BeginKnitrBlock{definition}[Risk Function]
\protect\hypertarget{def:unnamed-chunk-166}{}{\label{def:unnamed-chunk-166}
\iffalse (Risk Function) \fi }The \emph{risk function}, a.k.a.
\emph{generalization error}, or \emph{test error}, is the population
average loss of a predictor \(f\):

\begin{align}
  R(f):=E_P[l(f(x),y)].
\end{align}
\EndKnitrBlock{definition}

The best predictor, is the risk minimizer:

\begin{align}
  f^* := argmin_f \{R(f)\}.
  \label{eq:risk}  
\end{align}

To make things more explicit, \(f\) may be a linear function, and \(l\)
a squared error loss, in which case problem \eqref{eq:risk} collapses to

\begin{align}
  f^* := argmin_\beta \{ E_{P(x,y)}[(x'\beta-y)^2] \}.
\end{align}

Another fundamental problem is that we do not know the distribution of
all possible inputs and outputs, \(P\). We typically only have a sample
of \((x_i,y_i), i=1,\dots,n\). We thus state the \emph{empirical}
counterpart of \eqref{eq:risk}, which consists of minimizing the average
loss. This is known as the \emph{empirical risk miminization} problem
(ERM).

\BeginKnitrBlock{definition}[Empirical Risk]
\protect\hypertarget{def:unnamed-chunk-167}{}{\label{def:unnamed-chunk-167}
\iffalse (Empirical Risk) \fi }The \emph{empirical risk function},
a.k.a. \emph{in-sample error}, or \emph{train error}, is the sample
average loss of a predictor \(f\):

\begin{align}
  R_n(f):= \sum_i l(f(x_i),y_i).
\end{align}
\EndKnitrBlock{definition}

A good candidate predictor, \(\hat f\), is thus the \emph{empirical risk
minimizer}:

\begin{align}
  \hat f := argmin_f \{ R_n(f) \}.
  \label{eq:erm}  
\end{align}

Making things more explicit again by using a linear hypothesis with
squared loss, we see that the empirical risk minimization problem
collapses to an ordinary least-squares problem:

\begin{align}
  \hat f := argmin_\beta \{ \sum_i (x_i\beta - y_i)^2 \}.
\end{align}

When data samples are assumingly independent, then maximum likelihood
estimation is also an instance of ERM, when using the (negative) log
likelihood as the loss function.

If we don't assume any structure on the hypothesis, \(f\), then
\(\hat f\) from \eqref{eq:erm} will interpolate the data, and \(\hat f\)
will be a very bad predictor. We say, it will \emph{overfit} the
observed data, and will have bad performance on new data.

We have several ways to avoid overfitting:

\begin{enumerate}
\def\labelenumi{\arabic{enumi}.}
\tightlist
\item
  Restrict the hypothesis class \(\mathcal{F}\) (such as linear
  functions).
\item
  Penalize for the complexity of \(f\). The penalty denoted by
  \(\Vert f \Vert\).
\item
  Unbiased risk estimation, where we deal with the overfitted optimism
  of the empirical risk by debiasing it.
\end{enumerate}

\subsection{Common Hypothesis Classes}\label{common-hypothesis-classes}

Some common hypothesis classes, \(\mathcal{F}\), with restricted
complexity, are:

\begin{enumerate}
\def\labelenumi{\arabic{enumi}.}
\tightlist
\item
  \textbf{Linear hypotheses}: such as linear models, GLMs, and (linear)
  support vector machines (SVM).
\item
  \textbf{Neural networks}: a.k.a. \emph{feed-forward} neural nets,
  \emph{artificial} neural nets, and the celebrated class of \emph{deep}
  neural nets.
\item
  \textbf{Tree}: a.k.a. \emph{decision rules}, is a class of hypotheses
  which can be stated as ``if-then'' rules.
\item
  \textbf{Reproducing Kernel Hilbert Space}: a.k.a. RKHS, is a subset of
  ``the space of all functions\footnote{It is even a subset of the
    Hilbert space, itself a subset of the space of all functions.}''
  that is both large enough to capture very complicated relations, but
  small enough so that it is less prone to overfitting, and also
  surprisingly simple to compute with.
\item
  \textbf{Ensembles}: a ``meta'' hypothesis class, which consists of
  taking multiple hypotheses, possibly from different classes, and
  combining them.
\end{enumerate}

\subsection{Common Complexity
Penalties}\label{common-complexity-penalties}

The most common complexity penalty applies to classes that have a finite
dimensional parametric representation, such as the class of linear
predictors, parametrized via its coefficients \(\beta\). In such classes
we may penalize for the norm of the parameters. Common penalties
include:

\begin{enumerate}
\def\labelenumi{\arabic{enumi}.}
\tightlist
\item
  \textbf{Ridge penalty}: penalizing the \(l_2\) norm of the parameter.
  I.e. \(\Vert f \Vert=\Vert \beta \Vert_2^2=\sum_j \beta_j^2\).
\item
  \textbf{Lasso penalty}: penalizing the \(l_1\) norm of the parameter.
  I.e., \(\Vert f \Vert=\Vert \beta \Vert_1=\sum_j |\beta_j|\)
\item
  \textbf{Elastic net}: a combination of the lasso and ridge penalty.
  I.e.
  ,\(\Vert f \Vert= \alpha \Vert \beta \Vert_2^2 + (1-\alpha) \Vert \beta \Vert_1\).
\end{enumerate}

If the hypothesis class \(\mathcal{F}\) does not admit a finite
dimensional parametric representation, we may penalize it with some
functional norm such as \(\Vert f \Vert=\sqrt{\int f(t)^2 dt}\).

\subsection{Unbiased Risk Estimation}\label{unbiased-risk-estimation}

The fundamental problem of overfitting, is that the empirical risk,
\(R_n(\hat f)\), is downward biased to the population risk,
\(R(\hat f)\). Formally: \[ R_n(\hat f)<R_n(f^*) \] Why is that? Think
of estimating a population's mean with the sample minimum. It can be
done, but the minimum has to be debiased for it to estimate the
population mean. Unbiased estimation of \(R(f)\) broadly fall under: (a)
purely algorithmic \emph{resampling} based approaches, and (b) theory
driven estimators.

\begin{enumerate}
\def\labelenumi{\arabic{enumi}.}
\item
  \textbf{Train-Validate-Test}: The simplest form of validation is to
  split the data. A \emph{train} set to train/estimate/learn \(\hat f\).
  A \emph{validation} set to compute the out-of-sample expected loss,
  \(R(\hat f)\), and pick the best performing predictor. A \emph{test}
  sample to compute the out-of-sample performance of the selected
  hypothesis. This is a very simple approach, but it is very ``data
  inefficient'', thus motivating the next method.
\item
  \textbf{V-Fold Cross Validation}: By far the most popular risk
  estimation algorithm, in \emph{V-fold CV} we ``fold'' the data into
  \(V\) non-overlapping sets. For each of the \(V\) sets, we learn
  \(\hat f\) with the non-selected fold, and assess \(R(\hat f)\)) on
  the selected fold. We then aggregate results over the \(V\) folds,
  typically by averaging.
\item
  \textbf{AIC}: Akaike's information criterion (AIC) is a theory driven
  correction of the empirical risk, so that it is unbiased to the true
  risk. It is appropriate when using the likelihood loss.
\item
  \textbf{Cp}: Mallow's Cp is an instance of AIC for likelihood loss
  under normal noise.
\end{enumerate}

Other theory driven unbiased risk estimators include the \emph{Bayesian
Information Criterion} (BIC, aka SBC, aka SBIC), the \emph{Minimum
Description Length} (MDL), \emph{Vapnic's Structural Risk Minimization}
(SRM), the \emph{Deviance Information Criterion} (DIC), and the
\emph{Hannan-Quinn Information Criterion} (HQC).

Other resampling based unbiased risk estimators include resampling
\textbf{without replacement} algorithms like \emph{delete-d cross
validation} with its many variations, and \textbf{resampling with
replacement}, like the \emph{bootstrap}, with its many variations.

\subsection{Collecting the Pieces}\label{collecting-the-pieces}

An ERM problem with regularization will look like

\begin{align}
  \hat f := argmin_{f \in \mathcal{F}} \{ R_n(f)  + \lambda \Vert f \Vert \}.
  \label{eq:erm-regularized}  
\end{align}

Collecting ideas from the above sections, a typical supervised learning
pipeline will include: choosing the hypothesis class, choosing the
penalty function and level, unbiased risk estimator. We emphasize that
choosing the penalty function, \(\Vert f \Vert\) is not enough, and we
need to choose how ``hard'' to apply it. This if known as the
\emph{regularization level}, denoted by \(\lambda\) in
Eq.\eqref{eq:erm-regularized}.

Examples of such combos include:

\begin{enumerate}
\def\labelenumi{\arabic{enumi}.}
\tightlist
\item
  Linear regression, no penalty, train-validate test.
\item
  Linear regression, no penalty, AIC.
\item
  Linear regression, \(l_2\) penalty, V-fold CV. This combo is typically
  known as \emph{ridge regression}.
\item
  Linear regression, \(l_1\) penalty, V-fold CV. This combo is typically
  known as \emph{lasso regression}.
\item
  Linear regression, \(l_1\) and \(l_2\) penalty, V-fold CV. This combo
  is typically known as \emph{elastic net regression}.
\item
  Logistic regression, \(l_2\) penalty, V-fold CV.
\item
  SVM classification, \(l_2\) penalty, V-fold CV.
\item
  Deep network, no penalty, V-fold CV.
\item
  Etc.
\end{enumerate}

For fans of statistical hypothesis testing we will also emphasize:
Testing and prediction are related, but are not the same. \textbf{It is
indeed possible that we will want to ignore a significant predictor, and
add a non-significant one!} \citep{foster2004variable} Some authors will
use hypothesis testing as an initial screening of candidate predictors.
This is a useful heuristic, but that is all it is-- a heuristic.

\section{Supervised Learning in R}\label{supervised-learning-in-r}

At this point, we have a rich enough language to do supervised learning
with R.

In these examples, I will use two data sets from the
\textbf{ElemStatLearn} package: \texttt{spam} for categorical
predictions, and \texttt{prostate} for continuous predictions. In
\texttt{spam} we will try to decide if a mail is spam or not. In
\texttt{prostate} we will try to predict the size of a cancerous tumor.
You can now call \texttt{?prostate} and \texttt{?spam} to learn more
about these data sets.

Some boring pre-processing.

\begin{Shaded}
\begin{Highlighting}[]
\KeywordTok{library}\NormalTok{(ElemStatLearn) }
\KeywordTok{data}\NormalTok{(}\StringTok{"prostate"}\NormalTok{)}
\KeywordTok{data}\NormalTok{(}\StringTok{"spam"}\NormalTok{)}

\KeywordTok{library}\NormalTok{(magrittr) }\CommentTok{# for piping}

\CommentTok{# Preparing prostate data}
\NormalTok{prostate.train <-}\StringTok{ }\NormalTok{prostate[prostate$train, }\KeywordTok{names}\NormalTok{(prostate)!=}\StringTok{'train'}\NormalTok{]}
\NormalTok{prostate.test <-}\StringTok{ }\NormalTok{prostate[!prostate$train, }\KeywordTok{names}\NormalTok{(prostate)!=}\StringTok{'train'}\NormalTok{] }
\NormalTok{y.train <-}\StringTok{ }\NormalTok{prostate.train$lcavol}
\NormalTok{X.train <-}\StringTok{ }\KeywordTok{as.matrix}\NormalTok{(prostate.train[, }\KeywordTok{names}\NormalTok{(prostate.train)!=}\StringTok{'lcavol'}\NormalTok{] )}
\NormalTok{y.test <-}\StringTok{ }\NormalTok{prostate.test$lcavol }
\NormalTok{X.test <-}\StringTok{ }\KeywordTok{as.matrix}\NormalTok{(prostate.test[, }\KeywordTok{names}\NormalTok{(prostate.test)!=}\StringTok{'lcavol'}\NormalTok{] )}

\CommentTok{# Preparing spam data:}
\NormalTok{n <-}\StringTok{ }\KeywordTok{nrow}\NormalTok{(spam)}

\NormalTok{train.prop <-}\StringTok{ }\FloatTok{0.66}
\NormalTok{train.ind <-}\StringTok{ }\KeywordTok{c}\NormalTok{(}\OtherTok{TRUE}\NormalTok{,}\OtherTok{FALSE}\NormalTok{) %>%}\StringTok{  }
\StringTok{  }\KeywordTok{sample}\NormalTok{(}\DataTypeTok{size =} \NormalTok{n, }\DataTypeTok{prob =} \KeywordTok{c}\NormalTok{(train.prop,}\DecValTok{1}\NormalTok{-train.prop), }\DataTypeTok{replace=}\OtherTok{TRUE}\NormalTok{)}
\NormalTok{spam.train <-}\StringTok{ }\NormalTok{spam[train.ind,]}
\NormalTok{spam.test <-}\StringTok{ }\NormalTok{spam[!train.ind,]}

\NormalTok{y.train.spam <-}\StringTok{ }\NormalTok{spam.train$spam}
\NormalTok{X.train.spam <-}\StringTok{ }\KeywordTok{as.matrix}\NormalTok{(spam.train[,}\KeywordTok{names}\NormalTok{(spam.train)!=}\StringTok{'spam'}\NormalTok{] )}
\NormalTok{y.test.spam <-}\StringTok{ }\NormalTok{spam.test$spam}
\NormalTok{X.test.spam <-}\StringTok{  }\KeywordTok{as.matrix}\NormalTok{(spam.test[,}\KeywordTok{names}\NormalTok{(spam.test)!=}\StringTok{'spam'}\NormalTok{]) }

\NormalTok{spam.dummy <-}\StringTok{ }\NormalTok{spam}
\NormalTok{spam.dummy$spam <-}\StringTok{ }\KeywordTok{as.numeric}\NormalTok{(spam$spam==}\StringTok{'spam'}\NormalTok{) }
\NormalTok{spam.train.dummy <-}\StringTok{ }\NormalTok{spam.dummy[train.ind,]}
\NormalTok{spam.test.dummy <-}\StringTok{ }\NormalTok{spam.dummy[!train.ind,]}
\end{Highlighting}
\end{Shaded}

We also define some utility functions that we will require down the
road.

\begin{Shaded}
\begin{Highlighting}[]
\NormalTok{l2 <-}\StringTok{ }\NormalTok{function(x) x^}\DecValTok{2} \NormalTok\StringTok{ }\NormalTok{sum %>%}\StringTok{ }\NormalTok{sqrt }
\NormalTok{l1 <-}\StringTok{ }\NormalTok{function(x) }\KeywordTok{abs}\NormalTok{(x) %>%}\StringTok{ }\NormalTok{sum  }
\NormalTok{MSE <-}\StringTok{ }\NormalTok{function(x) x^}\DecValTok{2} \NormalTok\StringTok{ }\NormalTok{mean }
\NormalTok{missclassification <-}\StringTok{ }\NormalTok{function(tab) }\KeywordTok{sum}\NormalTok{(tab[}\KeywordTok{c}\NormalTok{(}\DecValTok{2}\NormalTok{,}\DecValTok{3}\NormalTok{)])/}\KeywordTok{sum}\NormalTok{(tab)}
\end{Highlighting}
\end{Shaded}

\subsection{Linear Models with Least Squares Loss}\label{least-squares}

Starting with OLS regression, and a train-test data approach. Notice the
better in-sample MSE than the out-of-sample. That is overfitting in
action.

\begin{Shaded}
\begin{Highlighting}[]
\NormalTok{ols}\FloatTok{.1} \NormalTok{<-}\StringTok{ }\KeywordTok{lm}\NormalTok{(lcavol~. ,}\DataTypeTok{data =} \NormalTok{prostate.train)}
\CommentTok{# Train error:}
\KeywordTok{MSE}\NormalTok{( }\KeywordTok{predict}\NormalTok{(ols}\FloatTok{.1}\NormalTok{)-}\StringTok{ }\NormalTok{prostate.train$lcavol) }
\end{Highlighting}
\end{Shaded}

\begin{verbatim}
## [1] 0.4383709
\end{verbatim}

\begin{Shaded}
\begin{Highlighting}[]
\CommentTok{# Test error:}
\KeywordTok{MSE}\NormalTok{( }\KeywordTok{predict}\NormalTok{(ols}\FloatTok{.1}\NormalTok{, }\DataTypeTok{newdata =} \NormalTok{prostate.test)-}\StringTok{ }\NormalTok{prostate.test$lcavol)}
\end{Highlighting}
\end{Shaded}

\begin{verbatim}
## [1] 0.5084068
\end{verbatim}

We now implement a V-fold CV, instead of our train-test approach. The
assignment of each observation to each fold is encoded in
\texttt{fold.assignment}. The following code is extremely inefficient,
but easy to read.

\begin{Shaded}
\begin{Highlighting}[]
\NormalTok{folds <-}\StringTok{ }\DecValTok{10}
\NormalTok{fold.assignment <-}\StringTok{ }\KeywordTok{sample}\NormalTok{(}\DecValTok{1}\NormalTok{:folds, }\KeywordTok{nrow}\NormalTok{(prostate), }\DataTypeTok{replace =} \OtherTok{TRUE}\NormalTok{)}
\NormalTok{errors <-}\StringTok{ }\OtherTok{NULL}

\NormalTok{for (k in }\DecValTok{1}\NormalTok{:folds)\{}
  \NormalTok{prostate.cross.train <-}\StringTok{ }\NormalTok{prostate[fold.assignment!=k,] }\CommentTok{# train subset}
  \NormalTok{prostate.cross.test <-}\StringTok{  }\NormalTok{prostate[fold.assignment==k,] }\CommentTok{# test subset}
  \NormalTok{.ols <-}\StringTok{ }\KeywordTok{lm}\NormalTok{(lcavol~. ,}\DataTypeTok{data =} \NormalTok{prostate.cross.train) }\CommentTok{# train}
  \NormalTok{.predictions <-}\StringTok{ }\KeywordTok{predict}\NormalTok{(.ols, }\DataTypeTok{newdata=}\NormalTok{prostate.cross.test)}
  \NormalTok{.errors <-}\StringTok{  }\NormalTok{.predictions -}\StringTok{ }\NormalTok{prostate.cross.test$lcavol }\CommentTok{# save prediction errors in the fold}
  \NormalTok{errors <-}\StringTok{ }\KeywordTok{c}\NormalTok{(errors, .errors) }\CommentTok{# aggregate error over folds.}
\NormalTok{\}}

\CommentTok{# Cross validated prediction error:}
\KeywordTok{MSE}\NormalTok{(errors)}
\end{Highlighting}
\end{Shaded}

\begin{verbatim}
## [1] 0.5442395
\end{verbatim}

Let's try all possible models, and choose the best performer with
respect to the Cp unbiased risk estimator. This is done with
\texttt{leaps::regsubsets}. We see that the best performer has 3
predictors.

\begin{Shaded}
\begin{Highlighting}[]
\KeywordTok{library}\NormalTok{(leaps)}
\NormalTok{regfit.full <-}\StringTok{ }\NormalTok{prostate.train %>%}\StringTok{ }
\StringTok{  }\KeywordTok{regsubsets}\NormalTok{(lcavol~.,}\DataTypeTok{data =} \NormalTok{., }\DataTypeTok{method =} \StringTok{'exhaustive'}\NormalTok{) }\CommentTok{# best subset selection}
\KeywordTok{plot}\NormalTok{(regfit.full, }\DataTypeTok{scale =} \StringTok{"Cp"}\NormalTok{)}
\end{Highlighting}
\end{Shaded}

\includegraphics[width=0.5\linewidth]{Rcourse_files/figure-latex/all subset-1}

Instead of the Cp criterion, we now compute the train and test errors
for all the possible predictor subsets\footnote{Example taken from
  \url{https://lagunita.stanford.edu/c4x/HumanitiesScience/StatLearning/asset/ch6.html}}.
In the resulting plot we can see overfitting in action.

\begin{Shaded}
\begin{Highlighting}[]
\NormalTok{model.n <-}\StringTok{ }\NormalTok{regfit.full %>%}\StringTok{ }\NormalTok{summary %>%}\StringTok{ }\NormalTok{length}
\NormalTok{X.train.named <-}\StringTok{ }\KeywordTok{model.matrix}\NormalTok{(lcavol ~}\StringTok{ }\NormalTok{., }\DataTypeTok{data =} \NormalTok{prostate.train ) }
\NormalTok{X.test.named <-}\StringTok{ }\KeywordTok{model.matrix}\NormalTok{(lcavol ~}\StringTok{ }\NormalTok{., }\DataTypeTok{data =} \NormalTok{prostate.test ) }


\NormalTok{val.errors <-}\StringTok{ }\KeywordTok{rep}\NormalTok{(}\OtherTok{NA}\NormalTok{, model.n)}
\NormalTok{train.errors <-}\StringTok{ }\KeywordTok{rep}\NormalTok{(}\OtherTok{NA}\NormalTok{, model.n)}
\NormalTok{for (i in }\DecValTok{1}\NormalTok{:model.n) \{}
    \NormalTok{coefi <-}\StringTok{ }\KeywordTok{coef}\NormalTok{(regfit.full, }\DataTypeTok{id =} \NormalTok{i)}
    
    \NormalTok{pred <-}\StringTok{  }\NormalTok{X.train.named[, }\KeywordTok{names}\NormalTok{(coefi)] %*%}\StringTok{ }\NormalTok{coefi }\CommentTok{# make in-sample predictions}
    \NormalTok{train.errors[i] <-}\StringTok{ }\KeywordTok{MSE}\NormalTok{(y.train -}\StringTok{ }\NormalTok{pred) }\CommentTok{# train errors}

    \NormalTok{pred <-}\StringTok{  }\NormalTok{X.test.named[, }\KeywordTok{names}\NormalTok{(coefi)] %*%}\StringTok{ }\NormalTok{coefi }\CommentTok{# make out-of-sample predictions}
    \NormalTok{val.errors[i] <-}\StringTok{ }\KeywordTok{MSE}\NormalTok{(y.test -}\StringTok{ }\NormalTok{pred) }\CommentTok{# test errors}
\NormalTok{\}}
\end{Highlighting}
\end{Shaded}

Plotting results.

\begin{Shaded}
\begin{Highlighting}[]
\KeywordTok{plot}\NormalTok{(train.errors, }\DataTypeTok{ylab =} \StringTok{"MSE"}\NormalTok{, }\DataTypeTok{pch =} \DecValTok{19}\NormalTok{, }\DataTypeTok{type =} \StringTok{"o"}\NormalTok{)}
\KeywordTok{points}\NormalTok{(val.errors, }\DataTypeTok{pch =} \DecValTok{19}\NormalTok{, }\DataTypeTok{type =} \StringTok{"b"}\NormalTok{, }\DataTypeTok{col=}\StringTok{"blue"}\NormalTok{)}
\KeywordTok{legend}\NormalTok{(}\StringTok{"topright"}\NormalTok{, }
       \DataTypeTok{legend =} \KeywordTok{c}\NormalTok{(}\StringTok{"Training"}\NormalTok{, }\StringTok{"Validation"}\NormalTok{), }
       \DataTypeTok{col =} \KeywordTok{c}\NormalTok{(}\StringTok{"black"}\NormalTok{, }\StringTok{"blue"}\NormalTok{), }
       \DataTypeTok{pch =} \DecValTok{19}\NormalTok{)}
\end{Highlighting}
\end{Shaded}

\includegraphics[width=0.5\linewidth]{Rcourse_files/figure-latex/unnamed-chunk-169-1}

Checking all possible models is computationally very hard. \emph{Forward
selection} is a greedy approach that adds one variable at a time, using
the AIC risk estimator. If AIC decreases, the variable is added.

\begin{Shaded}
\begin{Highlighting}[]
\NormalTok{ols}\FloatTok{.0} \NormalTok{<-}\StringTok{ }\KeywordTok{lm}\NormalTok{(lcavol~}\DecValTok{1} \NormalTok{,}\DataTypeTok{data =} \NormalTok{prostate.train)}
\NormalTok{model.scope <-}\StringTok{ }\KeywordTok{list}\NormalTok{(}\DataTypeTok{upper=}\NormalTok{ols}\FloatTok{.1}\NormalTok{, }\DataTypeTok{lower=}\NormalTok{ols}\FloatTok{.0}\NormalTok{)}
\KeywordTok{step}\NormalTok{(ols}\FloatTok{.0}\NormalTok{, }\DataTypeTok{scope=}\NormalTok{model.scope, }\DataTypeTok{direction=}\StringTok{'forward'}\NormalTok{, }\DataTypeTok{trace =} \OtherTok{TRUE}\NormalTok{)}
\end{Highlighting}
\end{Shaded}

\begin{verbatim}
## Start:  AIC=30.1
## lcavol ~ 1
## 
##           Df Sum of Sq     RSS     AIC
## + lpsa     1    54.776  47.130 -19.570
## + lcp      1    48.805  53.101 -11.578
## + svi      1    35.829  66.077   3.071
## + pgg45    1    23.789  78.117  14.285
## + gleason  1    18.529  83.377  18.651
## + lweight  1     9.186  92.720  25.768
## + age      1     8.354  93.552  26.366
## <none>                 101.906  30.097
## + lbph     1     0.407 101.499  31.829
## 
## Step:  AIC=-19.57
## lcavol ~ lpsa
## 
##           Df Sum of Sq    RSS     AIC
## + lcp      1   14.8895 32.240 -43.009
## + svi      1    5.0373 42.093 -25.143
## + gleason  1    3.5500 43.580 -22.817
## + pgg45    1    3.0503 44.080 -22.053
## + lbph     1    1.8389 45.291 -20.236
## + age      1    1.5329 45.597 -19.785
## <none>                 47.130 -19.570
## + lweight  1    0.4106 46.719 -18.156
## 
## Step:  AIC=-43.01
## lcavol ~ lpsa + lcp
## 
##           Df Sum of Sq    RSS     AIC
## <none>                 32.240 -43.009
## + age      1   0.92315 31.317 -42.955
## + pgg45    1   0.29594 31.944 -41.627
## + gleason  1   0.21500 32.025 -41.457
## + lbph     1   0.13904 32.101 -41.298
## + lweight  1   0.05504 32.185 -41.123
## + svi      1   0.02069 32.220 -41.052
\end{verbatim}

\begin{verbatim}
## 
## Call:
## lm(formula = lcavol ~ lpsa + lcp, data = prostate.train)
## 
## Coefficients:
## (Intercept)         lpsa          lcp  
##     0.08798      0.53369      0.38879
\end{verbatim}

We now learn a linear predictor on the \texttt{spam} data using, a least
squares loss, and train-test risk estimator.

\begin{Shaded}
\begin{Highlighting}[]
\CommentTok{# train the predictor}
\NormalTok{ols}\FloatTok{.2} \NormalTok{<-}\StringTok{ }\KeywordTok{lm}\NormalTok{(spam~., }\DataTypeTok{data =} \NormalTok{spam.train.dummy) }

\CommentTok{# make in-sample predictions}
\NormalTok{.predictions.train <-}\StringTok{ }\KeywordTok{predict}\NormalTok{(ols}\FloatTok{.2}\NormalTok{) >}\StringTok{ }\FloatTok{0.5} 
\CommentTok{# inspect the confusion matrix}
\NormalTok{(confusion.train <-}\StringTok{ }\KeywordTok{table}\NormalTok{(}\DataTypeTok{prediction=}\NormalTok{.predictions.train, }\DataTypeTok{truth=}\NormalTok{spam.train.dummy$spam)) }
\end{Highlighting}
\end{Shaded}

\begin{verbatim}
##           truth
## prediction    0    1
##      FALSE 1752  250
##      TRUE    79  915
\end{verbatim}

\begin{Shaded}
\begin{Highlighting}[]
\CommentTok{# compute the train (in sample) misclassification}
\KeywordTok{missclassification}\NormalTok{(confusion.train) }
\end{Highlighting}
\end{Shaded}

\begin{verbatim}
## [1] 0.1098131
\end{verbatim}

\begin{Shaded}
\begin{Highlighting}[]
\CommentTok{# make out-of-sample prediction}
\NormalTok{.predictions.test <-}\StringTok{ }\KeywordTok{predict}\NormalTok{(ols}\FloatTok{.2}\NormalTok{, }\DataTypeTok{newdata =} \NormalTok{spam.test.dummy) >}\StringTok{ }\FloatTok{0.5} 
\CommentTok{# inspect the confusion matrix}
\NormalTok{(confusion.test <-}\StringTok{ }\KeywordTok{table}\NormalTok{(}\DataTypeTok{prediction=}\NormalTok{.predictions.test, }\DataTypeTok{truth=}\NormalTok{spam.test.dummy$spam))}
\end{Highlighting}
\end{Shaded}

\begin{verbatim}
##           truth
## prediction   0   1
##      FALSE 915 135
##      TRUE   42 513
\end{verbatim}

\begin{Shaded}
\begin{Highlighting}[]
\CommentTok{# compute the train (in sample) misclassification}
\KeywordTok{missclassification}\NormalTok{(confusion.test)}
\end{Highlighting}
\end{Shaded}

\begin{verbatim}
## [1] 0.1102804
\end{verbatim}

The \texttt{glmnet} package is an excellent package that provides ridge,
lasso, and elastic net regularization, for all GLMs, so for linear
models in particular.

\begin{Shaded}
\begin{Highlighting}[]
\KeywordTok{suppressMessages}\NormalTok{(}\KeywordTok{library}\NormalTok{(glmnet))}
\NormalTok{ridge}\FloatTok{.2} \NormalTok{<-}\StringTok{ }\KeywordTok{glmnet}\NormalTok{(}\DataTypeTok{x=}\NormalTok{X.train, }\DataTypeTok{y=}\NormalTok{y.train, }\DataTypeTok{family =} \StringTok{'gaussian'}\NormalTok{, }\DataTypeTok{alpha =} \DecValTok{0}\NormalTok{)}

\CommentTok{# Train error:}
\KeywordTok{MSE}\NormalTok{( }\KeywordTok{predict}\NormalTok{(ridge}\FloatTok{.2}\NormalTok{, }\DataTypeTok{newx =}\NormalTok{X.train)-}\StringTok{ }\NormalTok{y.train)}
\end{Highlighting}
\end{Shaded}

\begin{verbatim}
## [1] 1.006028
\end{verbatim}

\begin{Shaded}
\begin{Highlighting}[]
\CommentTok{# Test error:}
\KeywordTok{MSE}\NormalTok{( }\KeywordTok{predict}\NormalTok{(ridge}\FloatTok{.2}\NormalTok{, }\DataTypeTok{newx =} \NormalTok{X.test)-}\StringTok{ }\NormalTok{y.test)}
\end{Highlighting}
\end{Shaded}

\begin{verbatim}
## [1] 0.7678264
\end{verbatim}

Things to note:

\begin{itemize}
\tightlist
\item
  The \texttt{alpha=0} parameters tells R to do ridge regression.
  Setting \(alpha=1\) will do lasso, and any other value, with return an
  elastic net with appropriate weights.
\item
  The `family=`gaussian' argument tells R to fit a linear model, with
  least squares loss.
\item
  The test error is \textbf{smaller} than the train error, which I
  attribute to the variability of the risk estimators.
\end{itemize}

\BeginKnitrBlock{remark}
\iffalse {Remark. } \fi The variability of risk estimator is a very
interesting problem, which recieved very little attention in the machine
learning literature. If this topic interests you, talk to me.
\EndKnitrBlock{remark}

We now use the lasso penalty.

\begin{Shaded}
\begin{Highlighting}[]
\NormalTok{lasso}\FloatTok{.1} \NormalTok{<-}\StringTok{ }\KeywordTok{glmnet}\NormalTok{(}\DataTypeTok{x=}\NormalTok{X.train, }\DataTypeTok{y=}\NormalTok{y.train, , }\DataTypeTok{family=}\StringTok{'gaussian'}\NormalTok{, }\DataTypeTok{alpha =} \DecValTok{1}\NormalTok{)}

\CommentTok{# Train error:}
\KeywordTok{MSE}\NormalTok{( }\KeywordTok{predict}\NormalTok{(lasso}\FloatTok{.1}\NormalTok{, }\DataTypeTok{newx =}\NormalTok{X.train)-}\StringTok{ }\NormalTok{y.train)}
\end{Highlighting}
\end{Shaded}

\begin{verbatim}
## [1] 0.5525279
\end{verbatim}

\begin{Shaded}
\begin{Highlighting}[]
\CommentTok{# Test error:}
\KeywordTok{MSE}\NormalTok{( }\KeywordTok{predict}\NormalTok{(lasso}\FloatTok{.1}\NormalTok{, }\DataTypeTok{newx =} \NormalTok{X.test)-}\StringTok{ }\NormalTok{y.test)}
\end{Highlighting}
\end{Shaded}

\begin{verbatim}
## [1] 0.5211263
\end{verbatim}

We now use \texttt{glmnet} for classification.

\begin{Shaded}
\begin{Highlighting}[]
\NormalTok{logistic}\FloatTok{.2} \NormalTok{<-}\StringTok{ }\KeywordTok{cv.glmnet}\NormalTok{(}\DataTypeTok{x=}\NormalTok{X.train.spam, }\DataTypeTok{y=}\NormalTok{y.train.spam, }\DataTypeTok{family =} \StringTok{"binomial"}\NormalTok{, }\DataTypeTok{alpha =} \DecValTok{0}\NormalTok{)}
\end{Highlighting}
\end{Shaded}

Things to note:

\begin{itemize}
\tightlist
\item
  We used \texttt{cv.glmnet} to do an automatic search for the optimal
  level of regularization (the \texttt{lambda} argument in
  \texttt{glmnet}) using V-fold CV.
\item
  We set \texttt{alpha=0} for ridge regression.
\end{itemize}

\begin{Shaded}
\begin{Highlighting}[]
\CommentTok{# Train confusion matrix:}
\NormalTok{.predictions.train <-}\StringTok{ }\KeywordTok{predict}\NormalTok{(logistic}\FloatTok{.2}\NormalTok{, }\DataTypeTok{newx =} \NormalTok{X.train.spam, }\DataTypeTok{type =} \StringTok{'class'}\NormalTok{) }
\NormalTok{(confusion.train <-}\StringTok{ }\KeywordTok{table}\NormalTok{(}\DataTypeTok{prediction=}\NormalTok{.predictions.train, }\DataTypeTok{truth=}\NormalTok{spam.train$spam))}
\end{Highlighting}
\end{Shaded}

\begin{verbatim}
##           truth
## prediction email spam
##      email  1748  178
##      spam     83  987
\end{verbatim}

\begin{Shaded}
\begin{Highlighting}[]
\CommentTok{# Train misclassification error}
\KeywordTok{missclassification}\NormalTok{(confusion.train)}
\end{Highlighting}
\end{Shaded}

\begin{verbatim}
## [1] 0.08711615
\end{verbatim}

\begin{Shaded}
\begin{Highlighting}[]
\CommentTok{# Test confusion matrix:}
\NormalTok{.predictions.test <-}\StringTok{ }\KeywordTok{predict}\NormalTok{(logistic}\FloatTok{.2}\NormalTok{, }\DataTypeTok{newx =} \NormalTok{X.test.spam, }\DataTypeTok{type=}\StringTok{'class'}\NormalTok{) }
\NormalTok{(confusion.test <-}\StringTok{ }\KeywordTok{table}\NormalTok{(}\DataTypeTok{prediction=}\NormalTok{.predictions.test, }\DataTypeTok{truth=}\NormalTok{y.test.spam))}
\end{Highlighting}
\end{Shaded}

\begin{verbatim}
##           truth
## prediction email spam
##      email   914  103
##      spam     43  545
\end{verbatim}

\begin{Shaded}
\begin{Highlighting}[]
\CommentTok{# Test misclassification error:}
\KeywordTok{missclassification}\NormalTok{(confusion.test)}
\end{Highlighting}
\end{Shaded}

\begin{verbatim}
## [1] 0.09096573
\end{verbatim}

\subsection{SVM}\label{svm}

A support vector machine (SVM) is a linear hypothesis class with a
particular loss function known as a \emph{hinge loss}. We learn an SVM
with the \texttt{svm} function from the \textbf{e1071} package, which is
merely a wrapper for the \textbf{libsvm} C library, which is the most
popular implementation of SVM today.

\begin{Shaded}
\begin{Highlighting}[]
\KeywordTok{library}\NormalTok{(e1071)}
\NormalTok{svm}\FloatTok{.1} \NormalTok{<-}\StringTok{ }\KeywordTok{svm}\NormalTok{(spam~., }\DataTypeTok{data =} \NormalTok{spam.train)}

\CommentTok{# Train confusion matrix:}
\NormalTok{.predictions.train <-}\StringTok{ }\KeywordTok{predict}\NormalTok{(svm}\FloatTok{.1}\NormalTok{) }
\NormalTok{(confusion.train <-}\StringTok{ }\KeywordTok{table}\NormalTok{(}\DataTypeTok{prediction=}\NormalTok{.predictions.train, }\DataTypeTok{truth=}\NormalTok{spam.train$spam))}
\end{Highlighting}
\end{Shaded}

\begin{verbatim}
##           truth
## prediction email spam
##      email  1776  101
##      spam     55 1064
\end{verbatim}

\begin{Shaded}
\begin{Highlighting}[]
\KeywordTok{missclassification}\NormalTok{(confusion.train)}
\end{Highlighting}
\end{Shaded}

\begin{verbatim}
## [1] 0.05206943
\end{verbatim}

\begin{Shaded}
\begin{Highlighting}[]
\CommentTok{# Test confusion matrix:}
\NormalTok{.predictions.test <-}\StringTok{ }\KeywordTok{predict}\NormalTok{(svm}\FloatTok{.1}\NormalTok{, }\DataTypeTok{newdata =} \NormalTok{spam.test) }
\NormalTok{(confusion.test <-}\StringTok{ }\KeywordTok{table}\NormalTok{(}\DataTypeTok{prediction=}\NormalTok{.predictions.test, }\DataTypeTok{truth=}\NormalTok{spam.test$spam))}
\end{Highlighting}
\end{Shaded}

\begin{verbatim}
##           truth
## prediction email spam
##      email   919   66
##      spam     38  582
\end{verbatim}

\begin{Shaded}
\begin{Highlighting}[]
\KeywordTok{missclassification}\NormalTok{(confusion.test)}
\end{Highlighting}
\end{Shaded}

\begin{verbatim}
## [1] 0.06479751
\end{verbatim}

We can also use SVM for regression.

\begin{Shaded}
\begin{Highlighting}[]
\NormalTok{svm}\FloatTok{.2} \NormalTok{<-}\StringTok{ }\KeywordTok{svm}\NormalTok{(lcavol~., }\DataTypeTok{data =} \NormalTok{prostate.train)}

\CommentTok{# Train error:}
\KeywordTok{MSE}\NormalTok{( }\KeywordTok{predict}\NormalTok{(svm}\FloatTok{.2}\NormalTok{)-}\StringTok{ }\NormalTok{prostate.train$lcavol)}
\end{Highlighting}
\end{Shaded}

\begin{verbatim}
## [1] 0.3336868
\end{verbatim}

\begin{Shaded}
\begin{Highlighting}[]
\CommentTok{# Test error:}
\KeywordTok{MSE}\NormalTok{( }\KeywordTok{predict}\NormalTok{(svm}\FloatTok{.2}\NormalTok{, }\DataTypeTok{newdata =} \NormalTok{prostate.test)-}\StringTok{ }\NormalTok{prostate.test$lcavol)}
\end{Highlighting}
\end{Shaded}

\begin{verbatim}
## [1] 0.5633183
\end{verbatim}

\subsection{Neural Nets}\label{neural-nets}

Neural nets (non deep) can be fitted, for example, with the
\texttt{nnet} function in the \textbf{nnet} package. We start with a
nnet regression.

\begin{Shaded}
\begin{Highlighting}[]
\KeywordTok{library}\NormalTok{(nnet)}
\NormalTok{nnet}\FloatTok{.1} \NormalTok{<-}\StringTok{ }\KeywordTok{nnet}\NormalTok{(lcavol~., }\DataTypeTok{size=}\DecValTok{20}\NormalTok{, }\DataTypeTok{data=}\NormalTok{prostate.train, }\DataTypeTok{rang =} \FloatTok{0.1}\NormalTok{, }\DataTypeTok{decay =} \FloatTok{5e-4}\NormalTok{, }\DataTypeTok{maxit =} \DecValTok{1000}\NormalTok{, }\DataTypeTok{trace=}\OtherTok{FALSE}\NormalTok{)}

\CommentTok{# Train error:}
\KeywordTok{MSE}\NormalTok{( }\KeywordTok{predict}\NormalTok{(nnet}\FloatTok{.1}\NormalTok{)-}\StringTok{ }\NormalTok{prostate.train$lcavol)}
\end{Highlighting}
\end{Shaded}

\begin{verbatim}
## [1] 1.173998
\end{verbatim}

\begin{Shaded}
\begin{Highlighting}[]
\CommentTok{# Test error:}
\KeywordTok{MSE}\NormalTok{( }\KeywordTok{predict}\NormalTok{(nnet}\FloatTok{.1}\NormalTok{, }\DataTypeTok{newdata =} \NormalTok{prostate.test)-}\StringTok{ }\NormalTok{prostate.test$lcavol)}
\end{Highlighting}
\end{Shaded}

\begin{verbatim}
## [1] 1.435194
\end{verbatim}

And nnet classification.

\begin{Shaded}
\begin{Highlighting}[]
\NormalTok{nnet}\FloatTok{.2} \NormalTok{<-}\StringTok{ }\KeywordTok{nnet}\NormalTok{(spam~., }\DataTypeTok{size=}\DecValTok{5}\NormalTok{, }\DataTypeTok{data=}\NormalTok{spam.train, }\DataTypeTok{rang =} \FloatTok{0.1}\NormalTok{, }\DataTypeTok{decay =} \FloatTok{5e-4}\NormalTok{, }\DataTypeTok{maxit =} \DecValTok{1000}\NormalTok{, }\DataTypeTok{trace=}\OtherTok{FALSE}\NormalTok{)}

\CommentTok{# Train confusion matrix:}
\NormalTok{.predictions.train <-}\StringTok{ }\KeywordTok{predict}\NormalTok{(nnet}\FloatTok{.2}\NormalTok{, }\DataTypeTok{type=}\StringTok{'class'}\NormalTok{) }
\NormalTok{(confusion.train <-}\StringTok{ }\KeywordTok{table}\NormalTok{(}\DataTypeTok{prediction=}\NormalTok{.predictions.train, }\DataTypeTok{truth=}\NormalTok{spam.train$spam))}
\end{Highlighting}
\end{Shaded}

\begin{verbatim}
##           truth
## prediction email spam
##      email  1777   49
##      spam     54 1116
\end{verbatim}

\begin{Shaded}
\begin{Highlighting}[]
\KeywordTok{missclassification}\NormalTok{(confusion.train)}
\end{Highlighting}
\end{Shaded}

\begin{verbatim}
## [1] 0.03437917
\end{verbatim}

\begin{Shaded}
\begin{Highlighting}[]
\CommentTok{# Test confusion matrix:}
\NormalTok{.predictions.test <-}\StringTok{ }\KeywordTok{predict}\NormalTok{(nnet}\FloatTok{.2}\NormalTok{, }\DataTypeTok{newdata =} \NormalTok{spam.test, }\DataTypeTok{type=}\StringTok{'class'}\NormalTok{) }
\NormalTok{(confusion.test <-}\StringTok{ }\KeywordTok{table}\NormalTok{(}\DataTypeTok{prediction=}\NormalTok{.predictions.test, }\DataTypeTok{truth=}\NormalTok{spam.test$spam))}
\end{Highlighting}
\end{Shaded}

\begin{verbatim}
##           truth
## prediction email spam
##      email   910   50
##      spam     47  598
\end{verbatim}

\begin{Shaded}
\begin{Highlighting}[]
\KeywordTok{missclassification}\NormalTok{(confusion.test)}
\end{Highlighting}
\end{Shaded}

\begin{verbatim}
## [1] 0.06043614
\end{verbatim}

\subsection{Classification and Regression Trees
(CART)}\label{classification-and-regression-trees-cart}

A CART, is not a linear model. It partitions the feature space
\(\mathcal{X}\), thus creating a set of if-then rules for prediction or
classification. This view clarifies the name of the function
\texttt{rpart}, which \emph{recursively partitions} the feature space.

We start with a regression tree.

\begin{Shaded}
\begin{Highlighting}[]
\KeywordTok{library}\NormalTok{(rpart)}
\NormalTok{tree}\FloatTok{.1} \NormalTok{<-}\StringTok{ }\KeywordTok{rpart}\NormalTok{(lcavol~., }\DataTypeTok{data=}\NormalTok{prostate.train)}

\CommentTok{# Train error:}
\KeywordTok{MSE}\NormalTok{( }\KeywordTok{predict}\NormalTok{(tree}\FloatTok{.1}\NormalTok{)-}\StringTok{ }\NormalTok{prostate.train$lcavol)}
\end{Highlighting}
\end{Shaded}

\begin{verbatim}
## [1] 0.4909568
\end{verbatim}

\begin{Shaded}
\begin{Highlighting}[]
\CommentTok{# Test error:}
\KeywordTok{MSE}\NormalTok{( }\KeywordTok{predict}\NormalTok{(tree}\FloatTok{.1}\NormalTok{, }\DataTypeTok{newdata =} \NormalTok{prostate.test)-}\StringTok{ }\NormalTok{prostate.test$lcavol)}
\end{Highlighting}
\end{Shaded}

\begin{verbatim}
## [1] 0.5623316
\end{verbatim}

{[}TODO: plot with \texttt{rpart.plot}{]}

Tree are very prone to overfitting. To avoid this, we reduce a tree's
complexity by \emph{pruning} it. This is done with the \texttt{prune}
function (not demonstrated herein).

We now fit a classification tree.

\begin{Shaded}
\begin{Highlighting}[]
\NormalTok{tree}\FloatTok{.2} \NormalTok{<-}\StringTok{ }\KeywordTok{rpart}\NormalTok{(spam~., }\DataTypeTok{data=}\NormalTok{spam.train)}

\CommentTok{# Train confusion matrix:}
\NormalTok{.predictions.train <-}\StringTok{ }\KeywordTok{predict}\NormalTok{(tree}\FloatTok{.2}\NormalTok{, }\DataTypeTok{type=}\StringTok{'class'}\NormalTok{) }
\NormalTok{(confusion.train <-}\StringTok{ }\KeywordTok{table}\NormalTok{(}\DataTypeTok{prediction=}\NormalTok{.predictions.train, }\DataTypeTok{truth=}\NormalTok{spam.train$spam))}
\end{Highlighting}
\end{Shaded}

\begin{verbatim}
##           truth
## prediction email spam
##      email  1730  181
##      spam    101  984
\end{verbatim}

\begin{Shaded}
\begin{Highlighting}[]
\KeywordTok{missclassification}\NormalTok{(confusion.train)}
\end{Highlighting}
\end{Shaded}

\begin{verbatim}
## [1] 0.0941255
\end{verbatim}

\begin{Shaded}
\begin{Highlighting}[]
\CommentTok{# Test confusion matrix:}
\NormalTok{.predictions.test <-}\StringTok{ }\KeywordTok{predict}\NormalTok{(tree}\FloatTok{.2}\NormalTok{, }\DataTypeTok{newdata =} \NormalTok{spam.test, }\DataTypeTok{type=}\StringTok{'class'}\NormalTok{) }
\NormalTok{(confusion.test <-}\StringTok{ }\KeywordTok{table}\NormalTok{(}\DataTypeTok{prediction=}\NormalTok{.predictions.test, }\DataTypeTok{truth=}\NormalTok{spam.test$spam))}
\end{Highlighting}
\end{Shaded}

\begin{verbatim}
##           truth
## prediction email spam
##      email   882   97
##      spam     75  551
\end{verbatim}

\begin{Shaded}
\begin{Highlighting}[]
\KeywordTok{missclassification}\NormalTok{(confusion.test)}
\end{Highlighting}
\end{Shaded}

\begin{verbatim}
## [1] 0.1071651
\end{verbatim}

\subsection{K-nearest neighbour (KNN)}\label{k-nearest-neighbour-knn}

KNN is not an ERM problem. For completeness, we still show how to fit
such a hypothesis class.

\begin{Shaded}
\begin{Highlighting}[]
\KeywordTok{library}\NormalTok{(class)}
\NormalTok{knn}\FloatTok{.1} \NormalTok{<-}\StringTok{ }\KeywordTok{knn}\NormalTok{(}\DataTypeTok{train =} \NormalTok{X.train.spam, }\DataTypeTok{test =} \NormalTok{X.test.spam, }\DataTypeTok{cl =}\NormalTok{y.train.spam, }\DataTypeTok{k =} \DecValTok{1}\NormalTok{)}

\CommentTok{# Test confusion matrix:}
\NormalTok{.predictions.test <-}\StringTok{ }\NormalTok{knn}\FloatTok{.1} 
\NormalTok{(confusion.test <-}\StringTok{ }\KeywordTok{table}\NormalTok{(}\DataTypeTok{prediction=}\NormalTok{.predictions.test, }\DataTypeTok{truth=}\NormalTok{spam.test$spam))}
\end{Highlighting}
\end{Shaded}

\begin{verbatim}
##           truth
## prediction email spam
##      email   807  140
##      spam    150  508
\end{verbatim}

\begin{Shaded}
\begin{Highlighting}[]
\KeywordTok{missclassification}\NormalTok{(confusion.test)}
\end{Highlighting}
\end{Shaded}

\begin{verbatim}
## [1] 0.1806854
\end{verbatim}

\subsection{Linear Discriminant Analysis
(LDA)}\label{linear-discriminant-analysis-lda}

LDA is equivalent to least squares classification \ref{least-squares}.
There are, however, some dedicated functions to fit it.

\begin{Shaded}
\begin{Highlighting}[]
\KeywordTok{library}\NormalTok{(MASS) }
\NormalTok{lda}\FloatTok{.1} \NormalTok{<-}\StringTok{ }\KeywordTok{lda}\NormalTok{(spam~., spam.train)}

\CommentTok{# Train confusion matrix:}
\NormalTok{.predictions.train <-}\StringTok{ }\KeywordTok{predict}\NormalTok{(lda}\FloatTok{.1}\NormalTok{)$class}
\NormalTok{(confusion.train <-}\StringTok{ }\KeywordTok{table}\NormalTok{(}\DataTypeTok{prediction=}\NormalTok{.predictions.train, }\DataTypeTok{truth=}\NormalTok{spam.train$spam))}
\end{Highlighting}
\end{Shaded}

\begin{verbatim}
##           truth
## prediction email spam
##      email  1752  247
##      spam     79  918
\end{verbatim}

\begin{Shaded}
\begin{Highlighting}[]
\KeywordTok{missclassification}\NormalTok{(confusion.train)}
\end{Highlighting}
\end{Shaded}

\begin{verbatim}
## [1] 0.1088117
\end{verbatim}

\begin{Shaded}
\begin{Highlighting}[]
\CommentTok{# Test confusion matrix:}
\NormalTok{.predictions.test <-}\StringTok{ }\KeywordTok{predict}\NormalTok{(lda}\FloatTok{.1}\NormalTok{, }\DataTypeTok{newdata =} \NormalTok{spam.test)$class}
\NormalTok{(confusion.test <-}\StringTok{ }\KeywordTok{table}\NormalTok{(}\DataTypeTok{prediction=}\NormalTok{.predictions.test, }\DataTypeTok{truth=}\NormalTok{spam.test$spam))}
\end{Highlighting}
\end{Shaded}

\begin{verbatim}
##           truth
## prediction email spam
##      email   915  134
##      spam     42  514
\end{verbatim}

\begin{Shaded}
\begin{Highlighting}[]
\KeywordTok{missclassification}\NormalTok{(confusion.test)}
\end{Highlighting}
\end{Shaded}

\begin{verbatim}
## [1] 0.1096573
\end{verbatim}

\subsection{Naive Bayes}\label{naive-bayes}

A Naive-Bayes classifier is also not part of the ERM framework. It is,
however, very popular, so we present it.

\begin{Shaded}
\begin{Highlighting}[]
\KeywordTok{library}\NormalTok{(e1071)}
\NormalTok{nb}\FloatTok{.1} \NormalTok{<-}\StringTok{ }\KeywordTok{naiveBayes}\NormalTok{(spam~., }\DataTypeTok{data =} \NormalTok{spam.train)}

\CommentTok{# Train confusion matrix:}
\NormalTok{.predictions.train <-}\StringTok{ }\KeywordTok{predict}\NormalTok{(nb}\FloatTok{.1}\NormalTok{, }\DataTypeTok{newdata =} \NormalTok{spam.train)}
\NormalTok{(confusion.train <-}\StringTok{ }\KeywordTok{table}\NormalTok{(}\DataTypeTok{prediction=}\NormalTok{.predictions.train, }\DataTypeTok{truth=}\NormalTok{spam.train$spam))}
\end{Highlighting}
\end{Shaded}

\begin{verbatim}
##           truth
## prediction email spam
##      email  1025   71
##      spam    806 1094
\end{verbatim}

\begin{Shaded}
\begin{Highlighting}[]
\KeywordTok{missclassification}\NormalTok{(confusion.train)}
\end{Highlighting}
\end{Shaded}

\begin{verbatim}
## [1] 0.2927236
\end{verbatim}

\begin{Shaded}
\begin{Highlighting}[]
\CommentTok{# Test confusion matrix:}
\NormalTok{.predictions.test <-}\StringTok{ }\KeywordTok{predict}\NormalTok{(nb}\FloatTok{.1}\NormalTok{, }\DataTypeTok{newdata =} \NormalTok{spam.test)}
\NormalTok{(confusion.test <-}\StringTok{ }\KeywordTok{table}\NormalTok{(}\DataTypeTok{prediction=}\NormalTok{.predictions.test, }\DataTypeTok{truth=}\NormalTok{spam.test$spam))}
\end{Highlighting}
\end{Shaded}

\begin{verbatim}
##           truth
## prediction email spam
##      email   566   30
##      spam    391  618
\end{verbatim}

\begin{Shaded}
\begin{Highlighting}[]
\KeywordTok{missclassification}\NormalTok{(confusion.test)}
\end{Highlighting}
\end{Shaded}

\begin{verbatim}
## [1] 0.2623053
\end{verbatim}

\section{Bibliographic Notes}\label{bibliographic-notes-6}

The ultimate reference on (statistical) machine learning is
\citet{friedman2001elements}. For a softer introduction, see
\citet{james2013introduction}. A statistician will also like
\citet{ripley2007pattern}. For an R oriented view see
\citet{lantz2013machine}. For a very algorithmic view, see the seminal
\citet{leskovec2014mining} or \citet{conway2012machine}. For a much more
theoretical reference, see \citet{mohri2012foundations},
\citet{vapnik2013nature}, \citet{shalev2014understanding}. Terminology
taken from \citet{sammut2011encyclopedia}. For a review of resampling
based unbiased risk estimation (i.e.~cross validation) see the
exceptional review of \citet{arlot2010survey}.

\section{Practice Yourself}\label{practice-yourself-7}

\chapter{Unsupervised Learning}\label{unsupervised}

This chapter deals with machine learning problems which are
unsupervised. This means the machine has access to a set of inputs,
\(x\), but the desired outcome, \(y\) is not available. Clearly,
learning a relation between inputs and outcomes is impossible, but there
are still a lot of problems of interest. In particular, we may want to
find a compact representation of the inputs, be it for visualization of
further processing. This is the problem of \emph{dimensionality
reduction}. For the same reasons we may want to group similar inputs.
This is the problem of \emph{clustering}.

In the statistical terminology, and with some exceptions, this chapter
can be thought of as multivariate \textbf{exploratory} statistics. For
multivariate \textbf{inference}, see Chapter \ref{multivariate}.

\section{Dimensionality Reduction}\label{dim-reduce}

\BeginKnitrBlock{example}
\protect\hypertarget{ex:bmi}{}{\label{ex:bmi}}Consider the heights and
weights of a sample of individuals. The data may seemingly reside in
\(2\) dimensions but given the height, we have a pretty good guess of a
persons weight, and vice versa. We can thus state that heights and
weights are not really two dimensional, but roughly lay on a \(1\)
dimensional subspace of \(\mathbb{R}^2\).
\EndKnitrBlock{example}

\BeginKnitrBlock{example}
\protect\hypertarget{ex:iq}{}{\label{ex:iq}}Consider the correctness of the
answers to a questionnaire with \(p\) questions. The data may seemingly
reside in a \(p\) dimensional space, but assuming there is a thing as
``skill'', then given the correctness of a person's reply to a subset of
questions, we have a good idea how he scores on the rest. Put
differently, we don't really need a \(200\) question questionnaire--
\(100\) is more than enough. If skill is indeed a one dimensional
quality, then the questionnaire data should organize around a single
line in the \(p\) dimensional cube.
\EndKnitrBlock{example}

\BeginKnitrBlock{example}
\protect\hypertarget{ex:blind-signal}{}{\label{ex:blind-signal}}Consider
\(n\) microphones recording an individual. The digitized recording
consists of \(p\) samples. Are the recordings really a shapeless cloud
of \(n\) points in \(\mathbb{R}^p\)? Since they all record the same
sound, one would expect the \(n\) \(p\)-dimensional points to arrange
around the source sound bit: a single point in \(\mathbb{R}^p\). If
microphones have different distances to the source, volumes may differ.
We would thus expect the \(n\) points to arrange about a \textbf{line}
in \(\mathbb{R}^p\).
\EndKnitrBlock{example}

\subsection{Principal Component Analysis}\label{pca}

\emph{Principal Component Analysis} (PCA) is such a basic technique, it
has been rediscovered and renamed independently in many fields. It can
be found under the names of Discrete Karhunen--Loève Transform;
Hotteling Transform; Proper Orthogonal Decomposition; Eckart--Young
Theorem; Schmidt--Mirsky Theorem; Empirical Orthogonal Functions;
Empirical Eigenfunction Decomposition; Empirical Component Analysis;
Quasi-Harmonic Modes; Spectral Decomposition; Empirical Modal Analysis,
and possibly more\footnote{\url{http://en.wikipedia.org/wiki/Principal_component_analysis}}.
The many names are quite interesting as they offer an insight into the
different problems that led to PCA's (re)discovery.

Return to the BMI problem in Example \ref{ex:bmi}. Assume you wish to
give each individual a ``size score'', that is a \textbf{linear}
combination of height and weight: PCA does just that. It returns the
linear combination that has the largest variability, i.e., the
combination which best distinguishes between individuals.

The variance maximizing motivation above was the one that guided
\citet{hotelling1933analysis}. But \(30\) years before him,
\citet{pearson1901liii} derived the same procedure with a different
motivation in mind. Pearson was also trying to give each individual a
score. He did not care about variance maximization, however. He simply
wanted a small set of coordinates in some (linear) space that
approximates the original data well.

Before we proceed, we give an example to fix ideas. Consider the crime
rate data in \texttt{USArrests}, which encodes reported murder events,
assaults, rapes, and the urban population of each american state.

\begin{Shaded}
\begin{Highlighting}[]
\KeywordTok{head}\NormalTok{(USArrests)}
\end{Highlighting}
\end{Shaded}

\begin{verbatim}
##            Murder Assault UrbanPop Rape
## Alabama      13.2     236       58 21.2
## Alaska       10.0     263       48 44.5
## Arizona       8.1     294       80 31.0
## Arkansas      8.8     190       50 19.5
## California    9.0     276       91 40.6
## Colorado      7.9     204       78 38.7
\end{verbatim}

Following Hotelling's motivation, we may want to give each state a
``crimilality score''. We first remove the \texttt{UrbanPop} variable,
which does not encode crime levels. We then z-score each variable with
\texttt{scale}, and call PCA for a sequence of \(1,\dots,3\) criminality
scores that best separate between states.

\begin{Shaded}
\begin{Highlighting}[]
\NormalTok{USArrests}\FloatTok{.1} \NormalTok{<-}\StringTok{ }\NormalTok{USArrests[,-}\DecValTok{3}\NormalTok{] %>%}\StringTok{ }\NormalTok{scale }
\NormalTok{pca}\FloatTok{.1} \NormalTok{<-}\StringTok{ }\KeywordTok{prcomp}\NormalTok{(USArrests}\FloatTok{.1}\NormalTok{, }\DataTypeTok{scale =} \OtherTok{TRUE}\NormalTok{)}
\NormalTok{pca}\FloatTok{.1}
\end{Highlighting}
\end{Shaded}

\begin{verbatim}
## Standard deviations:
## [1] 1.5357670 0.6767949 0.4282154
## 
## Rotation:
##                PC1        PC2        PC3
## Murder  -0.5826006  0.5339532 -0.6127565
## Assault -0.6079818  0.2140236  0.7645600
## Rape    -0.5393836 -0.8179779 -0.1999436
\end{verbatim}

Things to note:

\begin{itemize}
\tightlist
\item
  Distinguishing between states, i.e., finding the variance maximizing
  scores, should be indifferent to the \textbf{average} of each
  variable. We also don't want the score to be sensitive to the
  measurement \textbf{scale}. We thus perform PCA in the z-score scale
  of each variable, obtained with the \texttt{scale} function.
\item
  PCA is performed with the \texttt{prcomp} function. It returns the
  contribution (weight) of the original variables, to the new crimeness
  score.\\
  These weights are called the \emph{loadings} (or \texttt{Rotations} in
  the \texttt{prcomp} output, which is rather confusing as we will later
  see).
\item
  The number of possible scores, is the same as the number of original
  variables in the data.
\item
  The new scores are called the \emph{principal components}, labeled
  \texttt{PC1},\ldots{},\texttt{PC3} in our output.
\item
  The loadings on PC1 tell us that the best separation between states is
  along the average crime rate. Why is this? Because all the \(3\) crime
  variables have a similar loading on PC1.
\item
  The other PCs are slightly harder to interpret, but it is an
  interesting exercise.
\end{itemize}

\textbf{If we now represent each state, not with its original \(4\)
variables, but only with the first \(2\) PCs (for example), we have
reduced the dimensionality of the data.}

\subsection{Dimensionality Reduction
Preliminaries}\label{dimensionality-reduction-preliminaries}

Before presenting methods other than PCA, we need some terminology.

\begin{itemize}
\tightlist
\item
  \textbf{Variable}: A.k.a. \emph{dimension}, or \emph{feature}, or
  \emph{column}.
\item
  \textbf{Data}: A.k.a. \emph{sample}, \emph{observations}. Will
  typically consist of \(n\), vectors of dimension \(p\). We typically
  denote the data as a \(n\times p\) matrix \(X\).
\item
  \textbf{Manifold}: A generalization of a linear space, which is
  regular enough so that, \textbf{locally}, it has all the properties of
  a linear space. We will denote an arbitrary manifold by
  \(\mathcal{M}\), and by \(\mathcal{M}_q\) a \(q\)
  dimensional\footnote{You are probably used to thinking of the
    \textbf{dimension} of linear spaces. We will not rigorously define
    what is the dimension of a manifold, but you may think of it as the
    number of free coordinates needed to navigate along the manifold.}
  manifold.
\item
  \textbf{Embedding}: Informally speaking: a ``shape preserving''
  mapping of a space into another.
\item
  \textbf{Linear Embedding}: An embedding done via a linear operation
  (thus representable by a matrix).
\item
  \textbf{Generative Model}: Known to statisticians as the
  \textbf{sampling distribution}. The assumed stochastic process that
  generated the observed \(X\).
\end{itemize}

There are many motivations for dimensionality reduction:

\begin{enumerate}
\def\labelenumi{\arabic{enumi}.}
\tightlist
\item
  \textbf{Scoring}: Give each observation an interpretable, simple score
  (Hotelling's motivation).
\item
  \textbf{Latent structure}: Recover unobservable information from
  indirect measurements. E.g: Blind signal reconstruction, CT scan,
  cryo-electron microscopy, etc.
\item
  \textbf{Signal to Noise}: Denoise measurements before further
  processing like clustering, supervised learning, etc.
\item
  \textbf{Compression}: Save on RAM ,CPU, and communication when
  operating on a lower dimensional representation of the data.
\end{enumerate}

\subsection{Latent Variable Generative
Approaches}\label{latent-variable-generative-approaches}

All generative approaches to dimensionality reduction will include a set
of latent/unobservable variables, which we can try to recover from the
observables \(X\). The unobservable variables will typically have a
lower dimension than the observables, thus, dimension is reduced. We
start with the simplest case of linear Factor Analysis.

\subsubsection{Factor Analysis (FA)}\label{factor-analysis-fa}

FA originates from the psychometric literature. We thus revisit the IQ
(actually g-factor\footnote{\url{https://en.wikipedia.org/wiki/G_factor_(psychometrics)}})
Example \ref{ex:iq}:

\BeginKnitrBlock{example}
\protect\hypertarget{ex:unnamed-chunk-178}{}{\label{ex:unnamed-chunk-178}}Assume
\(n\) respondents answer \(p\) quantitative questions:
\(x_i \in \mathbb{R}^p, i=1,\dots,n\). Also assume, their responses are
some linear function of a single personality attribute, \(s_i\). We can
think of \(s_i\) as the subject's ``intelligence''. We thus have

\begin{align}
    x_i = A s_i + \varepsilon_i
\end{align}

And in matrix notation:

\begin{align}
    X = S A+\varepsilon,
    \label{eq:factor}
\end{align}

where \(A\) is the \(q \times p\) matrix of factor loadings, and \(S\)
the \(n \times q\) matrix of latent personality traits. In our
particular example where \(q=1\), the problem is to recover the
unobservable intelligence scores, \(s_1,\dots,s_n\), from the observed
answers \(X\).
\EndKnitrBlock{example}

We may try to estimate \(S A\) by assuming some distribution on \(S\)
and \(\varepsilon\) and apply maximum likelihood. Under standard
assumptions on the distribution of \(S\) and \(\varepsilon\), recovering
\(S\) from \(\widehat{S A }\) is still impossible as there are
infinitely many such solutions. In the statistical parlance we say the
problem is \emph{non identifiable}, and in the applied mathematics
parlance we say the problem is \emph{ill posed}. To see this, consider
an orthogonal \emph{rotation} matrix \(R\) (\(R' R=I\)). For each such
\(R\): \(S A = S R' R A = S^* A^*\). While both solve
Eq.\eqref{eq:factor}, \(A\) and \(A^*\) may have very different
interpretations. This is why many researchers find FA an unsatisfactory
inference tool.

\BeginKnitrBlock{remark}
\iffalse {Remark. } \fi The non-uniqueness (non-identifiability) of the
FA solution under variable rotation is never mentioned in the PCA
context. Why is this? This is because the methods solve different
problems. The reason the solution to PCA is well defined is that PCA
does not seek a single \(S\) but rather a \textbf{sequence} of \(S_q\)
with dimensions growing from \(q=1\) to \(q=p\).
\EndKnitrBlock{remark}

\BeginKnitrBlock{remark}
\iffalse {Remark. } \fi In classical FA in Eq.\eqref{eq:factor} is clearly
an embedding to a linear space: the one spanned by \(S\). Under the
classical probabilistic assumptions on \(S\) and \(\varepsilon\) the
embedding itself is also linear, and is sometimes solved with PCA. Being
a generative model, there is no restriction for the embedding to be
linear, and there certainly exists sets of assumptions for which the FA
returns a non linear embedding into a linear space.
\EndKnitrBlock{remark}

The FA terminology is slightly different than PCA:

\begin{itemize}
\tightlist
\item
  \textbf{Factors}: The unobserved attributes \(S\). Akin to the
  \emph{principal components} in PCA.
\item
  \textbf{Loading}: The \(A\) matrix; the contribution of each factor to
  the observed \(X\).
\item
  \textbf{Rotation}: An arbitrary orthogonal re-combination of the
  factors, \(S\), and loadings, \(A\), which changes the interpretation
  of the result.
\end{itemize}

The FA literature offers several heuristics to ``fix'' the
identifiability problem of FA. These are known as \emph{rotations}, and
go under the names of \emph{Varimax}, \emph{Quartimax}, \emph{Equimax},
\emph{Oblimin}, \emph{Promax}, and possibly others.

\subsubsection{Independent Component Analysis
(ICA)}\label{independent-component-analysis-ica}

Like FA, \emph{independent compoent analysis} (ICA) is a family of
latent space models, thus, a \emph{meta-method}. It assumes data is
generated as some function of the latent variables \(S\). In many cases
this function is assumed to be linear in \(S\) so that ICA is compared,
if not confused, with PCA and even more so with FA.

The fundamental idea of ICA is that \(S\) has a joint distribution of
\textbf{non-Gaussian}, \textbf{independent} variables. This independence
assumption, solves the the non-uniqueness of \(S\) in FA.

Being a generative model, estimation of \(S\) can then be done using
maximum likelihood, or other estimation principles.

ICA is a popular technique in signal processing, where \(A\) is actually
the signal, such as sound in Example \ref{ex:blind-signal}. Recovering
\(A\) is thus recovering the original signals mixing in the recorded
\(X\).

\subsection{Purely Algorithmic
Approaches}\label{purely-algorithmic-approaches}

We now discuss dimensionality reduction approaches that are not stated
via their generative model, but rather, directly as an algorithm. This
does not mean that they cannot be cast via their generative model, but
rather they were not motivated as such.

\subsubsection{Multidimensional Scaling
(MDS)}\label{multidimensional-scaling-mds}

MDS can be thought of as a variation on PCA, that begins with the
\(n \times n\) graph\footnote{The term Graph is typically used in this
  context instead of Network. But a graph allows only yes/no relations,
  while a network, which is a weighted graph, allows a continuous
  measure of similarity (or dissimilarity). \emph{Network} is thus more
  appropriate than \emph{graph}.}\} of distances between data points,
and not the original \(n \times p\) data.

MDS aims at embedding a graph of distances, while preserving the
original distances. Basic results in graph/network theory
\citep{graham1988isometric} suggest that the geometry of a graph cannot
be preserved when embedding it into lower dimensions. The different
types of MDSs, such as \emph{Classical MDS}, and \emph{Sammon Mappings},
differ in the \emph{stress function} penalizing for geometric
distortion.

\subsubsection{Local Multidimensional Scaling (Local
MDS)}\label{local-multidimensional-scaling-local-mds}

\BeginKnitrBlock{example}
\protect\hypertarget{ex:non-euclidean}{}{\label{ex:non-euclidean}}Consider
data of coordinates on the globe. At short distances, constructing a
dissimilarity graph with Euclidean distances will capture the true
distance between points. At long distances, however, the Euclidean
distances as grossly inappropriate. A more extreme example is
coordinates on the brain's cerebral cortex. Being a highly folded
surface, the Euclidean distance between points is far from the true
geodesic distances along the cortex's surface\footnote{Then again, it is
  possible that the true distances are the white matter fibers
  connecting going within the cortex, in which case, Euclidean distances
  are more appropriate than geodesic distances. We put that aside for
  now.}.
\EndKnitrBlock{example}

Local MDS is aimed at solving the case where we don't know how to
properly measure distances. It is an algorithm that compounds both the
construction of the dissimilarity graph, and the embedding. The solution
of local MDS, as the name suggests, rests on the computation of
\emph{local} distances, where the Euclidean assumption may still be
plausible, and then aggregate many such local distances, before calling
upon regular MDS for the embedding.

Because local MDS ends with a regular MDS, it can be seen as a
non-linear embedding into a linear \(\mathcal{M}\).

Local MDS is not popular. Why is this? Because it makes no sense: If we
believe the points reside in a non-Euclidean space, thus motivating the
use of geodesic distances, why would we want to wrap up with regular
MDS, which embeds in a linear space?! It does offer, however, some
intuition to the following, more popular, algorithms.

\subsubsection{Isometric Feature Mapping (IsoMap)}\label{isomap}

Like localMDS, only that the embedding, and not only the computation of
the distances, is local.

\subsubsection{Local Linear Embedding
(LLE)}\label{local-linear-embedding-lle}

Very similar to IsoMap \ref{isomap}.

\subsubsection{Kernel PCA}\label{kernel-pca}

TODO

\subsubsection{Simplified Component Technique LASSO
(SCoTLASS)}\label{simplified-component-technique-lasso-scotlass}

TODO

\subsubsection{Sparse Principal Component Analysis
(sPCA)}\label{sparse-principal-component-analysis-spca}

TODO

\subsubsection{Sparse kernel principal component analysis
(skPCA)}\label{sparse-kernel-principal-component-analysis-skpca}

TODO

\subsection{Dimensionality Reduction in
R}\label{dimensionality-reduction-in-r}

\subsubsection{PCA}\label{pca-in-r}

We already saw the basics of PCA in \ref{pca}. The fitting is done with
the \texttt{prcomp} function. The \emph{bi-plot} is a useful way to
visualize the output of PCA.

\begin{Shaded}
\begin{Highlighting}[]
\KeywordTok{library}\NormalTok{(devtools)}
\CommentTok{# install_github("vqv/ggbiplot")}
\NormalTok{ggbiplot::}\KeywordTok{ggbiplot}\NormalTok{(pca}\FloatTok{.1}\NormalTok{) }
\end{Highlighting}
\end{Shaded}

\includegraphics[width=0.5\linewidth]{Rcourse_files/figure-latex/unnamed-chunk-181-1}

Things to note:

\begin{itemize}
\tightlist
\item
  We used the \texttt{ggbiplot} function from the \textbf{ggbiplot}
  (available from github, but not from CRAN), because it has a nicer
  output than \texttt{stats::biplot}.
\item
  The bi-plot also plots the loadings as arrows. The coordinates of the
  arrows belong to the weight of each of the original variables in each
  PC. For example, the x-value of each arrow is the loadings on the
  first PC (on the x-axis). Since the weights of Murder, Assault, and
  Rape are almost the same, we conclude that PC1 captures the average
  crime rate in each state.
\item
  The bi-plot plots each data point along its PCs.
\end{itemize}

The \emph{scree plot} depicts the quality of the approximation of \(X\)
as \(q\) grows. This is depicted using the proportion of variability in
\(X\) that is removed by each added PC. It is customary to choose \(q\)
as the first PC that has a relative low contribution to the
approximation of \(X\).

\begin{Shaded}
\begin{Highlighting}[]
\NormalTok{ggbiplot::}\KeywordTok{ggscreeplot}\NormalTok{(pca}\FloatTok{.1}\NormalTok{)}
\end{Highlighting}
\end{Shaded}

\includegraphics[width=0.5\linewidth]{Rcourse_files/figure-latex/unnamed-chunk-182-1}

See how the first PC captures the variability in the Assault levels and
Murder levels, with a single score.

\includegraphics[width=0.5\linewidth]{Rcourse_files/figure-latex/unnamed-chunk-183-1}

More implementations of PCA:

\begin{Shaded}
\begin{Highlighting}[]
\CommentTok{# FAST solutions:}
\NormalTok{gmodels::}\KeywordTok{fast.prcomp}\NormalTok{()}

\CommentTok{# More detail in output:}
\NormalTok{FactoMineR::}\KeywordTok{PCA}\NormalTok{()}

\CommentTok{# For flexibility in algorithms and visualization:}
\NormalTok{ade4::}\KeywordTok{dudi.pca}\NormalTok{()}

\CommentTok{# Another one...}
\NormalTok{amap::}\KeywordTok{acp}\NormalTok{()}
\end{Highlighting}
\end{Shaded}

\subsubsection{FA}\label{fa}

\begin{Shaded}
\begin{Highlighting}[]
\NormalTok{fa}\FloatTok{.1} \NormalTok{<-}\StringTok{ }\NormalTok{psych::}\KeywordTok{principal}\NormalTok{(USArrests}\FloatTok{.1}\NormalTok{, }\DataTypeTok{nfactors =} \DecValTok{2}\NormalTok{, }\DataTypeTok{rotate =} \StringTok{"none"}\NormalTok{)}
\NormalTok{fa}\FloatTok{.1}
\end{Highlighting}
\end{Shaded}

\begin{verbatim}
## Principal Components Analysis
## Call: psych::principal(r = USArrests.1, nfactors = 2, rotate = "none")
## Standardized loadings (pattern matrix) based upon correlation matrix
##          PC1   PC2   h2     u2 com
## Murder  0.89 -0.36 0.93 0.0688 1.3
## Assault 0.93 -0.14 0.89 0.1072 1.0
## Rape    0.83  0.55 0.99 0.0073 1.7
## 
##                        PC1  PC2
## SS loadings           2.36 0.46
## Proportion Var        0.79 0.15
## Cumulative Var        0.79 0.94
## Proportion Explained  0.84 0.16
## Cumulative Proportion 0.84 1.00
## 
## Mean item complexity =  1.4
## Test of the hypothesis that 2 components are sufficient.
## 
## The root mean square of the residuals (RMSR) is  0.05 
##  with the empirical chi square  0.87  with prob <  NA 
## 
## Fit based upon off diagonal values = 0.99
\end{verbatim}

\begin{Shaded}
\begin{Highlighting}[]
\KeywordTok{biplot}\NormalTok{(fa}\FloatTok{.1}\NormalTok{, }\DataTypeTok{labels =}  \KeywordTok{rownames}\NormalTok{(USArrests}\FloatTok{.1}\NormalTok{)) }
\end{Highlighting}
\end{Shaded}

\includegraphics[width=0.5\linewidth]{Rcourse_files/figure-latex/FA-1}

\begin{Shaded}
\begin{Highlighting}[]
\CommentTok{# Numeric comparison with PCA:}
\NormalTok{fa}\FloatTok{.1}\NormalTok{$loadings}
\end{Highlighting}
\end{Shaded}

\begin{verbatim}
## 
## Loadings:
##         PC1    PC2   
## Murder   0.895 -0.361
## Assault  0.934 -0.145
## Rape     0.828  0.554
## 
##                  PC1   PC2
## SS loadings    2.359 0.458
## Proportion Var 0.786 0.153
## Cumulative Var 0.786 0.939
\end{verbatim}

\begin{Shaded}
\begin{Highlighting}[]
\NormalTok{pca}\FloatTok{.1}\NormalTok{$rotation}
\end{Highlighting}
\end{Shaded}

\begin{verbatim}
##                PC1        PC2        PC3
## Murder  -0.5826006  0.5339532 -0.6127565
## Assault -0.6079818  0.2140236  0.7645600
## Rape    -0.5393836 -0.8179779 -0.1999436
\end{verbatim}

Things to note:

\begin{itemize}
\tightlist
\item
  We perform FA with the \texttt{psych::principal} function. The
  \texttt{Principal\ Component\ Analysis} title is due to the fact that
  FA without rotations, is equivalent to PCA.
\item
  The first factor (\texttt{fa.1\$loadings}) has different weights than
  the first PC (\texttt{pca.1\$rotation}) because of normalization. They
  are the same, however, in that the first PC, and the first factor,
  capture average crime levels.
\end{itemize}

Graphical model fans will like the following plot, where the
contribution of each variable to each factor is encoded in the width of
the arrow.

\begin{Shaded}
\begin{Highlighting}[]
\NormalTok{qgraph::}\KeywordTok{qgraph}\NormalTok{(fa}\FloatTok{.1}\NormalTok{)}
\end{Highlighting}
\end{Shaded}

\includegraphics[width=0.5\linewidth]{Rcourse_files/figure-latex/unnamed-chunk-184-1}

Let's add a rotation (Varimax), and note that the rotation has indeed
changed the loadings of the variables, thus the interpretation of the
factors.

\begin{Shaded}
\begin{Highlighting}[]
\NormalTok{fa}\FloatTok{.2} \NormalTok{<-}\StringTok{ }\NormalTok{psych::}\KeywordTok{principal}\NormalTok{(USArrests}\FloatTok{.1}\NormalTok{, }\DataTypeTok{nfactors =} \DecValTok{2}\NormalTok{, }\DataTypeTok{rotate =} \StringTok{"varimax"}\NormalTok{)}

\NormalTok{fa}\FloatTok{.2}\NormalTok{$loadings}
\end{Highlighting}
\end{Shaded}

\begin{verbatim}
## 
## Loadings:
##         RC1   RC2  
## Murder  0.930 0.257
## Assault 0.829 0.453
## Rape    0.321 0.943
## 
##                  RC1   RC2
## SS loadings    1.656 1.160
## Proportion Var 0.552 0.387
## Cumulative Var 0.552 0.939
\end{verbatim}

Things to note:

\begin{itemize}
\tightlist
\item
  FA with a rotation is no longer equivalent to PCA.
\item
  The rotated factors are now called \emph{rotated componentes}, and
  reported in \texttt{RC1} and \texttt{RC2}.
\end{itemize}

\subsubsection{ICA}\label{ica}

\begin{Shaded}
\begin{Highlighting}[]
\NormalTok{ica}\FloatTok{.1} \NormalTok{<-}\StringTok{ }\NormalTok{fastICA::}\KeywordTok{fastICA}\NormalTok{(USArrests}\FloatTok{.1}\NormalTok{, }\DataTypeTok{n.com=}\DecValTok{2}\NormalTok{) }\CommentTok{# Also performs projection pursuit}

\KeywordTok{plot}\NormalTok{(ica}\FloatTok{.1}\NormalTok{$S)}
\KeywordTok{abline}\NormalTok{(}\DataTypeTok{h=}\DecValTok{0}\NormalTok{, }\DataTypeTok{v=}\DecValTok{0}\NormalTok{, }\DataTypeTok{lty=}\DecValTok{2}\NormalTok{)}
\KeywordTok{text}\NormalTok{(ica}\FloatTok{.1}\NormalTok{$S, }\DataTypeTok{pos =} \DecValTok{4}\NormalTok{, }\DataTypeTok{labels =} \KeywordTok{rownames}\NormalTok{(USArrests}\FloatTok{.1}\NormalTok{))}

\CommentTok{# Compare with PCA (first two PCs):}
\KeywordTok{arrows}\NormalTok{(}\DataTypeTok{x0 =} \NormalTok{ica}\FloatTok{.1}\NormalTok{$S[,}\DecValTok{1}\NormalTok{], }\DataTypeTok{y0 =} \NormalTok{ica}\FloatTok{.1}\NormalTok{$S[,}\DecValTok{2}\NormalTok{], }\DataTypeTok{x1 =} \NormalTok{pca}\FloatTok{.1}\NormalTok{$x[,}\DecValTok{2}\NormalTok{], }\DataTypeTok{y1 =} \NormalTok{pca}\FloatTok{.1}\NormalTok{$x[,}\DecValTok{1}\NormalTok{], }\DataTypeTok{col=}\StringTok{'red'}\NormalTok{, }\DataTypeTok{pch=}\DecValTok{19}\NormalTok{, }\DataTypeTok{cex=}\FloatTok{0.5}\NormalTok{)}
\end{Highlighting}
\end{Shaded}

\includegraphics[width=0.5\linewidth]{Rcourse_files/figure-latex/ICA-1}

Things to note:

\begin{itemize}
\tightlist
\item
  ICA is fitted with \texttt{fastICA::fastICA}.
\item
  The ICA components, like any other rotated components, are different
  than the PCA components.
\end{itemize}

\subsubsection{MDS}\label{mds}

Classical MDS, also compared with PCA.

\begin{Shaded}
\begin{Highlighting}[]
\CommentTok{# We first need a dissimarity matrix/graph:}
\NormalTok{state.disimilarity <-}\StringTok{ }\KeywordTok{dist}\NormalTok{(USArrests}\FloatTok{.1}\NormalTok{)}

\NormalTok{mds}\FloatTok{.1} \NormalTok{<-}\StringTok{ }\KeywordTok{cmdscale}\NormalTok{(state.disimilarity)}

\KeywordTok{plot}\NormalTok{(mds}\FloatTok{.1}\NormalTok{, }\DataTypeTok{pch =} \DecValTok{19}\NormalTok{)}
\KeywordTok{abline}\NormalTok{(}\DataTypeTok{h=}\DecValTok{0}\NormalTok{, }\DataTypeTok{v=}\DecValTok{0}\NormalTok{, }\DataTypeTok{lty=}\DecValTok{2}\NormalTok{)}
\NormalTok{USArrests}\FloatTok{.2} \NormalTok{<-}\StringTok{ }\NormalTok{USArrests[,}\DecValTok{1}\NormalTok{:}\DecValTok{2}\NormalTok{] %>%}\StringTok{  }\NormalTok{scale}
\KeywordTok{text}\NormalTok{(mds}\FloatTok{.1}\NormalTok{, }\DataTypeTok{pos =} \DecValTok{4}\NormalTok{, }\DataTypeTok{labels =} \KeywordTok{rownames}\NormalTok{(USArrests}\FloatTok{.2}\NormalTok{), }\DataTypeTok{col =} \StringTok{'tomato'}\NormalTok{)}

\CommentTok{# Compare with PCA (first two PCs):}
\KeywordTok{points}\NormalTok{(pca}\FloatTok{.1}\NormalTok{$x[,}\DecValTok{1}\NormalTok{:}\DecValTok{2}\NormalTok{], }\DataTypeTok{col=}\StringTok{'red'}\NormalTok{, }\DataTypeTok{pch=}\DecValTok{19}\NormalTok{, }\DataTypeTok{cex=}\FloatTok{0.5}\NormalTok{)}
\end{Highlighting}
\end{Shaded}

\includegraphics[width=0.5\linewidth]{Rcourse_files/figure-latex/MDS-1}

Things to note:

\begin{itemize}
\tightlist
\item
  We first compute a dissimilarity graph with \texttt{dist}. See the
  \texttt{cluster::daisy} function for more dissimilarity measures.
\item
  We learn the MDS embedding with \texttt{cmdscale}.
\item
  The embedding of PCA is the same as classical MDS with Euclidean
  distances.
\end{itemize}

Let's try other strain functions for MDS, like Sammon's strain, and
compare it with the PCs.

\begin{Shaded}
\begin{Highlighting}[]
\NormalTok{mds}\FloatTok{.2} \NormalTok{<-}\StringTok{ }\NormalTok{MASS::}\KeywordTok{sammon}\NormalTok{(state.disimilarity, }\DataTypeTok{trace =} \OtherTok{FALSE}\NormalTok{)}
\KeywordTok{plot}\NormalTok{(mds}\FloatTok{.2}\NormalTok{$points, }\DataTypeTok{pch =} \DecValTok{19}\NormalTok{)}
\KeywordTok{abline}\NormalTok{(}\DataTypeTok{h=}\DecValTok{0}\NormalTok{, }\DataTypeTok{v=}\DecValTok{0}\NormalTok{, }\DataTypeTok{lty=}\DecValTok{2}\NormalTok{)}
\KeywordTok{text}\NormalTok{(mds}\FloatTok{.2}\NormalTok{$points, }\DataTypeTok{pos =} \DecValTok{4}\NormalTok{, }\DataTypeTok{labels =} \KeywordTok{rownames}\NormalTok{(USArrests}\FloatTok{.2}\NormalTok{))}

\CommentTok{# Compare with PCA (first two PCs):}
\KeywordTok{arrows}\NormalTok{(}
  \DataTypeTok{x0 =} \NormalTok{mds}\FloatTok{.2}\NormalTok{$points[,}\DecValTok{1}\NormalTok{], }\DataTypeTok{y0 =} \NormalTok{mds}\FloatTok{.2}\NormalTok{$points[,}\DecValTok{2}\NormalTok{], }
  \DataTypeTok{x1 =} \NormalTok{pca}\FloatTok{.1}\NormalTok{$x[,}\DecValTok{1}\NormalTok{], }\DataTypeTok{y1 =} \NormalTok{pca}\FloatTok{.1}\NormalTok{$x[,}\DecValTok{2}\NormalTok{], }
  \DataTypeTok{col=}\StringTok{'red'}\NormalTok{, }\DataTypeTok{pch=}\DecValTok{19}\NormalTok{, }\DataTypeTok{cex=}\FloatTok{0.5}\NormalTok{)}
\end{Highlighting}
\end{Shaded}

\includegraphics[width=0.5\linewidth]{Rcourse_files/figure-latex/SammonMDS-1}

Things to note:

\begin{itemize}
\tightlist
\item
  \texttt{MASS::sammon} does the embedding.
\item
  Sammon strain is different than PCA.
\end{itemize}

\subsubsection{Sparse PCA}\label{sparse-pca}

\begin{Shaded}
\begin{Highlighting}[]
\CommentTok{# Compute similarity graph}
\NormalTok{state.similarity <-}\StringTok{ }\NormalTok{MASS::}\KeywordTok{cov.rob}\NormalTok{(USArrests}\FloatTok{.1}\NormalTok{)$cov}

\NormalTok{spca1 <-}\StringTok{ }\NormalTok{elasticnet::}\KeywordTok{spca}\NormalTok{(state.similarity, }\DataTypeTok{K=}\DecValTok{2}\NormalTok{, }\DataTypeTok{type=}\StringTok{"Gram"}\NormalTok{, }\DataTypeTok{sparse=}\StringTok{"penalty"}\NormalTok{, }\DataTypeTok{trace=}\OtherTok{FALSE}\NormalTok{, }\DataTypeTok{para=}\KeywordTok{c}\NormalTok{(}\FloatTok{0.06}\NormalTok{,}\FloatTok{0.16}\NormalTok{))}
\NormalTok{spca1$loadings}
\end{Highlighting}
\end{Shaded}

\begin{verbatim}
##                 PC1 PC2
## Murder  -0.79639278   0
## Assault -0.60027920   0
## Rape    -0.07364392  -1
\end{verbatim}

\subsubsection{Kernel PCA}\label{kernel-pca-1}

\begin{Shaded}
\begin{Highlighting}[]
\NormalTok{kernlab::}\KeywordTok{kpca}\NormalTok{()}
\end{Highlighting}
\end{Shaded}

\section{Clustering}\label{cluster}

\BeginKnitrBlock{example}
\protect\hypertarget{ex:photos}{}{\label{ex:photos}}Consider the tagging of
your friends' pictures on Facebook. If you tagged some pictures,
Facebook may try to use a supervised approach to automatically label
photos. If you never tagged pictures, a supervised approach is
impossible. It is still possible, however, to group simiar pictures
together.
\EndKnitrBlock{example}

\BeginKnitrBlock{example}
\protect\hypertarget{ex:spam}{}{\label{ex:spam}}Consider the problem of spam
detection. It would be nice if each user could label several thousands
emails, to apply a supervised learning approach to spam detection. This
is an unrealistic demand, so a pre-clustering stage is useful: the user
only needs to tag a couple dozens of homogenous clusters, before solving
the supervised learning problem.
\EndKnitrBlock{example}

In clustering problems, we seek to group observations that are similar.

There are many motivations for clustering:

\begin{enumerate}
\def\labelenumi{\arabic{enumi}.}
\tightlist
\item
  \textbf{Understanding}: The most common use of clustering is probably
  as a an exploratory step, to identify homogeneous groups in the data.
\item
  \textbf{Dimensionality reduction}: Clustering may be seen as a method
  for dimensionality reduction. Unlike the approaches in the
  Dimensionality Reduction Section \ref{dim-reduce}, it does not
  compress \textbf{variables} but rather \textbf{observations}. Each
  group of homogeneous observations may then be represented as a single
  prototypical observation of the group.
\item
  \textbf{Pre-Labelling}: Clustering may be performed as a
  pre-processing step for supervised learning, when labeling all the
  samples is impossible due to ``budget'' constraints, like in Example
  \ref{ex:spam}. This is sometimes known as \emph{pre-clustering}.
\end{enumerate}

Clustering, like dimensionality reduction, may rely on some latent
variable generative model, or on purely algorithmic approaches.

\subsection{Latent Variable Generative
Approaches}\label{latent-variable-generative-approaches-1}

\subsubsection{Finite Mixture}\label{finite-mixture}

\BeginKnitrBlock{example}
\protect\hypertarget{ex:males-females}{}{\label{ex:males-females}}Consider
the distribution of heights. Heights have a nice bell shaped
distribution within each gender. If genders have not been recorded,
heights will be distributed like a \emph{mixture} of males and females.
The gender in this example, is a \emph{latent} variable taking \(K=2\)
levels: male and female.
\EndKnitrBlock{example}

A \emph{finite mixture} is the marginal distribution of \(K\) distinct
classes, when the class variable is \emph{latent}. This is useful for
clustering: We can assume the number of classes, \(K\), and the
distribution of each class. We then use maximum likelihood to fit the
mixture distribution, and finally, cluster by assigning observations to
the most probable class.

\subsection{Purely Algorithmic
Approaches}\label{purely-algorithmic-approaches-1}

\subsubsection{K-Means}\label{k-means}

The \emph{K-means} algorithm is possibly the most popular clustering
algorithm. The goal behind K-means clustering is finding a
representative point for each of K clusters, and assign each data point
to one of these clusters. As each cluster has a representative point,
this is also a \emph{prototype method}. The clusters are defined so that
they minimize the average Euclidean distance between all points to the
center of the cluster.

In K-means, the clusters are first defined, and then similarities
computed. This is thus a \emph{top-down} method.

K-means clustering requires the raw features \(X\) as inputs, and not
only a similarity graph. This is evident when examining the algorithm
below.

The k-means algorithm works as follows:

\begin{enumerate}
\def\labelenumi{\arabic{enumi}.}
\tightlist
\item
  Choose the number of clusters \(K\).
\item
  Arbitrarily assign points to clusters.
\item
  While clusters keep changing:

  \begin{enumerate}
  \def\labelenumii{\arabic{enumii}.}
  \tightlist
  \item
    Compute the cluster centers as the average of their points.
  \item
    Assign each point to its closest cluster center (in Euclidean
    distance).
  \end{enumerate}
\item
  Return Cluster assignments and means.
\end{enumerate}

\BeginKnitrBlock{remark}
\iffalse {Remark. } \fi If trained as a statistician, you may wonder-
what population quantity is K-means actually estimating? The estimand of
K-means is known as the \emph{K principal points}. Principal points are
points which are \emph{self consistent}, i.e., they are the mean of
their neighbourhood.
\EndKnitrBlock{remark}

\subsubsection{K-Means++}\label{k-means-1}

\emph{K-means++} is a fast version of K-means thanks to a smart
initialization.

\subsubsection{K-Medoids}\label{k-medoids}

If a Euclidean distance is inappropriate for a particular set of
variables, or that robustness to corrupt observations is required, or
that we wish to constrain the cluster centers to be actual observations,
then the \emph{K-Medoids} algorithm is an adaptation of K-means that
allows this. It is also known under the name \emph{partition around
medoids} (PAM) clustering, suggesting its relation to
\href{https://en.wikipedia.org/wiki/Graph_partition}{graph
partitioning}.

The k-medoids algorithm works as follows.

\begin{enumerate}
\def\labelenumi{\arabic{enumi}.}
\tightlist
\item
  Given a dissimilarity graph.
\item
  Choose the number of clusters \(K\).
\item
  Arbitrarily assign points to clusters.
\item
  While clusters keep changing:

  \begin{enumerate}
  \def\labelenumii{\arabic{enumii}.}
  \tightlist
  \item
    Within each cluster, set the center as the data point that minimizes
    the sum of distances to other points in the cluster.
  \item
    Assign each point to its closest cluster center.
  \end{enumerate}
\item
  Return Cluster assignments and centers.
\end{enumerate}

\BeginKnitrBlock{remark}
\iffalse {Remark. } \fi If trained as a statistician, you may wonder-
what population quantity is K-medoids actually estimating? The estimand
of K-medoids is the median of their neighbourhood. A delicate matter is
that quantiles are not easy to define for \textbf{multivariate}
variables so that the ``multivaraitre median'', may be a more subtle
quantity than you may think. See \citet{small1990survey}.
\EndKnitrBlock{remark}

\subsubsection{Hirarchial Clustering}\label{hirarchial-clustering}

Hierarchical clustering algorithms take dissimilarity graphs as inputs.
Hierarchical clustering is a class of greedy \emph{graph-partitioning}
algorithms. Being hierarchical by design, they have the attractive
property that the evolution of the clustering can be presented with a
\emph{dendogram}, i.e., a tree plot.\\
A particular advantage of these methods is that they do not require an
a-priori choice of the number of cluster (\(K\)).

Two main sub-classes of algorithms are \emph{agglomerative}, and
\emph{divisive}.

\emph{Agglomerative clustering} algorithms are \textbf{bottom-up}
algorithm which build clusters by joining smaller clusters. To decide
which clusters are joined at each iteration some measure of closeness
between clusters is required.

\begin{itemize}
\tightlist
\item
  \textbf{Single Linkage}: Cluster distance is defined by the distance
  between the two \textbf{closest} members.
\item
  \textbf{Complete Linkage}: Cluster distance is defined by the distance
  between the two \textbf{farthest} members.
\item
  \textbf{Group Average}: Cluster distance is defined by the
  \textbf{average} distance between members.
\item
  \textbf{Group Median}: Like Group Average, only using the median.
\end{itemize}

\emph{Divisive clustering} algorithms are \textbf{top-down} algorithm
which build clusters by splitting larger clusters.

\subsubsection{Fuzzy Clustering}\label{fuzzy-clustering}

Can be thought of as a purely algorithmic view of the finite-mixture in
Section \ref{finite-mixture}.

\subsection{Clustering in R}\label{clustering-in-r}

\subsubsection{K-Means}\label{k-means-2}

The following code is an adaptation from
\href{http://people.stat.sc.edu/Hitchcock/chapter6_R_examples.txt}{David
Hitchcock}.

\begin{Shaded}
\begin{Highlighting}[]
\NormalTok{k <-}\StringTok{ }\DecValTok{2}
\NormalTok{kmeans}\FloatTok{.1} \NormalTok{<-}\StringTok{ }\NormalTok{stats::}\KeywordTok{kmeans}\NormalTok{(USArrests}\FloatTok{.1}\NormalTok{, }\DataTypeTok{centers =} \NormalTok{k)}
\KeywordTok{head}\NormalTok{(kmeans}\FloatTok{.1}\NormalTok{$cluster) }\CommentTok{# cluster asignments}
\end{Highlighting}
\end{Shaded}

\begin{verbatim}
##    Alabama     Alaska    Arizona   Arkansas California   Colorado 
##          2          2          2          1          2          2
\end{verbatim}

\begin{Shaded}
\begin{Highlighting}[]
\KeywordTok{pairs}\NormalTok{(USArrests}\FloatTok{.1}\NormalTok{, }\DataTypeTok{panel=}\NormalTok{function(x,y) }\KeywordTok{text}\NormalTok{(x,y,kmeans}\FloatTok{.1}\NormalTok{$cluster))}
\end{Highlighting}
\end{Shaded}

\includegraphics[width=0.5\linewidth]{Rcourse_files/figure-latex/kmeans-1}

Things to note:

\begin{itemize}
\tightlist
\item
  The \texttt{stats::kmeans} function does the clustering.
\item
  The cluster assignment is given in the \texttt{cluster} element of the
  \texttt{stats::kmeans} output.
\item
  The visual inspection confirms that similar states have been assigned
  to the same cluster.
\end{itemize}

\subsubsection{K-Means ++}\label{k-means-3}

\emph{K-Means++} is a smart initialization for K-Means. The following
code is taken from the
\href{https://stat.ethz.ch/pipermail/r-help/2012-January/300051.html}{r-help}
mailing list.

\begin{Shaded}
\begin{Highlighting}[]
\CommentTok{# Write my own K-means++ function.}
\NormalTok{kmpp <-}\StringTok{ }\NormalTok{function(X, k) \{}
  
  \NormalTok{n <-}\StringTok{ }\KeywordTok{nrow}\NormalTok{(X)}
  \NormalTok{C <-}\StringTok{ }\KeywordTok{numeric}\NormalTok{(k)}
  \NormalTok{C[}\DecValTok{1}\NormalTok{] <-}\StringTok{ }\KeywordTok{sample}\NormalTok{(}\DecValTok{1}\NormalTok{:n, }\DecValTok{1}\NormalTok{)}
  
  \NormalTok{for (i in }\DecValTok{2}\NormalTok{:k) \{}
    \NormalTok{dm <-}\StringTok{ }\NormalTok{pracma::}\KeywordTok{distmat}\NormalTok{(X, X[C, ])}
    \NormalTok{pr <-}\StringTok{ }\KeywordTok{apply}\NormalTok{(dm, }\DecValTok{1}\NormalTok{, min); pr[C] <-}\StringTok{ }\DecValTok{0}
    \NormalTok{C[i] <-}\StringTok{ }\KeywordTok{sample}\NormalTok{(}\DecValTok{1}\NormalTok{:n, }\DecValTok{1}\NormalTok{, }\DataTypeTok{prob =} \NormalTok{pr)}
  \NormalTok{\}}
  
  \KeywordTok{kmeans}\NormalTok{(X, X[C, ])}
\NormalTok{\}}

\NormalTok{kmeans}\FloatTok{.2} \NormalTok{<-}\StringTok{ }\KeywordTok{kmpp}\NormalTok{(USArrests}\FloatTok{.1}\NormalTok{, k)}
\KeywordTok{head}\NormalTok{(kmeans}\FloatTok{.2}\NormalTok{$cluster)}
\end{Highlighting}
\end{Shaded}

\begin{verbatim}
##    Alabama     Alaska    Arizona   Arkansas California   Colorado 
##          1          1          1          2          1          1
\end{verbatim}

\subsubsection{K-Medoids}\label{k-medoids-1}

Start by growing a distance graph with \texttt{dist} and then partition
using \texttt{pam}.

\begin{Shaded}
\begin{Highlighting}[]
\NormalTok{state.disimilarity <-}\StringTok{ }\KeywordTok{dist}\NormalTok{(USArrests}\FloatTok{.1}\NormalTok{)}
\NormalTok{kmed}\FloatTok{.1} \NormalTok{<-}\StringTok{ }\NormalTok{cluster::}\KeywordTok{pam}\NormalTok{(}\DataTypeTok{x=} \NormalTok{state.disimilarity, }\DataTypeTok{k=}\DecValTok{2}\NormalTok{)}
\KeywordTok{head}\NormalTok{(kmed}\FloatTok{.1}\NormalTok{$clustering)}
\end{Highlighting}
\end{Shaded}

\begin{verbatim}
##    Alabama     Alaska    Arizona   Arkansas California   Colorado 
##          1          1          1          1          1          1
\end{verbatim}

\begin{Shaded}
\begin{Highlighting}[]
\KeywordTok{plot}\NormalTok{(pca}\FloatTok{.1}\NormalTok{$x[,}\DecValTok{1}\NormalTok{], pca}\FloatTok{.1}\NormalTok{$x[,}\DecValTok{2}\NormalTok{], }\DataTypeTok{xlab=}\StringTok{"PC 1"}\NormalTok{, }\DataTypeTok{ylab=}\StringTok{"PC 2"}\NormalTok{, }\DataTypeTok{type =}\StringTok{'n'}\NormalTok{, }\DataTypeTok{lwd=}\DecValTok{2}\NormalTok{)}
\KeywordTok{text}\NormalTok{(pca}\FloatTok{.1}\NormalTok{$x[,}\DecValTok{1}\NormalTok{], pca}\FloatTok{.1}\NormalTok{$x[,}\DecValTok{2}\NormalTok{], }\DataTypeTok{labels=}\KeywordTok{rownames}\NormalTok{(USArrests}\FloatTok{.1}\NormalTok{), }\DataTypeTok{cex=}\FloatTok{0.7}\NormalTok{, }\DataTypeTok{lwd=}\DecValTok{2}\NormalTok{, }\DataTypeTok{col=}\NormalTok{kmed}\FloatTok{.1}\NormalTok{$cluster)}
\end{Highlighting}
\end{Shaded}

\includegraphics[width=0.5\linewidth]{Rcourse_files/figure-latex/kmedoids-1}

Things to note:

\begin{itemize}
\tightlist
\item
  K-medoids starts with the computation of a dissimilarity graph, done
  by the \texttt{dist} function.
\item
  The clustering is done by the \texttt{cluster::pam} function.
\item
  Inspecting the output confirms that similar states have been assigned
  to the same cluster.
\item
  Many other similarity measures can be found in \texttt{proxy::dist()}.
\item
  See \texttt{cluster::clara()} for a big-data implementation of PAM.
\end{itemize}

\subsubsection{Hirarchial Clustering}\label{hirarchial-clustering-1}

We start with agglomerative clustering with single-linkage.

\begin{Shaded}
\begin{Highlighting}[]
\NormalTok{hirar}\FloatTok{.1} \NormalTok{<-}\StringTok{ }\KeywordTok{hclust}\NormalTok{(state.disimilarity, }\DataTypeTok{method=}\StringTok{'single'}\NormalTok{)}
\KeywordTok{plot}\NormalTok{(hirar}\FloatTok{.1}\NormalTok{, }\DataTypeTok{labels=}\KeywordTok{rownames}\NormalTok{(USArrests}\FloatTok{.1}\NormalTok{), }\DataTypeTok{ylab=}\StringTok{"Distance"}\NormalTok{)}
\end{Highlighting}
\end{Shaded}

\includegraphics[width=0.5\linewidth]{Rcourse_files/figure-latex/HirarchialClustering-1}

Things to note:

\begin{itemize}
\tightlist
\item
  The clustering is done with the \texttt{hclust} function.
\item
  We choose the single-linkage distance using the
  \texttt{method=\textquotesingle{}single\textquotesingle{}} argument.
\item
  We did not need to a-priori specify the number of clusters, \(K\),
  since all the possible \(K\)'s are included in the output tree.
\item
  The \texttt{plot} function has a particular method for \texttt{hclust}
  class objects, and plots them as dendograms.
\end{itemize}

We try other types of linkages, to verify that the indeed affect the
clustering. Starting with complete linkage.

\begin{Shaded}
\begin{Highlighting}[]
\NormalTok{hirar}\FloatTok{.2} \NormalTok{<-}\StringTok{ }\KeywordTok{hclust}\NormalTok{(state.disimilarity, }\DataTypeTok{method=}\StringTok{'complete'}\NormalTok{)}
\KeywordTok{plot}\NormalTok{(hirar}\FloatTok{.2}\NormalTok{, }\DataTypeTok{labels=}\KeywordTok{rownames}\NormalTok{(USArrests}\FloatTok{.1}\NormalTok{), }\DataTypeTok{ylab=}\StringTok{"Distance"}\NormalTok{)}
\end{Highlighting}
\end{Shaded}

\includegraphics[width=0.5\linewidth]{Rcourse_files/figure-latex/complete linkage-1}

Now with average linkage.

\begin{Shaded}
\begin{Highlighting}[]
\NormalTok{hirar}\FloatTok{.3} \NormalTok{<-}\StringTok{ }\KeywordTok{hclust}\NormalTok{(state.disimilarity, }\DataTypeTok{method=}\StringTok{'average'}\NormalTok{)}
\KeywordTok{plot}\NormalTok{(hirar}\FloatTok{.3}\NormalTok{, }\DataTypeTok{labels=}\KeywordTok{rownames}\NormalTok{(USArrests}\FloatTok{.1}\NormalTok{), }\DataTypeTok{ylab=}\StringTok{"Distance"}\NormalTok{)}
\end{Highlighting}
\end{Shaded}

\includegraphics[width=0.5\linewidth]{Rcourse_files/figure-latex/average linkage-1}

If we know how many clusters we want, we can use \texttt{cuttree} to get
the class assignments.

\begin{Shaded}
\begin{Highlighting}[]
\NormalTok{cut}\FloatTok{.2.2} \NormalTok{<-}\StringTok{ }\KeywordTok{cutree}\NormalTok{(hirar}\FloatTok{.2}\NormalTok{, }\DataTypeTok{k=}\DecValTok{2}\NormalTok{)}
\KeywordTok{head}\NormalTok{(cut}\FloatTok{.2.2}\NormalTok{)}
\end{Highlighting}
\end{Shaded}

\begin{verbatim}
##    Alabama     Alaska    Arizona   Arkansas California   Colorado 
##          1          1          1          2          1          1
\end{verbatim}

\section{Bibliographic Notes}\label{bibliographic-notes-7}

For more on PCA see my
\href{https://github.com/johnros/dim_reduce/blob/master/dim_reduce.pdf}{Dimensionality
Reduction Class Notes} and references therein. For more on everything,
see \citet{friedman2001elements}. For a softer introduction, see
\citet{james2013introduction}.

\section{Practice Yourself}\label{practice-yourself-8}

\chapter{Plotting}\label{plotting}

Whether you are doing EDA, or preparing your results for publication,
you need plots. R has many plotting mechanisms, allowing the user a
tremendous amount of flexibility, while abstracting away a lot of the
tedious details. To be concrete, many of the plots in R are simply
impossible to produce with Excel, SPSS, or SAS, and would take a
tremendous amount of work to produce with Python, Java and lower level
programming languages.

In this text, we will focus on two plotting packages. The basic
\textbf{graphics} package, distributed with the base R distribution, and
the \textbf{ggplot2} package.

Before going into the details of the plotting packages, we start with
some high-level philosophy. The \textbf{graphics} package originates
from the main-frame days. Computers had no graphical interface, and the
output of the plot was immediately sent to a printer. Once a plot has
been produced with the \textbf{graphics} package, just like a printed
output, it cannot be queried nor changed, except for further additions.

The philosophy of R is that \textbf{everyting is an object}. The
\textbf{graphics} package does not adhere to this philosophy, and indeed
it was soon augmented with the \textbf{grid} package \citep{Rlanguage},
that treats plots as objects. \textbf{grid} is a low level graphics
interface, and users may be more familiar with the \textbf{lattice}
package built upon it \citep{lattice}.

\textbf{lattice} is very powerful, but soon enough, it was overtaken in
popularity by the \textbf{ggplot2} package \citep{ggplot2}.
\textbf{ggplot2} was the PhD project of \href{http://hadley.nz/}{Hadley
Wickham}, a name to remember\ldots{} Two fundamental ideas underlay
\textbf{ggplot2}: (i) everything is an object, and (ii), plots can be
described by a small set of building blocks. The building blocks in
\textbf{ggplot2} are the ones stated by \citet{wilkinson2006grammar}.
The objects and grammar of \textbf{ggplot2} have later evolved to allow
more complicated plotting and in particular, interactive plotting.

Interactive plotting is a very important feature for EDA, and reporting.
The major leap in interactive plotting was made possible by the
advancement of web technologies, such as JavaScript. Why is this?
Because an interactive plot, or report, can be seen as a web-site.
Building upon the capabilities of JavaScript and your web browser to
provide the interactivity, greatly facilitates the development of such
plots, as the programmer can rely on the web-browsers capabilities for
interactivity.

\section{The graphics System}\label{the-graphics-system}

The R code from the Basics Chapter \ref{basics} is a demonstration of
the \textbf{graphics} package and system. We make a quick review of the
basics.

\subsection{Using Existing Plotting
Functions}\label{using-existing-plotting-functions}

\subsubsection{Scatter Plot}\label{scatter-plot}

A simple scatter plot.

\begin{Shaded}
\begin{Highlighting}[]
\KeywordTok{attach}\NormalTok{(trees)}
\KeywordTok{plot}\NormalTok{(Girth ~}\StringTok{ }\NormalTok{Height)}
\end{Highlighting}
\end{Shaded}

\includegraphics[width=0.5\linewidth]{Rcourse_files/figure-latex/unnamed-chunk-188-1}

Various types of plots.

\begin{Shaded}
\begin{Highlighting}[]
\NormalTok{par.old <-}\StringTok{ }\KeywordTok{par}\NormalTok{(}\DataTypeTok{no.readonly =} \OtherTok{TRUE}\NormalTok{)}
\KeywordTok{par}\NormalTok{(}\DataTypeTok{mfrow=}\KeywordTok{c}\NormalTok{(}\DecValTok{2}\NormalTok{,}\DecValTok{3}\NormalTok{))}
\KeywordTok{plot}\NormalTok{(Girth, }\DataTypeTok{type=}\StringTok{'h'}\NormalTok{, }\DataTypeTok{main=}\StringTok{"type='h'"}\NormalTok{) }
\KeywordTok{plot}\NormalTok{(Girth, }\DataTypeTok{type=}\StringTok{'o'}\NormalTok{, }\DataTypeTok{main=}\StringTok{"type='o'"}\NormalTok{) }
\KeywordTok{plot}\NormalTok{(Girth, }\DataTypeTok{type=}\StringTok{'l'}\NormalTok{, }\DataTypeTok{main=}\StringTok{"type='l'"}\NormalTok{)}
\KeywordTok{plot}\NormalTok{(Girth, }\DataTypeTok{type=}\StringTok{'s'}\NormalTok{, }\DataTypeTok{main=}\StringTok{"type='s'"}\NormalTok{)}
\KeywordTok{plot}\NormalTok{(Girth, }\DataTypeTok{type=}\StringTok{'b'}\NormalTok{, }\DataTypeTok{main=}\StringTok{"type='b'"}\NormalTok{)}
\KeywordTok{plot}\NormalTok{(Girth, }\DataTypeTok{type=}\StringTok{'p'}\NormalTok{, }\DataTypeTok{main=}\StringTok{"type='p'"}\NormalTok{)}
\end{Highlighting}
\end{Shaded}

\includegraphics[width=0.5\linewidth]{Rcourse_files/figure-latex/unnamed-chunk-189-1}

\begin{Shaded}
\begin{Highlighting}[]
\KeywordTok{par}\NormalTok{(par.old)}
\end{Highlighting}
\end{Shaded}

Things to note:

\begin{itemize}
\tightlist
\item
  The \texttt{par} command controls the plotting parameters.
  \texttt{mfrow=c(2,3)} is used to produce a matrix of plots with 2 rows
  and 3 columns.
\item
  The \texttt{par.old} object saves the original plotting setting. It is
  restored after plotting using \texttt{par(par.old)}.
\item
  The \texttt{type} argument controls the type of plot.
\item
  The \texttt{main} argument controls the title.
\item
  See \texttt{?plot} and \texttt{?par} for more options.
\end{itemize}

Control the plotting characters with the \texttt{pch} argument.

\begin{Shaded}
\begin{Highlighting}[]
\KeywordTok{plot}\NormalTok{(Girth, }\DataTypeTok{pch=}\StringTok{'+'}\NormalTok{, }\DataTypeTok{cex=}\DecValTok{3}\NormalTok{)}
\end{Highlighting}
\end{Shaded}

\includegraphics[width=0.5\linewidth]{Rcourse_files/figure-latex/unnamed-chunk-190-1}

Control the line's type with \texttt{lty} argument, and width with
\texttt{lwd}.

\begin{Shaded}
\begin{Highlighting}[]
\KeywordTok{par}\NormalTok{(}\DataTypeTok{mfrow=}\KeywordTok{c}\NormalTok{(}\DecValTok{2}\NormalTok{,}\DecValTok{3}\NormalTok{))}
\KeywordTok{plot}\NormalTok{(Girth, }\DataTypeTok{type=}\StringTok{'l'}\NormalTok{, }\DataTypeTok{lty=}\DecValTok{1}\NormalTok{, }\DataTypeTok{lwd=}\DecValTok{2}\NormalTok{)}
\KeywordTok{plot}\NormalTok{(Girth, }\DataTypeTok{type=}\StringTok{'l'}\NormalTok{, }\DataTypeTok{lty=}\DecValTok{2}\NormalTok{, }\DataTypeTok{lwd=}\DecValTok{2}\NormalTok{)}
\KeywordTok{plot}\NormalTok{(Girth, }\DataTypeTok{type=}\StringTok{'l'}\NormalTok{, }\DataTypeTok{lty=}\DecValTok{3}\NormalTok{, }\DataTypeTok{lwd=}\DecValTok{2}\NormalTok{)}
\KeywordTok{plot}\NormalTok{(Girth, }\DataTypeTok{type=}\StringTok{'l'}\NormalTok{, }\DataTypeTok{lty=}\DecValTok{4}\NormalTok{, }\DataTypeTok{lwd=}\DecValTok{2}\NormalTok{)}
\KeywordTok{plot}\NormalTok{(Girth, }\DataTypeTok{type=}\StringTok{'l'}\NormalTok{, }\DataTypeTok{lty=}\DecValTok{5}\NormalTok{, }\DataTypeTok{lwd=}\DecValTok{2}\NormalTok{)}
\KeywordTok{plot}\NormalTok{(Girth, }\DataTypeTok{type=}\StringTok{'l'}\NormalTok{, }\DataTypeTok{lty=}\DecValTok{6}\NormalTok{, }\DataTypeTok{lwd=}\DecValTok{2}\NormalTok{)}
\end{Highlighting}
\end{Shaded}

\includegraphics[width=0.5\linewidth]{Rcourse_files/figure-latex/unnamed-chunk-191-1}

Add line by slope and intercept with \texttt{abline}.

\begin{Shaded}
\begin{Highlighting}[]
\KeywordTok{plot}\NormalTok{(Girth)}
\KeywordTok{abline}\NormalTok{(}\DataTypeTok{v=}\DecValTok{14}\NormalTok{, }\DataTypeTok{col=}\StringTok{'red'}\NormalTok{) }\CommentTok{# vertical line at 14.}
\KeywordTok{abline}\NormalTok{(}\DataTypeTok{h=}\DecValTok{9}\NormalTok{, }\DataTypeTok{lty=}\DecValTok{4}\NormalTok{,}\DataTypeTok{lwd=}\DecValTok{4}\NormalTok{, }\DataTypeTok{col=}\StringTok{'pink'}\NormalTok{) }\CommentTok{# horizontal line at 9.}
\KeywordTok{abline}\NormalTok{(}\DataTypeTok{a =} \DecValTok{0}\NormalTok{, }\DataTypeTok{b=}\DecValTok{1}\NormalTok{) }\CommentTok{# linear line with intercept a=0, and slope b=1.}
\end{Highlighting}
\end{Shaded}

\includegraphics[width=0.5\linewidth]{Rcourse_files/figure-latex/unnamed-chunk-192-1}

\begin{Shaded}
\begin{Highlighting}[]
\KeywordTok{plot}\NormalTok{(Girth)}
\KeywordTok{points}\NormalTok{(}\DataTypeTok{x=}\DecValTok{1}\NormalTok{:}\DecValTok{30}\NormalTok{, }\DataTypeTok{y=}\KeywordTok{rep}\NormalTok{(}\DecValTok{12}\NormalTok{,}\DecValTok{30}\NormalTok{), }\DataTypeTok{cex=}\FloatTok{0.5}\NormalTok{, }\DataTypeTok{col=}\StringTok{'darkblue'}\NormalTok{)}
\KeywordTok{lines}\NormalTok{(}\DataTypeTok{x=}\KeywordTok{rep}\NormalTok{(}\KeywordTok{c}\NormalTok{(}\DecValTok{5}\NormalTok{,}\DecValTok{10}\NormalTok{), }\DecValTok{7}\NormalTok{), }\DataTypeTok{y=}\DecValTok{7}\NormalTok{:}\DecValTok{20}\NormalTok{, }\DataTypeTok{lty=}\DecValTok{2} \NormalTok{)}
\KeywordTok{lines}\NormalTok{(}\DataTypeTok{x=}\KeywordTok{rep}\NormalTok{(}\KeywordTok{c}\NormalTok{(}\DecValTok{5}\NormalTok{,}\DecValTok{10}\NormalTok{), }\DecValTok{7}\NormalTok{)+}\DecValTok{2}\NormalTok{, }\DataTypeTok{y=}\DecValTok{7}\NormalTok{:}\DecValTok{20}\NormalTok{, }\DataTypeTok{lty=}\DecValTok{2} \NormalTok{)}
\KeywordTok{lines}\NormalTok{(}\DataTypeTok{x=}\KeywordTok{rep}\NormalTok{(}\KeywordTok{c}\NormalTok{(}\DecValTok{5}\NormalTok{,}\DecValTok{10}\NormalTok{), }\DecValTok{7}\NormalTok{)+}\DecValTok{4}\NormalTok{, }\DataTypeTok{y=}\DecValTok{7}\NormalTok{:}\DecValTok{20}\NormalTok{, }\DataTypeTok{lty=}\DecValTok{2} \NormalTok{, }\DataTypeTok{col=}\StringTok{'darkgreen'}\NormalTok{)}
\KeywordTok{lines}\NormalTok{(}\DataTypeTok{x=}\KeywordTok{rep}\NormalTok{(}\KeywordTok{c}\NormalTok{(}\DecValTok{5}\NormalTok{,}\DecValTok{10}\NormalTok{), }\DecValTok{7}\NormalTok{)+}\DecValTok{6}\NormalTok{, }\DataTypeTok{y=}\DecValTok{7}\NormalTok{:}\DecValTok{20}\NormalTok{, }\DataTypeTok{lty=}\DecValTok{4} \NormalTok{, }\DataTypeTok{col=}\StringTok{'brown'}\NormalTok{, }\DataTypeTok{lwd=}\DecValTok{4}\NormalTok{)}
\end{Highlighting}
\end{Shaded}

\includegraphics[width=0.5\linewidth]{Rcourse_files/figure-latex/unnamed-chunk-193-1}

Things to note:

\begin{itemize}
\tightlist
\item
  \texttt{points} adds points on an existing plot.
\item
  \texttt{lines} adds lines on an existing plot.
\item
  \texttt{col} controls the color of the element. It takes names or
  numbers as argument.
\item
  \texttt{cex} controls the scale of the element. Defaults to
  \texttt{cex=1}.
\end{itemize}

Add other elements.

\begin{Shaded}
\begin{Highlighting}[]
\KeywordTok{plot}\NormalTok{(Girth)}
\KeywordTok{segments}\NormalTok{(}\DataTypeTok{x0=}\KeywordTok{rep}\NormalTok{(}\KeywordTok{c}\NormalTok{(}\DecValTok{5}\NormalTok{,}\DecValTok{10}\NormalTok{), }\DecValTok{7}\NormalTok{), }\DataTypeTok{y0=}\DecValTok{7}\NormalTok{:}\DecValTok{20}\NormalTok{, }\DataTypeTok{x1=}\KeywordTok{rep}\NormalTok{(}\KeywordTok{c}\NormalTok{(}\DecValTok{5}\NormalTok{,}\DecValTok{10}\NormalTok{), }\DecValTok{7}\NormalTok{)+}\DecValTok{2}\NormalTok{, }\DataTypeTok{y1=}\NormalTok{(}\DecValTok{7}\NormalTok{:}\DecValTok{20}\NormalTok{)+}\DecValTok{2} \NormalTok{)}
\KeywordTok{arrows}\NormalTok{(}\DataTypeTok{x0=}\DecValTok{13}\NormalTok{,}\DataTypeTok{y0=}\DecValTok{16}\NormalTok{,}\DataTypeTok{x1=}\DecValTok{16}\NormalTok{,}\DataTypeTok{y1=}\DecValTok{17}\NormalTok{, )}
\KeywordTok{rect}\NormalTok{(}\DataTypeTok{xleft=}\DecValTok{10}\NormalTok{, }\DataTypeTok{ybottom=}\DecValTok{12}\NormalTok{,  }\DataTypeTok{xright=}\DecValTok{12}\NormalTok{, }\DataTypeTok{ytop=}\DecValTok{16}\NormalTok{)}
\KeywordTok{polygon}\NormalTok{(}\DataTypeTok{x=}\KeywordTok{c}\NormalTok{(}\DecValTok{10}\NormalTok{,}\DecValTok{11}\NormalTok{,}\DecValTok{12}\NormalTok{,}\FloatTok{11.5}\NormalTok{,}\FloatTok{10.5}\NormalTok{), }\DataTypeTok{y=}\KeywordTok{c}\NormalTok{(}\DecValTok{9}\NormalTok{,}\FloatTok{9.5}\NormalTok{,}\DecValTok{10}\NormalTok{,}\FloatTok{10.5}\NormalTok{,}\FloatTok{9.8}\NormalTok{), }\DataTypeTok{col=}\StringTok{'grey'}\NormalTok{)}
\KeywordTok{title}\NormalTok{(}\DataTypeTok{main=}\StringTok{'This plot makes no sense'}\NormalTok{, }\DataTypeTok{sub=}\StringTok{'Or does it?'}\NormalTok{)}
\KeywordTok{mtext}\NormalTok{(}\StringTok{'Printing in the margins'}\NormalTok{, }\DataTypeTok{side=}\DecValTok{2}\NormalTok{)}
\KeywordTok{mtext}\NormalTok{(}\KeywordTok{expression}\NormalTok{(alpha==}\KeywordTok{log}\NormalTok{(f[i])), }\DataTypeTok{side=}\DecValTok{4}\NormalTok{)}
\end{Highlighting}
\end{Shaded}

\includegraphics[width=0.5\linewidth]{Rcourse_files/figure-latex/unnamed-chunk-194-1}

Things to note:

\begin{itemize}
\tightlist
\item
  The following functions add the elements they are named after:
  \texttt{segments}, \texttt{arrows}, \texttt{rect}, \texttt{polygon},
  \texttt{title}.
\item
  \texttt{mtext} adds mathematical text. For more information for
  mathematical annotation see \texttt{?plotmath}.
\end{itemize}

Add a legend.

\begin{Shaded}
\begin{Highlighting}[]
\KeywordTok{plot}\NormalTok{(Girth, }\DataTypeTok{pch=}\StringTok{'G'}\NormalTok{,}\DataTypeTok{ylim=}\KeywordTok{c}\NormalTok{(}\DecValTok{8}\NormalTok{,}\DecValTok{77}\NormalTok{), }\DataTypeTok{xlab=}\StringTok{'Tree number'}\NormalTok{, }\DataTypeTok{ylab=}\StringTok{''}\NormalTok{, }\DataTypeTok{type=}\StringTok{'b'}\NormalTok{, }\DataTypeTok{col=}\StringTok{'blue'}\NormalTok{)}
\KeywordTok{points}\NormalTok{(Volume, }\DataTypeTok{pch=}\StringTok{'V'}\NormalTok{, }\DataTypeTok{type=}\StringTok{'b'}\NormalTok{, }\DataTypeTok{col=}\StringTok{'red'}\NormalTok{)}
\KeywordTok{legend}\NormalTok{(}\DataTypeTok{x=}\DecValTok{2}\NormalTok{, }\DataTypeTok{y=}\DecValTok{70}\NormalTok{, }\DataTypeTok{legend=}\KeywordTok{c}\NormalTok{(}\StringTok{'Girth'}\NormalTok{, }\StringTok{'Volume'}\NormalTok{), }\DataTypeTok{pch=}\KeywordTok{c}\NormalTok{(}\StringTok{'G'}\NormalTok{,}\StringTok{'V'}\NormalTok{), }\DataTypeTok{col=}\KeywordTok{c}\NormalTok{(}\StringTok{'blue'}\NormalTok{,}\StringTok{'red'}\NormalTok{), }\DataTypeTok{bg=}\StringTok{'grey'}\NormalTok{)}
\end{Highlighting}
\end{Shaded}

\includegraphics[width=0.5\linewidth]{Rcourse_files/figure-latex/unnamed-chunk-195-1}

Adjusting Axes with \texttt{xlim} and \texttt{ylim}.

\begin{Shaded}
\begin{Highlighting}[]
\KeywordTok{plot}\NormalTok{(Girth, }\DataTypeTok{xlim=}\KeywordTok{c}\NormalTok{(}\DecValTok{0}\NormalTok{,}\DecValTok{15}\NormalTok{), }\DataTypeTok{ylim=}\KeywordTok{c}\NormalTok{(}\DecValTok{8}\NormalTok{,}\DecValTok{12}\NormalTok{))}
\end{Highlighting}
\end{Shaded}

\includegraphics[width=0.5\linewidth]{Rcourse_files/figure-latex/unnamed-chunk-196-1}

Use \texttt{layout} for complicated plot layouts.

\begin{Shaded}
\begin{Highlighting}[]
\NormalTok{A<-}\KeywordTok{matrix}\NormalTok{(}\KeywordTok{c}\NormalTok{(}\DecValTok{1}\NormalTok{,}\DecValTok{1}\NormalTok{,}\DecValTok{2}\NormalTok{,}\DecValTok{3}\NormalTok{,}\DecValTok{4}\NormalTok{,}\DecValTok{4}\NormalTok{,}\DecValTok{5}\NormalTok{,}\DecValTok{6}\NormalTok{), }\DataTypeTok{byrow=}\OtherTok{TRUE}\NormalTok{, }\DataTypeTok{ncol=}\DecValTok{2}\NormalTok{)}
\KeywordTok{layout}\NormalTok{(A,}\DataTypeTok{heights=}\KeywordTok{c}\NormalTok{(}\DecValTok{1}\NormalTok{/}\DecValTok{14}\NormalTok{,}\DecValTok{6}\NormalTok{/}\DecValTok{14}\NormalTok{,}\DecValTok{1}\NormalTok{/}\DecValTok{14}\NormalTok{,}\DecValTok{6}\NormalTok{/}\DecValTok{14}\NormalTok{))}

\NormalTok{oma.saved <-}\StringTok{ }\KeywordTok{par}\NormalTok{(}\StringTok{"oma"}\NormalTok{)}
\KeywordTok{par}\NormalTok{(}\DataTypeTok{oma =} \KeywordTok{rep.int}\NormalTok{(}\DecValTok{0}\NormalTok{, }\DecValTok{4}\NormalTok{))}
\KeywordTok{par}\NormalTok{(}\DataTypeTok{oma =} \NormalTok{oma.saved)}
\NormalTok{o.par <-}\StringTok{ }\KeywordTok{par}\NormalTok{(}\DataTypeTok{mar =} \KeywordTok{rep.int}\NormalTok{(}\DecValTok{0}\NormalTok{, }\DecValTok{4}\NormalTok{))}
\NormalTok{for (i in }\KeywordTok{seq_len}\NormalTok{(}\DecValTok{6}\NormalTok{)) \{}
    \KeywordTok{plot.new}\NormalTok{()}
    \KeywordTok{box}\NormalTok{()}
    \KeywordTok{text}\NormalTok{(}\FloatTok{0.5}\NormalTok{, }\FloatTok{0.5}\NormalTok{, }\KeywordTok{paste}\NormalTok{(}\StringTok{'Box no.'}\NormalTok{,i), }\DataTypeTok{cex=}\DecValTok{3}\NormalTok{)}
\NormalTok{\}}
\end{Highlighting}
\end{Shaded}

\includegraphics[width=0.5\linewidth]{Rcourse_files/figure-latex/unnamed-chunk-197-1}

Always detach.

\begin{Shaded}
\begin{Highlighting}[]
\KeywordTok{detach}\NormalTok{(trees)}
\end{Highlighting}
\end{Shaded}

\subsection{Fancy graphics Examples}\label{fancy-graphics-examples}

Building a line graph from scratch.

\begin{Shaded}
\begin{Highlighting}[]
\NormalTok{x =}\StringTok{ }\DecValTok{1995}\NormalTok{:}\DecValTok{2005}
\NormalTok{y =}\StringTok{ }\KeywordTok{c}\NormalTok{(}\FloatTok{81.1}\NormalTok{, }\FloatTok{83.1}\NormalTok{, }\FloatTok{84.3}\NormalTok{, }\FloatTok{85.2}\NormalTok{, }\FloatTok{85.4}\NormalTok{, }\FloatTok{86.5}\NormalTok{, }\FloatTok{88.3}\NormalTok{, }\FloatTok{88.6}\NormalTok{, }\FloatTok{90.8}\NormalTok{, }\FloatTok{91.1}\NormalTok{, }\FloatTok{91.3}\NormalTok{)}
\KeywordTok{plot.new}\NormalTok{()}
\KeywordTok{plot.window}\NormalTok{(}\DataTypeTok{xlim =} \KeywordTok{range}\NormalTok{(x), }\DataTypeTok{ylim =} \KeywordTok{range}\NormalTok{(y))}
\KeywordTok{abline}\NormalTok{(}\DataTypeTok{h =} \NormalTok{-}\DecValTok{4}\NormalTok{:}\DecValTok{4}\NormalTok{, }\DataTypeTok{v =} \NormalTok{-}\DecValTok{4}\NormalTok{:}\DecValTok{4}\NormalTok{, }\DataTypeTok{col =} \StringTok{"lightgrey"}\NormalTok{)}
\KeywordTok{lines}\NormalTok{(x, y, }\DataTypeTok{lwd =} \DecValTok{2}\NormalTok{)}
\KeywordTok{title}\NormalTok{(}\DataTypeTok{main =} \StringTok{"A Line Graph Example"}\NormalTok{,}
        \DataTypeTok{xlab =} \StringTok{"Time"}\NormalTok{,}
        \DataTypeTok{ylab =} \StringTok{"Quality of R Graphics"}\NormalTok{)}
\KeywordTok{axis}\NormalTok{(}\DecValTok{1}\NormalTok{)}
\KeywordTok{axis}\NormalTok{(}\DecValTok{2}\NormalTok{)}
\KeywordTok{box}\NormalTok{()}
\end{Highlighting}
\end{Shaded}

\includegraphics[width=0.5\linewidth]{Rcourse_files/figure-latex/unnamed-chunk-199-1}

Things to note:

\begin{itemize}
\tightlist
\item
  \texttt{plot.new} creates a new, empty, plotting device.
\item
  \texttt{plot.window} determines the limits of the plotting region.
\item
  \texttt{axis} adds the axes, and \texttt{box} the framing box.
\item
  The rest of the elements, you already know.
\end{itemize}

Rosette.

\begin{Shaded}
\begin{Highlighting}[]
\NormalTok{n =}\StringTok{ }\DecValTok{17}
\NormalTok{theta =}\StringTok{ }\KeywordTok{seq}\NormalTok{(}\DecValTok{0}\NormalTok{, }\DecValTok{2} \NormalTok{*}\StringTok{ }\NormalTok{pi, }\DataTypeTok{length =} \NormalTok{n +}\StringTok{ }\DecValTok{1}\NormalTok{)[}\DecValTok{1}\NormalTok{:n]}
\NormalTok{x =}\StringTok{ }\KeywordTok{sin}\NormalTok{(theta)}
\NormalTok{y =}\StringTok{ }\KeywordTok{cos}\NormalTok{(theta)}
\NormalTok{v1 =}\StringTok{ }\KeywordTok{rep}\NormalTok{(}\DecValTok{1}\NormalTok{:n, n)}
\NormalTok{v2 =}\StringTok{ }\KeywordTok{rep}\NormalTok{(}\DecValTok{1}\NormalTok{:n, }\KeywordTok{rep}\NormalTok{(n, n))}
\KeywordTok{plot.new}\NormalTok{()}
\KeywordTok{plot.window}\NormalTok{(}\DataTypeTok{xlim =} \KeywordTok{c}\NormalTok{(-}\DecValTok{1}\NormalTok{, }\DecValTok{1}\NormalTok{), }\DataTypeTok{ylim =} \KeywordTok{c}\NormalTok{(-}\DecValTok{1}\NormalTok{, }\DecValTok{1}\NormalTok{), }\DataTypeTok{asp =} \DecValTok{1}\NormalTok{)}
\KeywordTok{segments}\NormalTok{(x[v1], y[v1], x[v2], y[v2])}
\KeywordTok{box}\NormalTok{()}
\end{Highlighting}
\end{Shaded}

\includegraphics[width=0.5\linewidth]{Rcourse_files/figure-latex/unnamed-chunk-200-1}

Arrows.

\begin{Shaded}
\begin{Highlighting}[]
\KeywordTok{plot.new}\NormalTok{()}
\KeywordTok{plot.window}\NormalTok{(}\DataTypeTok{xlim =} \KeywordTok{c}\NormalTok{(}\DecValTok{0}\NormalTok{, }\DecValTok{1}\NormalTok{), }\DataTypeTok{ylim =} \KeywordTok{c}\NormalTok{(}\DecValTok{0}\NormalTok{, }\DecValTok{1}\NormalTok{))}
\KeywordTok{arrows}\NormalTok{(.}\DecValTok{05}\NormalTok{, .}\DecValTok{075}\NormalTok{, .}\DecValTok{45}\NormalTok{, .}\DecValTok{9}\NormalTok{, }\DataTypeTok{code =} \DecValTok{1}\NormalTok{)}
\KeywordTok{arrows}\NormalTok{(.}\DecValTok{55}\NormalTok{, .}\DecValTok{9}\NormalTok{, .}\DecValTok{95}\NormalTok{, .}\DecValTok{075}\NormalTok{, }\DataTypeTok{code =} \DecValTok{2}\NormalTok{)}
\KeywordTok{arrows}\NormalTok{(.}\DecValTok{1}\NormalTok{, }\DecValTok{0}\NormalTok{, .}\DecValTok{9}\NormalTok{, }\DecValTok{0}\NormalTok{, }\DataTypeTok{code =} \DecValTok{3}\NormalTok{)}
\KeywordTok{text}\NormalTok{(.}\DecValTok{5}\NormalTok{, }\DecValTok{1}\NormalTok{, }\StringTok{"A"}\NormalTok{, }\DataTypeTok{cex =} \FloatTok{1.5}\NormalTok{)}
\KeywordTok{text}\NormalTok{(}\DecValTok{0}\NormalTok{, }\DecValTok{0}\NormalTok{, }\StringTok{"B"}\NormalTok{, }\DataTypeTok{cex =} \FloatTok{1.5}\NormalTok{)}
\KeywordTok{text}\NormalTok{(}\DecValTok{1}\NormalTok{, }\DecValTok{0}\NormalTok{, }\StringTok{"C"}\NormalTok{, }\DataTypeTok{cex =} \FloatTok{1.5}\NormalTok{)}
\end{Highlighting}
\end{Shaded}

\includegraphics[width=0.5\linewidth]{Rcourse_files/figure-latex/unnamed-chunk-201-1}

Arrows as error bars.

\begin{Shaded}
\begin{Highlighting}[]
\NormalTok{x =}\StringTok{ }\DecValTok{1}\NormalTok{:}\DecValTok{10}
\NormalTok{y =}\StringTok{ }\KeywordTok{runif}\NormalTok{(}\DecValTok{10}\NormalTok{) +}\StringTok{ }\KeywordTok{rep}\NormalTok{(}\KeywordTok{c}\NormalTok{(}\DecValTok{5}\NormalTok{, }\FloatTok{6.5}\NormalTok{), }\KeywordTok{c}\NormalTok{(}\DecValTok{5}\NormalTok{, }\DecValTok{5}\NormalTok{))}
\NormalTok{yl =}\StringTok{ }\NormalTok{y -}\StringTok{ }\FloatTok{0.25} \NormalTok{-}\StringTok{ }\KeywordTok{runif}\NormalTok{(}\DecValTok{10}\NormalTok{)/}\DecValTok{3}
\NormalTok{yu =}\StringTok{ }\NormalTok{y +}\StringTok{ }\FloatTok{0.25} \NormalTok{+}\StringTok{ }\KeywordTok{runif}\NormalTok{(}\DecValTok{10}\NormalTok{)/}\DecValTok{3}
\KeywordTok{plot.new}\NormalTok{()}
\KeywordTok{plot.window}\NormalTok{(}\DataTypeTok{xlim =} \KeywordTok{c}\NormalTok{(}\FloatTok{0.5}\NormalTok{, }\FloatTok{10.5}\NormalTok{), }\DataTypeTok{ylim =} \KeywordTok{range}\NormalTok{(yl, yu))}
\KeywordTok{arrows}\NormalTok{(x, yl, x, yu, }\DataTypeTok{code =} \DecValTok{3}\NormalTok{, }\DataTypeTok{angle =} \DecValTok{90}\NormalTok{, }\DataTypeTok{length =} \NormalTok{.}\DecValTok{125}\NormalTok{)}
\KeywordTok{points}\NormalTok{(x, y, }\DataTypeTok{pch =} \DecValTok{19}\NormalTok{, }\DataTypeTok{cex =} \FloatTok{1.5}\NormalTok{)}
\KeywordTok{axis}\NormalTok{(}\DecValTok{1}\NormalTok{, }\DataTypeTok{at =} \DecValTok{1}\NormalTok{:}\DecValTok{10}\NormalTok{, }\DataTypeTok{labels =} \NormalTok{LETTERS[}\DecValTok{1}\NormalTok{:}\DecValTok{10}\NormalTok{])}
\KeywordTok{axis}\NormalTok{(}\DecValTok{2}\NormalTok{, }\DataTypeTok{las =} \DecValTok{1}\NormalTok{)}
\KeywordTok{box}\NormalTok{()}
\end{Highlighting}
\end{Shaded}

\includegraphics[width=0.5\linewidth]{Rcourse_files/figure-latex/unnamed-chunk-202-1}

A histogram is nothing but a bunch of rectangle elements.

\begin{Shaded}
\begin{Highlighting}[]
\KeywordTok{plot.new}\NormalTok{()}
\KeywordTok{plot.window}\NormalTok{(}\DataTypeTok{xlim =} \KeywordTok{c}\NormalTok{(}\DecValTok{0}\NormalTok{, }\DecValTok{5}\NormalTok{), }\DataTypeTok{ylim =} \KeywordTok{c}\NormalTok{(}\DecValTok{0}\NormalTok{, }\DecValTok{10}\NormalTok{))}
\KeywordTok{rect}\NormalTok{(}\DecValTok{0}\NormalTok{:}\DecValTok{4}\NormalTok{, }\DecValTok{0}\NormalTok{, }\DecValTok{1}\NormalTok{:}\DecValTok{5}\NormalTok{, }\KeywordTok{c}\NormalTok{(}\DecValTok{7}\NormalTok{, }\DecValTok{8}\NormalTok{, }\DecValTok{4}\NormalTok{, }\DecValTok{3}\NormalTok{), }\DataTypeTok{col =} \StringTok{"lightblue"}\NormalTok{)}
\KeywordTok{axis}\NormalTok{(}\DecValTok{1}\NormalTok{)}
\KeywordTok{axis}\NormalTok{(}\DecValTok{2}\NormalTok{, }\DataTypeTok{las =} \DecValTok{1}\NormalTok{)}
\end{Highlighting}
\end{Shaded}

\includegraphics[width=0.5\linewidth]{Rcourse_files/figure-latex/unnamed-chunk-203-1}

Spiral Squares.

\begin{Shaded}
\begin{Highlighting}[]
\KeywordTok{plot.new}\NormalTok{()}
\KeywordTok{plot.window}\NormalTok{(}\DataTypeTok{xlim =} \KeywordTok{c}\NormalTok{(-}\DecValTok{1}\NormalTok{, }\DecValTok{1}\NormalTok{), }\DataTypeTok{ylim =} \KeywordTok{c}\NormalTok{(-}\DecValTok{1}\NormalTok{, }\DecValTok{1}\NormalTok{), }\DataTypeTok{asp =} \DecValTok{1}\NormalTok{)}
\NormalTok{x =}\StringTok{ }\KeywordTok{c}\NormalTok{(-}\DecValTok{1}\NormalTok{, }\DecValTok{1}\NormalTok{, }\DecValTok{1}\NormalTok{, -}\DecValTok{1}\NormalTok{)}
\NormalTok{y =}\StringTok{ }\KeywordTok{c}\NormalTok{( }\DecValTok{1}\NormalTok{, }\DecValTok{1}\NormalTok{, -}\DecValTok{1}\NormalTok{, -}\DecValTok{1}\NormalTok{)}
\KeywordTok{polygon}\NormalTok{(x, y, }\DataTypeTok{col =} \StringTok{"cornsilk"}\NormalTok{)}
\NormalTok{vertex1 =}\StringTok{ }\KeywordTok{c}\NormalTok{(}\DecValTok{1}\NormalTok{, }\DecValTok{2}\NormalTok{, }\DecValTok{3}\NormalTok{, }\DecValTok{4}\NormalTok{)}
\NormalTok{vertex2 =}\StringTok{ }\KeywordTok{c}\NormalTok{(}\DecValTok{2}\NormalTok{, }\DecValTok{3}\NormalTok{, }\DecValTok{4}\NormalTok{, }\DecValTok{1}\NormalTok{)}
\NormalTok{for(i in }\DecValTok{1}\NormalTok{:}\DecValTok{50}\NormalTok{) \{}
    \NormalTok{x =}\StringTok{ }\FloatTok{0.9} \NormalTok{*}\StringTok{ }\NormalTok{x[vertex1] +}\StringTok{ }\FloatTok{0.1} \NormalTok{*}\StringTok{ }\NormalTok{x[vertex2]}
    \NormalTok{y =}\StringTok{ }\FloatTok{0.9} \NormalTok{*}\StringTok{ }\NormalTok{y[vertex1] +}\StringTok{ }\FloatTok{0.1} \NormalTok{*}\StringTok{ }\NormalTok{y[vertex2]}
    \KeywordTok{polygon}\NormalTok{(x, y, }\DataTypeTok{col =} \StringTok{"cornsilk"}\NormalTok{)}
\NormalTok{\}}
\end{Highlighting}
\end{Shaded}

\includegraphics[width=0.5\linewidth]{Rcourse_files/figure-latex/unnamed-chunk-204-1}

Circles are just dense polygons.

\begin{Shaded}
\begin{Highlighting}[]
\NormalTok{R =}\StringTok{ }\DecValTok{1}
\NormalTok{xc =}\StringTok{ }\DecValTok{0}
\NormalTok{yc =}\StringTok{ }\DecValTok{0}
\NormalTok{n =}\StringTok{ }\DecValTok{72}
\NormalTok{t =}\StringTok{ }\KeywordTok{seq}\NormalTok{(}\DecValTok{0}\NormalTok{, }\DecValTok{2} \NormalTok{*}\StringTok{ }\NormalTok{pi, }\DataTypeTok{length =} \NormalTok{n)[}\DecValTok{1}\NormalTok{:(n}\DecValTok{-1}\NormalTok{)]}
\NormalTok{x =}\StringTok{ }\NormalTok{xc +}\StringTok{ }\NormalTok{R *}\StringTok{ }\KeywordTok{cos}\NormalTok{(t)}
\NormalTok{y =}\StringTok{ }\NormalTok{yc +}\StringTok{ }\NormalTok{R *}\StringTok{ }\KeywordTok{sin}\NormalTok{(t)}
\KeywordTok{plot.new}\NormalTok{()}
\KeywordTok{plot.window}\NormalTok{(}\DataTypeTok{xlim =} \KeywordTok{range}\NormalTok{(x), }\DataTypeTok{ylim =} \KeywordTok{range}\NormalTok{(y), }\DataTypeTok{asp =} \DecValTok{1}\NormalTok{)}
\KeywordTok{polygon}\NormalTok{(x, y, }\DataTypeTok{col =} \StringTok{"lightblue"}\NormalTok{, }\DataTypeTok{border =} \StringTok{"navyblue"}\NormalTok{)}
\end{Highlighting}
\end{Shaded}

\includegraphics[width=0.5\linewidth]{Rcourse_files/figure-latex/unnamed-chunk-205-1}

Spiral- just a bunch of lines.

\begin{Shaded}
\begin{Highlighting}[]
\NormalTok{k =}\StringTok{ }\DecValTok{5}
\NormalTok{n =}\StringTok{ }\NormalTok{k *}\StringTok{ }\DecValTok{72}
\NormalTok{theta =}\StringTok{ }\KeywordTok{seq}\NormalTok{(}\DecValTok{0}\NormalTok{, k *}\StringTok{ }\DecValTok{2} \NormalTok{*}\StringTok{ }\NormalTok{pi, }\DataTypeTok{length =} \NormalTok{n)}
\NormalTok{R =}\StringTok{ }\NormalTok{.}\DecValTok{98}\NormalTok{^(}\DecValTok{1}\NormalTok{:n -}\StringTok{ }\DecValTok{1}\NormalTok{)}
\NormalTok{x =}\StringTok{ }\NormalTok{R *}\StringTok{ }\KeywordTok{cos}\NormalTok{(theta)}
\NormalTok{y =}\StringTok{ }\NormalTok{R *}\StringTok{ }\KeywordTok{sin}\NormalTok{(theta)}
\KeywordTok{plot.new}\NormalTok{()}
\KeywordTok{plot.window}\NormalTok{(}\DataTypeTok{xlim =} \KeywordTok{range}\NormalTok{(x), }\DataTypeTok{ylim =} \KeywordTok{range}\NormalTok{(y), }\DataTypeTok{asp =} \DecValTok{1}\NormalTok{)}
\KeywordTok{lines}\NormalTok{(x, y)}
\end{Highlighting}
\end{Shaded}

\includegraphics[width=0.5\linewidth]{Rcourse_files/figure-latex/unnamed-chunk-206-1}

\subsection{Exporting a Plot}\label{exporting-a-plot}

The pipeline for exporting graphics is similar to the export of data.
Instead of the \texttt{write.table} or \texttt{save} functions, we will
use the \texttt{pdf}, \texttt{tiff}, \texttt{png}, functions. Depending
on the type of desired output.

Check and set the working directory.

\begin{Shaded}
\begin{Highlighting}[]
\KeywordTok{getwd}\NormalTok{()}
\KeywordTok{setwd}\NormalTok{(}\StringTok{"/tmp/"}\NormalTok{)}
\end{Highlighting}
\end{Shaded}

Export tiff.

\begin{Shaded}
\begin{Highlighting}[]
\KeywordTok{tiff}\NormalTok{(}\DataTypeTok{filename=}\StringTok{'graphicExample.tiff'}\NormalTok{)}
\KeywordTok{plot}\NormalTok{(}\KeywordTok{rnorm}\NormalTok{(}\DecValTok{100}\NormalTok{))}
\KeywordTok{dev.off}\NormalTok{()}
\end{Highlighting}
\end{Shaded}

Things to note:

\begin{itemize}
\tightlist
\item
  The \texttt{tiff} function tells R to open a .tiff file, and write the
  output of a plot.
\item
  Only a single (the last) plot is saved.
\item
  \texttt{dev.off} to close the tiff device, and return the plotting to
  the R console (or RStudio).
\end{itemize}

If you want to produce several plots, you can use a counter in the
file's name. The counter uses the
\href{https://en.wikipedia.org/wiki/Printf_format_string}{printf} format
string.

\begin{Shaded}
\begin{Highlighting}[]
\KeywordTok{tiff}\NormalTok{(}\DataTypeTok{filename=}\StringTok{'graphicExample%d.tiff'}\NormalTok{) }\CommentTok{#Creates a sequence of files}
\KeywordTok{plot}\NormalTok{(}\KeywordTok{rnorm}\NormalTok{(}\DecValTok{100}\NormalTok{))}
\KeywordTok{boxplot}\NormalTok{(}\KeywordTok{rnorm}\NormalTok{(}\DecValTok{100}\NormalTok{))}
\KeywordTok{hist}\NormalTok{(}\KeywordTok{rnorm}\NormalTok{(}\DecValTok{100}\NormalTok{))}
\KeywordTok{dev.off}\NormalTok{()}
\end{Highlighting}
\end{Shaded}

\begin{verbatim}
## pdf 
##   2
\end{verbatim}

To see the list of all open devices use \texttt{dev.list()}. To close
\textbf{all} device, (not only the last one), use
\texttt{graphics.off()}.

See \texttt{?pdf} and \texttt{?jpeg} for more info.

\section{The ggplot2 System}\label{the-ggplot2-system}

The philosophy of \textbf{ggplot2} is very different from the
\textbf{graphics} device. Recall, in \textbf{ggplot2}, a plot is a
object. It can be queried, it can be changed, and among other things, it
can be plotted.

\textbf{ggplot2} provides a convenience function for many plots:
\texttt{qplot}. We take a non-typical approach by ignoring
\texttt{qplot}, and presenting the fundamental building blocks. Once the
building blocks have been understood, mastering \texttt{qplot} will be
easy.

The following is taken from
\href{http://www.ats.ucla.edu/stat/r/seminars/ggplot2_intro/ggplot2_intro.htm}{UCLA's
idre}.

A \textbf{ggplot2} object will have the following elements:

\begin{itemize}
\tightlist
\item
  \textbf{Data} are the variables mapped to aesthetic features of the
  graph.
\item
  \textbf{Aes} is the mapping between objects to their visualization.
\item
  \textbf{Geoms} are the objects/shapes you see on the graph.
\item
  \textbf{Stats} are statistical transformations that summarize data,
  such as the mean or confidence intervals.
\item
  \textbf{Scales} define which aesthetic values are mapped to data
  values. Legends and axes display these mappings.
\item
  \textbf{Coordiante systems} define the plane on which data are mapped
  on the graphic.
\item
  \textbf{Faceting} splits the data into subsets to create multiple
  variations of the same graph (paneling).
\end{itemize}

The \texttt{nlme::Milk} dataset has the protein level of various cows,
at various times, with various diets.

\begin{Shaded}
\begin{Highlighting}[]
\KeywordTok{library}\NormalTok{(nlme)}
\KeywordTok{data}\NormalTok{(Milk)}
\KeywordTok{head}\NormalTok{(Milk)}
\end{Highlighting}
\end{Shaded}

\begin{verbatim}
## Grouped Data: protein ~ Time | Cow
##   protein Time Cow   Diet
## 1    3.63    1 B01 barley
## 2    3.57    2 B01 barley
## 3    3.47    3 B01 barley
## 4    3.65    4 B01 barley
## 5    3.89    5 B01 barley
## 6    3.73    6 B01 barley
\end{verbatim}

\begin{Shaded}
\begin{Highlighting}[]
\KeywordTok{library}\NormalTok{(ggplot2)}
\KeywordTok{ggplot}\NormalTok{(}\DataTypeTok{data =} \NormalTok{Milk, }\KeywordTok{aes}\NormalTok{(}\DataTypeTok{x=}\NormalTok{Time, }\DataTypeTok{y=}\NormalTok{protein)) +}
\StringTok{  }\KeywordTok{geom_point}\NormalTok{()}
\end{Highlighting}
\end{Shaded}

\includegraphics[width=0.5\linewidth]{Rcourse_files/figure-latex/unnamed-chunk-211-1}

Things to note:

\begin{itemize}
\tightlist
\item
  The \texttt{ggplot} function is the constructor of the
  \textbf{ggplot2} object. If the object is not assigned, it is plotted.
\item
  The \texttt{aes} argument tells R that the \texttt{Time} variable in
  the \texttt{Milk} data is the x axis, and protein is y.
\item
  The \texttt{geom\_point} defines the \textbf{Geom}, i.e., it tells R
  to plot the points as they are (and not lines, histograms, etc.).
\item
  The \textbf{ggplot2} object is build by compounding its various
  elements separated by the \texttt{+} operator.
\item
  All the variables that we will need are assumed to be in the
  \texttt{Milk} data frame. This means that (a) the data needs to be a
  data frame (not a matrix for instance), and (b) we will not be able to
  use variables that are not in the \texttt{Milk} data frame.
\end{itemize}

Let's add some color.

\begin{Shaded}
\begin{Highlighting}[]
\KeywordTok{ggplot}\NormalTok{(}\DataTypeTok{data =} \NormalTok{Milk, }\KeywordTok{aes}\NormalTok{(}\DataTypeTok{x=}\NormalTok{Time, }\DataTypeTok{y=}\NormalTok{protein)) +}
\StringTok{  }\KeywordTok{geom_point}\NormalTok{(}\KeywordTok{aes}\NormalTok{(}\DataTypeTok{color=}\NormalTok{Diet))}
\end{Highlighting}
\end{Shaded}

\includegraphics[width=0.5\linewidth]{Rcourse_files/figure-latex/unnamed-chunk-212-1}

The \texttt{color} argument tells R to use the variable \texttt{Diet} as
the coloring. A legend is added by default. If we wanted a fixed color,
and not a variable dependent color, \texttt{color} would have been put
outside the \texttt{aes} function.

\begin{Shaded}
\begin{Highlighting}[]
\KeywordTok{ggplot}\NormalTok{(}\DataTypeTok{data =} \NormalTok{Milk, }\KeywordTok{aes}\NormalTok{(}\DataTypeTok{x=}\NormalTok{Time, }\DataTypeTok{y=}\NormalTok{protein)) +}
\StringTok{  }\KeywordTok{geom_point}\NormalTok{(}\DataTypeTok{color=}\StringTok{"green"}\NormalTok{)}
\end{Highlighting}
\end{Shaded}

\includegraphics[width=0.5\linewidth]{Rcourse_files/figure-latex/unnamed-chunk-213-1}

Let's save the \textbf{ggplot2} object so we can reuse it. Notice it is
not plotted.

\begin{Shaded}
\begin{Highlighting}[]
\NormalTok{p <-}\StringTok{ }\KeywordTok{ggplot}\NormalTok{(}\DataTypeTok{data =} \NormalTok{Milk, }\KeywordTok{aes}\NormalTok{(}\DataTypeTok{x=}\NormalTok{Time, }\DataTypeTok{y=}\NormalTok{protein)) +}
\StringTok{  }\KeywordTok{geom_point}\NormalTok{()}
\end{Highlighting}
\end{Shaded}

We can add \emph{layers} of new \emph{geoms} using the \texttt{+}
operator. Here, we add a smoothing line.

\begin{Shaded}
\begin{Highlighting}[]
\NormalTok{p +}\StringTok{ }\KeywordTok{geom_smooth}\NormalTok{(}\DataTypeTok{method =} \StringTok{'gam'}\NormalTok{)}
\end{Highlighting}
\end{Shaded}

\includegraphics[width=0.5\linewidth]{Rcourse_files/figure-latex/unnamed-chunk-215-1}

Things to note:

\begin{itemize}
\tightlist
\item
  The smoothing line is a layer added with the \texttt{geom\_smooth()}
  function.
\item
  Lacking any arguments, the new layer will inherit the \texttt{aes} of
  the original object, x and y variables in particular.
\end{itemize}

To split the plot along some variable, we use faceting, done with the
\texttt{facet\_wrap} function.

\begin{Shaded}
\begin{Highlighting}[]
\NormalTok{p +}\StringTok{ }\KeywordTok{facet_wrap}\NormalTok{(~Diet)}
\end{Highlighting}
\end{Shaded}

\includegraphics[width=0.5\linewidth]{Rcourse_files/figure-latex/unnamed-chunk-216-1}

Instead of faceting, we can add a layer of the mean of each
\texttt{Diet} subgroup, connected by lines.

\begin{Shaded}
\begin{Highlighting}[]
\NormalTok{p +}\StringTok{ }\KeywordTok{stat_summary}\NormalTok{(}\KeywordTok{aes}\NormalTok{(}\DataTypeTok{color=}\NormalTok{Diet), }\DataTypeTok{fun.y=}\StringTok{"mean"}\NormalTok{, }\DataTypeTok{geom=}\StringTok{"line"}\NormalTok{)}
\end{Highlighting}
\end{Shaded}

\includegraphics[width=0.5\linewidth]{Rcourse_files/figure-latex/unnamed-chunk-217-1}

Things to note:

\begin{itemize}
\tightlist
\item
  \texttt{stat\_summary} adds a statistical summary.
\item
  The summary is applied along \texttt{Diet} subgroups, because of the
  \texttt{color=Diet} aesthetic.
\item
  The summary to be applied is the mean, because of
  \texttt{fun.y="mean"}.
\item
  The group means are connected by lines, because of the
  \texttt{geom="line"} argument.
\end{itemize}

What layers can be added using the \textbf{geoms} family of functions?

\begin{itemize}
\tightlist
\item
  \texttt{geom\_bar}: bars with bases on the x-axis.
\item
  \texttt{geom\_boxplot}: boxes-and-whiskers.
\item
  \texttt{geom\_errorbar}: T-shaped error bars.
\item
  \texttt{geom\_histogram}: histogram.
\item
  \texttt{geom\_line}: lines.
\item
  \texttt{geom\_point}: points (scatterplot).
\item
  \texttt{geom\_ribbon}: bands spanning y-values across a range of
  x-values.
\item
  \texttt{geom\_smooth}: smoothed conditional means (e.g.~loess smooth).
\end{itemize}

To demonstrate the layers added with the \texttt{geoms\_*} functions, we
start with a histogram.

\begin{Shaded}
\begin{Highlighting}[]
\NormalTok{pro <-}\StringTok{ }\KeywordTok{ggplot}\NormalTok{(Milk, }\KeywordTok{aes}\NormalTok{(}\DataTypeTok{x=}\NormalTok{protein))}
\NormalTok{pro +}\StringTok{ }\KeywordTok{geom_histogram}\NormalTok{(}\DataTypeTok{bins=}\DecValTok{30}\NormalTok{)}
\end{Highlighting}
\end{Shaded}

\includegraphics[width=0.5\linewidth]{Rcourse_files/figure-latex/unnamed-chunk-218-1}

A bar plot.

\begin{Shaded}
\begin{Highlighting}[]
\KeywordTok{ggplot}\NormalTok{(Milk, }\KeywordTok{aes}\NormalTok{(}\DataTypeTok{x=}\NormalTok{Diet)) +}
\StringTok{  }\KeywordTok{geom_bar}\NormalTok{()}
\end{Highlighting}
\end{Shaded}

\includegraphics[width=0.5\linewidth]{Rcourse_files/figure-latex/unnamed-chunk-219-1}

A scatter plot.

\begin{Shaded}
\begin{Highlighting}[]
\NormalTok{tp <-}\StringTok{ }\KeywordTok{ggplot}\NormalTok{(Milk, }\KeywordTok{aes}\NormalTok{(}\DataTypeTok{x=}\NormalTok{Time, }\DataTypeTok{y=}\NormalTok{protein))}
\NormalTok{tp +}\StringTok{ }\KeywordTok{geom_point}\NormalTok{()}
\end{Highlighting}
\end{Shaded}

\includegraphics[width=0.5\linewidth]{Rcourse_files/figure-latex/unnamed-chunk-220-1}

A smooth regression plot, reusing the \texttt{tp} object.

\begin{Shaded}
\begin{Highlighting}[]
\NormalTok{tp +}\StringTok{ }\KeywordTok{geom_smooth}\NormalTok{(}\DataTypeTok{method=}\StringTok{'gam'}\NormalTok{)}
\end{Highlighting}
\end{Shaded}

\includegraphics[width=0.5\linewidth]{Rcourse_files/figure-latex/unnamed-chunk-221-1}

And now, a simple line plot, reusing the \texttt{tp} object, and
connecting lines along \texttt{Cow}.

\begin{Shaded}
\begin{Highlighting}[]
\NormalTok{tp +}\StringTok{ }\KeywordTok{geom_line}\NormalTok{(}\KeywordTok{aes}\NormalTok{(}\DataTypeTok{group=}\NormalTok{Cow))}
\end{Highlighting}
\end{Shaded}

\includegraphics[width=0.5\linewidth]{Rcourse_files/figure-latex/unnamed-chunk-222-1}

The line plot is completely incomprehensible. Better look at boxplots
along time (even if omitting the \texttt{Cow} information).

\begin{Shaded}
\begin{Highlighting}[]
\NormalTok{tp +}\StringTok{ }\KeywordTok{geom_boxplot}\NormalTok{(}\KeywordTok{aes}\NormalTok{(}\DataTypeTok{group=}\NormalTok{Time))}
\end{Highlighting}
\end{Shaded}

\includegraphics[width=0.5\linewidth]{Rcourse_files/figure-latex/unnamed-chunk-223-1}

We can do some statistics for each subgroup. The following will compute
the mean and standard errors of \texttt{protein} at each time point.

\begin{Shaded}
\begin{Highlighting}[]
\KeywordTok{ggplot}\NormalTok{(Milk, }\KeywordTok{aes}\NormalTok{(}\DataTypeTok{x=}\NormalTok{Time, }\DataTypeTok{y=}\NormalTok{protein)) +}
\StringTok{  }\KeywordTok{stat_summary}\NormalTok{(}\DataTypeTok{fun.data =} \StringTok{'mean_se'}\NormalTok{)}
\end{Highlighting}
\end{Shaded}

\includegraphics[width=0.5\linewidth]{Rcourse_files/figure-latex/unnamed-chunk-224-1}

Some popular statistical summaries, have gained their own functions:

\begin{itemize}
\tightlist
\item
  \texttt{mean\_cl\_boot}: mean and bootstrapped confidence interval
  (default 95\%).
\item
  \texttt{mean\_cl\_normal}: mean and Gaussian (t-distribution based)
  confidence interval (default 95\%).
\item
  \texttt{mean\_dsl}: mean plus or minus standard deviation times some
  constant (default constant=2).
\item
  \texttt{median\_hilow}: median and outer quantiles (default outer
  quantiles = 0.025 and 0.975).
\end{itemize}

For less popular statistical summaries, we may specify the statistical
function in \texttt{stat\_summary}. The median is a first example.

\begin{Shaded}
\begin{Highlighting}[]
\KeywordTok{ggplot}\NormalTok{(Milk, }\KeywordTok{aes}\NormalTok{(}\DataTypeTok{x=}\NormalTok{Time, }\DataTypeTok{y=}\NormalTok{protein)) +}
\StringTok{  }\KeywordTok{stat_summary}\NormalTok{(}\DataTypeTok{fun.y=}\StringTok{"median"}\NormalTok{, }\DataTypeTok{geom=}\StringTok{"point"}\NormalTok{)}
\end{Highlighting}
\end{Shaded}

\includegraphics[width=0.5\linewidth]{Rcourse_files/figure-latex/unnamed-chunk-225-1}

We can also define our own statistical summaries.

\begin{Shaded}
\begin{Highlighting}[]
\NormalTok{medianlog <-}\StringTok{ }\NormalTok{function(y) \{}\KeywordTok{median}\NormalTok{(}\KeywordTok{log}\NormalTok{(y))\}}
\KeywordTok{ggplot}\NormalTok{(Milk, }\KeywordTok{aes}\NormalTok{(}\DataTypeTok{x=}\NormalTok{Time, }\DataTypeTok{y=}\NormalTok{protein)) +}
\StringTok{  }\KeywordTok{stat_summary}\NormalTok{(}\DataTypeTok{fun.y=}\StringTok{"medianlog"}\NormalTok{, }\DataTypeTok{geom=}\StringTok{"line"}\NormalTok{)}
\end{Highlighting}
\end{Shaded}

\includegraphics[width=0.5\linewidth]{Rcourse_files/figure-latex/unnamed-chunk-226-1}

\textbf{Faceting} allows to split the plotting along some variable.
\texttt{face\_wrap} tells R to compute the number of columns and rows of
plots automatically.

\begin{Shaded}
\begin{Highlighting}[]
\KeywordTok{ggplot}\NormalTok{(Milk, }\KeywordTok{aes}\NormalTok{(}\DataTypeTok{x=}\NormalTok{protein, }\DataTypeTok{color=}\NormalTok{Diet)) +}
\StringTok{  }\KeywordTok{geom_density}\NormalTok{() +}
\StringTok{  }\KeywordTok{facet_wrap}\NormalTok{(~Time)}
\end{Highlighting}
\end{Shaded}

\includegraphics[width=0.5\linewidth]{Rcourse_files/figure-latex/unnamed-chunk-227-1}

\texttt{facet\_grid} forces the plot to appear allow rows or columns,
using the \texttt{\textasciitilde{}} syntax.

\begin{Shaded}
\begin{Highlighting}[]
\KeywordTok{ggplot}\NormalTok{(Milk, }\KeywordTok{aes}\NormalTok{(}\DataTypeTok{x=}\NormalTok{Time, }\DataTypeTok{y=}\NormalTok{protein)) +}
\StringTok{  }\KeywordTok{geom_point}\NormalTok{() +}
\StringTok{  }\KeywordTok{facet_grid}\NormalTok{(Diet~.)}
\end{Highlighting}
\end{Shaded}

\includegraphics[width=0.5\linewidth]{Rcourse_files/figure-latex/unnamed-chunk-228-1}

To control the looks of the plot, \textbf{ggplot2} uses \textbf{themes}.

\begin{Shaded}
\begin{Highlighting}[]
\KeywordTok{ggplot}\NormalTok{(Milk, }\KeywordTok{aes}\NormalTok{(}\DataTypeTok{x=}\NormalTok{Time, }\DataTypeTok{y=}\NormalTok{protein)) +}
\StringTok{  }\KeywordTok{geom_point}\NormalTok{() +}
\StringTok{  }\KeywordTok{theme}\NormalTok{(}\DataTypeTok{panel.background=}\KeywordTok{element_rect}\NormalTok{(}\DataTypeTok{fill=}\StringTok{"lightblue"}\NormalTok{))}
\end{Highlighting}
\end{Shaded}

\includegraphics[width=0.5\linewidth]{Rcourse_files/figure-latex/unnamed-chunk-229-1}

\begin{Shaded}
\begin{Highlighting}[]
\KeywordTok{ggplot}\NormalTok{(Milk, }\KeywordTok{aes}\NormalTok{(}\DataTypeTok{x=}\NormalTok{Time, }\DataTypeTok{y=}\NormalTok{protein)) +}
\StringTok{  }\KeywordTok{geom_point}\NormalTok{() +}
\StringTok{  }\KeywordTok{theme}\NormalTok{(}\DataTypeTok{panel.background=}\KeywordTok{element_blank}\NormalTok{(),}
        \DataTypeTok{axis.title.x=}\KeywordTok{element_blank}\NormalTok{())}
\end{Highlighting}
\end{Shaded}

\includegraphics[width=0.5\linewidth]{Rcourse_files/figure-latex/unnamed-chunk-230-1}

Saving plots can be done using the \texttt{pdf} function, but possibly
easier with the \texttt{ggsave} function.

Finally, what every user of \textbf{ggplot2} constantly uses, is the
online documentation at \url{http://docs.ggplot2.org}.

\section{Interactive Graphics}\label{interactive-graphics}

{[}TODO: improve intro{]}

As already mentioned, the recent and dramatic advancement in interactive
visualization was made possible by the advances in web technologies, and
the \href{https://d3js.org/}{D3.JS} JavaScript library in particular.
This is because it allows developers to rely on existing libraries
designed for web browsing instead of re-implementing interactive
visualizations. These libraries are more visually pleasing, and
computationally efficient, than anything they could have developed
themselves.

Some noteworthy interactive plotting systems are the following:

\begin{itemize}
\item
  \textbf{plotly}: The \textbf{plotly} package \citep{plotly} uses the
  (brilliant!) visualization framework of the \href{}{Plotly} company to
  provide local, or web-publishable, interactive graphics.
\item
  \textbf{dygraphs}: The \href{http://dygraphs.com/}{dygraphs}
  JavaScript library is intended for interactive visualization of time
  series (\texttt{xts} class objects). The \textbf{dygraphs} R package
  is an interface allowing the plotting of R objects with this library.
  For more information see
  \href{https://rstudio.github.io/dygraphs/}{here}.
\item
  \textbf{rCharts}: If you like the \textbf{lattice} plotting system,
  the \textbf{rCharts} package will allow you to produce interactive
  plots from R using the \textbf{lattice} syntax. For more information
  see \href{https://ramnathv.github.io/rCharts/}{here}.
\item
  \textbf{clickme}: Very similar to \textbf{rCharts}.
\item
  \textbf{googleVis}: TODO
\item
  Highcharter: TODO
\item
  Rbokeh: TODO
\item
  \textbf{HTML Widgets}: The \textbf{htmlwidgets} package does not
  provide visualization, but rather, it facilitates the creation of new
  interactive visualizations. This is because it handles all the
  technical details that are required to use R output within JavaScript
  visualization libraries. It is available
  \href{http://www.htmlwidgets.org/}{here}, with a demo gallery
  \href{http://gallery.htmlwidgets.org/}{here}.
\end{itemize}

\subsection{Plotly}\label{plotly}

TODO: remove messages

\begin{Shaded}
\begin{Highlighting}[]
\KeywordTok{library}\NormalTok{(plotly)}
\KeywordTok{set.seed}\NormalTok{(}\DecValTok{100}\NormalTok{)}
\NormalTok{d <-}\StringTok{ }\NormalTok{diamonds[}\KeywordTok{sample}\NormalTok{(}\KeywordTok{nrow}\NormalTok{(diamonds), }\DecValTok{1000}\NormalTok{), ]}
\KeywordTok{plot_ly}\NormalTok{(d, }\DataTypeTok{x =} \NormalTok{~carat, }\DataTypeTok{y =} \NormalTok{~price, }\DataTypeTok{color =} \NormalTok{~carat,}
        \DataTypeTok{size =} \NormalTok{~carat, }\DataTypeTok{text =} \NormalTok{~}\KeywordTok{paste}\NormalTok{(}\StringTok{"Clarity: "}\NormalTok{, clarity))}
\end{Highlighting}
\end{Shaded}

\begin{verbatim}
## No trace type specified:
##   Based on info supplied, a 'scatter' trace seems appropriate.
##   Read more about this trace type -> https://plot.ly/r/reference/#scatter
\end{verbatim}

\begin{verbatim}
## No scatter mode specifed:
##   Setting the mode to markers
##   Read more about this attribute -> https://plot.ly/r/reference/#scatter-mode
\end{verbatim}

\includegraphics[width=0.5\linewidth]{Rcourse_files/figure-latex/unnamed-chunk-231-1}

If you are comfortable with \textbf{ggplot2}, you may use the
\textbf{ggplot2} syntax, and export the final result to \textbf{plotly}.

\begin{Shaded}
\begin{Highlighting}[]
\NormalTok{p <-}\StringTok{ }\KeywordTok{ggplot}\NormalTok{(}\DataTypeTok{data =} \NormalTok{d, }\KeywordTok{aes}\NormalTok{(}\DataTypeTok{x =} \NormalTok{carat, }\DataTypeTok{y =} \NormalTok{price)) +}
\StringTok{  }\KeywordTok{geom_smooth}\NormalTok{(}\KeywordTok{aes}\NormalTok{(}\DataTypeTok{colour =} \NormalTok{cut, }\DataTypeTok{fill =} \NormalTok{cut), }\DataTypeTok{method =} \StringTok{'loess'}\NormalTok{) +}\StringTok{ }
\StringTok{  }\KeywordTok{facet_wrap}\NormalTok{(~}\StringTok{ }\NormalTok{cut)}

\KeywordTok{ggplotly}\NormalTok{(p)}
\end{Highlighting}
\end{Shaded}

\includegraphics[width=0.5\linewidth]{Rcourse_files/figure-latex/unnamed-chunk-232-1}

For more on \textbf{plotly} see \url{https://plot.ly/r/}.

\subsection{HTML Widgets}\label{html-widgets}

TODO

\section{Bibliographic Notes}\label{bibliographic-notes-8}

For the \textbf{graphics} package, see \citet{Rlanguage}. For
\textbf{ggplot2} see \citet{ggplot2}. A
\href{https://www.youtube.com/watch?v=9Objw9Tvhb4\&feature=youtu.be}{video}
by one of my heroes, \href{http://www.bcaffo.com/}{Brian Caffo},
discussing \textbf{graphics} vs. \textbf{ggplot2}.

\section{Practice Yourself}\label{practice-yourself-9}

\chapter{Reports}\label{report}

If you have ever written a report, you are probably familiar with the
process of preparing your figures in some software, say R, and then
copy-pasting into your text editor, say MS Word. While very popular,
this process is both tedious, and plain painful if your data has changed
and you need to update the report. Wouldn't it be nice if you could
produce figures and numbers from within the text of the report, and
everything else would be automated? It turns out it is possible. There
are actually several systems in R that allow this. We start with a brief
review.

\begin{enumerate}
\def\labelenumi{\arabic{enumi}.}
\item
  \textbf{Sweave}: \emph{LaTeX} is a markup language that compiles to
  \emph{Tex} programs that compile, in turn, to documents (typically PS
  or PDFs). If you never heard of it, it may be because you were born
  the the MS Windows+MS Word era. You should know, however, that
  \emph{LaTeX} was there much earlier, when computers were mainframes
  with text-only graphic devices. You should also know that \emph{LaTeX}
  is still very popular (in some communities) due to its very rich
  markup syntax, and beautiful output. \emph{Sweave}
  \citep{leisch2002sweave} is a compiler for \emph{LaTeX} that allows
  you do insert R commands in the \emph{LaTeX} source file, and get the
  result as part of the outputted PDF. It's name suggests just that: it
  allows to weave S\footnote{Recall, S was the original software from
    which R evolved.} output into the document, thus, Sweave.
\item
  \textbf{knitr}: \emph{Markdown} is a text editing syntax that, unlike
  \emph{LaTeX}, is aimed to be human-readable, but also compilable by a
  machine. If you ever tried to read HTML or Latex source files, you may
  understand why human-readability is a desirable property. There are
  many \emph{markdown} compilers. One of the most popular is Pandoc,
  written by the Berkeley philosopher(!) Jon MacFarlane. The
  availability of Pandoc gave \href{https://yihui.name/}{Yihui Xie}, a
  name to remember, the idea that it is time for Sweave to evolve. Yihui
  thus wrote \textbf{knitr} \citep{xie2015dynamic}, which allows to
  write human readable text in \emph{Rmarkdown}, a superset of
  \emph{markdown}, compile it with R and the compile it with Pandoc.
  Because Pandoc can compile to PDF, but also to HTML, and DOCX, among
  others, this means that you can write in Rmarkdown, and get output in
  almost all text formats out there.
\item
  \textbf{bookdown}: \textbf{Bookdown} \citep{xie2016bookdown} is an
  evolution of \textbf{knitr}, also written by Yihui Xie, now working
  for RStudio. The text you are now reading was actually written in
  \textbf{bookdown}. It deals with the particular needs of writing large
  documents, and cross referencing in particular (which is very
  challenging if you want the text to be human readable).
\item
  \textbf{Shiny}: Shiny is essentially a framework for quick
  web-development. It includes (i) an abstraction layer that specifies
  the layout of a web-site which is our report, (ii) the command to
  start a web server to deliver the site. For more on Shiny see
  \citet{shiny}.
\end{enumerate}

\section{knitr}\label{knitr}

\subsection{Installation}\label{installation}

To run \textbf{knitr} you will need to install the package.

\begin{Shaded}
\begin{Highlighting}[]
\KeywordTok{install.packages}\NormalTok{(}\StringTok{'knitr'}\NormalTok{)}
\end{Highlighting}
\end{Shaded}

It is also recommended that you use it within RStudio
(version\textgreater{}0.96), where you can easily create a new
\texttt{.Rmd} file.

\subsection{Pandoc Markdown}\label{pandoc-markdown}

Because \textbf{knitr} builds upon \emph{Pandoc markdown}, here is a
simple example of markdown text, to be used in a \texttt{.Rmd} file,
which can be created using the \emph{File-\textgreater{} New File
-\textgreater{} R Markdown} menu of RStudio.

Underscores or asterisks for \texttt{\_italics1\_} and
\texttt{*italics2*} return \emph{italics1} and \emph{italics2}. Double
underscores or asterisks for \texttt{\_\_bold1\_\_} and
\texttt{**bold2**} return \textbf{bold1} and \textbf{bold2}. Subscripts
are enclosed in tildes,
\texttt{like\textasciitilde{}this\textasciitilde{}}
(like\textsubscript{this}), and superscripts are enclosed in carets
\texttt{like\^{}this\^{}} (like\textsuperscript{this}).

For links use \texttt{{[}text{]}(link)}, like
\texttt{{[}my\ site{]}(www.john-ros.com)}. An image is the same as a
link, starting with an exclamation, like this
\texttt{!{[}image\ caption{]}(image\ path)}.

An itemized list simply starts with hyphens preceeded by a blank line
(don't forget that!):

\begin{verbatim}

- bullet
- bullet
    - second level bullet
    - second level bullet
\end{verbatim}

Compiles into:

\begin{itemize}
\tightlist
\item
  bullet
\item
  bullet

  \begin{itemize}
  \tightlist
  \item
    second level bullet
  \item
    second level bullet
  \end{itemize}
\end{itemize}

An enumerated list starts with an arbitrary number:

\begin{verbatim}
1. number
1. number
    1. second level number
    1. second level number
\end{verbatim}

Compiles into:

\begin{enumerate}
\def\labelenumi{\arabic{enumi}.}
\tightlist
\item
  number
\item
  number

  \begin{enumerate}
  \def\labelenumii{\arabic{enumii}.}
  \tightlist
  \item
    second level number
  \item
    second level number
  \end{enumerate}
\end{enumerate}

For more on markdown see
\url{https://bookdown.org/yihui/bookdown/markdown-syntax.html}.

\subsection{Rmarkdown}\label{rmarkdown}

\emph{Rmarkdown}, is an extension of \emph{markdown} due to RStudio,
that allows to incorporate R expressions in the text, that will be
evaluated at the time of compilation, and the output automatically
inserted in the outputted text. The output can be a \texttt{.PDF},
\texttt{.DOCX}, \texttt{.HTML} or others, thanks to the power of
\emph{Pandoc}.

The start of a code chunk is indicated by three backticks and the end of
a code chunk is indicated by three backticks. Here is an example.

\begin{verbatim}
```{r  eval=FALSE}
rnorm(10)
```
\end{verbatim}

This chunk will compile to the following output (after setting
\texttt{eval=FALSE} to \texttt{eval=TRUE}):

\begin{Shaded}
\begin{Highlighting}[]
\KeywordTok{rnorm}\NormalTok{(}\DecValTok{10}\NormalTok{)}
\end{Highlighting}
\end{Shaded}

\begin{verbatim}
##  [1] -1.4462875  0.3158558 -0.3427475 -1.9313531  0.2428210 -0.3627679
##  [7]  2.4327289  0.5920912 -0.5762008  0.4066282
\end{verbatim}

Things to note:

\begin{itemize}
\tightlist
\item
  The evaluated expression is added in a chunk of highlighted text,
  before the R output.
\item
  The output is prefixed with \texttt{\#\#}.
\item
  The \texttt{eval=} argument is not required, since it is set to
  \texttt{eval=TRUE} by default. It does demonstrate how to set the
  options of the code chunk.
\end{itemize}

In the same way, we may add a plot:

\begin{verbatim}
```{r  eval=FALSE}
plot(rnorm(10))
```
\end{verbatim}

which compiles into

\begin{Shaded}
\begin{Highlighting}[]
\KeywordTok{plot}\NormalTok{(}\KeywordTok{rnorm}\NormalTok{(}\DecValTok{10}\NormalTok{))}
\end{Highlighting}
\end{Shaded}

\includegraphics[width=0.5\linewidth]{Rcourse_files/figure-latex/unnamed-chunk-237-1}

TODO: more code chunk options.

You can also call r expressions inline. This is done with a single tick
and the \texttt{r} argument. For instance:

\begin{quote}
\texttt{`r\ rnorm(1)`} is a random Gaussian
\end{quote}

will output

\begin{quote}
0.3378953 is a random Gaussian.
\end{quote}

\subsection{Compiling}\label{compiling}

Once you have your \texttt{.Rmd} file written in RMarkdown,
\textbf{knitr} will take care of the compilation for you. You can call
the \texttt{knitr::knitr} function directly from some \texttt{.R} file,
or more conveniently, use the RStudio (0.96) Knit button above the text
editing window. The location of the output file will be presented in the
console.

\section{bookdown}\label{bookdown}

As previously stated, \textbf{bookdown} is an extension of
\textbf{knitr} intended for documents more complicated than simple
reports-- such as books. Just like \textbf{knitr}, the writing is done
in \textbf{RMarkdown}. Being an extension of \textbf{knitr},
\textbf{bookdown} does allow some markdowns that are not supported by
other compilers. In particular, it has a more powerful cross referencing
system.

\section{Shiny}\label{shiny}

\textbf{Shiny} \citep{shiny} is different than the previous systems,
because it sets up an interactive web-site, and not a static file. The
power of Shiny is that the layout of the web-site, and the settings of
the web-server, is made with several simple R commands, with no need for
web-programming. Once you have your app up and running, you can setup
your own Shiny server on the web, or publish it via
\href{https://www.shinyapps.io/}{Shinyapps.io}. The freemium versions of
the service can deal with a small amount of traffic. If you expect a lot
of traffic, you will probably need the paid versions.

\subsection{Installation}\label{installation-1}

To setup your first Shiny app, you will need the \textbf{shiny} package.
You will probably want RStudio, which facilitates the process.

\begin{Shaded}
\begin{Highlighting}[]
\KeywordTok{install.packages}\NormalTok{(}\StringTok{'shiny'}\NormalTok{)}
\end{Highlighting}
\end{Shaded}

Once installed, you can run an example app to get the feel of it.

\begin{Shaded}
\begin{Highlighting}[]
\KeywordTok{library}\NormalTok{(shiny)}
\KeywordTok{runExample}\NormalTok{(}\StringTok{"01_hello"}\NormalTok{)}
\end{Highlighting}
\end{Shaded}

Remember to press the \textbf{Stop} button in RStudio to stop the
web-server, and get back to RStudio.

\subsection{The Basics of Shiny}\label{the-basics-of-shiny}

Every Shiny app has two main building blocks.

\begin{enumerate}
\def\labelenumi{\arabic{enumi}.}
\tightlist
\item
  A user interface, specified via the \texttt{ui.R} file in the app's
  directory.
\item
  A server side, specified via the \texttt{server.R} file, in the app's
  directory.
\end{enumerate}

You can run the app via the \textbf{RunApp} button in the RStudio
interface, of by calling the app's directory with the \texttt{shinyApp}
or \texttt{runApp} functions-- the former designed for single-app
projects, and the latter, for multiple app projects.

\begin{Shaded}
\begin{Highlighting}[]
\NormalTok{shiny::}\KeywordTok{runApp}\NormalTok{(}\StringTok{"my_app"}\NormalTok{)}
\end{Highlighting}
\end{Shaded}

The site's layout, is specified via \emph{layout functions} in the
\texttt{iu.R} file. For instance, the function \texttt{sidebarLayout},
as the name suggest, will create a sidebar. More layouts are detailed in
the \href{http://shiny.rstudio.com/articles/layout-guide.html}{layout
guide}.

The active elements in the UI, that control your report, are known as
\emph{widgets}. Each widget will have a unique \texttt{inputId} so that
it's values can be sent from the UI to the server. More about widgets,
in the
\href{http://shiny.rstudio.com/gallery/widget-gallery.html}{widget
gallery}.

The \texttt{inputId} on the UI are mapped to \texttt{input} arguments on
the server side. The value of the \texttt{mytext} \texttt{inputId} can
be queried by the server using \texttt{input\$mytext}. These are called
\emph{reactive values}. The way the server ``listens'' to the UI, is
governed by a set of functions that must wrap the \texttt{input} object.
These are the \texttt{observe}, \texttt{reactive}, and
\texttt{reactive*} class of functions.

With \texttt{observe} the server will get triggered when any of the
reactive values change. With \texttt{observeEvent} the server will only
be triggered by specified reactive values. Using \texttt{observe} is
easier, and \texttt{observeEvent} is more prudent programming.

A \texttt{reactive} function is a function that gets triggered when a
reactive element changes. It is defined on the server side, and reside
within an \texttt{observe} function.

We now analyze the \texttt{1\_Hello} app using these ideas. Here is the
\texttt{io.R} file.

\begin{Shaded}
\begin{Highlighting}[]
\KeywordTok{library}\NormalTok{(shiny)}

\KeywordTok{shinyUI}\NormalTok{(}\KeywordTok{fluidPage}\NormalTok{(}

  \KeywordTok{titlePanel}\NormalTok{(}\StringTok{"Hello Shiny!"}\NormalTok{),}

  \KeywordTok{sidebarLayout}\NormalTok{(}
    \KeywordTok{sidebarPanel}\NormalTok{(}
      \KeywordTok{sliderInput}\NormalTok{(}\DataTypeTok{inputId =} \StringTok{"bins"}\NormalTok{,}
                  \DataTypeTok{label =} \StringTok{"Number of bins:"}\NormalTok{, }
                  \DataTypeTok{min =} \DecValTok{1}\NormalTok{,}
                  \DataTypeTok{max =} \DecValTok{50}\NormalTok{,}
                  \DataTypeTok{value =} \DecValTok{30}\NormalTok{)}
    \NormalTok{),}

    \KeywordTok{mainPanel}\NormalTok{(}
      \KeywordTok{plotOutput}\NormalTok{(}\DataTypeTok{outputId =} \StringTok{"distPlot"}\NormalTok{)}
    \NormalTok{)}
  \NormalTok{)}
\NormalTok{))}
\end{Highlighting}
\end{Shaded}

Here is the \texttt{server.R} file:

\begin{Shaded}
\begin{Highlighting}[]
\KeywordTok{library}\NormalTok{(shiny)}

\KeywordTok{shinyServer}\NormalTok{(function(input, output) \{}

  \NormalTok{output$distPlot <-}\StringTok{ }\KeywordTok{renderPlot}\NormalTok{(\{}
    \NormalTok{x    <-}\StringTok{ }\NormalTok{faithful[, }\DecValTok{2}\NormalTok{]  }\CommentTok{# Old Faithful Geyser data}
    \NormalTok{bins <-}\StringTok{ }\KeywordTok{seq}\NormalTok{(}\KeywordTok{min}\NormalTok{(x), }\KeywordTok{max}\NormalTok{(x), }\DataTypeTok{length.out =} \NormalTok{input$bins +}\StringTok{ }\DecValTok{1}\NormalTok{)}

    \KeywordTok{hist}\NormalTok{(x, }\DataTypeTok{breaks =} \NormalTok{bins, }\DataTypeTok{col =} \StringTok{'darkgray'}\NormalTok{, }\DataTypeTok{border =} \StringTok{'white'}\NormalTok{)}
  \NormalTok{\})}
\NormalTok{\})}
\end{Highlighting}
\end{Shaded}

Things to note:

\begin{itemize}
\tightlist
\item
  \texttt{ShinyUI} is a (deprecated) wrapper for the UI.
\item
  \texttt{fluidPage} ensures that the proportions of the elements adapt
  to the window side, thus, are fluid.
\item
  The building blocks of the layout are a title, and the body. The title
  is governed by \texttt{titlePanel}, and the body is governed by
  \texttt{sidebarLayout}. The \texttt{sidebarLayout} includes the
  \texttt{sidebarPanel} to control the sidebar, and the
  \texttt{mainPanel} for the main panel.
\item
  \texttt{sliderInput} calls a widget with a slider. Its
  \texttt{inputId} is \texttt{bins}, which is later used by the server
  within the \texttt{renderPlot} reactive function.
\item
  \texttt{plotOutput} specifies that the content of the
  \texttt{mainPanel} is a plot (\texttt{textOutput} for text). This
  expectation is satisfied on the server side with the
  \texttt{renderPlot} function (\texttt{renderText}).
\item
  \texttt{shinyServer} is a (deprecated) wrapper function for the
  server.
\item
  The server runs a function with an \texttt{input} and an
  \texttt{output}. The elements of \texttt{input} are the
  \texttt{inputId}s from the UI. The elements of the \texttt{output}
  will be called by the UI using their \texttt{outputId}.
\end{itemize}

This is the output.

\begin{Shaded}
\begin{Highlighting}[]
\NormalTok{knitr::}\KeywordTok{include_url}\NormalTok{(}\StringTok{'http://shiny.rstudio.com/gallery/example-01-hello.html'}\NormalTok{)}
\end{Highlighting}
\end{Shaded}

\href{http://shiny.rstudio.com/gallery/example-01-hello.html}{\includegraphics[width=0.5\linewidth]{Rcourse_files/figure-latex/unnamed-chunk-243-1} }

Here is another example, taken from the RStudio
\href{https://github.com/rstudio/shiny-examples/tree/master/006-tabsets}{Shiny
examples}.

\texttt{ui.R}:

\begin{Shaded}
\begin{Highlighting}[]
\KeywordTok{library}\NormalTok{(shiny)}

\KeywordTok{fluidPage}\NormalTok{(}
    
  \KeywordTok{titlePanel}\NormalTok{(}\StringTok{"Tabsets"}\NormalTok{),}
  
  \KeywordTok{sidebarLayout}\NormalTok{(}
    \KeywordTok{sidebarPanel}\NormalTok{(}
      \KeywordTok{radioButtons}\NormalTok{(}\DataTypeTok{inputId =} \StringTok{"dist"}\NormalTok{, }
                   \DataTypeTok{label =} \StringTok{"Distribution type:"}\NormalTok{,}
                   \KeywordTok{c}\NormalTok{(}\StringTok{"Normal"} \NormalTok{=}\StringTok{ "norm"}\NormalTok{,}
                     \StringTok{"Uniform"} \NormalTok{=}\StringTok{ "unif"}\NormalTok{,}
                     \StringTok{"Log-normal"} \NormalTok{=}\StringTok{ "lnorm"}\NormalTok{,}
                     \StringTok{"Exponential"} \NormalTok{=}\StringTok{ "exp"}\NormalTok{)),}
      \KeywordTok{br}\NormalTok{(),}
      
      \KeywordTok{sliderInput}\NormalTok{(}\DataTypeTok{inputId =} \StringTok{"n"}\NormalTok{, }
                  \DataTypeTok{label =} \StringTok{"Number of observations:"}\NormalTok{, }
                   \DataTypeTok{value =} \DecValTok{500}\NormalTok{,}
                   \DataTypeTok{min =} \DecValTok{1}\NormalTok{, }
                   \DataTypeTok{max =} \DecValTok{1000}\NormalTok{)}
    \NormalTok{),}
    
    \KeywordTok{mainPanel}\NormalTok{(}
      \KeywordTok{tabsetPanel}\NormalTok{(}\DataTypeTok{type =} \StringTok{"tabs"}\NormalTok{, }
        \KeywordTok{tabPanel}\NormalTok{(}\DataTypeTok{title =} \StringTok{"Plot"}\NormalTok{, }\KeywordTok{plotOutput}\NormalTok{(}\DataTypeTok{outputId =} \StringTok{"plot"}\NormalTok{)), }
        \KeywordTok{tabPanel}\NormalTok{(}\DataTypeTok{title =} \StringTok{"Summary"}\NormalTok{, }\KeywordTok{verbatimTextOutput}\NormalTok{(}\DataTypeTok{outputId =} \StringTok{"summary"}\NormalTok{)), }
        \KeywordTok{tabPanel}\NormalTok{(}\DataTypeTok{title =} \StringTok{"Table"}\NormalTok{, }\KeywordTok{tableOutput}\NormalTok{(}\DataTypeTok{outputId =} \StringTok{"table"}\NormalTok{))}
      \NormalTok{)}
    \NormalTok{)}
  \NormalTok{)}
\NormalTok{)}
\end{Highlighting}
\end{Shaded}

\texttt{server.R}:

\begin{Shaded}
\begin{Highlighting}[]
\KeywordTok{library}\NormalTok{(shiny)}

\CommentTok{# Define server logic for random distribution application}
\NormalTok{function(input, output) \{}
  
  \NormalTok{data <-}\StringTok{ }\KeywordTok{reactive}\NormalTok{(\{}
    \NormalTok{dist <-}\StringTok{ }\NormalTok{switch(input$dist,}
                   \DataTypeTok{norm =} \NormalTok{rnorm,}
                   \DataTypeTok{unif =} \NormalTok{runif,}
                   \DataTypeTok{lnorm =} \NormalTok{rlnorm,}
                   \DataTypeTok{exp =} \NormalTok{rexp,}
                   \NormalTok{rnorm)}
    
    \KeywordTok{dist}\NormalTok{(input$n)}
  \NormalTok{\})}
  
  \NormalTok{output$plot <-}\StringTok{ }\KeywordTok{renderPlot}\NormalTok{(\{}
    \NormalTok{dist <-}\StringTok{ }\NormalTok{input$dist}
    \NormalTok{n <-}\StringTok{ }\NormalTok{input$n}
    
    \KeywordTok{hist}\NormalTok{(}\KeywordTok{data}\NormalTok{(), }\DataTypeTok{main=}\KeywordTok{paste}\NormalTok{(}\StringTok{'r'}\NormalTok{, dist, }\StringTok{'('}\NormalTok{, n, }\StringTok{')'}\NormalTok{, }\DataTypeTok{sep=}\StringTok{''}\NormalTok{))}
  \NormalTok{\})}
  
  \NormalTok{output$summary <-}\StringTok{ }\KeywordTok{renderPrint}\NormalTok{(\{}
    \KeywordTok{summary}\NormalTok{(}\KeywordTok{data}\NormalTok{())}
  \NormalTok{\})}
  
  \NormalTok{output$table <-}\StringTok{ }\KeywordTok{renderTable}\NormalTok{(\{}
    \KeywordTok{data.frame}\NormalTok{(}\DataTypeTok{x=}\KeywordTok{data}\NormalTok{())}
  \NormalTok{\})}
  
\NormalTok{\}}
\end{Highlighting}
\end{Shaded}

Things to note:

\begin{itemize}
\tightlist
\item
  We reused the \texttt{sidebarLayout}.
\item
  As the name suggests, \texttt{radioButtons} is a widget that produces
  radio buttons, above the \texttt{sliderInput} widget. Note the
  different \texttt{inputId}s.
\item
  Different widgets are separated in \texttt{sidebarPanel} by commas.
\item
  \texttt{br()} produces extra vertical spacing.
\item
  \texttt{tabsetPanel} produces tabs in the main output panel.
  \texttt{tabPanel} governs the content of each panel. Notice the use of
  various output functions
  (\texttt{plotOutput},\texttt{verbatimTextOutput},
  \texttt{tableOutput}) with corresponding \texttt{outputId}s.
\item
  In \texttt{server.R} we see the usual \texttt{function(input,output)}.
\item
  The \texttt{reactive} function tells the server the trigger the
  function whenever \texttt{input} changes.
\item
  The \texttt{output} object is constructed outside the
  \texttt{reactive} function. See how the elements of \texttt{output}
  correspond to the \texttt{outputId}s in the UI.
\end{itemize}

This is the output:

\begin{Shaded}
\begin{Highlighting}[]
\NormalTok{knitr::}\KeywordTok{include_url}\NormalTok{(}\StringTok{'https://shiny.rstudio.com/gallery/tabsets.html'}\NormalTok{)}
\end{Highlighting}
\end{Shaded}

\href{https://shiny.rstudio.com/gallery/tabsets.html}{\includegraphics[width=0.5\linewidth]{Rcourse_files/figure-latex/unnamed-chunk-246-1} }

\subsection{Beyond the Basics}\label{beyond-the-basics}

Now that we have seen the basics, we may consider extensions to the
basic report.

\subsubsection{Widgets}\label{widgets}

\begin{itemize}
\tightlist
\item
  \texttt{actionButton} Action Button.
\item
  \texttt{checkboxGroupInput} A group of check boxes.
\item
  \texttt{checkboxInput} A single check box.
\item
  \texttt{dateInput} A calendar to aid date selection.
\item
  \texttt{dateRangeInput} A pair of calendars for selecting a date
  range.
\item
  \texttt{fileInput} A file upload control wizard.
\item
  \texttt{helpText} Help text that can be added to an input form.
\item
  \texttt{numericInput} A field to enter numbers.
\item
  \texttt{radioButtons} A set of radio buttons.
\item
  \texttt{selectInput} A box with choices to select from.
\item
  \texttt{sliderInput} A slider bar.
\item
  \texttt{submitButton} A submit button.
\item
  \texttt{textInput} A field to enter text.
\end{itemize}

See examples
\href{https://shiny.rstudio.com/gallery/widget-gallery.html}{here}.

\begin{Shaded}
\begin{Highlighting}[]
\NormalTok{knitr::}\KeywordTok{include_url}\NormalTok{(}\StringTok{'https://shiny.rstudio.com/gallery/widget-gallery.html'}\NormalTok{)}
\end{Highlighting}
\end{Shaded}

\href{https://shiny.rstudio.com/gallery/widget-gallery.html}{\includegraphics[width=0.5\linewidth]{Rcourse_files/figure-latex/unnamed-chunk-247-1} }

\subsubsection{Output Elements}\label{output-elements}

The \texttt{ui.R} output types.

\begin{itemize}
\tightlist
\item
  \texttt{htmlOutput} raw HTML.
\item
  \texttt{imageOutput} image.
\item
  \texttt{plotOutput} plot.
\item
  \texttt{tableOutput} table.
\item
  \texttt{textOutput} text.
\item
  \texttt{uiOutput} raw HTML.
\item
  \texttt{verbatimTextOutput} text.
\end{itemize}

The corresponding \texttt{server.R} renderers.

\begin{itemize}
\tightlist
\item
  \texttt{renderImage} images (saved as a link to a source file)
\item
  \texttt{renderPlot} plots
\item
  \texttt{renderPrint} any printed output
\item
  \texttt{renderTable} data frame, matrix, other table like structures
\item
  \texttt{renderText} character strings
\item
  \texttt{renderUI} a Shiny tag object or HTML
\end{itemize}

Your Shiny app can use any R object. The things to remember:

\begin{itemize}
\tightlist
\item
  The working directory of the app is the location of \texttt{server.R}.
\item
  The code before \texttt{shinyServer} is run only once.
\item
  The code inside `\texttt{shinyServer} is run whenever a reactive is
  triggered, and may thus slow things.
\end{itemize}

To keep learning, see the RStudio's
\href{http://shiny.rstudio.com/tutorial/}{tutorial}, and the
Biblipgraphic notes herein.

\section{Flexdashboard}\label{flexdashboard}

\url{http://rmarkdown.rstudio.com/flexdashboard/}

TODO: write section

\section{Bibliographic Notes}\label{bibliographic-notes-9}

For RMarkdown see \href{http://rmarkdown.rstudio.com/}{here}. For
everything on \textbf{knitr} see
\href{https://yihui.name/knitr/}{Yihui's blog}, or the book
\citet{xie2015dynamic}. For a \textbf{bookdown} manual, see
\citet{xie2016bookdown}. For a Shiny manual, see \citet{shiny}, the
\href{http://shiny.rstudio.com/tutorial/}{RStudio tutorial}, or
\href{http://zevross.com/blog/2016/04/19/r-powered-web-applications-with-shiny-a-tutorial-and-cheat-sheet-with-40-example-apps/}{Zev
Ross's} excellent guide. Video tutorials are available
\href{https://www.rstudio.com/resources/webinars/shiny-developer-conference/}{here}.

\section{Practice Yourself}\label{practice-yourself-10}

\chapter{The Hadleyverse}\label{hadley}

The \emph{Hadleyverse}, short for ``Hadley Wickham's universe'', is a
set of packages that make it easier to handle data. If you are
developing packages, you should be careful since using these packages
may create many dependencies and compatibility issues. If you are
analyzing data, and the portability of your functions to other users,
machines, and operating systems is not of a concern, you will LOVE these
packages. The term Hadleyverse refers to \textbf{all} of Hadley's
packages, but here, we mention only a useful subset, which can be
collectively installed via the \textbf{tidyverse} package:

\begin{itemize}
\tightlist
\item
  \textbf{ggplot2} for data visualization. See the Plotting Chapter
  \ref{plotting}.
\item
  \textbf{dplyr} for data manipulation.
\item
  \textbf{tidyr} for data tidying.
\item
  \textbf{readr} for data import.
\item
  \textbf{stringr} for character strings.
\item
  \textbf{anytime} for time data.
\end{itemize}

\section{readr}\label{readr}

The \textbf{readr} package \citep{readr} replaces base functions for
importing and exporting data such as \texttt{read.table}. It is faster,
with a cleaner syntax.

We will not go into the details and refer the reader to the official
documentation
\href{http://readr.tidyverse.org/articles/readr.html}{here} and the
\href{http://r4ds.had.co.nz/data-import.html}{R for data sciecne} book.

\section{dplyr}\label{dplyr}

When you think of data frame operations, think \textbf{dplyr}
\citep{dplyr}. Notable utilities in the package include:

\begin{itemize}
\tightlist
\item
  \texttt{select()} Select columns from a data frame.
\item
  \texttt{filter()} Filter rows according to some condition(s).
\item
  \texttt{arrange()} Sort / Re-order rows in a data frame.
\item
  \texttt{mutate()} Create new columns or transform existing ones.
\item
  \texttt{group\_by()} Group a data frame by some factor(s) usually in
  conjunction to summary.
\item
  \texttt{summarize()} Summarize some values from the data frame or
  across groups.
\item
  \texttt{inner\_join(x,y,by="col")}return all rows from `x' where there
  are matching values in `x', and all columns from `x' and `y'. If there
  are multiple matches between `x' and `y', all combination of the
  matches are returned.
\item
  \texttt{left\_join(x,y,by="col")} return all rows from `x', and all
  columns from `x' and `y'. Rows in `x' with no match in `y' will have
  `NA' values in the new columns. If there are multiple matches between
  `x' and `y', all combinations of the matches are returned.
\item
  \texttt{right\_join(x,y,by="col")} return all rows from `y', and all
  columns from `x' and y. Rows in `y' with no match in `x' will have
  `NA' values in the new columns. If there are multiple matches between
  `x' and `y', all combinations of the matches are returned.
\item
  \texttt{anti\_join(x,y,by="col")} return all rows from `x' where there
  are not matching values in `y', keeping just columns from `x'.
\end{itemize}

The following example involve \texttt{data.frame} objects, but
\textbf{dplyr} can handle other classes. In particular
\texttt{data.table}s from the \textbf{data.table} package
\citep{datatable}, which is designed for very large data sets.

\textbf{dplyr} can work with data stored in a database. In which case,
it will convert your command to the appropriate SQL syntax, and issue it
to the database. This has the advantage that (a) you do not need to know
the specific SQL implementation of your database, and (b), you can enjoy
the optimized algorithms provided by the database supplier. For more on
this, see the
\href{https://cran.r-project.org/web/packages/dplyr/vignettes/databases.html}{databses
vignette}.

The following examples are taken from
\href{https://github.com/justmarkham/dplyr-tutorial/blob/master/dplyr-tutorial.Rmd}{Kevin
Markham}. The \texttt{nycflights13::flights} has delay data for US
flights.

\begin{Shaded}
\begin{Highlighting}[]
\KeywordTok{library}\NormalTok{(nycflights13)}
\NormalTok{flights}
\end{Highlighting}
\end{Shaded}

\begin{verbatim}
## # A tibble: 336,776 × 19
##     year month   day dep_time sched_dep_time dep_delay arr_time
##    <int> <int> <int>    <int>          <int>     <dbl>    <int>
## 1   2013     1     1      517            515         2      830
## 2   2013     1     1      533            529         4      850
## 3   2013     1     1      542            540         2      923
## 4   2013     1     1      544            545        -1     1004
## 5   2013     1     1      554            600        -6      812
## 6   2013     1     1      554            558        -4      740
## 7   2013     1     1      555            600        -5      913
## 8   2013     1     1      557            600        -3      709
## 9   2013     1     1      557            600        -3      838
## 10  2013     1     1      558            600        -2      753
## # ... with 336,766 more rows, and 12 more variables: sched_arr_time <int>,
## #   arr_delay <dbl>, carrier <chr>, flight <int>, tailnum <chr>,
## #   origin <chr>, dest <chr>, air_time <dbl>, distance <dbl>, hour <dbl>,
## #   minute <dbl>, time_hour <dttm>
\end{verbatim}

The data is of class \texttt{tbl\_df} which is an extension of the
\texttt{data.frame} class, designed for large data sets. Notice that the
printing of \texttt{flights} is short, even without calling the
\texttt{head} function. This is a feature of the \texttt{tbl\_df} class
( \texttt{print(data.frame)} would try to load all the data, thus take a
long time).

\begin{Shaded}
\begin{Highlighting}[]
\KeywordTok{class}\NormalTok{(flights) }\CommentTok{# a tbl_df is an extension of the data.frame class}
\end{Highlighting}
\end{Shaded}

\begin{verbatim}
## [1] "tbl_df"     "tbl"        "data.frame"
\end{verbatim}

Let's filter the observations from the first day of the first month.
Notice how much better (i.e.~readable) is the \textbf{dplyr} syntax,
with piping, compared to the basic syntax.

\begin{Shaded}
\begin{Highlighting}[]
\NormalTok{flights[flights$month ==}\StringTok{ }\DecValTok{1} \NormalTok{&}\StringTok{ }\NormalTok{flights$day ==}\StringTok{ }\DecValTok{1}\NormalTok{, ] }\CommentTok{# old style}

\KeywordTok{library}\NormalTok{(dplyr) }
\KeywordTok{filter}\NormalTok{(flights, month ==}\StringTok{ }\DecValTok{1}\NormalTok{, day ==}\StringTok{ }\DecValTok{1}\NormalTok{) }\CommentTok{#dplyr style}
\NormalTok{flights %>%}\StringTok{ }\KeywordTok{filter}\NormalTok{(month ==}\StringTok{ }\DecValTok{1}\NormalTok{, day ==}\StringTok{ }\DecValTok{1}\NormalTok{) }\CommentTok{# dplyr with piping.}
\end{Highlighting}
\end{Shaded}

More filtering.

\begin{Shaded}
\begin{Highlighting}[]
\KeywordTok{filter}\NormalTok{(flights, month ==}\StringTok{ }\DecValTok{1} \NormalTok{|}\StringTok{ }\NormalTok{month ==}\StringTok{ }\DecValTok{2}\NormalTok{) }\CommentTok{# First OR second month.}
\KeywordTok{slice}\NormalTok{(flights, }\DecValTok{1}\NormalTok{:}\DecValTok{10}\NormalTok{) }\CommentTok{# selects first ten rows.}

\KeywordTok{arrange}\NormalTok{(flights, year, month, day) }\CommentTok{# sort}
\KeywordTok{arrange}\NormalTok{(flights, }\KeywordTok{desc}\NormalTok{(arr_delay)) }\CommentTok{# sort descending}

\KeywordTok{select}\NormalTok{(flights, year, month, day) }\CommentTok{# select columns year, month, and day}
\KeywordTok{select}\NormalTok{(flights, year:day) }\CommentTok{# select column range}
\KeywordTok{select}\NormalTok{(flights, -(year:day)) }\CommentTok{# drop columns}
\KeywordTok{rename}\NormalTok{(flights, }\DataTypeTok{tail_num =} \NormalTok{tailnum) }\CommentTok{# rename column}

\CommentTok{# add a new computed colume}
\KeywordTok{mutate}\NormalTok{(flights,}
  \DataTypeTok{gain =} \NormalTok{arr_delay -}\StringTok{ }\NormalTok{dep_delay,}
  \DataTypeTok{speed =} \NormalTok{distance /}\StringTok{ }\NormalTok{air_time *}\StringTok{ }\DecValTok{60}\NormalTok{) }

\CommentTok{# you can refer to columns you just created! (gain)}
\KeywordTok{mutate}\NormalTok{(flights,}
  \DataTypeTok{gain =} \NormalTok{arr_delay -}\StringTok{ }\NormalTok{dep_delay,}
  \DataTypeTok{gain_per_hour =} \NormalTok{gain /}\StringTok{ }\NormalTok{(air_time /}\StringTok{ }\DecValTok{60}\NormalTok{)}
\NormalTok{)}

\CommentTok{# keep only new variables, not all data frame.}
\KeywordTok{transmute}\NormalTok{(flights,}
  \DataTypeTok{gain =} \NormalTok{arr_delay -}\StringTok{ }\NormalTok{dep_delay,}
  \DataTypeTok{gain_per_hour =} \NormalTok{gain /}\StringTok{ }\NormalTok{(air_time /}\StringTok{ }\DecValTok{60}\NormalTok{)}
\NormalTok{)}

\CommentTok{# simple statistics}
\KeywordTok{summarise}\NormalTok{(flights,}
  \DataTypeTok{delay =} \KeywordTok{mean}\NormalTok{(dep_delay, }\DataTypeTok{na.rm =} \OtherTok{TRUE}\NormalTok{)}
  \NormalTok{)}
\CommentTok{# random subsample }
\KeywordTok{sample_n}\NormalTok{(flights, }\DecValTok{10}\NormalTok{) }
\KeywordTok{sample_frac}\NormalTok{(flights, }\FloatTok{0.01}\NormalTok{) }
\end{Highlighting}
\end{Shaded}

We now perform operations on subgroups. we group observations along the
plane's tail number (\texttt{tailnum}), and compute the count, average
distance traveled, and average delay. We group with \texttt{group\_by},
and compute subgroup statistics with \texttt{summarise}.

\begin{Shaded}
\begin{Highlighting}[]
\NormalTok{by_tailnum <-}\StringTok{ }\KeywordTok{group_by}\NormalTok{(flights, tailnum)}

\NormalTok{delay <-}\StringTok{ }\KeywordTok{summarise}\NormalTok{(by_tailnum,}
  \DataTypeTok{count =} \KeywordTok{n}\NormalTok{(),}
  \DataTypeTok{avg.dist =} \KeywordTok{mean}\NormalTok{(distance, }\DataTypeTok{na.rm =} \OtherTok{TRUE}\NormalTok{),}
  \DataTypeTok{avg.delay =} \KeywordTok{mean}\NormalTok{(arr_delay, }\DataTypeTok{na.rm =} \OtherTok{TRUE}\NormalTok{))}

\NormalTok{delay}
\end{Highlighting}
\end{Shaded}

\begin{verbatim}
## # A tibble: 4,044 × 4
##    tailnum count avg.dist  avg.delay
##      <chr> <int>    <dbl>      <dbl>
## 1   D942DN     4 854.5000 31.5000000
## 2   N0EGMQ   371 676.1887  9.9829545
## 3   N10156   153 757.9477 12.7172414
## 4   N102UW    48 535.8750  2.9375000
## 5   N103US    46 535.1957 -6.9347826
## 6   N104UW    47 535.2553  1.8043478
## 7   N10575   289 519.7024 20.6914498
## 8   N105UW    45 524.8444 -0.2666667
## 9   N107US    41 528.7073 -5.7317073
## 10  N108UW    60 534.5000 -1.2500000
## # ... with 4,034 more rows
\end{verbatim}

We can group along several variables, with a hierarchy. We then collapse
the hierarchy one by one.

\begin{Shaded}
\begin{Highlighting}[]
\NormalTok{daily <-}\StringTok{ }\KeywordTok{group_by}\NormalTok{(flights, year, month, day)}
\NormalTok{per_day   <-}\StringTok{ }\KeywordTok{summarise}\NormalTok{(daily, }\DataTypeTok{flights =} \KeywordTok{n}\NormalTok{())}
\NormalTok{per_month <-}\StringTok{ }\KeywordTok{summarise}\NormalTok{(per_day, }\DataTypeTok{flights =} \KeywordTok{sum}\NormalTok{(flights))}
\NormalTok{per_year  <-}\StringTok{ }\KeywordTok{summarise}\NormalTok{(per_month, }\DataTypeTok{flights =} \KeywordTok{sum}\NormalTok{(flights))}
\end{Highlighting}
\end{Shaded}

Things to note:

\begin{itemize}
\tightlist
\item
  Every call to \texttt{summarise} collapses one level in the hierarchy
  of grouping. The output of \texttt{group\_by} recalls the hierarchy of
  aggregation, and collapses along this hierarchy.
\end{itemize}

We can use \textbf{dplyr} for two table operations, i.e., \emph{joins}.
For this, we join the flight data, with the airplane data in
\texttt{airplanes}.

\begin{Shaded}
\begin{Highlighting}[]
\KeywordTok{library}\NormalTok{(dplyr) }
\NormalTok{airlines  }
\end{Highlighting}
\end{Shaded}

\begin{verbatim}
## # A tibble: 16 × 2
##    carrier                        name
##      <chr>                       <chr>
## 1       9E           Endeavor Air Inc.
## 2       AA      American Airlines Inc.
## 3       AS        Alaska Airlines Inc.
## 4       B6             JetBlue Airways
## 5       DL        Delta Air Lines Inc.
## 6       EV    ExpressJet Airlines Inc.
## 7       F9      Frontier Airlines Inc.
## 8       FL AirTran Airways Corporation
## 9       HA      Hawaiian Airlines Inc.
## 10      MQ                   Envoy Air
## 11      OO       SkyWest Airlines Inc.
## 12      UA       United Air Lines Inc.
## 13      US             US Airways Inc.
## 14      VX              Virgin America
## 15      WN      Southwest Airlines Co.
## 16      YV          Mesa Airlines Inc.
\end{verbatim}

\begin{Shaded}
\begin{Highlighting}[]
\CommentTok{# select the subset of interesting flight data. }
\NormalTok{flights2 <-}\StringTok{ }\NormalTok{flights %>%}\StringTok{ }\KeywordTok{select}\NormalTok{(year:day, hour, origin, dest, tailnum, carrier) }

\CommentTok{# join on left table with automatic matching.}
\NormalTok{flights2 %>%}\StringTok{ }\KeywordTok{left_join}\NormalTok{(airlines) }
\end{Highlighting}
\end{Shaded}

\begin{verbatim}
## Joining, by = "carrier"
\end{verbatim}

\begin{verbatim}
## # A tibble: 336,776 × 9
##     year month   day  hour origin  dest tailnum carrier
##    <int> <int> <int> <dbl>  <chr> <chr>   <chr>   <chr>
## 1   2013     1     1     5    EWR   IAH  N14228      UA
## 2   2013     1     1     5    LGA   IAH  N24211      UA
## 3   2013     1     1     5    JFK   MIA  N619AA      AA
## 4   2013     1     1     5    JFK   BQN  N804JB      B6
## 5   2013     1     1     6    LGA   ATL  N668DN      DL
## 6   2013     1     1     5    EWR   ORD  N39463      UA
## 7   2013     1     1     6    EWR   FLL  N516JB      B6
## 8   2013     1     1     6    LGA   IAD  N829AS      EV
## 9   2013     1     1     6    JFK   MCO  N593JB      B6
## 10  2013     1     1     6    LGA   ORD  N3ALAA      AA
## # ... with 336,766 more rows, and 1 more variables: name <chr>
\end{verbatim}

\begin{Shaded}
\begin{Highlighting}[]
\NormalTok{flights2 %>%}\StringTok{ }\KeywordTok{left_join}\NormalTok{(weather) }
\end{Highlighting}
\end{Shaded}

\begin{verbatim}
## Joining, by = c("year", "month", "day", "hour", "origin")
\end{verbatim}

\begin{verbatim}
## # A tibble: 336,776 × 18
##     year month   day  hour origin  dest tailnum carrier  temp  dewp humid
##    <dbl> <dbl> <int> <dbl>  <chr> <chr>   <chr>   <chr> <dbl> <dbl> <dbl>
## 1   2013     1     1     5    EWR   IAH  N14228      UA    NA    NA    NA
## 2   2013     1     1     5    LGA   IAH  N24211      UA    NA    NA    NA
## 3   2013     1     1     5    JFK   MIA  N619AA      AA    NA    NA    NA
## 4   2013     1     1     5    JFK   BQN  N804JB      B6    NA    NA    NA
## 5   2013     1     1     6    LGA   ATL  N668DN      DL 39.92 26.06 57.33
## 6   2013     1     1     5    EWR   ORD  N39463      UA    NA    NA    NA
## 7   2013     1     1     6    EWR   FLL  N516JB      B6 39.02 26.06 59.37
## 8   2013     1     1     6    LGA   IAD  N829AS      EV 39.92 26.06 57.33
## 9   2013     1     1     6    JFK   MCO  N593JB      B6 39.02 26.06 59.37
## 10  2013     1     1     6    LGA   ORD  N3ALAA      AA 39.92 26.06 57.33
## # ... with 336,766 more rows, and 7 more variables: wind_dir <dbl>,
## #   wind_speed <dbl>, wind_gust <dbl>, precip <dbl>, pressure <dbl>,
## #   visib <dbl>, time_hour <dttm>
\end{verbatim}

\begin{Shaded}
\begin{Highlighting}[]
\CommentTok{# join with named matching}
\NormalTok{flights2 %>%}\StringTok{ }\KeywordTok{left_join}\NormalTok{(planes, }\DataTypeTok{by =} \StringTok{"tailnum"}\NormalTok{) }
\end{Highlighting}
\end{Shaded}

\begin{verbatim}
## # A tibble: 336,776 × 16
##    year.x month   day  hour origin  dest tailnum carrier year.y
##     <int> <int> <int> <dbl>  <chr> <chr>   <chr>   <chr>  <int>
## 1    2013     1     1     5    EWR   IAH  N14228      UA   1999
## 2    2013     1     1     5    LGA   IAH  N24211      UA   1998
## 3    2013     1     1     5    JFK   MIA  N619AA      AA   1990
## 4    2013     1     1     5    JFK   BQN  N804JB      B6   2012
## 5    2013     1     1     6    LGA   ATL  N668DN      DL   1991
## 6    2013     1     1     5    EWR   ORD  N39463      UA   2012
## 7    2013     1     1     6    EWR   FLL  N516JB      B6   2000
## 8    2013     1     1     6    LGA   IAD  N829AS      EV   1998
## 9    2013     1     1     6    JFK   MCO  N593JB      B6   2004
## 10   2013     1     1     6    LGA   ORD  N3ALAA      AA     NA
## # ... with 336,766 more rows, and 7 more variables: type <chr>,
## #   manufacturer <chr>, model <chr>, engines <int>, seats <int>,
## #   speed <int>, engine <chr>
\end{verbatim}

\begin{Shaded}
\begin{Highlighting}[]
\CommentTok{# join with explicit column matching}
\NormalTok{flights2 %>%}\StringTok{ }\KeywordTok{left_join}\NormalTok{(airports, }\DataTypeTok{by=} \KeywordTok{c}\NormalTok{(}\StringTok{"dest"} \NormalTok{=}\StringTok{ "faa"}\NormalTok{)) }
\end{Highlighting}
\end{Shaded}

\begin{verbatim}
## # A tibble: 336,776 × 15
##     year month   day  hour origin  dest tailnum carrier
##    <int> <int> <int> <dbl>  <chr> <chr>   <chr>   <chr>
## 1   2013     1     1     5    EWR   IAH  N14228      UA
## 2   2013     1     1     5    LGA   IAH  N24211      UA
## 3   2013     1     1     5    JFK   MIA  N619AA      AA
## 4   2013     1     1     5    JFK   BQN  N804JB      B6
## 5   2013     1     1     6    LGA   ATL  N668DN      DL
## 6   2013     1     1     5    EWR   ORD  N39463      UA
## 7   2013     1     1     6    EWR   FLL  N516JB      B6
## 8   2013     1     1     6    LGA   IAD  N829AS      EV
## 9   2013     1     1     6    JFK   MCO  N593JB      B6
## 10  2013     1     1     6    LGA   ORD  N3ALAA      AA
## # ... with 336,766 more rows, and 7 more variables: name <chr>, lat <dbl>,
## #   lon <dbl>, alt <int>, tz <dbl>, dst <chr>, tzone <chr>
\end{verbatim}

Types of join with SQL equivalent.

\begin{Shaded}
\begin{Highlighting}[]
\CommentTok{# Create simple data}
\NormalTok{(df1 <-}\StringTok{ }\KeywordTok{data_frame}\NormalTok{(}\DataTypeTok{x =} \KeywordTok{c}\NormalTok{(}\DecValTok{1}\NormalTok{, }\DecValTok{2}\NormalTok{), }\DataTypeTok{y =} \DecValTok{2}\NormalTok{:}\DecValTok{1}\NormalTok{))}
\end{Highlighting}
\end{Shaded}

\begin{verbatim}
## # A tibble: 2 × 2
##       x     y
##   <dbl> <int>
## 1     1     2
## 2     2     1
\end{verbatim}

\begin{Shaded}
\begin{Highlighting}[]
\NormalTok{(df2 <-}\StringTok{ }\KeywordTok{data_frame}\NormalTok{(}\DataTypeTok{x =} \KeywordTok{c}\NormalTok{(}\DecValTok{1}\NormalTok{, }\DecValTok{3}\NormalTok{), }\DataTypeTok{a =} \DecValTok{10}\NormalTok{, }\DataTypeTok{b =} \StringTok{"a"}\NormalTok{))}
\end{Highlighting}
\end{Shaded}

\begin{verbatim}
## # A tibble: 2 × 3
##       x     a     b
##   <dbl> <dbl> <chr>
## 1     1    10     a
## 2     3    10     a
\end{verbatim}

\begin{Shaded}
\begin{Highlighting}[]
\CommentTok{# Return only matched rows}
\NormalTok{df1 %>%}\StringTok{ }\KeywordTok{inner_join}\NormalTok{(df2) }\CommentTok{# SELECT * FROM x JOIN y ON x.a = y.a}
\end{Highlighting}
\end{Shaded}

\begin{verbatim}
## Joining, by = "x"
\end{verbatim}

\begin{verbatim}
## # A tibble: 1 × 4
##       x     y     a     b
##   <dbl> <int> <dbl> <chr>
## 1     1     2    10     a
\end{verbatim}

\begin{Shaded}
\begin{Highlighting}[]
\CommentTok{# Return all rows in df1.}
\NormalTok{df1 %>%}\StringTok{ }\KeywordTok{left_join}\NormalTok{(df2) }\CommentTok{# SELECT * FROM x LEFT JOIN y ON x.a = y.a}
\end{Highlighting}
\end{Shaded}

\begin{verbatim}
## Joining, by = "x"
\end{verbatim}

\begin{verbatim}
## # A tibble: 2 × 4
##       x     y     a     b
##   <dbl> <int> <dbl> <chr>
## 1     1     2    10     a
## 2     2     1    NA  <NA>
\end{verbatim}

\begin{Shaded}
\begin{Highlighting}[]
\CommentTok{# Return all rows in df2.}
\NormalTok{df1 %>%}\StringTok{ }\KeywordTok{right_join}\NormalTok{(df2) }\CommentTok{# SELECT * FROM x RIGHT JOIN y ON x.a = y.a}
\end{Highlighting}
\end{Shaded}

\begin{verbatim}
## Joining, by = "x"
\end{verbatim}

\begin{verbatim}
## # A tibble: 2 × 4
##       x     y     a     b
##   <dbl> <int> <dbl> <chr>
## 1     1     2    10     a
## 2     3    NA    10     a
\end{verbatim}

\begin{Shaded}
\begin{Highlighting}[]
\CommentTok{# Return all rows. }
\NormalTok{df1 %>%}\StringTok{ }\KeywordTok{full_join}\NormalTok{(df2) }\CommentTok{# SELECT * FROM x FULL JOIN y ON x.a = y.a}
\end{Highlighting}
\end{Shaded}

\begin{verbatim}
## Joining, by = "x"
\end{verbatim}

\begin{verbatim}
## # A tibble: 3 × 4
##       x     y     a     b
##   <dbl> <int> <dbl> <chr>
## 1     1     2    10     a
## 2     2     1    NA  <NA>
## 3     3    NA    10     a
\end{verbatim}

\begin{Shaded}
\begin{Highlighting}[]
\CommentTok{# Like left_join, but returning only columns in df1}
\NormalTok{df1 %>%}\StringTok{ }\KeywordTok{semi_join}\NormalTok{(df2, }\DataTypeTok{by =} \StringTok{"x"}\NormalTok{)  }\CommentTok{# SELECT * FROM x WHERE EXISTS (SELECT 1 FROM y WHERE x.a = y.a)}
\end{Highlighting}
\end{Shaded}

\begin{verbatim}
## # A tibble: 1 × 2
##       x     y
##   <dbl> <int>
## 1     1     2
\end{verbatim}

\section{tidyr}\label{tidyr}

\section{reshape2}\label{reshape2}

\section{stringr}\label{stringr}

\section{anytime}\label{anytime}

\section{Biblipgraphic Notes}\label{biblipgraphic-notes-1}

\section{Practice Yourself}\label{practice-yourself-11}

\chapter{Sparse Representations}\label{sparse}

Analyzing ``bigdata'' in R is a challenge because the workspace is
memory resident, i.e., all your objects are stored in RAM. As a rule of
thumb, fitting models requires about 5 times the size of the data. This
means that if you have 1 GB of data, you might need about 5 GB to fit a
linear models. We will discuss how to compute \emph{out of RAM} in the
Memory Efficiency Chapter \ref{memory}. In this chapter, we discuss
efficient representations of your data, so that it takes less memory.
The fundamental idea, is that if your data is \emph{sparse}, i.e., there
are many zero entries in your data, then a naive \texttt{data.frame} or
\texttt{matrix} will consume memory for all these zeroes. If, however,
you have many recurring zeroes, it is more efficient to save only the
non-zero entries.

When we say \emph{data}, we actually mean the \texttt{model.matrix}. The
\texttt{model.matrix} is a matrix that R grows, converting all your
factors to numeric variables that can be computed with. \emph{Dummy
coding} of your factors, for instance, is something that is done in your
\texttt{model.matrix}. If you have a factor with many levels, you can
imagine that after dummy coding it, many zeroes will be present.

The \textbf{Matrix} package replaces the \texttt{matrix} class, with
several sparse representations of matrix objects.

When using sparse representation, and the \textbf{Matrix} package, you
will need an implementation of your favorite model fitting algorithm
(e.g. \texttt{lm}) that is adapted to these sparse representations;
otherwise, R will cast the sparse matrix into a regular (non-sparse)
matrix, and you will have saved nothing in RAM.

\BeginKnitrBlock{remark}
\iffalse {Remark. } \fi If you are familiar with MATLAB you should know
that one of the great capabilities of MATLAB, is the excellent treatment
of sparse matrices with the \texttt{sparse} function.
\EndKnitrBlock{remark}

Before we go into details, here is a simple example. We will create a
factor of letters with the \texttt{letters} function. Clearly, this
factor can take only \(26\) values. This means that \(25/26\) of the
\texttt{model.matrix} will be zeroes after dummy coding. We will compare
the memory footprint of the naive \texttt{model.matrix} with the sparse
representation of the same matrix.

\begin{Shaded}
\begin{Highlighting}[]
\KeywordTok{library}\NormalTok{(magrittr)}
\NormalTok{reps <-}\StringTok{ }\FloatTok{1e6} \CommentTok{# number of samples}
\NormalTok{y<-}\KeywordTok{rnorm}\NormalTok{(reps)}
\NormalTok{x<-}\StringTok{ }\NormalTok{letters %>%}\StringTok{ }
\StringTok{  }\KeywordTok{sample}\NormalTok{(reps, }\DataTypeTok{replace=}\OtherTok{TRUE}\NormalTok{) %>%}\StringTok{ }
\StringTok{  }\NormalTok{factor}
\end{Highlighting}
\end{Shaded}

The object \texttt{x} is a factor of letters:

\begin{Shaded}
\begin{Highlighting}[]
\KeywordTok{head}\NormalTok{(x)}
\end{Highlighting}
\end{Shaded}

\begin{verbatim}
## [1] n x z f a i
## Levels: a b c d e f g h i j k l m n o p q r s t u v w x y z
\end{verbatim}

We dummy code \texttt{x} with the \texttt{model.matrix} function.

\begin{Shaded}
\begin{Highlighting}[]
\NormalTok{X}\FloatTok{.1} \NormalTok{<-}\StringTok{ }\KeywordTok{model.matrix}\NormalTok{(~x}\DecValTok{-1}\NormalTok{)}
\KeywordTok{head}\NormalTok{(X}\FloatTok{.1}\NormalTok{)}
\end{Highlighting}
\end{Shaded}

\begin{verbatim}
##   xa xb xc xd xe xf xg xh xi xj xk xl xm xn xo xp xq xr xs xt xu xv xw xx
## 1  0  0  0  0  0  0  0  0  0  0  0  0  0  1  0  0  0  0  0  0  0  0  0  0
## 2  0  0  0  0  0  0  0  0  0  0  0  0  0  0  0  0  0  0  0  0  0  0  0  1
## 3  0  0  0  0  0  0  0  0  0  0  0  0  0  0  0  0  0  0  0  0  0  0  0  0
## 4  0  0  0  0  0  1  0  0  0  0  0  0  0  0  0  0  0  0  0  0  0  0  0  0
## 5  1  0  0  0  0  0  0  0  0  0  0  0  0  0  0  0  0  0  0  0  0  0  0  0
## 6  0  0  0  0  0  0  0  0  1  0  0  0  0  0  0  0  0  0  0  0  0  0  0  0
##   xy xz
## 1  0  0
## 2  0  0
## 3  0  1
## 4  0  0
## 5  0  0
## 6  0  0
\end{verbatim}

We call \textbf{MatrixModels} for an implementation of
\texttt{model.matrix} that supports sparse representations.

\begin{Shaded}
\begin{Highlighting}[]
\KeywordTok{suppressPackageStartupMessages}\NormalTok{(}\KeywordTok{library}\NormalTok{(MatrixModels))}
\NormalTok{X}\FloatTok{.2}\NormalTok{<-}\StringTok{ }\KeywordTok{as}\NormalTok{(x,}\StringTok{"sparseMatrix"}\NormalTok{) %>%}\StringTok{ }\NormalTok{t }\CommentTok{# Makes sparse dummy model.matrix}
\KeywordTok{head}\NormalTok{(X}\FloatTok{.2}\NormalTok{)}
\end{Highlighting}
\end{Shaded}

\begin{verbatim}
## 6 x 26 sparse Matrix of class "dgCMatrix"
\end{verbatim}

\begin{verbatim}
##    [[ suppressing 26 column names 'a', 'b', 'c' ... ]]
\end{verbatim}

\begin{verbatim}
##                                                         
## [1,] . . . . . . . . . . . . . 1 . . . . . . . . . . . .
## [2,] . . . . . . . . . . . . . . . . . . . . . . . 1 . .
## [3,] . . . . . . . . . . . . . . . . . . . . . . . . . 1
## [4,] . . . . . 1 . . . . . . . . . . . . . . . . . . . .
## [5,] 1 . . . . . . . . . . . . . . . . . . . . . . . . .
## [6,] . . . . . . . . 1 . . . . . . . . . . . . . . . . .
\end{verbatim}

Notice that the matrices have the same dimensions:

\begin{Shaded}
\begin{Highlighting}[]
\KeywordTok{dim}\NormalTok{(X}\FloatTok{.1}\NormalTok{)}
\end{Highlighting}
\end{Shaded}

\begin{verbatim}
## [1] 1000000      26
\end{verbatim}

\begin{Shaded}
\begin{Highlighting}[]
\KeywordTok{dim}\NormalTok{(X}\FloatTok{.2}\NormalTok{)}
\end{Highlighting}
\end{Shaded}

\begin{verbatim}
## [1] 1000000      26
\end{verbatim}

The memory footprint of the matrices, given by the
\texttt{pryr::object\_size} function, are very very different.

\begin{Shaded}
\begin{Highlighting}[]
\NormalTok{pryr::}\KeywordTok{object_size}\NormalTok{(X}\FloatTok{.1}\NormalTok{)}
\end{Highlighting}
\end{Shaded}

\begin{verbatim}
## 264 MB
\end{verbatim}

\begin{Shaded}
\begin{Highlighting}[]
\NormalTok{pryr::}\KeywordTok{object_size}\NormalTok{(X}\FloatTok{.2}\NormalTok{)}
\end{Highlighting}
\end{Shaded}

\begin{verbatim}
## 12 MB
\end{verbatim}

Things to note:

\begin{itemize}
\tightlist
\item
  The sparse representation takes a whole lot less memory than the non
  sparse.
\item
  The \texttt{as(,"sparseMatrix")} function grows the dummy variable
  representation of the factor \texttt{x}.
\item
  The \textbf{pryr} package provides many facilities for inspecting the
  memory footprint of your objects and code.
\end{itemize}

With a sparse representation, we not only saved on RAM, but also on the
computing time of fitting a model. Here is the timing of a non sparse
representation:

\begin{Shaded}
\begin{Highlighting}[]
\KeywordTok{system.time}\NormalTok{(lm}\FloatTok{.1} \NormalTok{<-}\StringTok{ }\KeywordTok{lm}\NormalTok{(y ~}\StringTok{ }\NormalTok{X}\FloatTok{.1}\NormalTok{)) }
\end{Highlighting}
\end{Shaded}

\begin{verbatim}
##    user  system elapsed 
##   3.048   0.124   3.172
\end{verbatim}

Well actually, \texttt{lm} is a wrapper for the \texttt{lm.fit}
function. If we override all the overhead of \texttt{lm}, and call
\texttt{lm.fit} directly, we gain some time:

\begin{Shaded}
\begin{Highlighting}[]
\KeywordTok{system.time}\NormalTok{(lm}\FloatTok{.1} \NormalTok{<-}\StringTok{ }\KeywordTok{lm.fit}\NormalTok{(}\DataTypeTok{y=}\NormalTok{y, }\DataTypeTok{x=}\NormalTok{X}\FloatTok{.1}\NormalTok{))}
\end{Highlighting}
\end{Shaded}

\begin{verbatim}
##    user  system elapsed 
##   1.196   0.028   1.232
\end{verbatim}

We now do the same with the sparse representation:

\begin{Shaded}
\begin{Highlighting}[]
\KeywordTok{system.time}\NormalTok{(lm}\FloatTok{.2} \NormalTok{<-}\StringTok{ }\NormalTok{MatrixModels:::}\KeywordTok{lm.fit.sparse}\NormalTok{(X}\FloatTok{.2}\NormalTok{,y))}
\end{Highlighting}
\end{Shaded}

\begin{verbatim}
##    user  system elapsed 
##   0.212   0.004   0.215
\end{verbatim}

It is only left to verify that the returned coefficients are the same:

\begin{Shaded}
\begin{Highlighting}[]
\KeywordTok{all.equal}\NormalTok{(lm}\FloatTok{.2}\NormalTok{, }\KeywordTok{unname}\NormalTok{(lm}\FloatTok{.1}\NormalTok{$coefficients), }\DataTypeTok{tolerance =} \FloatTok{1e-12}\NormalTok{)}
\end{Highlighting}
\end{Shaded}

\begin{verbatim}
## [1] TRUE
\end{verbatim}

You can also visualize the non zero entries, i.e., the sparsity
structure.

\begin{Shaded}
\begin{Highlighting}[]
\KeywordTok{image}\NormalTok{(X}\FloatTok{.2}\NormalTok{[}\DecValTok{1}\NormalTok{:}\DecValTok{26}\NormalTok{,}\DecValTok{1}\NormalTok{:}\DecValTok{26}\NormalTok{])}
\end{Highlighting}
\end{Shaded}

\includegraphics[width=0.5\linewidth]{Rcourse_files/figure-latex/unnamed-chunk-261-1}

\section{Sparse Matrix
Representations}\label{sparse-matrix-representations}

We first distinguish between the two main goals of the efficient
representation: (i) efficient writing, i.e., modification; (ii)
efficient reading, i.e., access. For our purposes, we will typically
want efficient reading, since the \texttt{model.matrix} will not change
while a model is being fitted.

Representations designed for writing include the \emph{dictionary of
keys}, \emph{list of lists}, and a \emph{coordinate list}.
Representations designed for efficient reading include the
\emph{compressed sparse row} and \emph{compressed sparse column}.

\subsection{Coordinate List Representation}\label{coo}

A \emph{coordinate list representation}, also known as \emph{COO}, or
\emph{triplet represantation} is simply a list of the non zero entries.
Each element in the list is a triplet of the column, row, and value, of
each non-zero entry in the matrix.

\subsection{Compressed Column Oriented
Representation}\label{compressed-column-oriented-representation}

A \emph{compressed column oriented representation}, also known as
\emph{compressed sparse column}, or \emph{CSC}, where the
\textbf{column} index is similar to COO, but instead of saving the row
indexes, we save the locations in the column index vectors where the row
index has to increase. The following figure may clarify this simple
idea.

\begin{figure}[htbp]
\centering
\includegraphics{art/crc.png}
\caption{The CSC data structure. From \citet{shah2004sparse}. Remember
that MATLAB is written in C, where the indexing starts at \(0\), and not
\(1\).}
\end{figure}

The nature of statistical applications is such, that CSC representation
is typically the most economical, justifying its popularity.

\subsection{Compressed Row Oriented
Representation}\label{compressed-row-oriented-representation}

A \emph{compressed row oriented representation}, also known as
\emph{compressed sparse row}, or \emph{CSR}, is very similar to CSC,
after switching the role of rows and columns. CSR is much less popular
than CSC.

\subsection{Sparse Algorithms}\label{sparse-algorithms}

We will go into the details of some algorithms in the Numerical Linear
Algebra Chapter \ref{algebra}. For our current purposes two things need
to be emphasized:

\begin{enumerate}
\def\labelenumi{\arabic{enumi}.}
\item
  A mathematician may write \(Ax=b \Rightarrow x=A^{-1}b\). A computer,
  however, would \textbf{never} compute \(A^{-1}\) in order to find
  \(x\), but rather use one of many endlessly many numerical algorithms.
\item
  Working with sparse representations requires using a function that is
  aware of the representation you are using.
\end{enumerate}

\section{Sparse Matrices and Sparse Models in
R}\label{sparse-matrices-and-sparse-models-in-r}

\subsection{The Matrix Package}\label{the-matrix-package}

The \textbf{Matrix} package provides facilities to deal with real
(stored as double precision), logical and so-called ``pattern'' (binary)
dense and sparse matrices. There are provisions to provide integer and
complex (stored as double precision complex) matrices.

The sparse matrix classes include:

\begin{itemize}
\tightlist
\item
  \texttt{TsparseMatrix}: a virtual class of the various sparse matrices
  in triplet representation.
\item
  \texttt{CsparseMatrix}: a virtual class of the various sparse matrices
  in CSC representation.
\item
  \texttt{RsparseMatrix}: a virtual class of the various sparse matrices
  in CSR representation.
\end{itemize}

For matrices of real numbers, stored in \emph{double precision}, the
\textbf{Matrix} package provides the following (non virtual) classes:

\begin{itemize}
\tightlist
\item
  \texttt{dgTMatrix}: a \textbf{general} sparse matrix of
  \textbf{doubles}, in \textbf{triplet} representation.
\item
  \texttt{dgCMatrix}: a \textbf{general} sparse matrix of
  \textbf{doubles}, in \textbf{CSC} representation.
\item
  \texttt{dsCMatrix}: a \textbf{symmetric} sparse matrix of
  \textbf{doubles}, in \textbf{CSC} representation.
\item
  \texttt{dtCMatrix}: a \textbf{triangular} sparse matrix of
  \textbf{doubles}, in \textbf{CSC} representation.
\end{itemize}

Why bother with distinguishing between the different shapes of the
matrix? Because the more structure is assumed on a matrix, the more our
(statistical) algorithms can be optimized. For our purposes
\texttt{dgCMatrix} will be the most useful.

\subsection{The glmnet Package}\label{the-glmnet-package}

As previously stated, an efficient storage of the \texttt{model.matrix}
is half of the story. We now need implementations of our favorite
statistical algorithms that make use of this representation. At the time
of writing, a very useful package that does that is the \textbf{glmnet}
package, which allows to fit linear models, generalized linear models,
with ridge, lasso, and elastic net regularization. The \textbf{glmnet}
package allows all of this, using the sparse matrices of the
\textbf{Matrix} package.

The following example is taken from
\href{http://www.johnmyleswhite.com/notebook/2011/10/31/using-sparse-matrices-in-r/}{John
Myles White's blog}, and compares the runtime of fitting an OLS model,
using \texttt{glmnet} with both sparse and dense matrix representations.

\begin{Shaded}
\begin{Highlighting}[]
\KeywordTok{suppressPackageStartupMessages}\NormalTok{(}\KeywordTok{library}\NormalTok{(}\StringTok{'glmnet'}\NormalTok{))}

\KeywordTok{set.seed}\NormalTok{(}\DecValTok{1}\NormalTok{)}
\NormalTok{performance <-}\StringTok{ }\KeywordTok{data.frame}\NormalTok{()}
 
\NormalTok{for (sim in }\DecValTok{1}\NormalTok{:}\DecValTok{10}\NormalTok{)\{}
  \NormalTok{n <-}\StringTok{ }\DecValTok{10000}
  \NormalTok{p <-}\StringTok{ }\DecValTok{500}
 
  \NormalTok{nzc <-}\StringTok{ }\KeywordTok{trunc}\NormalTok{(p /}\StringTok{ }\DecValTok{10}\NormalTok{)}
 
  \NormalTok{x <-}\StringTok{ }\KeywordTok{matrix}\NormalTok{(}\KeywordTok{rnorm}\NormalTok{(n *}\StringTok{ }\NormalTok{p), n, p) }\CommentTok{#make a dense matrix}
  \NormalTok{iz <-}\StringTok{ }\KeywordTok{sample}\NormalTok{(}\DecValTok{1}\NormalTok{:(n *}\StringTok{ }\NormalTok{p),}
               \DataTypeTok{size =} \NormalTok{n *}\StringTok{ }\NormalTok{p *}\StringTok{ }\FloatTok{0.85}\NormalTok{,}
               \DataTypeTok{replace =} \OtherTok{FALSE}\NormalTok{)}
  \NormalTok{x[iz] <-}\StringTok{ }\DecValTok{0} \CommentTok{# sparsify by injecting zeroes}
  \NormalTok{sx <-}\StringTok{ }\KeywordTok{Matrix}\NormalTok{(x, }\DataTypeTok{sparse =} \OtherTok{TRUE}\NormalTok{) }\CommentTok{# save as a sparse object}
 
  \NormalTok{beta <-}\StringTok{ }\KeywordTok{rnorm}\NormalTok{(nzc)}
  \NormalTok{fx <-}\StringTok{ }\NormalTok{x[, }\KeywordTok{seq}\NormalTok{(nzc)] %*%}\StringTok{ }\NormalTok{beta}
 
  \NormalTok{eps <-}\StringTok{ }\KeywordTok{rnorm}\NormalTok{(n)}
  \NormalTok{y <-}\StringTok{ }\NormalTok{fx +}\StringTok{ }\NormalTok{eps }\CommentTok{# make data}
 
  \CommentTok{# Now to the actual model fitting:}
  \NormalTok{sparse.times <-}\StringTok{ }\KeywordTok{system.time}\NormalTok{(fit1 <-}\StringTok{ }\KeywordTok{glmnet}\NormalTok{(sx, y)) }\CommentTok{# sparse glmnet}
  \NormalTok{full.times <-}\StringTok{ }\KeywordTok{system.time}\NormalTok{(fit2 <-}\StringTok{ }\KeywordTok{glmnet}\NormalTok{(x, y)) }\CommentTok{# dense glmnet}
  
  \NormalTok{sparse.size <-}\StringTok{ }\KeywordTok{as.numeric}\NormalTok{(}\KeywordTok{object.size}\NormalTok{(sx))}
  \NormalTok{full.size <-}\StringTok{ }\KeywordTok{as.numeric}\NormalTok{(}\KeywordTok{object.size}\NormalTok{(x))}
 
  \NormalTok{performance <-}\StringTok{ }\KeywordTok{rbind}\NormalTok{(performance, }\KeywordTok{data.frame}\NormalTok{(}\DataTypeTok{Format =} \StringTok{'Sparse'}\NormalTok{,}
                                                \DataTypeTok{UserTime =} \NormalTok{sparse.times[}\DecValTok{1}\NormalTok{],}
                                               \DataTypeTok{SystemTime =} \NormalTok{sparse.times[}\DecValTok{2}\NormalTok{],}
                                               \DataTypeTok{ElapsedTime =} \NormalTok{sparse.times[}\DecValTok{3}\NormalTok{],}
                                               \DataTypeTok{Size =} \NormalTok{sparse.size))}
  \NormalTok{performance <-}\StringTok{ }\KeywordTok{rbind}\NormalTok{(performance, }\KeywordTok{data.frame}\NormalTok{(}\DataTypeTok{Format =} \StringTok{'Full'}\NormalTok{,}
                                               \DataTypeTok{UserTime =} \NormalTok{full.times[}\DecValTok{1}\NormalTok{],}
                                               \DataTypeTok{SystemTime =} \NormalTok{full.times[}\DecValTok{2}\NormalTok{],}
                                               \DataTypeTok{ElapsedTime =} \NormalTok{full.times[}\DecValTok{3}\NormalTok{],}
                                               \DataTypeTok{Size =} \NormalTok{full.size))}
\NormalTok{\}}
\end{Highlighting}
\end{Shaded}

Things to note:

\begin{itemize}
\tightlist
\item
  The simulation calls \texttt{glmnet} twice. Once with the non-sparse
  object \texttt{x}, and once with its sparse version \texttt{sx}.
\item
  The degree of sparsity of \texttt{sx} is \(85\%\). We know this
  because we ``injected'' zeroes in \(0.85\) of the locations of
  \texttt{x}.
\item
  Because \texttt{y} is continuous \texttt{glmnet} will fit a simple OLS
  model. We will see later how to use it to fit GLMs and use lasso,
  ridge, and elastic-net regularization.
\end{itemize}

We now inspect the computing time, and the memory footprint, only to
discover that sparse representations make a BIG difference.

\begin{Shaded}
\begin{Highlighting}[]
\KeywordTok{suppressPackageStartupMessages}\NormalTok{(}\KeywordTok{library}\NormalTok{(}\StringTok{'ggplot2'}\NormalTok{))}
\KeywordTok{ggplot}\NormalTok{(performance, }\KeywordTok{aes}\NormalTok{(}\DataTypeTok{x =} \NormalTok{Format, }\DataTypeTok{y =} \NormalTok{ElapsedTime, }\DataTypeTok{fill =} \NormalTok{Format)) +}
\StringTok{  }\KeywordTok{stat_summary}\NormalTok{(}\DataTypeTok{fun.data =} \StringTok{'mean_cl_boot'}\NormalTok{, }\DataTypeTok{geom =} \StringTok{'bar'}\NormalTok{) +}
\StringTok{  }\KeywordTok{stat_summary}\NormalTok{(}\DataTypeTok{fun.data =} \StringTok{'mean_cl_boot'}\NormalTok{, }\DataTypeTok{geom =} \StringTok{'errorbar'}\NormalTok{) +}
\StringTok{  }\KeywordTok{ylab}\NormalTok{(}\StringTok{'Elapsed Time in Seconds'}\NormalTok{) }
\end{Highlighting}
\end{Shaded}

\includegraphics[width=0.5\linewidth]{Rcourse_files/figure-latex/unnamed-chunk-262-1}

\begin{Shaded}
\begin{Highlighting}[]
\KeywordTok{ggplot}\NormalTok{(performance, }\KeywordTok{aes}\NormalTok{(}\DataTypeTok{x =} \NormalTok{Format, }\DataTypeTok{y =} \NormalTok{Size /}\StringTok{ }\DecValTok{1000000}\NormalTok{, }\DataTypeTok{fill =} \NormalTok{Format)) +}
\StringTok{  }\KeywordTok{stat_summary}\NormalTok{(}\DataTypeTok{fun.data =} \StringTok{'mean_cl_boot'}\NormalTok{, }\DataTypeTok{geom =} \StringTok{'bar'}\NormalTok{) +}
\StringTok{  }\KeywordTok{stat_summary}\NormalTok{(}\DataTypeTok{fun.data =} \StringTok{'mean_cl_boot'}\NormalTok{, }\DataTypeTok{geom =} \StringTok{'errorbar'}\NormalTok{) +}
\StringTok{  }\KeywordTok{ylab}\NormalTok{(}\StringTok{'Matrix Size in MB'}\NormalTok{) }
\end{Highlighting}
\end{Shaded}

\includegraphics[width=0.5\linewidth]{Rcourse_files/figure-latex/unnamed-chunk-263-1}

How do we perform other types of regression with the \textbf{glmnet}? We
just need to use the \texttt{family} and \texttt{alpha} arguments of
\texttt{glmnet::glmnet}. The \texttt{family} argument governs the type
of GLM to fit: logistic, Poisson, probit, or other types of GLM. The
\texttt{alpha} argument controls the type of regularization. Set to
\texttt{alpha=0} for ridge, \texttt{alpha=1} for lasso, and any value in
between for elastic-net regularization.

\section{Bibliographic Notes}\label{bibliographic-notes-10}

The best place to start reading on sparse representations and algorithms
is the
\href{http://svitsrv25.epfl.ch/R-doc/library/Matrix/doc/}{vignettes} of
the \textbf{Matrix} package. \citet{gilbert1992sparse} is also a great
read for some general background. For the theory on solving sparse
linear systems see \citet{davis2006direct}. For general numerical linear
algebra see \citet{gentle2012numerical}.

\section{Practice Yourself}\label{practice-yourself-12}

\chapter{Memory Efficiency}\label{memory}

As put by \citet{kane2013scalable}, it was quite puzzling when very few
of the competitors, for the Million dollars prize in the
\href{https://en.wikipedia.org/wiki/Netflix_Prize}{Netflix challenge},
were statisticians. This is perhaps because the statistical community
historically uses SAS, SPSS, and R. The first two tools are very well
equipped to deal with big data, but are very unfriendly when trying to
implement a new method. R, on the other hand, is very friendly for
innovation, but was not equipped to deal with the large data sets of the
Netflix challenge. A lot has changed in R since 2006. This is the topic
of this chapter.

As we have seen in the Sparsity Chapter \ref{sparse}, an efficient
representation of your data in RAM will reduce computing time, and will
allow you to fit models that would otherwise require tremendous amounts
of RAM. Not all problems are sparse however. It is also possible that
your data does not fit in RAM, even if sparse. There are several
scenarios to consider:

\begin{enumerate}
\def\labelenumi{\arabic{enumi}.}
\tightlist
\item
  Your data fits in RAM, but is too big to compute with.
\item
  Your data does not fit in RAM, but fits in your local storage (HD,
  SSD, etc.)
\item
  Your data does not fit in your local storage.
\end{enumerate}

If your data fits in RAM, but is too large to compute with, a solution
is to replace the algorithm you are using. Instead of computing with the
whole data, your algorithm will compute with parts of the data, also
called \emph{chunks}, or \emph{batches}. These algorithms are known as
\emph{external memory algorithms} (EMA).

If your data does not fit in RAM, but fits in your local storage, you
have two options. The first is to save your data in a \emph{database
management system} (DBMS). This will allow you to use the algorithms
provided by your DBMS, or let R use an EMA while ``chunking'' from your
DBMS. Alternatively, and preferably, you may avoid using a DBMS, and
work with the data directly form your local storage by saving your data
in some efficient manner.

Finally, if your data does not fit on you local storage, you will need
some external storage solution such as a distributed DBMS, or
distributed file system.

\BeginKnitrBlock{remark}
\iffalse {Remark. } \fi If you use Linux, you may be better of than
Windows users. Linux will allow you to compute with larger datasets
using its \emph{swap file} that extends RAM using your HD or SSD. On the
other hand, relying on the swap file is a BAD practice since it is much
slower than RAM, and you can typically do much better using the tricks
of this chapter. Also, while I LOVE Linux, I would never dare to
recommend switching to Linux just to deal with memory contraints.
\EndKnitrBlock{remark}

\section{Efficient Computing from
RAM}\label{efficient-computing-from-ram}

If our data can fit in RAM, but is still too large to compute with it
(recall that fitting a model requires roughly 5-10 times more memory
than saving it), there are several facilities to be used. The first, is
the sparse representation discussed in Chapter \ref{sparse}, which is
relevant when you have factors, which will typically map to sparse model
matrices. Another way is to use \emph{external memory algorithms} (EMA).

The \texttt{biglm::biglm} function provides an EMA for linear
regression. The following if taken from the function's example.

\begin{Shaded}
\begin{Highlighting}[]
\KeywordTok{data}\NormalTok{(trees)}
\NormalTok{ff<-}\KeywordTok{log}\NormalTok{(Volume)~}\KeywordTok{log}\NormalTok{(Girth)+}\KeywordTok{log}\NormalTok{(Height)}

\NormalTok{chunk1<-trees[}\DecValTok{1}\NormalTok{:}\DecValTok{10}\NormalTok{,]}
\NormalTok{chunk2<-trees[}\DecValTok{11}\NormalTok{:}\DecValTok{20}\NormalTok{,]}
\NormalTok{chunk3<-trees[}\DecValTok{21}\NormalTok{:}\DecValTok{31}\NormalTok{,]}

\KeywordTok{library}\NormalTok{(biglm)}
\NormalTok{a <-}\StringTok{ }\KeywordTok{biglm}\NormalTok{(ff,chunk1)}
\NormalTok{a <-}\StringTok{ }\KeywordTok{update}\NormalTok{(a,chunk2)}
\NormalTok{a <-}\StringTok{ }\KeywordTok{update}\NormalTok{(a,chunk3)}

\KeywordTok{coef}\NormalTok{(a)}
\end{Highlighting}
\end{Shaded}

\begin{verbatim}
## (Intercept)  log(Girth) log(Height) 
##   -6.631617    1.982650    1.117123
\end{verbatim}

Things to note:

\begin{itemize}
\tightlist
\item
  The data has been chunked along rows.
\item
  The initial fit is done with the \texttt{biglm} function.
\item
  The model is updated with further chunks using the \texttt{update}
  function.
\end{itemize}

We now compare it to the in-memory version of \texttt{lm} to verify the
results are the same.

\begin{Shaded}
\begin{Highlighting}[]
\NormalTok{b <-}\StringTok{ }\KeywordTok{lm}\NormalTok{(ff, }\DataTypeTok{data=}\NormalTok{trees)}
\KeywordTok{rbind}\NormalTok{(}\KeywordTok{coef}\NormalTok{(a),}\KeywordTok{coef}\NormalTok{(b))}
\end{Highlighting}
\end{Shaded}

\begin{verbatim}
##      (Intercept) log(Girth) log(Height)
## [1,]   -6.631617    1.98265    1.117123
## [2,]   -6.631617    1.98265    1.117123
\end{verbatim}

Other packages that follow these lines, particularly with classification
using SVMs, are \textbf{LiblineaR}, and \textbf{RSofia}.

\subsection{Summary Statistics from
RAM}\label{summary-statistics-from-ram}

If you are not going to do any model fitting, and all you want is
efficient filtering, selection and summary statistics, then a lot of my
warnings above are irrelevant. For these purposes, the facilities
provided by \textbf{base}, \textbf{stats}, and \textbf{dplyr} are
probably enough. If the data is large, however, these facilities may be
too slow. If your data fits into RAM, but speed bothers you, take a look
at the \textbf{data.table} package. The syntax is less friendly than
\textbf{dplyr}, but \textbf{data.table} is BLAZING FAST compared to
competitors. Here is a little benchmark\footnote{The code was
  contributed by Liad Shekel.}.

First, we setup the data.

\begin{Shaded}
\begin{Highlighting}[]
\KeywordTok{library}\NormalTok{(data.table)}

\NormalTok{n <-}\StringTok{ }\FloatTok{1e6} \CommentTok{# number of rows}
\NormalTok{k <-}\StringTok{ }\KeywordTok{c}\NormalTok{(}\DecValTok{200}\NormalTok{,}\DecValTok{500}\NormalTok{) }\CommentTok{# number of distinct values for each 'group_by' variable}
\NormalTok{p <-}\StringTok{ }\DecValTok{3} \CommentTok{# number of variables to summarize}

\NormalTok{L1 <-}\StringTok{ }\KeywordTok{sapply}\NormalTok{(k, function(x) }\KeywordTok{as.character}\NormalTok{(}\KeywordTok{sample}\NormalTok{(}\DecValTok{1}\NormalTok{:x, n, }\DataTypeTok{replace =} \OtherTok{TRUE}\NormalTok{) ))}
\NormalTok{L2 <-}\StringTok{ }\KeywordTok{sapply}\NormalTok{(}\DecValTok{1}\NormalTok{:p, function(x) }\KeywordTok{rnorm}\NormalTok{(n) )}

\NormalTok{tbl <-}\StringTok{ }\KeywordTok{data.table}\NormalTok{(L1,L2) %>%}\StringTok{ }
\StringTok{  }\KeywordTok{setnames}\NormalTok{(}\KeywordTok{c}\NormalTok{(}\KeywordTok{paste}\NormalTok{(}\StringTok{"v"}\NormalTok{,}\DecValTok{1}\NormalTok{:}\KeywordTok{length}\NormalTok{(k),}\DataTypeTok{sep=}\StringTok{""}\NormalTok{), }\KeywordTok{paste}\NormalTok{(}\StringTok{"x"}\NormalTok{,}\DecValTok{1}\NormalTok{:p,}\DataTypeTok{sep=}\StringTok{""}\NormalTok{) ))}

\NormalTok{tbl_dt <-}\StringTok{ }\NormalTok{tbl}
\NormalTok{tbl_df <-}\StringTok{ }\NormalTok{tbl %>%}\StringTok{ }\NormalTok{as.data.frame}
\end{Highlighting}
\end{Shaded}

We compare the aggregation speeds. Here is the timing for
\textbf{dplyr}.

\begin{Shaded}
\begin{Highlighting}[]
\KeywordTok{system.time}\NormalTok{( tbl_df %>%}\StringTok{ }
\StringTok{               }\KeywordTok{group_by}\NormalTok{(v1,v2) %>%}\StringTok{ }
\StringTok{               }\KeywordTok{summarize}\NormalTok{(}
                 \DataTypeTok{x1 =} \KeywordTok{sum}\NormalTok{(}\KeywordTok{abs}\NormalTok{(x1)), }
                 \DataTypeTok{x2 =} \KeywordTok{sum}\NormalTok{(}\KeywordTok{abs}\NormalTok{(x2)), }
                 \DataTypeTok{x3 =} \KeywordTok{sum}\NormalTok{(}\KeywordTok{abs}\NormalTok{(x3)) }
                 \NormalTok{)}
             \NormalTok{)}
\end{Highlighting}
\end{Shaded}

\begin{verbatim}
##    user  system elapsed 
##   6.892   0.000   6.898
\end{verbatim}

And now the timing for \textbf{data.table}.

\begin{Shaded}
\begin{Highlighting}[]
\KeywordTok{system.time}\NormalTok{( }
  \NormalTok{tbl_dt[ ,  .( }\DataTypeTok{x1 =} \KeywordTok{sum}\NormalTok{(}\KeywordTok{abs}\NormalTok{(x1)), }\DataTypeTok{x2 =} \KeywordTok{sum}\NormalTok{(}\KeywordTok{abs}\NormalTok{(x2)), }\DataTypeTok{x3 =} \KeywordTok{sum}\NormalTok{(}\KeywordTok{abs}\NormalTok{(x3)) ), .(v1,v2)]}
  \NormalTok{)}
\end{Highlighting}
\end{Shaded}

\begin{verbatim}
##    user  system elapsed 
##   0.304   0.004   0.309
\end{verbatim}

The winner is obvious. Let's compare filtering (i.e.~row subsets,
i.e.~SQL's SELECT).

\begin{Shaded}
\begin{Highlighting}[]
\KeywordTok{system.time}\NormalTok{( }
  \NormalTok{tbl_df %>%}\StringTok{ }\KeywordTok{filter}\NormalTok{(v1 ==}\StringTok{ "1"}\NormalTok{) }
  \NormalTok{)}
\end{Highlighting}
\end{Shaded}

\begin{verbatim}
##    user  system elapsed 
##   0.012   0.000   0.012
\end{verbatim}

\begin{Shaded}
\begin{Highlighting}[]
\KeywordTok{system.time}\NormalTok{( }
  \NormalTok{tbl_dt[v1 ==}\StringTok{ "1"}\NormalTok{] }
  \NormalTok{)}
\end{Highlighting}
\end{Shaded}

\begin{verbatim}
##    user  system elapsed 
##   0.016   0.000   0.015
\end{verbatim}

\section{Computing from a Database}\label{computing-from-a-database}

The early solutions to oversized data relied on storing your data in
some DBMS such as \emph{MySQL}, \emph{PostgresSQL}, \emph{SQLite},
\emph{H2}, \emph{Oracle}, etc. Several R packages provide interfaces to
these DBMSs, such as \textbf{sqldf}, \textbf{RDBI}, \textbf{RSQite}.
Some will even include the DBMS as part of the package itself.

Storing your data in a DBMS has the advantage that you can typically
rely on DBMS providers to include very efficient algorithms for the
queries they support. On the downside, SQL queries may include a lot of
summary statistics, but will rarely include model fitting\footnote{This
  is slowly changing. Indeed, Microsoft's SQL Server 2016 is already
  providing
  \href{https://blogs.technet.microsoft.com/dataplatforminsider/2016/03/29/in-database-advanced-analytics-with-r-in-sql-server-2016/}{in-database-analytics},
  and other will surely follow.}. This means that even for simple things
like linear models, you will have to revert to R's facilities--
typically some sort of EMA with chunking from the DBMS. For this reason,
and others, we prefer to compute from efficient file structures, as
described in Section \ref{file-structure}.

If, however, you have a powerful DBMS around, or you only need summary
statistics, or you are an SQL master, keep reading.

The package \textbf{RSQLite} includes an SQLite server, which we now
setup for demonstration. The package \textbf{dplyr}, discussed in the
Hadleyverse Chapter \ref{hadley}, will take care of translating the
\textbf{dplyr} syntax, to the SQL syntax of the DBMS. The following
example is taken from the \textbf{dplyr}
\href{https://cran.r-project.org/web/packages/dplyr/vignettes/databases.html}{Databases
vignette}.

\begin{Shaded}
\begin{Highlighting}[]
\KeywordTok{library}\NormalTok{(RSQLite)}
\KeywordTok{library}\NormalTok{(dplyr)}

\KeywordTok{file.remove}\NormalTok{(}\StringTok{'my_db.sqlite3'}\NormalTok{)}
\end{Highlighting}
\end{Shaded}

\begin{verbatim}
## [1] TRUE
\end{verbatim}

\begin{Shaded}
\begin{Highlighting}[]
\NormalTok{my_db <-}\StringTok{ }\KeywordTok{src_sqlite}\NormalTok{(}\DataTypeTok{path =} \StringTok{"my_db.sqlite3"}\NormalTok{, }\DataTypeTok{create =} \OtherTok{TRUE}\NormalTok{)}

\KeywordTok{library}\NormalTok{(nycflights13)}
\NormalTok{flights_sqlite <-}\StringTok{ }\KeywordTok{copy_to}\NormalTok{(}
  \DataTypeTok{dest=} \NormalTok{my_db, }
  \DataTypeTok{df=} \NormalTok{flights, }
  \DataTypeTok{temporary =} \OtherTok{FALSE}\NormalTok{, }
  \DataTypeTok{indexes =} \KeywordTok{list}\NormalTok{(}\KeywordTok{c}\NormalTok{(}\StringTok{"year"}\NormalTok{, }\StringTok{"month"}\NormalTok{, }\StringTok{"day"}\NormalTok{), }\StringTok{"carrier"}\NormalTok{, }\StringTok{"tailnum"}\NormalTok{))}
\end{Highlighting}
\end{Shaded}

Things to note:

\begin{itemize}
\tightlist
\item
  \texttt{src\_sqlite} to start an empty table, managed by SQLite, at
  the desired path.
\item
  \texttt{copy\_to} copies data from R to the database.
\item
  Typically, setting up a DBMS like this makes no sense, since it
  requires loading the data into RAM, which is precisely what we want to
  avoid.
\end{itemize}

We can now start querying the DBMS.

\begin{Shaded}
\begin{Highlighting}[]
\KeywordTok{select}\NormalTok{(flights_sqlite, year:day, dep_delay, arr_delay)}
\end{Highlighting}
\end{Shaded}

\begin{verbatim}
## Source:   query [?? x 5]
## Database: sqlite 3.11.1 [my_db.sqlite3]
## 
##     year month   day dep_delay arr_delay
##    <int> <int> <int>     <dbl>     <dbl>
## 1   2013     1     1         2        11
## 2   2013     1     1         4        20
## 3   2013     1     1         2        33
## 4   2013     1     1        -1       -18
## 5   2013     1     1        -6       -25
## 6   2013     1     1        -4        12
## 7   2013     1     1        -5        19
## 8   2013     1     1        -3       -14
## 9   2013     1     1        -3        -8
## 10  2013     1     1        -2         8
## # ... with more rows
\end{verbatim}

\begin{Shaded}
\begin{Highlighting}[]
\KeywordTok{filter}\NormalTok{(flights_sqlite, dep_delay >}\StringTok{ }\DecValTok{240}\NormalTok{)}
\end{Highlighting}
\end{Shaded}

\begin{verbatim}
## Source:   query [?? x 19]
## Database: sqlite 3.11.1 [my_db.sqlite3]
## 
##     year month   day dep_time sched_dep_time dep_delay arr_time
##    <int> <int> <int>    <int>          <int>     <dbl>    <int>
## 1   2013     1     1      848           1835       853     1001
## 2   2013     1     1     1815           1325       290     2120
## 3   2013     1     1     1842           1422       260     1958
## 4   2013     1     1     2115           1700       255     2330
## 5   2013     1     1     2205           1720       285       46
## 6   2013     1     1     2343           1724       379      314
## 7   2013     1     2     1332            904       268     1616
## 8   2013     1     2     1412            838       334     1710
## 9   2013     1     2     1607           1030       337     2003
## 10  2013     1     2     2131           1512       379     2340
## # ... with more rows, and 12 more variables: sched_arr_time <int>,
## #   arr_delay <dbl>, carrier <chr>, flight <int>, tailnum <chr>,
## #   origin <chr>, dest <chr>, air_time <dbl>, distance <dbl>, hour <dbl>,
## #   minute <dbl>, time_hour <dbl>
\end{verbatim}

\begin{Shaded}
\begin{Highlighting}[]
\KeywordTok{summarise}\NormalTok{(flights_sqlite, }\DataTypeTok{delay =} \KeywordTok{mean}\NormalTok{(dep_time))}
\end{Highlighting}
\end{Shaded}

\begin{verbatim}
## Source:   query [?? x 1]
## Database: sqlite 3.11.1 [my_db.sqlite3]
## 
##     delay
##     <dbl>
## 1 1349.11
\end{verbatim}

\section{Computing From Efficient File
Structrures}\label{file-structure}

It is possible to save your data on your storage device, without the
DBMS layer to manage it. This has several advantages:

\begin{itemize}
\tightlist
\item
  You don't need to manage a DBMS.
\item
  You don't have the computational overhead of the DBMS.
\item
  You may optimize the file structure for statistical modelling, and not
  for join and summary operations, as in relational DBMSs.
\end{itemize}

There are several facilities that allow you to save and compute directly
from your storage:

\begin{enumerate}
\def\labelenumi{\arabic{enumi}.}
\item
  \textbf{Memory Mapping}: Where RAM addresses are mapped to a file on
  your storage. This extends the RAM to the capacity of your storage
  (HD, SSD,\ldots{}). Performance slightly deteriorates, but the access
  is typically very fast. This approach is implemented in the
  \textbf{bigmemory} package.
\item
  \textbf{Efficient Binaries}: Where the data is stored as a file on the
  storage device. The file is binary, with a well designed structure, so
  that chunking is easy. This approach is implemented in the \textbf{ff}
  package, and the commercial \textbf{RevoScaleR} package.
\end{enumerate}

Your algorithms need to be aware of the facility you are using. For this
reason each facility ( \textbf{bigmemory}, \textbf{ff},
\textbf{RevoScaleR},\ldots{}) has an eco-system of packages that
implement various statistical methods using that facility. As a general
rule, you can see which package builds on a package using the
\emph{Reverse Depends} entry in the package description. For the
\textbf{bigmemory} package, for instance,
\href{https://cran.r-project.org/web/packages/bigmemory/index.html}{we
can see} that the packages \textbf{bigalgebra}, \textbf{biganalytics},
\textbf{bigFastlm}, \textbf{biglasso}, \textbf{bigpca},
\textbf{bigtabulate}, \textbf{GHap}, and \textbf{oem}, build upon it. We
can expect this list to expand.

Here is a benchmark result, from \citet{wang2015statistical}. It can be
seen that \textbf{ff} and \textbf{bigmemory} have similar performance,
while \textbf{RevoScaleR} (RRE in the figure) outperforms them. This has
to do both with the efficiency of the binary representation, but also
because \textbf{RevoScaleR} is inherently parallel. More on this in the
Parallelization Chapter \ref{parallel}.
\includegraphics{art/benchmark.png}

\subsection{bigmemory}\label{bigmemory}

We now demonstrate the workflow of the \textbf{bigmemory} package. We
will see that \textbf{bigmemory}, with it's \texttt{big.matrix} object
is a very powerful mechanism. If you deal with big numeric matrices, you
will find it very useful. If you deal with big data frames, or any other
non-numeric matrix, \textbf{bigmemory} may not be the appropriate tool,
and you should try \textbf{ff}, or the commercial \textbf{RevoScaleR}.

\begin{Shaded}
\begin{Highlighting}[]
\CommentTok{# download.file("http://www.cms.gov/Research-Statistics-Data-and-Systems/Statistics-Trends-and-Reports/BSAPUFS/Downloads/2010_Carrier_PUF.zip", "2010_Carrier_PUF.zip")}
\CommentTok{# unzip(zipfile="2010_Carrier_PUF.zip")}

\KeywordTok{library}\NormalTok{(}\StringTok{"bigmemory"}\NormalTok{)}
\NormalTok{x <-}\StringTok{ }\KeywordTok{read.big.matrix}\NormalTok{(}\StringTok{"2010_BSA_Carrier_PUF.csv"}\NormalTok{, }\DataTypeTok{header =} \OtherTok{TRUE}\NormalTok{, }
                     \DataTypeTok{backingfile =} \StringTok{"airline.bin"}\NormalTok{, }
                     \DataTypeTok{descriptorfile =} \StringTok{"airline.desc"}\NormalTok{, }
                     \DataTypeTok{type =} \StringTok{"integer"}\NormalTok{)}
\KeywordTok{dim}\NormalTok{(x)}
\end{Highlighting}
\end{Shaded}

\begin{verbatim}
## [1] 2801660      11
\end{verbatim}

\begin{Shaded}
\begin{Highlighting}[]
\NormalTok{pryr::}\KeywordTok{object_size}\NormalTok{(x)}
\end{Highlighting}
\end{Shaded}

\begin{verbatim}
## 616 B
\end{verbatim}

\begin{Shaded}
\begin{Highlighting}[]
\KeywordTok{class}\NormalTok{(x)}
\end{Highlighting}
\end{Shaded}

\begin{verbatim}
## [1] "big.matrix"
## attr(,"package")
## [1] "bigmemory"
\end{verbatim}

Things to note:

\begin{itemize}
\tightlist
\item
  The basic building block of the \textbf{bigmemory} ecosystem, is the
  \texttt{big.matrix} class, we constructed with
  \texttt{read.big.matrix}.
\item
  \texttt{read.big.matrix} handles the import to R, and the saving to a
  memory mapped file. The implementation is such that at no point does R
  hold the data in RAM.
\item
  The memory mapped file will be there after the session is over. It can
  thus be called by other R sessions using
  \texttt{attach.big.matrix("airline.desc")}. This will be useful when
  parallelizing.
\item
  \texttt{pryr::object\_size} return the size of the object. Since
  \texttt{x} holds only the memory mappings, it is much smaller than the
  100MB of data that it holds.
\end{itemize}

We can now start computing with the data. Many statistical procedures
for the \texttt{big.matrix} object are provided by the
\textbf{biganalytics} package. In particular, the
\texttt{biglm.big.matrix} and \texttt{bigglm.big.matrix} functions,
provide an interface from \texttt{big.matrix} objects, to the EMA linear
models in \texttt{biglm::biglm} and \texttt{biglm::bigglm}.

\begin{Shaded}
\begin{Highlighting}[]
\KeywordTok{library}\NormalTok{(biganalytics)}
\NormalTok{biglm}\FloatTok{.2} \NormalTok{<-}\StringTok{ }\KeywordTok{bigglm.big.matrix}\NormalTok{(BENE_SEX_IDENT_CD~CAR_LINE_HCPCS_CD, }\DataTypeTok{data=}\NormalTok{x)}
\KeywordTok{coef}\NormalTok{(biglm}\FloatTok{.2}\NormalTok{)}
\end{Highlighting}
\end{Shaded}

\begin{verbatim}
##       (Intercept) CAR_LINE_HCPCS_CD 
##      1.537848e+00      1.210282e-07
\end{verbatim}

Other notable packages that operate with \texttt{big.matrix} objects
include:

\begin{itemize}
\tightlist
\item
  \textbf{bigtabulate}: Extend the bigmemory package with `table',
  `tapply', and `split' support for `big.matrix' objects.
\item
  \textbf{bigalgebra}: For matrix operation.
\item
  \textbf{bigpca}: principle components analysis (PCA), or singular
  value decomposition (SVD).
\item
  \textbf{bigFastlm}: for (fast) linear models.
\item
  \textbf{biglasso}: extends lasso and elastic nets.
\item
  \textbf{GHap}: Haplotype calling from phased SNP data.
\end{itemize}

\section{ff}\label{ff}

The \textbf{ff} packages replaces R's in-RAM storage mechanism with
on-disk (efficient) storage. Unlike \textbf{bigmemory}, \textbf{ff}
supports all of R vector types such as factors, and not only numeric.
Unlike \texttt{big.matrix}, which deals with (numeric) matrices, the
\texttt{ffdf} class can deal with data frames.

Here is an example. First open a connection to the file, without
actually importing it using the \texttt{LaF::laf\_open\_csv} function.

\begin{Shaded}
\begin{Highlighting}[]
\NormalTok{.dat <-}\StringTok{ }\NormalTok{LaF::}\KeywordTok{laf_open_csv}\NormalTok{(}\DataTypeTok{filename =} \StringTok{"2010_BSA_Carrier_PUF.csv"}\NormalTok{,}
                    \DataTypeTok{column_types =} \KeywordTok{c}\NormalTok{(}\StringTok{"integer"}\NormalTok{, }\StringTok{"integer"}\NormalTok{, }\StringTok{"categorical"}\NormalTok{, }\StringTok{"categorical"}\NormalTok{, }\StringTok{"categorical"}\NormalTok{, }\StringTok{"integer"}\NormalTok{, }\StringTok{"integer"}\NormalTok{, }\StringTok{"categorical"}\NormalTok{, }\StringTok{"integer"}\NormalTok{, }\StringTok{"integer"}\NormalTok{, }\StringTok{"integer"}\NormalTok{), }
                    \DataTypeTok{column_names =} \KeywordTok{c}\NormalTok{(}\StringTok{"sex"}\NormalTok{, }\StringTok{"age"}\NormalTok{, }\StringTok{"diagnose"}\NormalTok{, }\StringTok{"healthcare.procedure"}\NormalTok{, }\StringTok{"typeofservice"}\NormalTok{, }\StringTok{"service.count"}\NormalTok{, }\StringTok{"provider.type"}\NormalTok{, }\StringTok{"servicesprocessed"}\NormalTok{, }\StringTok{"place.served"}\NormalTok{, }\StringTok{"payment"}\NormalTok{, }\StringTok{"carrierline.count"}\NormalTok{), }
                    \DataTypeTok{skip =} \DecValTok{1}\NormalTok{)}
\end{Highlighting}
\end{Shaded}

Now write the data to local storage as an ff data frame, using
\texttt{laf\_to\_ffdf}.

\begin{Shaded}
\begin{Highlighting}[]
\NormalTok{data.ffdf <-}\StringTok{ }\NormalTok{ffbase::}\KeywordTok{laf_to_ffdf}\NormalTok{(}\DataTypeTok{laf =} \NormalTok{.dat)}
\KeywordTok{head}\NormalTok{(data.ffdf)}
\end{Highlighting}
\end{Shaded}

\begin{verbatim}
## ffdf (all open) dim=c(2801660,6), dimorder=c(1,2) row.names=NULL
## ffdf virtual mapping
##                              PhysicalName VirtualVmode PhysicalVmode  AsIs
## sex                                   sex      integer       integer FALSE
## age                                   age      integer       integer FALSE
## diagnose                         diagnose      integer       integer FALSE
## healthcare.procedure healthcare.procedure      integer       integer FALSE
## typeofservice               typeofservice      integer       integer FALSE
## service.count               service.count      integer       integer FALSE
##                      VirtualIsMatrix PhysicalIsMatrix PhysicalElementNo
## sex                            FALSE            FALSE                 1
## age                            FALSE            FALSE                 2
## diagnose                       FALSE            FALSE                 3
## healthcare.procedure           FALSE            FALSE                 4
## typeofservice                  FALSE            FALSE                 5
## service.count                  FALSE            FALSE                 6
##                      PhysicalFirstCol PhysicalLastCol PhysicalIsOpen
## sex                                 1               1           TRUE
## age                                 1               1           TRUE
## diagnose                            1               1           TRUE
## healthcare.procedure                1               1           TRUE
## typeofservice                       1               1           TRUE
## service.count                       1               1           TRUE
## ffdf data
##           sex   age diagnose healthcare.procedure typeofservice
## 1       1     1        NA                   99213         M1B  
## 2       1     1        NA                   A0425         O1A  
## 3       1     1        NA                   A0425         O1A  
## 4       1     1        NA                   A0425         O1A  
## 5       1     1        NA                   A0425         O1A  
## 6       1     1        NA                   A0425         O1A  
## 7       1     1        NA                   A0425         O1A  
## 8       1     1        NA                   A0425         O1A  
## :           :     :        :                    :             :
## 2801653 2     6        V82                  85025         T1D  
## 2801654 2     6        V82                  87186         T1H  
## 2801655 2     6        V82                  99213         M1B  
## 2801656 2     6        V82                  99213         M1B  
## 2801657 2     6        V82                  A0429         O1A  
## 2801658 2     6        V82                  G0328         T1H  
## 2801659 2     6        V86                  80053         T1B  
## 2801660 2     6        V88                  76856         I3B  
##         service.count
## 1               1    
## 2               1    
## 3               1    
## 4               2    
## 5               2    
## 6               3    
## 7               3    
## 8               4    
## :                   :
## 2801653         1    
## 2801654         1    
## 2801655         1    
## 2801656         1    
## 2801657         1    
## 2801658         1    
## 2801659         1    
## 2801660         1
\end{verbatim}

We can verify that the \texttt{ffdf} data frame has a small RAM
footprint.

\begin{Shaded}
\begin{Highlighting}[]
\NormalTok{pryr::}\KeywordTok{object_size}\NormalTok{(data.ffdf)}
\end{Highlighting}
\end{Shaded}

\begin{verbatim}
## 343 kB
\end{verbatim}

The \textbf{ffbase} package provides several statistical tools to
compute with \texttt{ff} class objects. Here is simple table.

\begin{Shaded}
\begin{Highlighting}[]
\NormalTok{ffbase:::}\KeywordTok{table.ff}\NormalTok{(data.ffdf$age) }
\end{Highlighting}
\end{Shaded}

\begin{verbatim}
## 
##      1      2      3      4      5      6 
## 517717 495315 492851 457643 419429 418705
\end{verbatim}

The EMA implementation of \texttt{biglm::biglm} and
\texttt{biglm::bigglm} have their \textbf{ff} versions.

\begin{Shaded}
\begin{Highlighting}[]
\KeywordTok{library}\NormalTok{(biglm)}
\NormalTok{mymodel.ffdf <-}\StringTok{ }\KeywordTok{biglm}\NormalTok{(payment ~}\StringTok{ }\KeywordTok{factor}\NormalTok{(sex) +}\StringTok{ }\KeywordTok{factor}\NormalTok{(age) +}\StringTok{ }\NormalTok{place.served, }
                              \DataTypeTok{data =} \NormalTok{data.ffdf)}
\KeywordTok{summary}\NormalTok{(mymodel.ffdf)}
\end{Highlighting}
\end{Shaded}

\begin{verbatim}
## Large data regression model: biglm(payment ~ factor(sex) + factor(age) + place.served, data = data.ffdf)
## Sample size =  2801660 
##                 Coef    (95%     CI)     SE      p
## (Intercept)  97.3313 96.6412 98.0214 0.3450 0.0000
## factor(sex)2 -4.2272 -4.7169 -3.7375 0.2449 0.0000
## factor(age)2  3.8067  2.9966  4.6168 0.4050 0.0000
## factor(age)3  4.5958  3.7847  5.4070 0.4056 0.0000
## factor(age)4  3.8517  3.0248  4.6787 0.4135 0.0000
## factor(age)5  1.0498  0.2030  1.8965 0.4234 0.0132
## factor(age)6 -4.8313 -5.6788 -3.9837 0.4238 0.0000
## place.served -0.6132 -0.6253 -0.6012 0.0060 0.0000
\end{verbatim}

Things to note:

\begin{itemize}
\tightlist
\item
  \texttt{biglm::biglm} notices the input of of class \texttt{ffdf} and
  calls the appropriate implementation.
\item
  The model formula,
  \texttt{payment\ \textasciitilde{}\ factor(sex)\ +\ factor(age)\ +\ place.served},
  includes factors which cause no difficulty.
\item
  You cannot inspect the factor coding (dummy? effect?) using
  \texttt{model.matrix}. This is because EMAs never really construct the
  whole matrix, let alone, save it in memory.
\end{itemize}

\section{Computing from a Distributed File
System}\label{computing-from-a-distributed-file-system}

If your data is SOOO big that it cannot fit on your local storage, you
will need a distributed file system or DBMS. We do not cover this topic
here, and refer the reader to the \textbf{RHipe}, \textbf{RHadoop}, and
\textbf{RSpark} packages and references therein.

\section{Bibliographic Notes}\label{bibliographic-notes-11}

An absolute SUPERB review on computing with big data is
\citet{wang2015statistical}, and references therein
(\citet{kane2013scalable} in particular). For an up-to-date list of the
packages that deal with memory constraints, see the \textbf{Large memory
and out-of-memory data} section in the High Performance Computing
\href{https://cran.r-project.org/web/views/HighPerformanceComputing.html}{R
task view}.

\section{Practice Yourself}\label{practice-yourself-13}

\chapter{Parallel Computing}\label{parallel}

You would think that because you have an expensive multicore computer
your computations will speed up. Well, no. At least not if you don't
make sure they do. By default, no matter how many cores you have, the
operating system will allocate each R session to a single core.

For starters, we need to distinguish between two types of parallelism:

\begin{enumerate}
\def\labelenumi{\arabic{enumi}.}
\tightlist
\item
  \textbf{Explicit parallelism}: where the user handles the
  parallelisation.
\item
  \textbf{Implicit parallelism}: where the parallelisation is abstracted
  away from the user.
\end{enumerate}

Clearly, implicit parallelism is more desirable, but the state of
mathematical computing is such that no sufficiently general implicit
parallelism framework exists. The
\href{https://www.r-consortium.org/projects/awarded-projects}{R
Consortium} is currently financing a major project for a A ``Unified
Framework For Distributed Computing in R'' so we can expect things to
change soon. In the meanwhile, most of the parallel implementations are
explicit.

\section{Explicit Parallelism}\label{explicit-parallelism}

R provides many frameworks for explicit parallelism. Because the
parallelism is initiated by the user, we first need to decide
\textbf{when to parallelize?} As a rule of thumb, you want to
parallelise when you encounter a CPU bottleneck, and not a memory
bottleneck. Memory bottlenecks are released with sparsity (Chapter
\ref{sparse}), or efficient memory usage (Chapter \ref{memory}).

Several ways to diagnose your bottleneck include:

\begin{itemize}
\tightlist
\item
  Keep your Windows Task Manager, or Linux \texttt{top} open, and look
  for the CPU load, and RAM loads.
\item
  The computation takes a long time, and when you stop it pressing ESC,
  R is immediately responsive. If it is not immediately responsive, you
  have a memory bottleneck.
\item
  Profile your code. See Hadley's
  \href{http://adv-r.had.co.nz/Profiling.html}{guide}.
\end{itemize}

For reasons detailed in \citet{kane2013scalable}, we will present the
\textbf{foreach} parallelisation package \citep{foreach}. It will allow
us to:

\begin{enumerate}
\def\labelenumi{\arabic{enumi}.}
\item
  Decouple between our parallel algorithm and the parallelisation
  mechanism: we write parallelisable code once, and can then switch the
  underlying parallelisation mechanism.
\item
  Combine with the \texttt{big.matrix} object from Chapter \ref{memory}
  for \emph{shared memory parallisation}: all the machines may see the
  same data, so that we don't need to export objects from machine to
  machine.
\end{enumerate}

What do we mean by ``switch the underlying parallesation mechanism''? It
means there are several packages that will handle communication between
machines. Some are very general and will work on any cluster. Some are
more specific and will work only on a single multicore machine (not a
cluster) with a particular operating system. These mechanisms include
\textbf{multicore}, \textbf{snow}, \textbf{parallel}, and \textbf{Rmpi}.
The compatibility between these mechanisms and \textbf{foreach} is
provided by another set of packages: \textbf{doMC} , \textbf{doMPI},
\textbf{doRedis}, \textbf{doParallel}, and \textbf{doSNOW}.

\BeginKnitrBlock{remark}
\iffalse {Remark. } \fi I personally prefer the \textbf{multicore}
mechanism, with the \textbf{doMC} adapter for \textbf{foreach}. I will
not use this combo, however, because \textbf{multicore} will not work on
Windows machines. I will thus use the more general \textbf{snow} and
\textbf{doParallel} combo. If you do happen to run on Linux, or Unix,
you will want to replace all \textbf{doParallel} functionality with
\textbf{doMC}.
\EndKnitrBlock{remark}

Let's start with a simple example, taken from
\href{http://debian.mc.vanderbilt.edu/R/CRAN/web/packages/doParallel/vignettes/gettingstartedParallel.pdf}{``Getting
Started with doParallel and foreach''}.

\begin{Shaded}
\begin{Highlighting}[]
\KeywordTok{library}\NormalTok{(doParallel)}
\NormalTok{cl <-}\StringTok{ }\KeywordTok{makeCluster}\NormalTok{(}\DecValTok{2}\NormalTok{)}
\KeywordTok{registerDoParallel}\NormalTok{(cl)}
\NormalTok{result <-}\StringTok{ }\KeywordTok{foreach}\NormalTok{(}\DataTypeTok{i=}\DecValTok{1}\NormalTok{:}\DecValTok{3}\NormalTok{) %dopar%}\StringTok{ }\KeywordTok{sqrt}\NormalTok{(i)}
\KeywordTok{class}\NormalTok{(result)}
\end{Highlighting}
\end{Shaded}

\begin{verbatim}
## [1] "list"
\end{verbatim}

\begin{Shaded}
\begin{Highlighting}[]
\NormalTok{result}
\end{Highlighting}
\end{Shaded}

\begin{verbatim}
## [[1]]
## [1] 1
## 
## [[2]]
## [1] 1.414214
## 
## [[3]]
## [1] 1.732051
\end{verbatim}

Things to note:

\begin{itemize}
\tightlist
\item
  \texttt{makeCluster} creates an object with the information our
  cluster. On a single machine it is very simple. On a cluster of
  machines, you will need to specify the i.p.~addresses or other
  identifiers of the machines.
\item
  \texttt{registerDoParallel} is used to inform the \textbf{foreach}
  package of the presence of our cluster.
\item
  The \texttt{foreach} function handles the looping. In particular note
  the \texttt{\%dopar} operator that ensures that looping is in
  parallel. \texttt{\%dopar\%} can be replaced by \texttt{\%do\%} if you
  want serial looping (like the \texttt{for} loop), for instance, for
  debugging.
\item
  The output of the various machines is collected by \texttt{foreach} to
  a list object.
\item
  In this simple example, no data is shared between machines so we are
  not putting the shared memory capabilities to the test.
\item
  We can check how many workers were involved using the
  \texttt{getDoParWorkers()} function.
\item
  We can check the parallelisation mechanism used with the
  \texttt{getDoParName()} function.
\end{itemize}

Here is a more involved example. We now try to make
\href{https://en.wikipedia.org/wiki/Bootstrapping_(statistics)}{Bootstrap}
inference on the coefficients of a logistic regression. Bootstrapping
means that in each iteration, we resample the data, and refit the model.

\begin{Shaded}
\begin{Highlighting}[]
\NormalTok{x <-}\StringTok{ }\NormalTok{iris[}\KeywordTok{which}\NormalTok{(iris[,}\DecValTok{5}\NormalTok{] !=}\StringTok{ "setosa"}\NormalTok{), }\KeywordTok{c}\NormalTok{(}\DecValTok{1}\NormalTok{,}\DecValTok{5}\NormalTok{)]}
\NormalTok{trials <-}\StringTok{ }\FloatTok{1e4}
\NormalTok{ptime <-}\StringTok{ }\KeywordTok{system.time}\NormalTok{(\{}
 \NormalTok{r <-}\StringTok{ }\KeywordTok{foreach}\NormalTok{(}\KeywordTok{icount}\NormalTok{(trials), }\DataTypeTok{.combine=}\NormalTok{cbind) %dopar%}\StringTok{ }\NormalTok{\{}
 \NormalTok{ind <-}\StringTok{ }\KeywordTok{sample}\NormalTok{(}\DecValTok{100}\NormalTok{, }\DecValTok{100}\NormalTok{, }\DataTypeTok{replace=}\OtherTok{TRUE}\NormalTok{)}
 \NormalTok{result1 <-}\StringTok{ }\KeywordTok{glm}\NormalTok{(x[ind,}\DecValTok{2}\NormalTok{]~x[ind,}\DecValTok{1}\NormalTok{], }\DataTypeTok{family=}\KeywordTok{binomial}\NormalTok{(logit))}
 \KeywordTok{coefficients}\NormalTok{(result1)}
 \NormalTok{\}}
 \NormalTok{\})[}\DecValTok{3}\NormalTok{]}
\end{Highlighting}
\end{Shaded}

\begin{verbatim}
## Warning: closing unused connection 10 (<-localhost:11481)
\end{verbatim}

\begin{verbatim}
## Warning: closing unused connection 9 (<-localhost:11481)
\end{verbatim}

\begin{verbatim}
## Warning: closing unused connection 8 (<-localhost:11481)
\end{verbatim}

\begin{verbatim}
## Warning: closing unused connection 7 (<-localhost:11481)
\end{verbatim}

\begin{Shaded}
\begin{Highlighting}[]
\NormalTok{ptime}
\end{Highlighting}
\end{Shaded}

\begin{verbatim}
## elapsed 
##  24.491
\end{verbatim}

Things to note:

\begin{itemize}
\tightlist
\item
  As usual, we use the \texttt{foreach} function with the
  \texttt{\%dopar\%} operator to loop in parallel.
\item
  The \texttt{icounts} function generates a counter.
\item
  The \texttt{.combine=cbind} argument tells the \texttt{foreach}
  function how to combine the output of different machines, so that the
  returned object is not the default list.
\end{itemize}

How long would that have taken in a simple (serial) loop? We only need
to replace \texttt{\%dopar\%} with \texttt{\%do\%} to test.

\begin{Shaded}
\begin{Highlighting}[]
\NormalTok{stime <-}\StringTok{ }\KeywordTok{system.time}\NormalTok{(\{}
 \NormalTok{r <-}\StringTok{ }\KeywordTok{foreach}\NormalTok{(}\KeywordTok{icount}\NormalTok{(trials), }\DataTypeTok{.combine=}\NormalTok{cbind) %do%}\StringTok{ }\NormalTok{\{}
 \NormalTok{ind <-}\StringTok{ }\KeywordTok{sample}\NormalTok{(}\DecValTok{100}\NormalTok{, }\DecValTok{100}\NormalTok{, }\DataTypeTok{replace=}\OtherTok{TRUE}\NormalTok{)}
 \NormalTok{result1 <-}\StringTok{ }\KeywordTok{glm}\NormalTok{(x[ind,}\DecValTok{2}\NormalTok{]~x[ind,}\DecValTok{1}\NormalTok{], }\DataTypeTok{family=}\KeywordTok{binomial}\NormalTok{(logit))}
 \KeywordTok{coefficients}\NormalTok{(result1)}
 \NormalTok{\}}
 \NormalTok{\})[}\DecValTok{3}\NormalTok{]}
\NormalTok{stime}
\end{Highlighting}
\end{Shaded}

\begin{verbatim}
## elapsed 
##  34.684
\end{verbatim}

Yes. Parallelising is clearly faster.

Let's see how we can combine the power of \textbf{bigmemory} and
\textbf{foreach} by creating a file mapped \texttt{big.matrix} object,
which is shared by all machines. The following example is taken from
\citet{kane2013scalable}, and uses the \texttt{big.matrix} object we
created in Chapter \ref{memory}.

\begin{Shaded}
\begin{Highlighting}[]
\KeywordTok{library}\NormalTok{(bigmemory)}
\NormalTok{x <-}\StringTok{ }\KeywordTok{attach.big.matrix}\NormalTok{(}\StringTok{"airline.desc"}\NormalTok{)}

\KeywordTok{library}\NormalTok{(foreach)}
\KeywordTok{library}\NormalTok{(doSNOW)}
\NormalTok{cl <-}\StringTok{ }\KeywordTok{makeSOCKcluster}\NormalTok{(}\KeywordTok{rep}\NormalTok{(}\StringTok{"localhost"}\NormalTok{, }\DecValTok{4}\NormalTok{)) }\CommentTok{# make a cluster of 4 machines}
\KeywordTok{registerDoSNOW}\NormalTok{(cl) }\CommentTok{# register machines for foreach()}
\end{Highlighting}
\end{Shaded}

Get a ``description'' of the \texttt{big.matrix} object that will be
used to call it from each machine.

\begin{Shaded}
\begin{Highlighting}[]
\NormalTok{xdesc <-}\StringTok{ }\KeywordTok{describe}\NormalTok{(x) }
\end{Highlighting}
\end{Shaded}

Split the data along values of \texttt{BENE\_AGE\_CAT\_CD}.

\begin{Shaded}
\begin{Highlighting}[]
\NormalTok{G <-}\StringTok{ }\KeywordTok{split}\NormalTok{(}\DecValTok{1}\NormalTok{:}\KeywordTok{nrow}\NormalTok{(x), x[, }\StringTok{"BENE_AGE_CAT_CD"}\NormalTok{]) }
\end{Highlighting}
\end{Shaded}

Define a function that computes quantiles of
\texttt{CAR\_LINE\_ICD9\_DGNS\_CD}.

\begin{Shaded}
\begin{Highlighting}[]
\NormalTok{GetDepQuantiles <-}\StringTok{ }\NormalTok{function(rows, data) \{}
 \KeywordTok{quantile}\NormalTok{(data[rows, }\StringTok{"CAR_LINE_ICD9_DGNS_CD"}\NormalTok{], }\DataTypeTok{probs =} \KeywordTok{c}\NormalTok{(}\FloatTok{0.5}\NormalTok{, }\FloatTok{0.9}\NormalTok{, }\FloatTok{0.99}\NormalTok{),}
 \DataTypeTok{na.rm =} \OtherTok{TRUE}\NormalTok{)}
\NormalTok{\}}
\end{Highlighting}
\end{Shaded}

We are all set up to loop, in parallel, and compute quantiles of
\texttt{CAR\_LINE\_ICD9\_DGNS\_CD} for each value of
\texttt{BENE\_AGE\_CAT\_CD}.

\begin{Shaded}
\begin{Highlighting}[]
\NormalTok{qs <-}\StringTok{ }\KeywordTok{foreach}\NormalTok{(}\DataTypeTok{g =} \NormalTok{G, }\DataTypeTok{.combine =} \NormalTok{rbind) %dopar%}\StringTok{ }\NormalTok{\{}
 \KeywordTok{require}\NormalTok{(}\StringTok{"bigmemory"}\NormalTok{)}
 \NormalTok{x <-}\StringTok{ }\KeywordTok{attach.big.matrix}\NormalTok{(xdesc)}
 \KeywordTok{GetDepQuantiles}\NormalTok{(}\DataTypeTok{rows =} \NormalTok{g, }\DataTypeTok{data =} \NormalTok{x)}
\NormalTok{\}}
\NormalTok{qs}
\end{Highlighting}
\end{Shaded}

\begin{verbatim}
##          50% 90% 99%
## result.1 558 793 996
## result.2 518 789 996
## result.3 514 789 996
## result.4 511 789 996
## result.5 511 790 996
## result.6 518 796 995
\end{verbatim}

\section{Implicit Parallelism}\label{implicit-parallelism}

We will not elaborate on implicit parallelism except mentioning the
following:

\begin{itemize}
\tightlist
\item
  You can enjoy parallel linear algebra by replacing the linear algebra
  libraries with BLAS and LAPACK as described
  \href{https://www.r-bloggers.com/faster-r-through-better-blas/}{here}.
\item
  You should read the ``Parallel computing: Implicit parallelism''
  section in the excellent
  \href{https://cran.r-project.org/web/views/HighPerformanceComputing.html}{High
  Performance Computing} task view, for the latest developments in
  implicit parallelism.
\end{itemize}

\section{Bibliographic Notes}\label{bibliographic-notes-12}

For a brief and excellent explanation on parallel computing in R see
\citet{schmidberger2009state}. For a full review see
\citet{chapple2016mastering}. For an up-to-date list of packages
supporting parallel programming see the High Performance Computing
\href{https://cran.r-project.org/web/views/HighPerformanceComputing.html}{R
task view}.

\section{Practice Yourself}\label{practice-yourself-14}

\chapter{Numerical Linear Algebra}\label{algebra}

In your algebra courses you would write \(Ax=b\) and solve
\(x=A^{-1}b\). This is useful to understand the algebraic properties of
\(x\), but a computer would never recover \(x\) that way. Even the
computation of the sample variance,
\(S^2(x)=(n-1)^{-1}\sum (x_i-\bar x)^2\) is not solved that way in a
computer, because of numerical and speed considerations.

In this chapter, we discuss several ways a computer solves systems of
linear equations, with their application to statistics, namely, to OLS
problems.

\section{LU Factorization}\label{lu-factorization}

\BeginKnitrBlock{definition}[LU Factorization]
\protect\hypertarget{def:lu}{}{\label{def:lu} \iffalse (LU Factorization)
\fi }For some matrix \(A\), the LU factorization is defined as

\begin{align}
 A = L U 
\end{align}

where L is unit lower triangular and U is upper triangular.
\EndKnitrBlock{definition}

The LU factorization is essentially the matrix notation for the
\href{https://en.wikipedia.org/wiki/Gaussian_elimination}{Gaussian
elimination} you did in your introductory algebra courses.

For a square \(n \times n\) matrix, the LU factorization requires
\(n^3/3\) operations, and stores \(n^2+n\) elements in memory.

\section{Cholesky Factorization}\label{cholesky-factorization}

\BeginKnitrBlock{definition}[Non Negative Matrix]
\protect\hypertarget{def:nonnegative}{}{\label{def:nonnegative}
\iffalse (Non Negative Matrix) \fi }A matrix \(A\) is said to be
\emph{non-negative} if \(x'Ax \geq 0\) for all \(x\).
\EndKnitrBlock{definition}

Seeing the matrix \(A\) as a function, non-negative matrices can be
thought of as functions that generalize the \emph{squaring} operation.

\BeginKnitrBlock{definition}[Cholesky Factorization]
\protect\hypertarget{def:cholesky}{}{\label{def:cholesky} \iffalse (Cholesky
Factorization) \fi }For some non-negative matrix \(A\), the Cholesky
factorization is defined as

\begin{align}
 A = T'T 
\end{align}

where T is upper triangular with positive diagonal elements.
\EndKnitrBlock{definition}

For obvious reasons, the Cholesky factorization is known as the
\emph{square root} of a matrix.

Because Cholesky is less general than LU, it is also more efficient. It
can be computed in \(n^3/6\) operations, and requires storing
\(n(n+1)/2\) elements.

\section{QR Factorization}\label{qr-factorization}

\BeginKnitrBlock{definition}[QR Factorization]
\protect\hypertarget{def:qr}{}{\label{def:qr} \iffalse (QR Factorization)
\fi }For some matrix \(A\), the QR factorization is defined as

\begin{align}
 A = QR 
\end{align}

where Q is orthogonal and R is upper triangular.
\EndKnitrBlock{definition}

The QR factorization is very useful to solve the OLS problem as we will
see in \ref{solving-ols}. The QR factorization takes \(2n^3/3\)
operations to compute. Three major methods for computing the QR
factorization exist. These rely on \emph{Householder transformations},
\emph{Givens transformations}, and a (modified) \emph{Gram-Schmidt
procedure} \citep{gentle2012numerical}.

\section{Singular Value
Factorization}\label{singular-value-factorization}

\BeginKnitrBlock{definition}[SVD]
\protect\hypertarget{def:svd}{}{\label{def:svd} \iffalse (SVD) \fi }For an
arbitrary \(n\times m\) matrix \(A\), the \emph{singular valued
decomposition} (SVD), is defined as

\begin{align}
 A = U \Sigma V' 
\end{align}

where U is an orthonormal \(n \times n\) matrix, V is an \(m \times m\)
orthonormal matrix, and \(\Sigma\) is diagonal.
\EndKnitrBlock{definition}

The SVD factorization is very useful for algebraic analysis, but less so
for computations. This is because it is (typically) solved via the QR
factorization.

\section{Iterative Methods}\label{iterative-methods}

The various matrix factorizations above may be used to solve a system of
linear equations, and in particular, the OLS problem. There is, however,
a very different approach to solving systems of linear equations. This
approach relies on the fact that solutions of linear systems of
equations, can be cast as optimization problems: simply find \(x\) by
minimizing \(\Vert Ax-b \Vert\).

Some methods for solving (convex) optimization problems are reviewed in
the Convex Optimization Chapter \ref@(convex). For our purposes we will
just mention that historically (this means in the \texttt{lm} function,
and in the LAPACK numerical libraries) the factorization approach was
preferred, and now optimization approaches are preferred. This is
because the optimization approach is more numerically stable, and easier
to parallelize.

\section{Solving the OLS Problem}\label{solving-ols}

Recalling the OLS problem in Eq.\eqref{eq:ols-matrix}: we wish to find
\(\beta\) such that\\

\begin{align}
  \hat \beta:= argmin_\beta \{ \Vert y-X\beta \Vert^2_2 \}.
\end{align}

The solution, \(\hat \beta\) that solves this problem has to satisfy

\begin{align}
  X'X \beta = X'y.
  \label{eq:normal-equations}
\end{align}

Eq.\eqref{eq:normal-equations} are known as the \emph{normal equations}.
The normal equations are the link between the OLS problem, and the
matrix factorization discussed above.

Using the QR decomposition in the normal equations we have that

\begin{align*}
  \hat \beta = R_{(1:p,1:p)}^{-1} y,
\end{align*}

where \((R_{n\times p})=(R_{(1:p,1:p)},0_{(p+1:n,1:p)})\) is the

\section{Bibliographic Notes}\label{bibliographic-notes-13}

For an excellent introduction to numerical algorithms in statistics, see
\citet{weihs2013foundations}. For an emphasis on numerical linear
algebra, see \citet{gentle2012numerical}, and \citet{golub2012matrix}.

\section{Practice Yourself}\label{practice-yourself-15}

\chapter{Convex Optimization}\label{convex}

\section{Bibliographic Notes}\label{bibliographic-notes-14}

\section{Practice Yourself}\label{practice-yourself-16}

\chapter{RCpp}\label{rcpp}

\section{Bibliographic Notes}\label{bibliographic-notes-15}

\section{Practice Yourself}\label{practice-yourself-17}

\chapter{Debugging Tools}\label{debugging}

\section{Bibliographic Notes}\label{bibliographic-notes-16}

\section{Practice Yourself}\label{practice-yourself-18}

\chapter{Bibliography}\label{bib}

\bibliography{bib.bib}

\end{document}
